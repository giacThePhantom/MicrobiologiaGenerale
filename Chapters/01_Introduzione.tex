\chapter{Introduzione}
I microbi sono organismi unicellulari origine di tutte le forme di vita, mostrano una grande differenza tra di loro, maggiore di quella esistente tra piante
e animali, sono enormemente numerosi e ubiquitari. Trasformano e riciclano la materia organica e influenzano il clima. Hanno relaizoni simbiotiche con 
animali, piante e altri microorganismi. Alcuni sono patogeni. Possono sopravvivere a condizioni estreme:
\begin{itemize}
\item $5$ megarad di radiazioni gamma.
\item pH estremi: da $0$ a $11.4$.
\item Temperature estreme: da $-15$ a $121$ gradi centigradi.
\item Pressione idrostatica di $1300$ ATM.
\item Pressione osmotica corrispondente a $5.2$ di NaCl.
\end{itemize}
Si trovano sulla terra da molto prima della nascita di organismi pluricellulari. 
\section{Albero della vita}
Si chiama LUCA il last universal common ancestor, l'antenato comune a tutti i tre regni della vita, un organismo termofilo. Si nota come confrontando rRNA conservati gli eucarioti e 
gli archea si trovano vicini tra di loro mentre i batteri sono enormemente diversificati rispetto agli altri due. Oltre all'evoluzione in verticale nell'albero della vita possono 
accadere degli scambi in orizzontale tra specie molto distanti tra di loro. 
\subsection{Alcuni organismi}
\begin{itemize}
	\item I funghi sono eucarioti dotati di cellule con nucleo incapaci di fotosintesi ma possiedono parete cellulare. 
	\item Le muffe sono funghi filamentosi multicellulari, crescono come lunghi filamenti detti hyphae alle cui estremit\`a sono presenti spore. Possono riprodursi sia sessualmente
		che asessualmente, alcuni uccidono i batteri come Penicillum chrysogenum da cui si estrae la penicillina.
	\item I lieviti sono piccoli organismi unicellulari che si riproducono asessualmente per gemmazione. 
	\item Protozoi: sono eucarioti unicellulari che si muovono attraverso pseupodi, ciglia e flagelli in acqua o all'interno di organismi.
	\item Le alghe sono organismi unicellulari o pluricellulari che svolgono fotosintesi, producono emulsificanti e una gelatina usata per i terreni di coltura. 
\end{itemize}
\subsection{Batteri e archea}
Batteri e archea sono organismi procarioti, ovvero non hanno nucleo cellulare, possiedono una parete cellulare polisaccaride di peptidoglicano. Svolgono una
riproduzione asessuata e sono tipicamente dalle 10 alle 100 volte pi\`u piccoli delle cellule eucariote, nell'ordine dei micrometri. Gli archea sono presenti in ambienti inospitali ed
estremi. I batteri permettono la degradazione degli animali e il riciclo del materiale che li compone. Differiscono tra di loro per le propriet\`a chimiche della parete cellulare e 
delle membrane, i batteri sono sensibili agli antibiotici e possono essere patogeni, mentre gli archea non lo sono e utilizzano enzimi per produrre proteine e acidi nucleici simili 
agli eucarioti. 
\subsection{Virus}
I virus sono acellulari e costituiti da un materiale genetico a DNA o RNA, di un capside proteico e eventualmente di un ulteriore strato lipidico. Dipendono
dalla cellula ospite per la loro riproduzione e per questo non possono essere definiti come organismi viventi. 
\subsection{Caratteristiche degli esseri viventi}
\begin{itemize}
	\item Sono dotati di un metabolismo: la cellula \`e un sistema aperto che assume sostanze nutritive dall'esterno, le elabora ed espelle il materiale di scarto.
	\item Si riproducono e crescono: le sostanze chimiche assunte dall'ambiente sono utilizzate dalle cellule per fabbricarne di nuove.
	\item Sono differenziate: presentano strutture specializzate. 
	\item Comunicano attraverso sostanze chimiche rilasciate nell'ambiente. 
	\item Si muovono.
	\item Si evolvono: acquisiscono nuove propriet\`a biologiche, gli alberi filogenetici mostrano le relazioni evolutive tra le cellule. 
\end{itemize}
\subsection{Caratterizzazione dei microbi}
\begin{tabular}{|c|c|c|c|}
\hline
& Individuo & Popolazione & Comunit\`a \\
\hline
Ecologia & \makecell{Fisiologia:\\ differente espressione\\ di geni in risposta\\ a cambiamenti} & \makecell{Demografica:\\ nascita, morte,\\ immigrazione, 
emigrazione} & \makecell{Ecologia comunitaria:\\ interazioni interspecie che \\danno forma a struttura e\\ funzione della comunit\`a}\\
\hline
Genomica & \makecell{Mappatura fine\\ di singoli genomi} & \makecell{Genomica della popolazione:\\ analisi genomica comparativa\\ per determinare 
variazioni} & \makecell{Metagenomica:\\ potenziale genetico \\dei membri della comunit\`a}\\
\hline
Genetica & \makecell{Genetica dei batteri:\\ ruolo dei geni\\ sotto certe variazioni} & \makecell{Genetica della popolazione:\\ frequenza della 
distribuzione\\ degli alleli} & \makecell{Genetica comunitaria:\\ interazione tra la composizione\\ genetica della comunit\`a e le\\ propriet\`a della \\
comunit\`a ecologica}\\
\hline
\end{tabular}
\section{La macchina cellulare}
Le condizioni necessarie affinch\`e la cellula possa riprodursi comprendono un adeguato supporto energetico e la presenza di precursori per la sintesi di 
nuove macromolecole. Le istruzioni codificate nel genoma devono essere replicate in modo che ogni cellula figlia possa riceverne una copia. Infine i geni
devono essere espressi attraverso trascrizione e traduzione per formare le proteine e le macromolecole necessarie per dare origine a una nuova cellula.
\subsection{Impatto dei microbi sulle attivit\`a umane}
I microbi svolgono un ruolo fondamentale in varie attivit\`a umane:
\begin{itemize}
\item Agricoltura: fissazione di $N_2$ ($N_2\rightarrow 2NH_3$), necessario per il ciclo dei nutrienti, permettono ai ruminanti di consumare erba.
\item Cibo: preservazione del cibo, creazione di cibi fermentati e additivi.
\item Alcuni sono agenti patogeni.
\item Creazione di biofuels, bioremediation nel caso di petrolio disperso nell'ambiente e microbial mining.
\item Biotecnologie: produzione di organismi geneticamente modificati, produzione di prodotti farmaceutici, terapia genetica per certe malattie. 
\end{itemize}
\subsection{Ricombinazione del DNA}
I microbi sono utilizzati per ricombinare il DNA. Il DNA plasmidico e quello del donatore possono essere tagliati attraverso un'endonucleasi di restrizione
in modo da ottenere frammenti compatibili. Mescolando e legando il plasmide linearizzato il DNA estraneo digerito i frammenti sono incorporati nel 
plasmide formando un plasmide ricombinante che viene inserito in cellule batteriche. Quando si riproduce viene riprodotto anche il DNA estraneo. Se il 
donatore contiene un gene questo pu\`o essere espresso producendo una proteina eterologa. 
\section{Microrganismi come modello}
I microrganismi sono stati ampiamente utilizzati per la ricerca in quanto si replicano velocemente, sono economici da coltivare e hanno strutture 
relativamente semplici. Sono stati pertanto utilizzati per studiare i processi cellulari come replicazione del DNA, trascrizione e traduzione. 
\subsection{Il conflitto sulla generazione spontanea}
Fino all'esperimento di Redi si credeva che gli organismi viventi potessero svilupparsi da materia non vivente o in decomposizione. Questa teoria viene
confutata ponendo della carne in putrefazione in tre vasi: uno scoperto (con conseguenza di deposito di larve di mosca), uno sigillato (che rimase senza 
larve) e uno coperto da una garza (su cui le mosche, attratte dall'odore deposero le larve). Un altro esperimento \`e quello di Pasteur in cui si prende un infusione in un pallone a 
collo di cigno e la si sterilizza. L'apertura del collo permette il passaggio dell'aria fino al punto pi\`u basso dove si formano dei sedimenti. Si nota come l'infusione rimane sterile
fino a quando non la si fa entrare in contatto con i sedimenti. 
\subsection{Postulati di Koch}
\begin{enumerate}
\item Il microrganismo deve essere presente in tutti gli individui affetti dalla malattia e assente in quelli sani.
\item Il microrganismo deve essere isolato dall'individuo affetto e, posto in coltura, deve dare origine a una popolazione cellulare omogenea.
\item L'inoculo di una cultura pura del microrganismo in individui sani pu\`o causare la comparsa della malattia di cui \`e ritenuto responsabile. 
\item Il microrganismo deve essere reisolato dall'organismo infetto sperimentalmente in cui la malattia sia insorta.
\end{enumerate}
\subsubsection{I postulati di Koch molecolari}
\begin{enumerate}
\item Il gene implicato nella patogenicit\`a o virulenza deve trovarsi in tutti i ceppi patogeni di una data specie ed essere assente dalle specie non 
patogene.
\item L'inattivazione selettiva del gene deve portare a una diminuzione misurabile della patogenicit\`a o virulenza.
\item La complementazione o reversione della mutazione deve ripristinare il livello originale di patogenicit\`a o virulenza. Parimenti l'introduzione del
gene in un ceppo non patogeno lo trasforma in patogeno.
\end{enumerate}
\section{I batteri}
\subsection{Composizione elementare}
Le cellule batteriche sono composte per l'$8\%$ da idrogeno (H), per il $20\%$ da ossigeno (O), per il $50\%$ da carbonio (C), per il $14\%$ da azoto (N), 
per il $3\%$ da fosforo (P) e per l'$1\%$ da zolfo (S). Se lo zolfo si trova unicamente nelle proteine e il fosforo in proteine e lipidi e polisaccaridi gli
altri sono presenti in tutte le macromolecole che formano la cellula che sono:
\begin{itemize}
\item Polisaccaridi semplici e complessi per il $7\%$.
\item Lipidi e lipopolisaccaridi per l'$11\%$. 
\item Acidi nucleici per il $23\%$.
\item Proteine per il $55\%$.
\end{itemize}
\subsection{Strutture e loro funzioni}
\subsubsection{Membrana plasmatica}
La membrana plasmatica \`e una barriera dotata di permeabilit\`a selettiva. \`E il confine fisico della cellula, si occupa del trasporto di nutrienti e 
prodotti di rifiuto, \`e sede di molti processi metabolici come respirazione e fotosintesi e si occupa di rilevare gli stimoli ambientali per la 
chemiotassi.
\subsubsection{Vacuolo gassoso}
Il vacuolo gassoso garantisce la propriet\`a di galleggiamento in ambienti acquosi.
\subsubsection{Ribosomi}
I ribosomi si occupano della sintesi proteica. Sono composti principalmente da RNA e proteine.
\subsubsection{Corpi d'inclusione}
I corpi d'inclusione svolgono il compito di riserva di carbonio, fostato e altre sostanze. Sono molto variabili, composti tipicamente da carboidrati, 
lipidi, proteine e sostanze inorganiche.
\subsubsection{Nucleoide}
Il nucleoide \`e il sito del materiale genetico (DNA).
\subsubsection{Spazio periplasmatico}
Lo spazio periplasmatico contiene enzimi idrolitici e proteine per l'assorbimento dei nutrienti e il loro utilizzo metabolico. Composto da fosfolipidi e 
proteine. 
\subsubsection{Parete cellulare}
La parete cellulare conferisce ai batteri la loro forma caratteristica e li protegge dalla lisi in soluzione ipotoniche. Composta principalmente da 
peptidoglicano (mureina).
\subsubsection{Capsule e strati mucosi}
Le capsule e gli strati mucosi offrono resistenza alla fagocitosi e aderenza alle superfici. Sono composti da polisaccaridi o polipeptidi. 
\subsubsection{Fimbrie e pili}
Le fimbrie e pili permettono adesione alle superfici e coniugazione batterica (pili sessuali). Sono composti da proteine.
\subsubsection{Flagelli}
I flagelli si occupano del movimento. Sono composti da proteine.
\subsubsection{Endospora}
Le endospore consentono la sopravvivenza in condizioni ambientali molto avverse. 
\subsection{Classificazione}
Le pareti cellulari dei batteri assumono due tipi di strutture caratteristiche, permettendo una loro classificazione in Gram-negativi e Gram-positivi. Le differenze strutturali sono
responsabili del diverso comportamento rispetto al Gram-staining. Durante l'esperimento un complesso cristallino insolubile e violetto viene fatto formare nella cellula e poi 
estratto dall'alcol nei Gram-negativi ma non nei Gram-positivi. Questo avviene in quanto i secondi possiedono una parete cellulare molto spessa che quando disidratata dall'alcol chiude
i pori e non permette l'uscita del complesso cristallino. Nei primi invece l'alcol penetra facilmente la membrana esterna estraendo poi il complesso cristallino. 
