\chapter{La struttura della cellula}
\section{La parete cellulare}
Tutti i batteri possiedono una parete cellulare che svolge vari ruoli:
\begin{itemize}
	\item Strutturale: conferisce forma e rigidit\`a alla cellula.
	\item Protettivo: impedisce la lisi in un ambiente tipicamente ipotonico in quanto il citoplasma possiede un alta concentrazione di soluti e crea una pressione osmotica 
		nell'ordine delle $2$\si{atm}.
\end{itemize}
Oltre a permettere di capire i processi vitali di un procariote lo studio della parete cellulare ha permesso la sintesi di diversi antibiotici che la attaccano specificatamente in 
quanto assente nelle cellule dell'organismo infetto.
\subsection{Peptidoglicano}
La parete cellulare dei batteri \`e composta principalmente da peptidoglicano (mureina), un polisaccaride complesso composto da l'\emph{N-acetilglucosamide} (\emph{NAM}) e da 
\emph{N-acetilmuramico} (\emph{NAG}) che si alterano formando lunghi polisaccaridi formando un legame $\beta-1,4$. La sintesi delle catene avviene in zone concentrate, favorendo 
la formazione di fogli che circondano la cellula. Tra le catene si formano legami crociati tra $4$ amminoacidi legatisi in posizioni specifiche al \emph{NAM} creando ponti tetrapeptidici
la cui composizione \`e variabile (tipicamente \emph{L-alanina}, \emph{D-alanina}, \emph{D-acido glutammico} e \emph{acido diamino-pimelico}). 
\begin{itemize}
	\item Nei Gram negativi il legame crociato si forma tra \emph{DAP} e \emph{D-alanina}.
	\item Nei gram positivi il legame crociato si forma attraverso un ponte peptidico tipicamente di $5$ glicine.
\end{itemize}
\subsection{Gram positivi} 
I Gram positivi possiedono una parete costituita da uno strato solido di peptidoglicano anche $50$ volte pi\`u spesso e meno elaborato rispetto a quello dei Gram negativi. La
parete cellulare \`e formata per il $90\%$ da fogli di peptidoglicano sovrapposti. Si trovano nella parete oltre a proteine di trasporto e strutturali acidi teicoici e lipoteicoici. 
Questi sono formati da catene di alcol e zucchero come glicerolo o ribitolo unite da legami fosfodiesterici grazie a gruppi fosfato. Sono spesso associati ad altri zuccheri come
glucosio e \emph{D-alanina}. Si dicono lipoteicoici gli acidi teicoici contenenti glicerolo e pertanto legati con i lipidi della membrana citoplasmatica. Queste molecole possiedono una
carica negativa e sono in parte responsabili della carica elettrica negativa della superficie cellulare. Sono inoltre fondamentali per il trasporto di ioni calcio e magnesio. Gli
acidi lipoteicoici in particolare ancorano la parete cellulare alla membrana citoplasmatica. 
\subsection{Gram negativi}
I Gram negativi sono formati da due membrane, una interna e una esterna e nello spazio tra di esse, detto periplasmatico, si trova la parete cellulare di peptidoglicano. 
\subsubsection{Spazio periplasmatico}
Lo spazio periplasmatico \`e largo circa $15\si{\nano\metre}$, ha una consistenza simile ad un gel e contiene:
\begin{itemize}
	\item Enzimi idrolitici: con ruolo nella degradazione iniziale delle molecole alimentari.
	\item Proteine leganti: iniziano il processo di trasporto dei substrati.
	\item Chemiorecettori: che governano la risposta chemiotassica.
	\item Acqua.
	\item Nutrienti.
\end{itemize}
La maggior parte di questi elementi raggiunge il periplasma attraverso un sistema di esportazione di proteine presente nella membrana citoplasmatica. 
\subsubsection{Membrana esterna}
La membrana esterna presenta una forte resistenza contro composti chimici dannosi ed \`e formata nel lato interno da fosfolipidi e proteine, mentre in quello esterno da lipopolisaccaridi
\emph{LPS}. Questo \`e formato da tre parti:
\begin{itemize}
	\item Lipide $A$: un glicolipide formato da un disaccaride \emph{NAG} legato a fosfati ed acidi grassi grazie ad un legame estere-amminico. Questi disaccaridi sono legati al core
		polisaccaridico attraverso il chetodeossiottonato \emph{KDO}.
	\item Core polisaccaridico: oltre il legame con \emph{KDO} \`e formato da zuccheri a $7$ atomi di carbonio \emph{eptosi}, glucosio, galattosio e \emph{NAG}. 
	\item Polisaccaride $O$-specifico o antigene $O$: \`e una catena di zuccheri lunga fino a $40$ residui legata al core contenente galattosio, ramnosio e mannosio oltre a uno 
		o pi\`u dideossizuccheri (abequosio, colitosio, paratosio o tivelosio) legati a formare catene lunge $4$-$5$ elementi, spesso ramificate. La loro ripetizione porta alla
		sua formazione. 
\end{itemize}
\paragraph{Porine}
Le porine sono proteine transmembrana che attraversano completamente la membrana esterna e la rendono parzialmente permeabile a piccole molecole idrofiliche. Sono formate da tre 
subunit\`a identiche ciascuna con un poro di $1\si{\nano\metre}$ che non permette la fuoriuscita di enzimi del periplasma. Le porine si dividono in specifiche o aspecifiche in base
alla selettivit\`a della molecola a cui permettono il passaggio. Tipicamente le prime formano canali d'acqua attraverso i quali pu\`o passare qualsiasi piccola molecola, mentre le
seconde presentano un sito di legame specifico. 
\paragraph{Lipiproteine}
Le lipoproteine sono presenti nello strato interno della membrana esterna e servono da giunzione fra questa e il peptidoglicano.
\subsection{Lisozima}
L’enzima lisozima è una proteina che può rompere i legami $\beta(1, 4)$ degli zuccheri NAM-NAG.
\begin{itemize}
\item In una soluzione diluita (ipotonica) la degradazione della parete con il lisozima rilascia il protoplasto che va incontro a lisi in seguito 
all’ingresso di acqua nella cellula
\item In una soluzione isotonica non c’è nessun movimento netto di acqua tra ambiente e protoplasti ed essi rimangono stabili.
\end{itemize}
Esistono archea e micoplasmi che si trovano naturalmente nella forma di protoplasti in quanto vivono in habitat osmoticamente protetti e non necessitano una protezione contro la 
pressione osmotica. Sono pertanto spesso parassiti. 
\section{La membrana citoplasmatica}
La membrana citoplasmatica \`e spessa $8\si{\nano\metre}$ e circonda il citoplasma separandolo dall’ambiente. Nel caso in cui la membrana citoplasmatica sia compromessa, l’integrità 
della cellula viene distrutta, il citoplasma si disperde e di conseguenza il batterio muore. \`E una barriera altamente selettiva e funziona anche come sito di ancoraggio per proteine.
L'isolamento che fornisce con l'ambiente permette la creazione di un gradiente ionico negativo all'interno e positivo all'esterno che viene usato per generare energia o forza
proton-motrice. 
\subsection{La struttura}
La struttura generale della membrana è quella di un doppio strato fosfolipidico. I fosfolipidi sono molecole composte da una coda idrofobica (acidi grassi) e una testa idrofila 
(glicerolo-fosfato).
Quando i fosfolipidi si aggregano in una soluzione acquosa, formano naturalmente bistrati. In una membrana 
fosfolipidica, le code puntano verso il centro del bistrato per formare un ambiente idrofobico, e le teste rimangono esposte 
all'ambiente esterno o al citoplasma. Gli acidi grassi comuni nella membrana citoplasmatica hanno catene da  14 a 20 atomi di carbonio.
Le membrane citoplasmiche di alcuni batteri sono rafforzate da molecole simili allo sterolo chiamate opanoidi. Gli steroli sono molecole rigide e planari 
che funzionano per rafforzare le membrane delle cellule eucariote, e gli opanoidi svolgono una funzione simile nei batteri.
La membrana contiene inoltre un alto numero di proteine, che hanno tipicamente superfici idrofobe in regioni che attraversano la membrana e superfici 
idrofile in regioni che sono a contatto con l'ambiente e il citoplasma. Molte di esse sono saldamente incorporate nella membrana e sono chiamate proteine 
integrali. Altre hanno una porzione ancorata nelle regioni della membrana e dell'extramembrana che puntano dentro o fuori la cellula. Altre proteine ancora, 
chiamate proteine della membrana periferica, non sono incorporate nella membrana, ma rimangono comunque associate alla sua superficie. Alcune di queste 
proteine periferiche sono le lipoproteine, molecole che contengono una coda lipidica che ancora la proteina alla membrana. Le proteine della membrana 
periferica in genere interagiscono con le proteine integrali della membrana in importanti processi cellulari come il metabolismo energetico e il trasporto. 
Spesso le proteine che devono interagire tra loro in qualche processo sono tipicamente raggruppate in cluster per consentire loro di rimanere adiacenti 
l'una all'altra nell'ambiente semifluido della  membrana.
\section{Membrana e parete degli archaea}
In contrasto con i lipidi di batteri ed Eukarya in cui ci sono legami estere tra acidi grassi e glicerolo, i lipidi degli Archaea contengono legami etere 
tra glicerolo e le loro catene laterali idrofobiche (catena alifatica). I lipidi degli Archaea mancano quindi di acidi grassi, di per sé, anche se le catene 
laterali idrofobiche svolgono lo stesso ruolo funzionale degli acidi grassi. 
Le principali differenze sono le catene isopreniche e il fatto che alcune sono 
a monostrato invece che bistrato. A differenza dei bistrati lipidici, le membrane monostrato lipidiche sono estremamente resistenti al calore e sono quindi 
ampiamente distribuite tra Archaea ipertermofili, organismi che crescono a temperature superiori a 80 gradi centigradi. Esistono anche membrane con una 
miscela di carattere bistrato e monostrato, con alcuni dei gruppi idrofobici opposti covalentmente legati e altri no. Un’altra differenza tra i batteri e 
gli Archaea è la frequente assenza sia di una parete cellullare, sia di una membrana esterna, che vengono sostuititi da una grande varietà di diverse pareti 
cellullari, alcune delle quali hanno delle bio-componenti molto simili a quelle dei batteri, come polisaccaridi, proteine e glicoproteine. Alcuni archaea 
metanogeni (producono metano) hanno una parete costituita da pseudopeptidoglicano, costituita da NAG e, al posto di avere il NAM, ha 
l’acido N-acetiltalosaminuronica (NAT). Questo porta ad avere legami $\beta(1, 3)$, che li rende insensibili al lisozima e 
alla penicillina, che invece nei batteri distruggerebbero il peptidoglicano o ne arresterebbero la sua biosintesi. Alcuni archaea possiedono ulteriori polisaccaridi diversi anche dallo 
pseudopeptidoglicano.

\section{Flagelli}
I flagelli sono lunghe e sottili strutture di forma elicoidale che si estendono oltre il corpo del microorganismo.
La loro funzione principale \`e la locomozione: esercitano un movimento rotatorio e funzionano come organi propulsori.
La distanza tra curve adiacenti dell'elica \`e costante ma differisce per ogni specie.
	
	\subsection{Tipologie di flagelli}
	Esistono diversi tipi di flagelli:
	\begin{itemize}
		\item Peritrico: comprende i microorganismi con moltissimi flagelli che ricoprono tutto il corpo.
		\item Polari: si trovano a un'estremit\`a del microorganismo in numero variabile.
		\item Lofotrichi: gli organismi hanno pi\`u di un flagello in un'unica estremit\`a.
			A differenza dei polari possiedono pi\`u di un flagello nell'estremit\`a.
	\end{itemize}

	\subsection{Struttura}
	I flagelli sporgono dalla cellula per una lunghezza fino a $3$ volte del corpo che li contiene e sono divisi in:
	\begin{itemize}
		\item Filamento: struttura elicoidale formata da flagellina.
		\item Uncino: struttura incurvata a gancio che connette il filamento al corpo basale.
		\item Corpo basale: \`e formato da $15$ proteine aggregate a fare un bastoncello a cui sono attaccati anelli che tengono ancorata la struttura e le permettono di ruotare.
	\end{itemize}
	Gli anelli dei corpi basali sono $2$ nei Gram$+$ e $4$ nei Gram$-$, $2$ nella membrana plasmatica esterna e $2$ nella parete cellulare.

		\subsubsection{Corpo basale - Gram$\mathbf{-}$}
		Il corpo basale nei Gram$-$ \`e formato da $4$ anelli:
		\begin{multicols}{2}
			\begin{itemize}
				\item Anello $C$: lega il citoplasma.
				\item Anello $MS$: si trova all'interno della membrana citoplasmatica.
				\item Anello $P$: ancora la parete cellulare.
				\item Anello $L$: lipopolisaccaride.
			\end{itemize}
		\end{multicols}
		Attraverso questi anelli passa il bastoncello che completa il corpo basale.
		Due proteine di membrana circondano la struttura di base:
		\begin{multicols}{2}
			\begin{itemize}
				\item \emph{Mot}: le proteine motrici, pompano protoni nello spazio periplasmatico e utilizzano l'energia generata dal gradiente elettrochimico per creare una forza motrice con velocit\`a molto elevate.
				\item \emph{Fli}: funzionano da invertitore, ribaltano il senso di rotazione del flagello in risposta a segnali intracellulari.
			\end{itemize}
		\end{multicols}

	\subsection{Movimento flagellare}
	Il motore flagellare \`e composto da un rotore e uno statore.
	Il rotore \`e composto dagli anelli $C$, $MS$ e $P$, mentre lo statore dalle proteine \emph{Mot}.
	Il movimento del flagello viene impartito dal corpo basale e l'energia viene fornita dalla forza proton-motrice.
	Gli ioni \emph{$H^+$} attraversano \emph{Mot} conferendo il moto rotatorio al flagello.
	Per compiere un giro sono necessari \num{1000} protoni.

	\subsection{Sintesi flagellare}
	Nei Gram$-$ per la sintesi flagellare avviene in sequenza:
	\begin{multicols}{3}
		\begin{enumerate}
			\item Formazione degli anelli $MS$ e $C$ nella membrana interna.
			\item Formazione del bastoncello e sua chiusura.
			\item Formazione anello $P$.
			\item Formazione anello $L$.
			\item Formazione Hook.
			\item Formazione Cap.
			\item \num{20000} proteine di flagellina attraversano l'hook e formano il filamento.
		\end{enumerate}
	\end{multicols}

		\subsubsection{Cap}
		Il cap controlla la sintesi del filamento che avviene dall'alto verso il basso aiutando l'organizzazione della flagellina.

		\subsubsection{Flagellina}
		La flagellina \`e la proteina monomerica che forma il filamento.
		Viene formata all'interno del microorganismo e trasportata all'esterno andando a formare il flagello disponendosi a forma di spirale.
		Il filamento rimane cavo all'interno.
		Ha inoltre propriet\`a antigeniche ed \`e rigida.

	\subsection{Tipologie di movimento flagellare}
	\begin{multicols}{2}
		\begin{itemize}
			\item Peritrico: alterna tumble e run in base se si vuole avvicinare ad un nutriente o allontanarsi da un repellente.
				La direzionalit\`a della cellula \`e data dal fascio dei flagelli lungo lo stesso asse che ruota in senso orario.
				Affinch\`e avvengano tumble avviene una rotazione antioraria.
			\item Polare: microorganismi dotati di un unico flagello hanno la possibilit\`a di ruotarlo in entrambi i sensi: in uno vanno avanti nell'altro indietro.
				Altri possono muoversi in un unico senso con movimento aleatorio.
		\end{itemize}
	\end{multicols}

		\subsubsection{Velocit\`a}
		I flagelli permettono spostamenti al secondo fino a $60$ volte la lunghezza del corpo.

	\subsection{Spirocheti}
	Gli spirocheti sono una classe di microorganismi a forma di spirale e Gram$-$.
	I loro flagelli si trovano all'interno del corpo tra periplasma e membrana esterna.
	Vengono detti filamenti assiali e percorrono il corpo formando una spirale ancorata alla membrana interna.
	La rotazione sincrona dei filamenti fa muovere l'intero corpo.
	Gli spirocheti sono pertanto in grado di muoversi in ambienti viscosi come nei liquidi interni del corpo umano e molti di essi sono patogeni.

	\subsection{Motilit\`a batterica}

		\subsubsection{Motilit\`a per scivolamento}
		I batteri possono muoversi anche in altri modi oltre a quello che utilizza i flagelli.
		Nei batteri bastoncellari o filamentosi la motilit\`a per scivolamento non usa strutture esterne, ma la cellula si muove lungo il suo asse maggiore.
		Questo movimento \`e pi\`u lento rispetto a quello flagellare e richiede una superficie solida.
		Nei cianobatteri \`e accompagnato dalla secrezione di una sostanza mucosa o \emph{slime}, un polisaccaride che fa aderire meglio il microorganismo alla superficie solida.
		In altri batteri si trovano proteine di membrana citoplasmatica che utilizzano l'energia rilasciata dalle pompe di ioni \emph{$H^+$} per subire cambiamenti conformazionali che vengono trasmessi alle proteine di membrana esterna e consentono lo scivolamento della cellula sulla superficie.

		\subsubsection{Chemiotassi}
		Si intende per chemiostassi il movimento dovuto alla presenza di un attraente o repellente nell'ambiente segnalato grazie a chemiorecettori.
		In presenza di attraente i moti di avanzamento sono orientati e il numero di run \`e maggiore rispetto a quello di tumble, mentre in presenza di repellente avviene l'inverso.
		Si possono scoprire nuove forme per combattere i patogeni analizzando i loro repellenti.

		\subsubsection{Fototassi}
		Si intende per fototassi il movimento di microorganismi verso la luce.
		\`E in risposta a uno stimolo che un batterio riceve se esposto a una specifica lunghezza d'onda.
		Si orientano per ottenere il livello di luce ottimale.
		Il responsabile di tale risposta \`e il fotorecettore.

		\subsubsection{Aerotassi}
		Si intende per aerotassi l'avvicinamento o allontanamento da fonti ricche di \emph{$O_2$}.
		Esistono batteri che necessitano di ossigeno per vivere, mentre altri che ne sono inerti o per cui \`e tossico.

		\subsubsection{Osmotassi}
		Si intende per osmotassi il movimento di allontanamento o avvicinamento ad alte concentrazioni ioniche.

\section{Strutture esterne}

	\subsection{Glicocalice}
	Il glicocalice \`e una struttura variabile costituita principalmente da polisaccaridi e polipeptidi.
	\`E un polisaccaride ad elevato peso molecolare, eteropolimerico od omopolimerico.

		\subsubsection{Capsula}
		La capsula \`e un tipo di glicocalice con struttura rigida, solida e forte.
		\`E presente nei patogeni grazie alla quale riescono ad aderire a molti substrati: circonda le difese immunologiche degli eucarioti in quanto gli zuccheri che la compongono sono simili a quelli presenti nelle cellule degli eucarioti.
		Pu\`o superare le dimensioni del batterio ed \`e la principale linea di difesa contro la fagocitosi.

		\subsubsection{Strato mucoso}
		Lo strato mucoso \`e un tipo di glicocalice con struttura pi\`u morbida e meno attaccata alla cellula.
		Pu\`o essere disciolto in acqua e la sua funzione fondamentale \`e di protezione dalla essiccazione oltre a conferire capacit\`a adesive.

	\subsection{Fimbrie}
	Le fimbrie sono filamenti molto pi\`u corti e numerosi che non partecipano al processo di locomozione ma a quello di aderenza ad altre cellule o per la creazione di biofilm.

	\subsection{Pili}
	I pili sono costituiti da proteine diverse, sono pi\`u lunghi delle fimbrie e sono presenti in poche quantit\`a.
	La loro funzione \`e quella di connettersi ad un altro microorganismo e scambiare materiale genetico attraverso coniugaizone o per iniettare tossine attraverso i sistemi di secrezione.
	La capacit\`a di adesione \`e mediata dalle adesine, proteine presenti alle estremit\`a.
	Sono sintetizzati quando la cellula li necessita.
	Sono responsabili di un tipo di variazione genetica e sono pi\`u lunghi del corpo batterico.
	Permettono una motilit\`a agganciandosi ad una cellula nelle vicinanze e depolimerizzandosi all'interno della cellula madre permettono l'avvicinamento con motilit\`a contrattile.
	Crescono dall'interno verso l'esterno.

	\subsection{Biofilm}
	I biofilm sono masse viscose di batteri raggruppati in un'unica zona.
	Le cellule hanno un comportamento diverso rispetto a quando sono isolate.
	La massa \`e circondata da fimbrie e polisaccaridi.
	Nei biofilm i batteri sono pi\`u resistenti ad antibiotici in quanto \`e presente una barriera supplementare.

		\subsubsection{Quorum sensing}
		Per produrre i biofilm i batteri devono essere in grado di comunicare tra loro per ottenere una struttura tridimensionale adatta.
		Questo avviene attraverso il rilascio di proteine captate da recettori specifici.
		Queste vie di segnale permettono ai microorganismi di capire quando il loro numero \`e sufficiente per creare un biofilm.
		I recettori saturi infatti indicano un numero ottimale per la creazione di un biofilm.
		Si smette di produrre le molecole di segnale si inizia a formare il biofilm.

\section{Inclusioni cellulari}
Le inclusioni cellulari sono strutture batteriche interne composte da lipidi, proteine, glicogeno e amido.
Sono riserve di nutrienti e presenti in condizioni favorevoli come previsione.
Arrivano a occupare i $\frac{3}{4}$ della cellula.
Isolano il materiale di riserva dal citoplasma.
Sono circondate da membrana lipidica.
	
	\subsection{Tipologie di inclusioni cellulari}

		\subsubsection{Acido \emph{poli-$\mathbf{\beta}$-idrossibutirrico} \emph{PBH}}
		Il \emph{PHB} \`e formato da monomeri dell'acido che si legano tramite legami estere formando polimeri che si aggregano in granuli.
		Sono sintetizzati in eccesso di carbonio e utilizzati per produrre \emph{ATP}.
		
		\subsubsection{Glicogeno}
		Il glicoceno \`e un polimero del glucosio, funge da riserva di energia e di carbonio.

		\subsubsection{Polifosfati}
		I polifosfati sono sorgenti di fosfati per sintesi di acidi nucleici e di fosfolipidi per la membrana citoplasmatica.


		\subsubsection{Globuli di zolfo elementare}
		I globuli di zolfo elementare si formano per ossidazione del solfuro di idrogeno accoppiate con reazioni di metabolismo energetico e fotosintesi.
		Quando lo zolfo si esaurisce si ossidano.
		Si trova nel periplasma.


		\subsubsection{Magnetosomi}
		I magnetosomi sono particelle cristalline costituite da magnetite \emph{$Fe_3O_4$} e creano un dipolo magnetico permanente.
			I batteri che li contengono svolgono magnetotassi.


		\subsubsection{Vescicole gassose}
		Le vescicole gassose sono strutture di natura proteica, vuote e rigide nel citoplasma.
		Conferiscono alla cellula la capacit\`a di galleggiare diminuendone la densit\`a.
		La membrana proteica \`e impermeabile all'acqua.
		Permettono al batterio di aggiustare la posizione verticale nella colonna d'acqua.
		Sono formate da \emph{GvpA} e \emph{GvpC}: la prima idrofobica ricca di $\beta$-foglietti, mentre la seconda ricca di $\alpha$-eliche.
		\emph{GvpA} si allinea in nastri paralleli formando una superficie impermeabile rafforzata da \emph{GvpC} attraverso legami crociati.

\section{Endospore}
Le endospore sono strutture uniche che alcuni batteri sono in grado di produrre.
Nella sporulazione una cellula trasforma s\`e stessa in un'endospora a causa di ambiente ostile.
L'endospora \`e molto resistente a calore, agenti chimici e radiazioni.

	\subsection{Struttura}
	L'endospora \`e formata da:
	\begin{multicols}{2}
		\begin{itemize}
			\item Esosporio: lo strato pi\`u esterno, sottile e delicato.
			\item Tunica: parete della spora, pi\`u strati proteici.
			\item Corteccia: peptidoglicano lasso.
			\item Core: parete cellulare, membrana citoplasmatica, nucleoide e ribosomi.
		\end{itemize}
	\end{multicols}
	Il metabolismo viene rallentato di molto.
	
		\subsubsection{Acido dipicolinico}
		Una sostanza chimica loro caratteristica \`e l'acido dipicolinico che forma un complesso con gli ioni calcio  che riduce la presenza d'acqua all'interno dell'endospora.
		Si intercala inoltre con il DNA rendendolo pi\`u resistente alla denaturazione per riscaldamento.

		\subsubsection{Small acid-soluble spore protein \emph{SASP}}
		Le \emph{SASP} legano il DNA e lo proteggono da UV e sostanze chimiche.
		Sono fonte di carbonio e forniscono l'energia per la germinazione.
		Proteggono il DNA modificandolo nella forma $A$, resistente alla formazione di dimeri e pirimidina

		\subsubsection{Gram$\mathbf{-}$}
		Nei pochi gram $-$ che producono endospore si trova il core, la corteccia sostituisce il peptidoglicano e si trova dopo il coat.
		Nella corteccia viene contenuto l'acido dipicolinico che la protegge e conferisce rigidit\`a.
		Compatta il DNA nel core per renderlo pi\`u resistente al caldo.

	\subsection{Formazione}
	Un segnale extracellulare interpretato dalla cellula attiva una serie di geni: il DNA si replica in due cromosomi identici in cui un'estremit\`a rimane legata alla membrana cellulare.
	Si allungano lungo l'asse della cellula e inizia a formarsi una seconda membrana.
	Crea un'invaginazione all'interno del batterio fino a creare due compartimenti distinti.
	Un DNA viene digerito, mentre intorno all'altro si crea una seconda membrana.
	Si forma la corteccia e il coat.
	L'endospora viene rilasciata dalla cellula madre.

	\subsection{Riattivazione della cellula}
	L'endospora pu\`o rimanere inattiva per molti anni, ma pu\`o essere riconvertita a cellula vegetativa molto rapidamente.
	Questo processo avviene in tre stadi.

		\subsubsection{Attivazione}
		L'attivazione \`e innescata dal riscaldamento di endospore appena formate o dalla presenza di nutrienti specifici.
		
		\subsubsection{Germinazione}
		Durante la germinazione l'endospora perde l'acido dipocolinico in complesso con il calcio e i componenti della corteccia.

		\subsubsection{Esocrescita}
		Durante l'esocrescita avviene un rigonfiamento dovuto all'assorbimento di acqua e alla sintesi di nuovi RNA, DNA e proteine.
