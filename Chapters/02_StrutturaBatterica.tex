\chapter{La struttura della cellula}
\section{La parete cellulare}
Tutti i batteri possiedono una parete cellulare che svolge vari ruoli:
\begin{itemize}
	\item Strutturale: conferisce forma e rigidit\`a alla cellula.
	\item Protettivo: impedisce la lisi in un ambiente tipicamente ipotonico in quanto il citoplasma possiede un alta concentrazione di soluti e crea una pressione osmotica 
		nell'ordine delle $2$\si{atm}.
\end{itemize}
Oltre a permettere di capire i processi vitali di un procariote lo studio della parete cellulare ha permesso la sintesi di diversi antibiotici che la attaccano specificatamente in 
quanto assente nelle cellule dell'organismo infetto.
\subsection{Peptidoglicano}
La parete cellulare dei batteri \`e composta principalmente da peptidoglicano (mureina), un polisaccaride complesso composto da l'\emph{N-acetilglucosamide} (\emph{NAM}) e da 
\emph{N-acetilmuramico} (\emph{NAG}) che si alterano formando lunghi polisaccaridi formando un legame $\beta-1,4$. La sintesi delle catene avviene in zone concentrate, favorendo 
la formazione di fogli che circondano la cellula. Tra le catene si formano legami crociati tra $4$ amminoacidi legatisi in posizioni specifiche al \emph{NAM} creando ponti tetrapeptidici
la cui composizione \`e variabile (tipicamente \emph{L-alanina}, \emph{D-alanina}, \emph{D-acido glutammico} e \emph{acido diamino-pimelico}). 
\begin{itemize}
	\item Nei Gram negativi il legame crociato si forma tra \emph{DAP} e \emph{D-alanina}.
	\item Nei gram positivi il legame crociato si forma attraverso un ponte peptidico tipicamente di $5$ glicine.
\end{itemize}
\subsection{Gram positivi} 
I Gram positivi possiedono una parete costituita da uno strato solido di peptidoglicano anche $50$ volte pi\`u spesso e meno elaborato rispetto a quello dei Gram negativi. La
parete cellulare \`e formata per il $90\%$ da fogli di peptidoglicano sovrapposti. Si trovano nella parete oltre a proteine di trasporto e strutturali acidi teicoici e lipoteicoici. 
Questi sono formati da catene di alcol e zucchero come glicerolo o ribitolo unite da legami fosfodiesterici grazie a gruppi fosfato. Sono spesso associati ad altri zuccheri come
glucosio e \emph{D-alanina}. Si dicono lipoteicoici gli acidi teicoici contenenti glicerolo e pertanto legati con i lipidi della membrana citoplasmatica. Queste molecole possiedono una
carica negativa e sono in parte responsabili della carica elettrica negativa della superficie cellulare. Sono inoltre fondamentali per il trasporto di ioni calcio e magnesio. Gli
acidi lipoteicoici in particolare ancorano la parete cellulare alla membrana citoplasmatica. 
\subsection{Gram negativi}
I Gram negativi sono formati da due membrane, una interna e una esterna e nello spazio tra di esse, detto periplasmatico, si trova la parete cellulare di peptidoglicano. 
\subsubsection{Spazio periplasmatico}
Lo spazio periplasmatico \`e largo circa $15\si{\nano\metre}$, ha una consistenza simile ad un gel e contiene:
\begin{itemize}
	\item Enzimi idrolitici: con ruolo nella degradazione iniziale delle molecole alimentari.
	\item Proteine leganti: iniziano il processo di trasporto dei substrati.
	\item Chemiorecettori: che governano la risposta chemiotassica.
	\item Acqua.
	\item Nutrienti.
\end{itemize}
La maggior parte di questi elementi raggiunge il periplasma attraverso un sistema di esportazione di proteine presente nella membrana citoplasmatica. 
\subsubsection{Membrana esterna}
La membrana esterna presenta una forte resistenza contro composti chimici dannosi ed \`e formata nel lato interno da fosfolipidi e proteine, mentre in quello esterno da lipopolisaccaridi
\emph{LPS}. Questo \`e formato da tre parti:
\begin{itemize}
	\item Lipide $A$: un glicolipide formato da un disaccaride \emph{NAG} legato a fosfati ed acidi grassi grazie ad un legame estere-amminico. Questi disaccaridi sono legati al core
		polisaccaridico attraverso il chetodeossiottonato \emph{KDO}.
	\item Core polisaccaridico: oltre il legame con \emph{KDO} \`e formato da zuccheri a $7$ atomi di carbonio \emph{eptosi}, glucosio, galattosio e \emph{NAG}. 
	\item Polisaccaride $O$-specifico o antigene $O$: \`e una catena di zuccheri lunga fino a $40$ residui legata al core contenente galattosio, ramnosio e mannosio oltre a uno 
		o pi\`u dideossizuccheri (abequosio, colitosio, paratosio o tivelosio) legati a formare catene lunge $4$-$5$ elementi, spesso ramificate. La loro ripetizione porta alla
		sua formazione. 
\end{itemize}
\paragraph{Porine}
Le porine sono proteine transmembrana che attraversano completamente la membrana esterna e la rendono parzialmente permeabile a piccole molecole idrofiliche. Sono formate da tre 
subunit\`a identiche ciascuna con un poro di $1\si{\nano\metre}$ che non permette la fuoriuscita di enzimi del periplasma. Le porine si dividono in specifiche o specifiche in base
alla selettivit\`a della molecola a cui permettono il passaggio. Tipicamente le prime formano canali d'acqua attraverso i quali pu\`o passare qualsiasi piccola molecola, mentre le
seconde presentano un sito di legame specifico. 
\paragraph{Lipiproteine}
Le lipoproteine sono presenti nello strato interno della membrana esterna e servono da giunzione fra questa e il peptidoglicano.
\subsection{Lisozima}
L’enzima lisozima è una proteina che può rompere i legami $\beta(1, 4)$ degli zuccheri NAM-NAG.
\begin{itemize}
\item In una soluzione diluita (ipotonica) la degradazione della parete con il lisozima rilascia il protoplasto che va incontro a lisi in seguito 
all’ingresso di acqua nella cellula
\item In una soluzione isotonica non c’è nessun movimento netto di acqua tra ambiente e protoplasti ed essi rimangono stabili.
\end{itemize}
Esistono archea e micoplasmi che si trovano naturalmente nella forma di protoplasti in quanto vivono in habitat osmoticamente protetti e non necessitano una protezione contro la 
pressione osmotica. Sono pertanto spesso parassiti. 
\section{La membrana citoplasmatica}
La membrana citoplasmatica \`e spessa $8\si{\nano\metre}$ e circonda il citoplasma separandolo dall’ambiente. Nel caso in cui la membrana citoplasmatica sia compromessa, l’integrità 
della cellula viene distrutta, il citoplasma si disperde e di conseguenza il batterio muore. \`E una barriera altamente selettiva e funziona anche come sito di ancoraggio per proteine.
L'isolamento che fornisce con l'ambiente permette la creazione di un gradiente ionico negativo all'interno e positivo all'esterno che viene usato per generare energia o forza
proton-motrice. 
\subsection{La struttura}
La struttura generale della membrana è quella di un doppio strato fosfolipidico. I fosfolipidi sono molecole composte da una coda idrofobica (acidi grassi) e una testa idrofila 
(glicerolo-fosfato) .
Quando i fosfolipidi si aggregano in una soluzione acquosa, formano naturalmente bistrati. In una membrana 
fosfolipidica, le code puntano verso il centro del bistrato per formare un ambiente idrofobico, e le teste rimangono esposte 
all'ambiente esterno o al citoplasma. Gli acidi grassi comuni nella membrana citoplasmatica hanno catene da  14 a 20 atomi di carbonio.
Le membrane citoplasmiche di alcuni batteri sono rafforzate da molecole simili allo sterolo chiamate opanoidi. Gli steroli sono molecole rigide e planari 
che funzionano per rafforzare le membrane delle cellule eucariote, e gli opanoidi svolgono una funzione simile nei batteri.
La membrana contiene inoltre un alto numero di proteine, che hanno tipicamente superfici idrofobe in regioni che attraversano la membrana e superfici 
idrofile in regioni che sono a contatto con l'ambiente e il citoplasma. Molte di esse sono saldamente incorporate nella membrana e sono chiamate proteine 
integrali. Altre hanno una porzione ancorata nelle regioni della membrana e dell'extramembrana che puntano dentro o fuori la cellula. Altre proteine ancora, 
chiamate proteine della membrana periferica, non sono incorporate nella membrana, ma rimangono comunque associate alla sua superficie. Alcune di queste 
proteine periferiche sono le lipoproteine, molecole che contengono una coda lipidica che ancora la proteina alla membrana. Le proteine della membrana 
periferica in genere interagiscono con le proteine integrali della membrana in importanti processi cellulari come il metabolismo energetico e il trasporto. 
Spesso le proteine che devono interagire tra loro in qualche processo sono tipicamente raggruppate in cluster per consentire loro di rimanere adiacenti 
l'una all'altra nell'ambiente semifluido della  membrana.
\section{Membrana e parete degli archaea}
In contrasto con i lipidi di batteri ed Eukarya in cui ci sono legami estere tra acidi grassi e glicerolo, i lipidi degli Archaea contengono legami etere 
tra glicerolo e le loro catene laterali idrofobiche (catena alifatica). I lipidi degli Archaea mancano quindi di acidi grassi, di per sé, anche se le catene 
laterali idrofobiche svolgono lo stesso ruolo funzionale degli acidi grassi. 
Le principali differenze sono le catene isopreniche e il fatto che alcune sono 
a monostrato invece che bistrato. A differenza dei bistrati lipidici, le membrane monostrato lipidiche sono estremamente resistenti al calore e sono quindi 
ampiamente distribuite tra Archaea ipertermofili, organismi che crescono a temperature superiori a 80 gradi centigradi. Esistono anche membrane con una 
miscela di carattere bistrato e monostrato, con alcuni dei gruppi idrofobici opposti covalentmente legati e altri no. Un’altra differenza tra i batteri e 
gli Archaea è la frequente assenza sia di una parete cellullare, sia di una membrana esterna, che vengono sostuititi da una grande varietà di diverse pareti 
cellullari, alcune delle quali hanno delle bio-componenti molto simili a quelle dei batteri, come polisaccaridi, proteine e glicoproteine. Alcuni archaea 
metanogeni (producono metano) hanno una parete costituita da pseudopeptidoglicano, costituita da NAG e, al posto di avere il NAM, ha 
l’acido N-acetiltalosaminuronica (NAT). Questo porta ad avere legami $\beta(1, 3)$, che li rende insensibili al lisozima e 
alla penicillina, che invece nei batteri distruggerebbero il peptidoglicano o ne arresterebbero la sua biosintesi. Alcuni archaea possiedono ulteriori polisaccaridi diversi anche dallo 
pseudopeptidoglicano.
\section{Flagelli}

\subsection{Struttura}

\subsection{Movimento flagellare}

\subsection{Sintesi flagellare}

\subsection{Movimento flagellare}

\subsection{Spirocheti}

\section{Motilit\`a batterica}

\section{Strutture esterne}
\subsection{Glicocalice}

\subsection{Fimbrie}

\subsection{Pili}

\subsection{Biofilm}

\subsubsection{Quorum sensing}

\section{Inclusioni cellulari}

\section{Endospore}

\subsection{Struttura}

\subsection{Formazione}




\section{Locomozione microbica}
Molti batteri hanno la capacità di potersi muovere sotto il loro controllo, spesso con l’aiuto di strutture chiamate flagelli che 
permettono loro di rispondere a degli stimoli ambientali, detti tassi. I due principali tipi di movimento sono nuotare e moto a 
scorrimento (glinding). 
\subsection{Il flagello}
\begin{figure}[H]
	\includegraphics[width=\textwidth]{Pictures/14.png}
\end{figure}

I flagelli batterici sono lunghe e sottili appendici libere da un'estremità e attaccate alla cellula all'altra estremità. I flagelli batterici sono così 
sottili ($15$-$20 nm$, a seconda della specie) che un singolo flagello non può essere visto dalla microscopia ottica a meno che non sia macchiato per aumentarne 
il diametro. Tuttavia, i flagelli sono facilmente visibili con il microscopio elettronico. I flagelli differiscono per numero e posizione  a seconda della 
specie. Possono assumere un’organizzazione peritrica (peritrichous) (a) quando numerosi flagelli 
sono distribuiti su tutta la superficie cellulare. Flagelli polari si trovano soltanto alle estremità della cellula (b). L’organizzazione lofotrica 
(lophotrichous) (c) indica la presenza di più di un flagello polare. \\
\newpage
\begin{multicols}{2}
\begin{figure}[H]
	\includegraphics[width=0.48\textwidth]{Pictures/15.png}
\end{figure}
	
\begin{figure}[H]
	\includegraphics[width=0.48\textwidth]{Pictures/16.png}
\end{figure}
\end{multicols}

Il flagello \`e  un’elica la cui la distanza tra curve adiacenti è 
costante. Questa lunghezza d'onda è una caratteristica che differisce da specie a specie. \`E composto principalmente da tre parti: il filamento, l’uncino 
(hook) e la base. Mentre le prime due hanno una composizione chimica e una struttura molto simile tra i vari batteri, il modo in cui il 
flagello è ancorato alla parete e alla membrana cellullare cambia a seconda che il batterio sia gram-positivo o gram-negativo. Il filamento è composto da 
molte copie della proteina flagellina. La forma e la lunghezza d'onda del flagello sono in parte determinate dalla struttura della proteina flagellina e in 
dalla direzione di rotazione del filamento. La sequenza di amminoacidi della flagellina è altamente conservata in specie di batteri, 
suggerendo che la motilità dei flagelli si è evoluta presto e ha radici profonde all'interno di questo dominio. L’uncino è chimicamente diverso dal 
filamento ed è costituito da un solo tipo di proteina. La base è ancorata alla membrana citoplasmatica e alla parete cellulare e ha una funzione meccanica, con un funzionamento analogo 
a quello di un motore a propellenza. Il motore consiste in un’asta centrale, chiamata bastoncello, che 
attraversa diversi anelli. Nei batteri gram-positivi, che non hanno una membrana esterna, possiedono solo la coppia di anelli più interna.
 Intorno 
all'anello interno e ancorato nella membrana citoplasmatica c'è una serie di proteine chiamate proteine Mot. Un insieme finale di proteine, chiamate 
proteine Fli, funzionano come l'interruttore del motore, invertendo la direzione di rotazione del flagello in risposta ai segnali intracellulari. Nei 
batteri gram-negativi, un anello esterno, chiamato anello L, è ancorato nello strato di lipopolissaccaride. Un secondo anello, chiamato anello P, è ancorato 
nello strato di peptidoglicano della parete cellulare. Una terza serie di anelli, chiamati anelli MS e C, si trovano rispettivamente all'interno della 
membrana citoplasmatica e del citoplasma.
\subsection{Il movimento flagellare}
\begin{wrapfigure}{l}{0.5\textwidth}
  \begin{center}
    \includegraphics[width=0.48\textwidth]{Pictures/17.png}
  \end{center}
\end{wrapfigure}
Il flagello è un piccolo motore rotativo. Questi motori contengono due componenti principali: il rotore e lo statore. Nel motore del flagello, il rotore è 
costituito dall'asta centrale e dagli anelli L, P, C e MS. Collettivamente, queste strutture costituiscono il corpo basale. Lo statore è costituito dalle 
proteine Mot che circondano il corpo basale e generano momento. La rotazione del flagello è impartita dal corpo basale. L'energia necessaria per la 
rotazione del flagello proviene dalla forza protomotrice. Il movimento dei protoni attraverso la membrana citoplasmica attraverso il complesso Mot aziona la 
rotazione del flagello. Circa 1000 protoni sono traslocati ad ogni rotazione del flagello. In questo modello di turbina protonica, i protoni che scorrono 
attraverso i canali delle proteine Mot esercitano forze elettrostatiche su cariche disposte ad elica sulle proteine del rotore. Le attrazioni tra cariche 
positive e negative causano quindi la rotazione del corpo basale man mano che i protoni scorrono attraverso le proteine Mot.
\subsubsection{Il movimento flagellare nelle diverse categorie}
\begin{wrapfigure}{l}{0.5\textwidth}
  \begin{center}
    \includegraphics[width=0.48\textwidth]{Pictures/18.png}
  \end{center}
\end{wrapfigure}
Viene descritto il movimento in procarioti peritrichous e polarizzati.
\begin{itemize}
	\item[(a)] Peritrichous:il movimento in avanti \`e causato da tutti i flagelli che routano in senso antiorario (CCW) in un fascio. La rotazione oraria (CW) causa un 
		``ruzzolamento" (tumble). della cellula, che le permette di modificare la direzione all'inizio di una nuova rotazione dei flagelli in senso antiorario. 
	\item[(b)] Polare: la cellula cambia direzione invertendo la rotazione flagellare passando dal tirare allo spingere la cellula o, con flagelli unidirezionali, fermandosi periodicamente per riorientarsi e successivamente 
		muovendosi in avanti attraverso una rotazione in senso orario. 
\end{itemize}
La freccia gialla mostra la direzione in cui la cellula si sta muovendo.
\subsection{La sintesi flagellare}
La sintesi del flagello non avviene alla base, come ad esempio i capelli umani, ma dalla punta. Per primo viene sintetizzato l’anello MS ed inserito nella 
membrana citoplasmatica. Successivamente altre proteine di ancoraggio vengono sintetizzate insieme all’uncino, prima che il filamento cominci a formarsi. Le 
subunità di flagellina, sintetizzate nel citoplasma, vengono estruse attraverso un canale di $3 nm$ all'interno del filamento e si aggiungono alla fine per 
formare il flagello maturo. Un ``cap" proteico è presente alla fine del flagello in crescita, che aiuta le molecole di flagellina che si sono diffuse 
attraverso il canale del filamento ad assemblarsi in modo corretto. 
\begin{figure}[H]
	\includegraphics[width=\textwidth]{Pictures/19.png}
\end{figure}
\subsection{I flagelli negli spirocheti}
Nei batteri ``spirocheti”, ossia che assumono una forma a spirale, i flagelli sono assiali e si trovano tra la membrana citoplasmatica e quella esterna (sono 
gram-negativi). La rotazione dell’endoflagello porta l’intera cellula a ruotare su se stessa in un movimento elicoidale che consente lo spostamento anche in 
ambienti molto viscosi.
\begin{figure}[H]
	\includegraphics[width=\textwidth]{Pictures/20.png}
\end{figure}
