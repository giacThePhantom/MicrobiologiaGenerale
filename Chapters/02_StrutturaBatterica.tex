\chapter{La struttura della cellula}
\section{La parete cellulare}
Tutti i batteri dispongono di una parete cellulare, la quale assume un ruolo strutturale, conferendo la forma specifica della specie e rigidità, e 
protettivo, ad esempio contro la lisi. Il citoplasma dei batteri infatti contiene un’alta concentrazione di soluti, creando una pressione osmotica 
significativa dell’ordine di 2atm. Lo studio della parete cellulare è stato importante non solo per capire 
ulteriormente i processi vitali di una cellula procariote, ma anche per la sintesi di antibiotici. Questi ultimi infatti distruggono la parete cellulare causando la morte dell'agente patogeno senza causare effetti secondari sul paziente
in quanto le cellule umane sono prive di essa.
\subsection{Peptidoglicano}
\begin{multicols}{2}
\begin{figure}[H]
	\includegraphics[width=0.5\textwidth]{Pictures/Peptidoglicano.png}
\end{figure}    
\columnbreak
\begin{figure}[H]
	\includegraphics[width=0.5\textwidth]{Pictures/Peptidoglicano2.png}
\end{figure}       
\end{multicols}
La parete cellulare dei batteri possiede uno strato di peptidoglicanto (mureina) che conferisce rigidit\`a. \`E
un polisaccaride composto principalmente da due derivati dello zucchero, N-acetilglucosamina (NAG) e acido N-acetilmuramico (NAM), entrambi strutturalmente 
simili al glucosio. Lunghe catene di peptidoglicano (formate da NAM e NAG covalentemente associate) vengono sintetizzate vicine tra di loro, creando un 
foglio che circonda la cellula. Sono connesse da quattro amminoacidi (solitmaente L-Alanina, D- Alanina D-Acido glutammico e Acido diamino-pimelico (DAP)) che formano legami crociati 
diretti o indiretti (con ponte peptidico) a seconda della natura della cellula. Nei batteri gram-negativi il legame crociato è formato da un legame peptidico dal gruppo amminico di DAP di una 
catena con il gruppo carbossile dell’ultimo D-Alanina dell’altra catena. Nei batteri gram-positivi invece, il legame spesso avviene tramite un corto ponte 
peptidico il cui numero e tipo di amminoacidi varia da specie a specie. 
\subsection{Parete cellulare nei gram-positivi e gram-negativi}
\begin{wrapfigure}{l}{0.6\textwidth}
  \begin{center}
    \includegraphics[width=0.58\textwidth]{Pictures/3.png}
  \end{center}
\end{wrapfigure}
Le pareti cellulari nel mondo dei batteri assumono principalmente due strutture diverse, che hanno portato ad una classificazione che consiste nel 
suddividere i microrganismi in gram-positivi e gram-negativi.Le differenze strutturali tra le pareti cellulari dei batteri gram-positivi e gram-negativi sono responsabili della reazione diversa alla 
reazione di gram-staining. Durante il  Gram staining un complesso cristallino insolubile e violetto si forma nella cellula. Questo complesso viene estratto dall'alcol dai batteri gram-negativi ma non dai gram-positivi.
Come si \`e visto i batteri gram-positivi hanno una parete cellulare molto spessa che consiste principalmente di peptidoglicano. Durante il Gram-staining tale parete \`e disidratata dall'alcol causando la chiusura dei pori della parete e 
impedendo al complesso cristallino di uscire. In contrasto nei batteri gram-negativi l'alcol penetra facilmente la membrana esterna ricca di lipidi ed estrae il complesso cristallino dalla cellula. Dopo il trattamento con l'alcol le 
cellule gram-negative sono quasi invisibili a meno che siano contro-macchiate da un secondo colorante, parte della procedura standard durante il Gram stain.
\subsubsection{Parete cellulare nei gram+} 
\begin{wrapfigure}{l}{0.5\textwidth}
  \begin{center}
    \includegraphics[width=0.48\textwidth]{Pictures/4.png}
  \end{center}
\end{wrapfigure}
La prima differenza che si nota tra le due classificazioni di batteri è la quantità di peptidoglicano che possiedono: i gram-positivi sono infatti ricoperti da un 
solido strato di peptidoglicano che è anche 50 volte più spesso di quello dei gram-negativi, che invece ne possiedono solo un piccolo strato chiuso tra due membrane cellulari.
Nei gram-positivi il $90\%$ della parete cellulare \`e formato da diversi ``fogli” di peptidoglicano sovrapposti (alcuni presentano un solo strato). 
Si pensa che il peptidoglicano sia sintetizzato in ``tubi” di $50nm$ di dimensione, i quali contengono al 
loro interno legami crociati di fili di glicano. Mentre questi tubi vengono sintetizzati, formano legami tra di loro aumentando la stabilit\`a della struttura. 
Oltre alle proteine che svolgono la funzione di trasporto e strutturale i gram-positivi presentano sulla parete cellulare anche acidi teicoici e lipoteicoici. Il 
termine ``acidi teichoici" comprende tutte le pareti cellulari, la membrana citoplasmica e i polimeri capsulari composti da fosfato glicerol o fosfato di 
ribitolo (pag73 pdf??????). Gli acidi teichoici sono legati covalentemente all’acido muramico nella parete. Poiché i fosfati sono caricati negativamente, gli acidi 
teichoici sono in parte responsabili della carica elettrica negativa complessiva della superficie cellulare. Hanno anche la funzione di legare 
$Ca_2^+$ e $Mg_2^+$ per l'eventuale trasporto nella cellula. Alcuni di essi sono legati covalentemente ai lipidi della membrana, e questi sono chiamati 
acidi lipoteichoici. Gli acidi lipoteicoici assumono quindi la funzione di ancoraggio della parete alla membrana citoplasmatica. Microplasma, 
batteri patogeni legati ai batteri gram-positivi che causano diverse malattie infettive degli esseri umani e di altri animali, thermoplasma e simili, 
specie di Archaea riescono a vivere anche se non possiedono una parete cellulare. Questi organismi sono in grado di sopravvivere senza parete cellulare perché contengono 
membrane citoplasmiche insolitamente forti o perché vivono in habitat osmoticamente protetti come il corpo animale. La maggior parte dei microplasma hanno 
steroli nelle loro membrane citoplasmiche, e queste molecole conferiscono forza e rigidità alla membrana, come fanno nelle membrane citoplasmiche delle 
cellule eucariotiche. Anche le membrane dei termoplasma contengono molecole chiamate lipolicani che svolgono una funzione di rafforzamento simile.
\subsubsection{Parete cellullare nei gram-}
\begin{figure}[H]
	\includegraphics[width=\textwidth]{Pictures/5.png}
\end{figure} 
\newpage
\begin{wrapfigure}{l}{0.5\textwidth}
  \begin{center}
    \includegraphics[width=0.48\textwidth]{Pictures/6.png}
  \end{center}
\end{wrapfigure}
La parete cellulare dei gram-negativi è formata principalmente da una membrana esterna, mentre un sottile strato di peptidoglicano si trova in mezzo alle due 
membrane. La membrana esterna non è composta solo da fosfolipidi e proteine come quella citoplasmatica, ma anche da polisaccaridi, che si 
uniscono ai lipidi per formare una struttura complessa. Per questo motivo la membrana esterna viene spesso chiamata strato lipopolisaccaride, o semplicemente 
LPS. Sebbene permeabile a piccole molecole, la membrana esterna è impermeabile alle proteine e ad altre molecole molto grandi. Infatti, una delle principali 
funzioni della membrana esterna è quella di impedire che le proteine le cui attività si verificano al di fuori della membrana citoplasmatica si diffondano 
all’esterno dalla cellula. Queste proteine sono presenti in una regione chiamata periplasma. Questo spazio, situato tra la superficie esterna della membrana 
citoplasmica e la superficie interna della membrana esterna, è largo circa $15nm$. Il periplasma ha una consistenza simile a quella di un gel, dovuta 
all’alta concentrazione di proteine. A seconda dell'organismo, il periplasma può contenere diverse classi di proteine:
\begin{itemize}
	\item Enzimi idrolitici: svolgono un ruolo nella degradazione iniziale delle molecole alimentari.
	\item Proteine leganti: iniziano il processo di trasporto dei substrati.
	\item chemiorecettori: proteine che governano la risposta chemiotassica.
\end{itemize}
La maggior parte di esse raggiunge il periplasma tramite un sistema di esportazione di proteine presente nella membrana citoplasmatica. La membrana esterna è solo in parte impermeabile alle piccole molecole per la presenza di 
 porine, proteine che fungono da canali per l’entrata e l’uscita di soluti. Sono note diverse porine, tra cui classi specifiche e non specifiche. Le 
porine non specifiche formano canali pieni d'acqua attraverso i quali può passare qualsiasi piccola sostanza, mentre quelle specifiche contengono un 
sito di legame per solo una o un piccolo gruppo di sostanze strutturalmente correlate. Strutturalmente, le porine sono proteine transmembrana composte da 
tre sotto unità identiche. Oltre ai tre canali principali che si vengono a formare, nel centro della porina si trova un piccolo foro di circa $1nm$ di 
diametro attraverso il quale possono viaggiare molecole estremamente piccole. 
\subsection{Lisozima}
\begin{multicols}{2}
\begin{figure}[H]
	\includegraphics[width=0.5\textwidth]{Pictures/7.png}
\end{figure} 
L’enzima lisozima è una proteina che può rompere i legami $\beta(1, 4)$ degli zuccheri NAM-NAG.
\begin{itemize}
\item[(a)] In una soluzione diluita (ipotonica) la degradazione della parete con il lisozima rilascia il protoplasto che va incontro a lisi in seguito 
all’ingresso di acqua nella cellula
\item[(b)] In una soluzione isotonica non c’è nessun movimento netto di acqua tra ambiente e protoplasti ed essi rimangono stabili.
\end{itemize}
\end{multicols}
\section{La membrana citoplasmatica}
\begin{figure}[H]
	\includegraphics[width=\textwidth]{Pictures/8.png}
\end{figure}
La membrana citoplasmatica circonda il citoplasma e lo separa dall’ambiente. Nel caso in cui la membrana citoplasmatica sia compromessa, l’integrità della 
cellula viene distrutta, il citoplasma si disperde e di conseguenza il batterio muore. La membrana non è così rigida da conferire un’adeguata protezione 
contro la lisi, ma la sua mobilità le permette di svolgere la sua funzione principale: la permeabilità selettiva. Non solo, la membrana funge 
anche da sito di ancoraggio per proteine che sono coinvolte in alcuna attività e da conservatrice di energia.
\subsection{La struttura}
La struttura generale della membrana è quella di un doppio strato fosfolipidico. I fosfolipidi sono molecole composte da una coda idrofobica (acidi grassi) e una testa idrofila (glicerolo-fosfato) (Figura 2.14).
Quando i fosfolipidi si aggregano in una soluzione acquosa, formano naturalmente bistrati. In una membrana 
fosfolipidica, le code puntano verso il centro del bistrato per formare un ambiente idrofobico, e le teste rimangono esposte 
all'ambiente esterno o al citoplasma. Gli acidi grassi comuni nella membrana citoplasmatica hanno catene da  14 a 20 atomi di carbonio.
Le membrane citoplasmiche di alcuni batteri sono rafforzate da molecole simili allo sterolo chiamate opanoidi. Gli steroli sono molecole rigide e planari 
che funzionano per rafforzare le membrane delle cellule eucariote, e gli opanoidi svolgono una funzione simile nei batteri.
La membrana contiene inoltre un alto numero di proteine, che hanno tipicamente superfici idrofobe in regioni che attraversano la membrana e superfici 
idrofile in regioni che sono a contatto con l'ambiente e il citoplasma. Molte di esse sono saldamente incorporate nella membrana e sono chiamate proteine 
integrali. Altre hanno una porzione ancorata nelle regioni della membrana e dell'extramembrana che puntano dentro o fuori la cellula. Altre proteine ancora, 
chiamate proteine della membrana periferica, non sono incorporate nella membrana, ma rimangono comunque associate alla sua superfice. Alcune di queste 
proteine periferiche sono le lipoproteine, molecole che contengono una coda lipidica che ancora la proteina alla membrana. Le proteine della membrana 
periferica in genere interagiscono con le proteine integrali della membrana in importanti processi cellulari come il metabolismo energetico e il trasporto. 
Spesso le proteine che devono interagire tra loro in qualche processo sono tipicamente raggruppate in cluster per consentire loro di rimanere adiacenti 
l'una all'altra nell'ambiente semifluido della  membrana.
\begin{multicols}{2}
\begin{figure}[H]
	\includegraphics[width=0.5\textwidth]{Pictures/9.png}
\end{figure}
\begin{figure}[H]
	\includegraphics[width=0.5\textwidth]{Pictures/10.png}
\end{figure}
\end{multicols}
\section{Membrana e parete degli archaea}
In contrasto con i lipidi di batteri ed Eukarya in cui ci sono legami estere tra acidi grassi e glicerolo, i lipidi degli Archaea contengono legami etere 
tra glicerolo e le loro catene laterali idrofobiche (catena alifatica). I lipidi degli Archaea mancano quindi di acidi grassi, di per sé, anche se le catene 
laterali idrofobiche svolgono lo stesso ruolo funzionale degli acidi grassi. 
\begin{figure}[H]
	\includegraphics[width=\textwidth]{Pictures/11.png}
\end{figure}
\begin{figure}[H]
	\includegraphics[width=\textwidth]{Pictures/12.png}
\end{figure}
\begin{wrapfigure}{l}{0.5\textwidth}
  \begin{center}
    \includegraphics[width=0.48\textwidth]{Pictures/13.png}
  \end{center}
\end{wrapfigure}
Le principali differenze sono le catene isopreniche e il fatto che alcune sono 
a monostrato invece che bistrato. A differenza dei bistrati lipidici, le membrane monostrato lipidiche sono estremamente resistenti al calore e sono quindi 
ampiamente distribuite tra Archaea ipertermofili, organismi che crescono a temperature superiori a 80 gradi centigradi. Esistono anche membrane con una 
miscela di carattere bistrato e monostrato, con alcuni dei gruppi idrofobici opposti covalentmente legati e altri no. Un’altra differenza tra i batteri e 
gli Archaea è la frequente assenza sia di una parete cellullare, sia di una membrana esterna, che vengono sostuititi da una grande varietà di diverse pareti 
cellullari, alcune delle quali hanno delle bio-componenti molto simili a quelle dei batteri, come polisaccaridi, proteine e glicoproteine. Alcuni archaea 
metanogeni (producono metano) hanno una parete costituita da pseudopeptidoglicano, costituita da NAG e, al posto di avere il NAM, ha 
l’acido N-acetiltalosaminuronica (NAT). Questo porta ad avere legami $\beta(1, 3)$, che li rende insensibili al lisozima e 
alla penicillina, che invece nei batteri distruggerebbero il peptidoglicano o ne arresterebbero la sua biosintesi. Alcuni archaea possiedono ulteriori polisaccaridi diversi anche dallo pseudopeptidoglicano.
\section{Locomozione microbica}
Molti batteri hanno la capacità di potersi muovere sotto il loro controllo, spesso con l’aiuto di strutture chiamate flagelli che 
permettono loro di rispondere a degli stimoli ambientali, detti tassi. I due principali tipi di movimento sono nuotare e moto a 
scorrimento (glinding). 
\subsection{Il flagello}
\begin{figure}[H]
	\includegraphics[width=\textwidth]{Pictures/14.png}
\end{figure}

I flagelli batterici sono lunghe e sottili appendici libere da un'estremità e attaccate alla cellula all'altra estremità. I flagelli batterici sono così 
sottili ($15$-$20 nm$, a seconda della specie) che un singolo flagello non può essere visto dalla microscopia ottica a meno che non sia macchiato per aumentarne 
il diametro. Tuttavia, i flagelli sono facilmente visibili con il microscopio elettronico. I flagelli differiscono per numero e posizione  a seconda della 
specie. Possono assumere un’organizzazione peritrica (peritrichous) (a) quando numerosi flagelli 
sono distribuiti su tutta la superficie cellulare. Flagelli polari si trovano soltanto alle estremità della cellula (b). L’organizzazione lofotrica 
(lophotrichous) (c) indica la presenza di più di un flagello polare. \\
\newpage
\begin{multicols}{2}
\begin{figure}[H]
	\includegraphics[width=0.48\textwidth]{Pictures/15.png}
\end{figure}
	
\begin{figure}[H]
	\includegraphics[width=0.48\textwidth]{Pictures/16.png}
\end{figure}
\end{multicols}

Il flagello \`e  un’elica la cui la distanza tra curve adiacenti è 
costante. Questa lunghezza d'onda è una caratteristica che differisce da specie a specie. \`E composto principalmente da tre parti: il filamento, l’uncino 
(hook) e la base. Mentre le prime due hanno una composizione chimica e una struttura molto simile tra i vari batteri, il modo in cui il 
flagello è ancorato alla parete e alla membrana cellullare cambia a seconda che il batterio sia gram-positivo o gram-negativo. Il filamento è composto da 
molte copie della proteina flagellina. La forma e la lunghezza d'onda del flagello sono in parte determinate dalla struttura della proteina flagellina e in 
dalla direzione di rotazione del filamento. La sequenza di amminoacidi della flagellina è altamente conservata in specie di batteri, 
suggerendo che la motilità dei flagelli si è evoluta presto e ha radici profonde all'interno di questo dominio. L’uncino è chimicamente diverso dal 
filamento ed è costituito da un solo tipo di proteina. La base è ancorata alla membrana citoplasmatica e alla parete cellulare e ha una funzione meccanica, con un funzionamento analogo 
a quello di un motore a propellenza. Il motore consiste in un’asta centrale, chiamata bastoncello, che 
attraversa diversi anelli. Nei batteri gram-positivi, che non hanno una membrana esterna, possiedono solo la coppia di anelli più interna.
 Intorno 
all'anello interno e ancorato nella membrana citoplasmatica c'è una serie di proteine chiamate proteine Mot. Un insieme finale di proteine, chiamate 
proteine Fli, funzionano come l'interruttore del motore, invertendo la direzione di rotazione del flagello in risposta ai segnali intracellulari. Nei 
batteri gram-negativi, un anello esterno, chiamato anello L, è ancorato nello strato di lipopolissaccaride. Un secondo anello, chiamato anello P, è ancorato 
nello strato di peptidoglicano della parete cellulare. Una terza serie di anelli, chiamati anelli MS e C, si trovano rispettivamente all'interno della 
membrana citoplasmatica e del citoplasma.
\subsection{Il movimento flagellare}
\begin{wrapfigure}{l}{0.5\textwidth}
  \begin{center}
    \includegraphics[width=0.48\textwidth]{Pictures/17.png}
  \end{center}
\end{wrapfigure}
Il flagello è un piccolo motore rotativo. Questi motori contengono due componenti principali: il rotore e lo statore. Nel motore del flagello, il rotore è 
costituito dall'asta centrale e dagli anelli L, P, C e MS. Collettivamente, queste strutture costituiscono il corpo basale. Lo statore è costituito dalle 
proteine Mot che circondano il corpo basale e generano momento. La rotazione del flagello è impartita dal corpo basale. L'energia necessaria per la 
rotazione del flagello proviene dalla forza protomotrice. Il movimento dei protoni attraverso la membrana citoplasmica attraverso il complesso Mot aziona la 
rotazione del flagello. Circa 1000 protoni sono traslocati ad ogni rotazione del flagello. In questo modello di turbina protonica, i protoni che scorrono 
attraverso i canali delle proteine Mot esercitano forze elettrostatiche su cariche disposte ad elica sulle proteine del rotore. Le attrazioni tra cariche 
positive e negative causano quindi la rotazione del corpo basale man mano che i protoni scorrono attraverso le proteine Mot.
\subsubsection{Il movimento flagellare nelle diverse categorie}
\begin{wrapfigure}{l}{0.5\textwidth}
  \begin{center}
    \includegraphics[width=0.48\textwidth]{Pictures/18.png}
  \end{center}
\end{wrapfigure}
Viene descritto il movimento in procarioti peritrichous e polarizzati.
\begin{itemize}
	\item[(a)] Peritrichous:il movimento in avanti \`e causato da tutti i flagelli che routano in senso antiorario (CCW) in un fascio. La rotazione oraria (CW) causa un 
		``ruzzolamento" (tumble). della cellula, che le permette di modificare la direzione all'inizio di una nuova rotazione dei flagelli in senso antiorario. 
	\item[(b)] Polare: la cellula cambia direzione invertendo la rotazione flagellare passando dal tirare allo spingere la cellula o, con flagelli unidirezionali, fermandosi periodicamente per riorientarsi e successivamente 
		muovendosi in avanti attraverso una rotazione in senso orario. 
\end{itemize}
La freccia gialla mostra la direzione in cui la cellula si sta muovendo.
\subsection{La sintesi flagellare}
La sintesi del flagello non avviene alla base, come ad esempio i capelli umani, ma dalla punta. Per primo viene sintetizzato l’anello MS ed inserito nella 
membrana citoplasmatica. Successivamente altre proteine di ancoraggio vengono sintetizzate insieme all’uncino, prima che il filamento cominci a formarsi. Le 
subunità di flagellina, sintetizzate nel citoplasma, vengono estruse attraverso un canale di $3 nm$ all'interno del filamento e si aggiungono alla fine per 
formare il flagello maturo. Un ``cap" proteico è presente alla fine del flagello in crescita, che aiuta le molecole di flagellina che si sono diffuse 
attraverso il canale del filamento ad assemblarsi in modo corretto. 
\begin{figure}[H]
	\includegraphics[width=\textwidth]{Pictures/19.png}
\end{figure}
\subsection{I flagelli negli spirocheti}
Nei batteri ``spirocheti”, ossia che assumono una forma a spirale, i flagelli sono assiali e si trovano tra la membrana citoplasmatica e quella esterna (sono 
gram-negativi). La rotazione dell’endoflagello porta l’intera cellula a ruotare su se stessa in un movimento elicoidale che consente lo spostamento anche in 
ambienti molto viscosi.
\begin{figure}[H]
	\includegraphics[width=\textwidth]{Pictures/20.png}
\end{figure}
