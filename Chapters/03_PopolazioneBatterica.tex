\chapter{Popolazione batterica}

\section{Crescita batterica}
Per crescita batterica si intende il processo di duplicazione e moltiplicazione del numero di batteri.
Questo processo \`e detto scissione binaria: da una cellula madre vengono generate due cellule figlie identiche.
Il processo \`e di tipo mitotico e non si trova variabilit\`a genetia.

	\subsection{Processo di scissione binaria}
	La scissione binaria inizia con la duplicazione del DNA che viene segregato in due cromosomi attaccati alla membrana.
	Il DNA rimane comunque diffuso nella cellula.
	Successivamente la cellula si allunga fino al doppio della sua lunghezza media e si replicano tutte le macromolecole contenute in essa.
	Avviene la formazione di un setto che consente la separazione dei due batteri.
	I due cromosomi rimangono attaccati alla membrana.

	\subsection{Divisoma}
	Nella formazione dell'apparato di divisione \emph{divisoma} sono coinvolte molte proteine dette \emph{Fts} (filamentose sensibili alla temperatura).
	\begin{multicols}{2}
		\begin{itemize}
			\item \emph{FtsZ}: forma l'anello del divisoma.
				Formano un polimero di \num{10000} monomeri.
				Il processo di aggregazione e depolimerizzazione richiede idrolisi di \emph{GTP}.
			\item \emph{ZipA}: ancora l'anello di \emph{FtsZ} alla membrana interna.
			\item \emph{FtsA}: un'\emph{ATPasi} sulla membrana citoplasmatica, fornisce l'energia per l'assemblaggio delle proteine del divisoma.
			\item \emph{Fts1}: collega la membrana interna al peptidoglicano e media la sua biosintesi.
			\item \emph{FtsK}: separa le due cellule figlie.
		\end{itemize}
	\end{multicols}

	\subsection{Sintesi della parete cellulare}
	Prima della divisione cellulare \`e necessaria la sintesi di nuovo materiale parietale.
	Questo deve essere aggiunti senza che vi sia perdita dell'integrit\`a cellulare e delle molecole interna.
	\emph{Wall band} dividono la membrana vecchia da quella nuova e la sintesi del peptidoglicano avviene da parte di autolisine, che causano piccole rotture nella parete cellulare del batterio, spezzando parzialmente il peptidoglicano a partire dall'anello \emph{FtsZ}.
	Il materiale viene aggiunto attraverso queste fessure a cui viene trasportato dal bactoprenolo, una molecola lipidica con funzioni di trasporto.
	Il bactoprenolo trasporta \emph{NAM, NAG} e un pentapeptide.
	Attraverso flippingi precursori attraversano la membrana e nel periplasma interagisce con enzimi che inseriscono i precursori attraverso transpeptidazione.

		\subsubsection{Transpeptidazione}
		La transpeptidazione forma legami crociati tra residui amminoacilici di \emph{NAM}.
		Nei Gram$-$ avviene tra \emph{DAP} (acido diaminopimelico) e due residui di \emph{D-alanina},  uno dei quali viene eliminato per fornire energia.
		Nei Gram$+$ i legami crociati avvengono attraverso ponti pentaglicinic tra una \emph{L-lisina} e una \emph{D-alanina}.

\section{Crescita di una popolazione batterica}
La crescita di una popolazione batterica avviene con andamento esponenziale di fattore $2$.

	\subsection{Tempo di generazione}
	Si intende per tempo di generazione il tempo necessario per la duplicazione di una popolazione batterica.
	Il tempo medio di generazione per batteri in condizioni ottimali \`e di $30$ minuti.

		\subsubsection{Rappresentazione}
		La crescita viene normalmente rappresentata in scala semilogaritmica, con il tempo rappresentato linearmente e il numero di batteri logaritmicamente.

		\subsubsection{Calcolo dei dati}
		Si consideri:
		\begin{multicols}{3}
			\begin{itemize}
				\item $N$ il numero finale di cellule.
				\item $N_0$ il numero iniziale.
				\item $n$ il numero di generazioni.
			\end{itemize}
		\end{multicols}
		Allora:
		\[N=N_02^n\]
		\[n=\dfrac{\log N - \log N_0}{\log 2}\]
	
	\subsection{Fasi di crescita}
	Una coltura batterica presenta quattro fasi durante la sua crescita.

		\subsubsection{Fase di latenza}
		Nella fase di latenza o lag la cellula in condizione adatta non si divide per un certo periodo.
		Questo avviene in quanto si adatta all'ambiente, duplica le macromolecole.
		Non \`e presentata da cellule prelevate da una fase esponenziale.
		La si trova in cellule prelevate da una fase stazionaria, trasferite in un terreno pi\`u povero o batteri che hanno subito danni.

		\subsubsection{Fase esponenziale}
		Nella fase esponenziale avviene un aumento esponenziale dei batteri.
		Durante questa fase si trovano nel migliore stato di salute e pertanto gli esperimenti su di essi vengono svolti in questa fase.

		\subsubsection{Fase stazionaria}
		Durante la fase stazionaria non ci sono cambiamenti significativi nel numero di batteri.
		Inizia a causa di mancanza di nutrienti nel terreno e dell'accumulo dei prodotti di scarto.
		La scissione binaria \`e lenta e con velocit\`a comparabile alla morte delle cellule.
		I due processi si trovano pertanto in equilibrio dinamico.

		\subsubsection{Fase di morte}
		Durante la fase di morte avviene una diminuzione esponenziale del numero della popolazione con velocit\`a minore rispetto alla fase esponenziale.
		Si verifica a causa della mancanza di nutrienti o dell'accumulo di sostanze tossiche.

\section{Metodi di conta delle colonie batteriche}
	
	\subsection{Conta totale}
	Per effettuare una conta totale si prende un campione di coltura e lo si pone su un vetrino conta cellule con vari compartimenti a quadrato.
	Questo viene osservato al microscopio e si conta il numero di cellule in questi quadrati.
	Si convertono le misure in densit\`a.
		
		\subsubsection{Limiti}
		\begin{multicols}{2}
			\begin{itemize}
				\item Non si distinguono cellule vive e morte.
				\item Non possono essere osservati piccoli organismi.
				\item Non \`e accurato per ambienti in cui i batteri sono presenti in bassa densit\`a.
				\item Difficile ottenere stime precise.
				\item Si devono immobilizzare le cellule prima della conta.
			\end{itemize}
		\end{multicols}

	\subsection{Conta vitale}
	La conta vitale consiste nel determinare il numero di cellule in grado di formare una colonia o \emph{CFU}.
	Per farlo si piastra in superficie o per inclusione un volume nodo di coltura cellulare e lo si fa incubare.
	Successivamente si contano le colonie formate.

		\subsubsection{Diluizioni}
		Spesso vengono fatti diluizioni per riuscire a contare con maggiore precisione il numero di colonie formate: per una conta accurata si considera un massimo di $300$ colonie per campione.
		Naturalmente per avere la densit\`a iniziale si deve tenere conto delle diluizioni.

		\subsubsection{Cause di errori}
		\begin{multicols}{2}
			\begin{itemize}
				\item Periodo di incubazione sbagliato.
				\item Presenza di colonie vicine e indistinguibili.
				\item Colonie piccole e poco visibili.
				\item Pipettamento non accurato.
				\item Terreno di coltura non adatto.
			\end{itemize}
		\end{multicols}

	\subsection{Filtrazione su membrana}
	Questa misura d\`a una stima di batteri presenti in ambienti dove la loro concentrazione \`e molto bassa.
	Si filtra il liquido attraverso una membrana che cattura i batteri.
	Dopo il filtraggio si incuba e contano le colonie che si formano.
	Si deve tern conto della quantit\`a di liquido diluito.

	\subsection{Most Probable Number}
	Il most probable number \emph{MPN} \`e un metodo statistico utilizzzato per stimare cellule poco concentrate e difficili da far crescere in laboratorio.
	Essendo che maggiore il numero batteri maggiore il numero di diluizioni necessarie per ridurre il loro numero a zero si pongono in delle provette diluizioni seriali della coltura.
	Si fa incubare e si inserisce in ogni provetta un indicatore di $pH$: il $pH$ acido \`e indicatore di metabolismo attivo e presenza microbica.
	I tubi positivi ordinati per grado di diluizione creano un codice che letto su una tabella di riferimento ritorna la densit\`a batterica.

	\subsection{Misura della torbidit\`a}
	La misura della torbidit\`a avviene attraverso uno spettrofotometro: una fonte di luce attraversa un campione in una cuvetta con del terreno di coltura.
	Una fotocellula misura la quantit\`a di luce che \`e riuscita ad attraversarlo. 
	La misura calcolata \`e la densita ottica (OD) il logaritmo della luce incidente diviso il valore della luce non deviata. 
	$OD$ \`e un valore compreso tra $0$ e $1$, dove $0$ indica assenza di cellule mentre $1$ il massimo rilevabile.
	Pu\`o generare una sovrastima a basse concentrazioni in quanto non distingue tra cellule vive e cellule morte.
	In caso di alti valori di $OD$ la misura tende a perdere di precisione in quanto i raggi di luce in alte concentrazioni possono rimbalzare pi\`u volte su cellule diverse e colpire lo stesso il sensore. 
	Quando si raggiunge un OD di $0.6$-$0.7$ si diluisce la coltura. 
	Si dovr\`a tenere conto della diluizione moltiplicando il suo inverso al valore di $OD$.

\section{Sistemi di coltivazione microbica}

	\subsection{Spread-plate method}
	Nel piastramento in superficie si aggiunge un volume noto di una coltura su un terreno preparato, si mescola e incuba.

	\subsection{Pour-plate method}
	Nel piastramento per inclusione viene prima aggiunta la colonia e poi l'agar e si mescola dolcemente.
	\`E possibile utilizzare un volume maggiore di batteri e si formano pi\`u colonie anche all'interno dell'agar.

	\subsection{Chemostato}
	Il chemostato \`e un sistema aperto utilizzato per volumi molto grandi (si dice in batch per una coltura in un sistema chiuso e rapidamente si arriva a saturazione).
	Consente di mantenere la coltura in crescita esponenziale per tempi indefiniti. 
	Un rubinetto regola la sterilit\`a e un altro elimina l'overflow. 
	Viene pertanto continuamente introdotto terreno fresco e espulso terreno esausto insieme alle cellule. 
	All'equilibrio il volume del chemostato, il numero di densit\`a e la concentrazione dei nutrienti rimangono costanti.
	Questo viene detto steady state.
	La velocit\`a di crescita \`e pertanto determinata dalla velocit\` a di flusso e dalla concentrazione del nutriente limitante. 
	La velocit\`a di flusso deve essere abbastanza lenta da non dilavare la coltura ma abbastanza veloce da non impedire al nutriente limitante di sostenere i bisogni metabolici della popolazione.

\section{Effetti ambientali sulla crescita microbica}

	\subsection{Temperatura}
	La temperatura \`e il fattore che influenza maggiormente la crescita di un microorganismo.
	Per ciascuno di essi esiste una temperatura ottimale alla quale cresce.
	I grafici che mettono in relazione crescita batterica con temperatura presentano spesso lo stesso andamento, cambiando solo il range di temperatura relativo.
	La temperatura ottimale \`e sempre pi\`u vicina al massimo.

		\subsubsection{Motivazioni}
			
			\paragraph{Minimo}
			Al minimo avviene una gelificazione della membrana: i processi di trasporto sono cos\`i lenti che non pu\`o avvenire crescita e non si forma un gradiente protonico.

			\paragraph{Tra minimo e ottimo}
			Con l'aumento della temperatura le reazioni enzimatiche avvengono a tassi sempre pi\`u veloci.

			\paragraph{Ottimo}
			Le reazioni enzimatiche avvengo al tasso maggiore possibile.

			\paragraph{Massimo}
			Al massimo avviene la denaturazione delle proteine, collassa la membrana citoplasmatica e la lisi termica.

		\subsubsection{Classificazione degli organismi}
		\begin{multicols}{2}
			\begin{itemize}
				\item Psicrofili: temperatura ottimale $\le 15\si{\degree}$.
					Enzimi con maggiore quantit\`a di $\alpha$-eliche e maggiori amminoacidi polari.
					Meno legami deboli e meno interazioni tra domini proteici.
					Acidi grassi insaturi nella frazione lipidica della membrana la rende pi\`u fluida.
					Glicerolo o dimetillsolfossido \emph{DMSO} proteggono le cellule dalla formazione di cristalli di ghiaccio.
				\item Mesofili: vivono a temperature intermedie.
				\item Termofili: temperatura ottimale $\ge 45\si{\degree}\ \le 80\si{\degree}$.
					La stabilit\`a al calore \`e dovuta alla sostituzione di amminoacidi critici che determinano una struttura quaternaria pi\`u resistente.
					Contengono pi\`u amminoacidi polari e pi\`u legami ionici.
				\item Ipertermofili: si trovano solo negli archea e hanno una temperatura ottimale $\ge 80\si{\degree}$/
					Non contengono acidi grassi nelle membrane ma idrocarburi \emph{$C_40$}.
					Le membrane sono fatte da un unico monostrato lipidico. 
			\end{itemize}
		\end{multicols}

	\subsection{$\mathbf{pH}$}
	Il $pH$ o concentrazione di \emph{$H_3O^+$} influisce sulla crescita degli organismi.
	La maggior parte cresce in un $pH$ compreso tra $6$ e $8$, ma esistono estremofili.
	La maggior parte dei microorganismi vive in un intervallo di $2$-$3$ unit\`a.

		\subsubsection{Acidofili}
		Gli acidofili possono sopportare concentrazioni di \emph{$H^+$} molto elevate.
		I funghi sono pi\`u acido-tolleranti dei batteri.
		Le memgrane cellulari si lisano a $pH$ vicini alla neutralit\`a.
		Ricevono energia attraverso forza proton motrice derivata da pompe \emph{$H^+$}.

		\subsubsection{Alcalofili}
		Gli alcalofili sopportano concentrazioni di \emph{$H^+$} molto basse come nei laghi alcalini e sono per lo pi\`u archea.
		Si dividono in due classi rispetto al $pH$ interno che pu\`o essere mantenuto vicino alla neutralit\`a o no.
		Nel primo caso si crea un forte gradiente di ioni.
		Ricevono energia attraverso forza proton motrice derivata da pompe \emph{$Na^+$}.

	\subsection{Concentrazione \emph{NaCl}}
	Nella maggior parte dei microorganismi il citoplasma ha una concentrazione di soluti pi\`u elevata dell'ambiente esterno in modo che l'acqua tenda a penetrare all'interno della cellula.

		\subsubsection{Non alofili}
		Gli organismi non alofili non sopportano valori anche bassi di \emph{NaCl}.
		
		\subsubsection{Alotolleranti}
		Gli organismi alotolleranti crescono a valori medi di \emph{NaCl} ($1$-$15\%$).

		\subsubsection{Alofili}
		Gli organismi alofili crescono a concentrazioni medio-alte di \emph{NaCl}.

		\subsubsection{Alofili estremi}
		Gli organismi alofili estremi non possono crescere se nell'ambiente non \`e presente una certa concentrazione di \emph{NaCl} ($15$-$30\%$).

	\subsection{Attivit\`a dell'acqua}
	Si intende per attivit\`a dell'acqua $a_w$ la quantit\`a di acqua disponibile per l'organismo.
	Ovvero un indice relativo alla quantit\`a d'acqua che in un determinato prodotto \`e libera da legami con altri componenti
	\[a_w=\dfrac{P}{P_0}\]
	Dove:
	\begin{multicols}{2}
		\begin{itemize}
			\item $P$ \`e la pressione di vapore del prodotto.
			\item $P_0$ \`e la pressione di vapore dell'acqua pura per una stessa temperatura.
		\end{itemize}
	\end{multicols}
	Non sono noti organismi in grado di crescere dove l'attivit\`a dell'acqua \`e $\le 0.55$.
		
		\subsubsection{Osmoregolazione}
		In condizioni di bassa attivit\`a dell'acqua la pu\`o ottenere solo aumentando la concentrazione dei soluti al suo interno attraverso osmoregolazione.
		Questo viene ottenuto trasferendo ioni inorganici all'interno o sintetizzando un soluto organico.
		I pi\`u utilizzati o soluti compatibili sono amminoacidi, zuccheri o alcol.
		Una cellula che cresce in ambiente disidratato fa entrare e produce soluti in modo che l'acqua entri in essa per osmosi.

	\subsection{Ossigeno}

		\subsubsection{Classificazione}
		Gli organismi possono essere divisi in base alla loro risposta in presenza di ossigeno.
		\begin{center}
			\begin{tabular}{|c|c|c|c|}
				\hline
				\multicolumn{2}{|c|}{Gruppo} & Relazione con \emph{$O_2$} & Tipo di metabolismo\\
				\hline
				\multirow{3}{*}{Aerobi} & Obbligati & Richiesto & Respirazione aerobica\\
				\cline{2-4}
							& Facoltativi & \makecell{Non richiesto\\ma preferito} & \makecell{Respirazione aerobica\\anaerobica\\fermentazione}\\
				\cline{2-4}
							& Microaerofili & \makecell{Richiesto\\ma a livelli inferiori\\rispetto a quelli \\atmosferici} & Respirazione aerobica\\
				\hline
				\multirow{2}{*}{Anaerobi} & Aerotolleranti & \makecell{Non richiesto\\non preferito} & Fermentazione\\
				\cline{2-4}
							  & Obbligati & \makecell{Non tollerata\\o letale} & \makecell{Respirazione anaerobica\\fermentazione}\\
				\hline
			\end{tabular}
		\end{center}

		\subsubsection{Determinazione dello stato di areobiosi}
		Per determinare se un organismo \`e aerobio o anaerobio si pongono tubi di coltura con indicatore redox di colore rosa quando ossidato e incolore quando ridotto.
		L'aria entra nel tubo e arriva fino al terreno e se ne trova sempre meno aumentando la profondit\`a.
		La posizione di crescita lungo il tubo determina lo stato di aerobiosi.

		\subsubsection{Forme tossiche dell'ossigeno}
		Esistono delle forme tossiche dell'ossigeno come superosside o perosside di idrogeno.
		Questi sono detti radicali liberi e si formano durante la reazione di creazione dell'acqua.
		\begin{align*}
			O_2 + e^- &\rightarrow O_2^- &\text{Superosside}\\
			O_2^- + e^- + 2H^+ &\rightarrow H_2O_2 &\text{Perosside di idrogeno}\\
			H_2O_2 + e^- + H^+ &\rightarrow H_20 + OH\mathbf{\cdot} &\text{Radicale idrossile}\\
			OH\mathbf{\cdot} + e^- + H^+ &\rightarrow H_2O &\text{Acqua}
		\end{align*}

		I batteri che riescono a vivere in aerobiosi possiedono enzimi che distruggono i composti tossici dell'ossigeno:
		\begin{multicols}{2}
			\begin{itemize}
				\item Catalasi.
				\item Perossidasi.
				\item Superosside dismutasi.
				\item Superosside riduttasi.
			\end{itemize}
		\end{multicols}

	\subsection{Pressione}
	I barofili sono batteri in grado di resistere a forti pressioni atmosferiche.
