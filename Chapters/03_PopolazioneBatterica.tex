\chapter{Popolazione batterica}
Il grafico a slide 29 mostra le scale logaritmiche (andamento in moltiplicazioni di $10$) e aritmetiche per mostrare l'andamento della popolazione batterica. Il tempo di generazione 
pu\`o essere calcolato come:
\begin{itemize}
	\item N il numero finale di cellule.
	\item $N_0$ il numero iniziale di cellule.
	\item $n$ il numero di generazioni.
	\item $N=N_02^n$.
	\item $n=\frac{\log N - \log N_0}{\log 2}$.
\end{itemize}
La popolazione ha un andamento in quattro fasi: latenza (assenza di crescita significativa, le cellule costituiscono tutte le operazioni necessarie alla crecita), esponenziale (crescita 
massima), stazionaria (la popolazione si stabilizza per mancanza di spazio e nutrienti, il numero di cellule che si dividono \`e uguale a quello di quelle che muoiono), morte (
esponenziale ma pi\`u lenta della fase di crescita, causata da assenza di nutrienti e l'accumulo di materiale di scarto tossico). 
\subsection{Conta vitale}
Piastre con lo spread-plate method con terreno solido con un volume di colura ($100 \mu l$), dopo una notte si osserva il numero di colonie. Nel pour-plate method si aggiunge il terreno
successivamente alla colonia. Per determinare quante colonie sono presenti si fa una diluizione seriale: si prende un $ml$ di coltura e lo si aggiunge a $9ml$ di terreno, ripetendo 
l'operazione fino a che si riesce a stimare la popolazione microbica (tra le $30$ e le $300$ colonie per avere una conta affidabile). Alla conta si moltiplica per l'inverso della 
diluizione. \`E importante ovviamente segnare i tempi in quanto la popolaizone varia nel tempo. Misura solo le cellule vive.
\subsection{Misura della torbidit\`a}
Avviene attraverso uno spettrofotometro: una fonte di luce \`e presente e attraversa un campione in una cuvetta con del terreno di coltura. Una fotocellula misura la quanit\`a di luce
che \`e riuscita ad attraversare la cuvetta. La misura calcolata \`e la densita ottica (OD) il logaritmo della luce incidente diviso il valore della luce non deviata. Valore compreso 
tra $0$ e $1$. Pu\`o generare una sovrastima in quanto non distingue tra cellule vive e cellule morte. A differenza della conta vitale in caso di alti valori di OD la misura tende a 
perdere di precisione in quanto i raggi di luce in alte concentrazioni possono rimbalzare pi\`u volte e colpire lo stesso il sensore. Quando si raggiunge un OD di 0.6 0.7 si diluisce
la coltura. 
Quando si lavora con campioni ambientali bisogna concentrare il campione, pertanto si filtra a vuoto con filtro di membrana che trattiene le cellule batteriche in quanto pi\`u grandi
dei pori e le rende osservabili e vengono messe su un terreno di coltura. 
\subsection{Most Probable Number}
Utilizzata per stimare la popolazione microbica all'interno di un campione. Un metodo statistico basato sul fatto che maggiore la concentrazione della popolazione, maggiori le diluizioni
necessarie per portare questo numero a $0$. Prendo un campione, lo diluisco serialmente e partendo da ciascuna diluizioni si incuba in 5 tubi un millilitro con un indicatore di pH che
ritorna giallo quando il pH si abbassa, indicamento di metabolismo attivo e di presenza microbica. Si contano quanti tubi hanno cambiato colore. Nel campione pi\`u concentrato ci sono
pi\`u campioni positivi. E si ottiene un codice, il numero di tubi positivi per ciascuna diluizione. Che pu\`o essere letto attraverso una tabella di riferimento. 
\section{Sistemi di coltivazione microbica}
\subsection{Chemostato}
Utilizzati per volumi molto grandi (si dice in batch per una coltura in un sistema chiuso e rapidamente si arriva a saturazione), questo \`e un sistema che consente di mantenere la 
colura in crescita esponenziale per tempi indefiniti. C'\`e la necessit\`a di un rubinetto che regola la sterilit\`a e un altro che elimina l'overflow. C'\`e sempre terreno fresco che
entra e esausto che esce insieme alle cellule. All'equilibrio il volume del chemostato, il numero di densit\`a e la concentrazione dei nutrianti rimangono costanti. La velocit\`a di 
crescita \`e pertanto determinata dalla velocit\` a di flusso, ovvero la concentrazione del nutriente limitante. 
INSERIRE GRAFO BATCH
INSERIRE GRAFO CHEMOSTATO
