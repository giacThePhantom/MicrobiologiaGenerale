\chapter{Meccanismi di trasporto cellulare}
La membrana citoplasmatica svolge numerose e fondamentali funzioni: 
\begin{itemize}
    \item \textbf{Barriera di permeabilit\`a}: previene la dispersione di sostanze e funziona come una porta di controllo per il trasporto di nutrienti verso l'interno e di sostanze di scarto verso l'esterno della cellula; 
    \item \textbf{Sito di ancoraggio}: sito di ancoraggio di molte proteine coinvolte nel trasporto, nella bioenergetica e nella chemiotassi; 
    \item \textbf{Conservazione dell'energia}: sito di generazione e di utilizzazione della forza proton-motrice.
\end{itemize}
\section{Tipologie di trasporto}
Ci sono 3 tipi di trasporto principali:
\begin{enumerate}
    \item \textbf{Diffusione passiva}: le molecole possono passare liberamente e secondo gradiente;
    \item \textbf{Diffusione facilitata}: le molecole possono passare ma hanno bisogno di particolari canali. Vanno anche esse secondo gradiente;
    \item \textbf{Trasporto attivo}: molecole che necessitano un canale appropriato e l'apporto di ATP, dato che si muovono contro-gradiente.
\end{enumerate}
\subsection{Diffusione passiva}
Le molecole si muovono dalla regione a pi\`u alta concentrazione a quella a pi\`u bassa per agitazione termica. Si possono diffondere liberamente molecole di H$_2$O, O$_2$ e CO$_2$. 
\subsection{Diffusione facilitata}
\`E simile alla diffusione passiva:
\begin{itemize}
    \item Movimento di molecole non energia dipendente; 
    \item Direzione del movimento da alta a bassa concentrazione; 
    \item Il valore del gradiente di concentrazione incide sul tasso di assorbimento.
\end{itemize}
Differisce dalla diffusione passiva:
\begin{itemize}
    \item Uso di molecole trasportatrici (carrier o permeasi); 
    \item Un minore gradiente di concentrazione \`e necessario per un significativo assorbimento delle molecole; 
    \item trasporta glicerolo, zuccheri ed aminoacidi.
\end{itemize}
Esistono vari tipi di diffusione facilitata:
\begin{itemize}
    \item \textbf{Non specifico} $\xrightarrow{}$ alcune proteine consentono il passaggio di molecole di una certa dimensione o carica elettrica; 
    \item \textbf{Specifico} $\xrightarrow{}$ le proteine permeasi sono pi\`u specifiche grazie ad un sito di legame per il substrato del trasportatore.
\end{itemize}
\subsubsection{Osmosi}
L'osmosi \`e un esempio di diffusione facilitata. \`E la diffusione di acqua attraverso una membrana selettivamente permeabile, dal compartimento con minor concentrazione di soluti vero quello con la maggior concentrazione di soluti. 
\\Una membrana separé due soluzione con diverse concentrazioni. Questa membrana \`e permeabile all'acqua ma non hai soluti. L'acqua si sposta verso il compartimento con maggiore concentrazione in soluti. Quindi la pressione osmotica contrasta la forza di gravit\`a.
In base alla differenza di concentrazione si possono definire tre tipologie di ambienti: 
\begin{itemize}
    \item \textbf{Isotonica} $\xrightarrow{}$ nessun movimento netto di acqua; 
    \item \textbf{Ipertonica} $\xrightarrow{}$ uscita di acqua causando la diminuzione del volume cellulare; 
    \item \textbf{Ipotonica} $\xrightarrow{}$ ingresso di acqua nella cellula che porta la lisi delle cellule prime di parete.
\end{itemize}
Dato che i batteri presentano una parete cellulare sono in grado di resistere bene ai vari stress osmotici.
\subsection{Trasporto attivo}
Il trasporto attivo richiede energia per movimentare le sostanze contro il proprio gradiente di concentrazione. Questa energia viene formata dall'idrolisi dell'ATP o dalla forza proton-motrice. 
\\Possono essere di diverso tipo:
\begin{itemize}
    \item \textbf{Uniporto} $\xrightarrow{}$ portano un solo tipo di sostanza in una sola direzione; 
    \item \textbf{Antiporto} $\xrightarrow{}$ portano due tipi di sostanze in direzioni opposte (una fuori e l'altra dentro); 
    \item \textbf{Simporto} $\xrightarrow{}$ portano due molecole diverse nella stessa direzione.
\end{itemize}
Un esempio \`e l'assunzione del lattosio in \textit{E. coli} tramite Lac permeasi. 
\\La lac permeasi richiede energia per importare il lattosio nella cellula. Man mano che trasporta il lattosio, l'energia della forza proton-motrice si riduce a causa del trasporto simultaneo di protoni all'interno della cellula.
\\\\Esistono tre tipi di movimenti: 
\begin{enumerate}
    \item \textbf{Trasporto semplice} $\xrightarrow{}$ guidato dall'energia associata alla forza proton-motrice; 
    \item \textbf{Traslocazione di gruppo} $\xrightarrow{}$ modificazione chimica della sostanza trasportata guidata dal fosfoenolpiruvato. Per esempio, viene utilizzato per far entrare nella cellula il glucosio. Il glucosio viene trasportato attraverso un canale e modificato chimicamente con l'aggiunta di un gruppo fosfato. Nella fosforilazione del glucosio a glucosio 6-P \`e il primo stadio del suo metabolismo cellulare: la glicolisi. Il sistema prepara il glucosio in modo che possa essere immediatamente assunto in una via metabolica; 
    \item \textbf{Sistema ABC (ATP - Binding Cassette)} $\xrightarrow{}$ coinvolge le proteine periplasmatiche di legame e l'energia proviene dall'ATP. Vengono impiegati tre componenti:
    \begin{itemize}
        \item proteine di legame periplasmatiche; 
        \item proteine integrali di membrana; 
        \item proteine per l'idrolisi dell'ATP.
    \end{itemize}
    Le proteine di legame periplasmatiche mostrano un'elevata affinit\`a per il loro substrato; una volta sequestrato, esso forma un complesso che interagisce con la proteina integrale di membrana. \`E un processo altamente specifico per vari tipi di substrato, come zuccheri, aminoacidi, solfati, fosfati, metalli, e sensibile.
\end{enumerate}
I gram-negativi richiedono un sistema di trasporto pi\`u complicato rispetto a quello dei gram-positivi. Per esempio il trasporto di ferro nei gram-negativi richiede un sistema ABC e l'uso della forza proton-motrice (attraversamento della membrana esterna tramite un recettore TonB-dipendente). Questo \`e un sistema combinato, dove:
\begin{enumerate}
    \item ABC serve per l'attraversamento della membrana interna; 
    \item Enterochelina riconosce il ferro e lo passa alla proteina sottostante.
\end{enumerate}
\section{Sistemi di traslocazione nei procarioti}
La traslocazione \`e il trasferimento di una proteina da un compartimento ad un altro. Molte proteine necessitano di essere trasportate fuori dalla cellula o di essere inserite in modo specifico nella membrana. Nei procarioti l'esportazione delle proteine avviene attraverso l'attivit\`a di proteine chiamate traslocasi. 
\\Il riconoscimento di una proteina da parte del sistema Sec avviene prima che venga ultimata la sintesi del ribosoma. La prima regione a venire sintetizzata a livello della porzione N-terminale delle proteine \`e un sito di riconoscimento che si chiama sequenza segnale. La sequenza segnale \`e formata da 15-30 aminoacidi:
\begin{itemize}
    \item Porzione idrofila basica; 
    \item Regione idrofoba; 
    \item Porzione polare.
\end{itemize}
Nel momento in cui la proteina viene portate fuori, viene rimossa la sequenza segnale.
\subsection{Via di secrezione dipendente da Sec}
Questa via di secrezione \`e formata da 3 componenti:
\begin{itemize}
    \item Complesso transmembrana; 
    \item "Motore" citoplasmatico che fornisce l'energia; 
    \item Sistema citoplasmatico che riconosce le proteine da trasportare.
\end{itemize}
Normalmente questa via \`e composta da vari passaggi: 
\begin{enumerate}
    \item Il ribosoma traduce l'mRNA per la proteina da trasportare; 
    \item SecB lega la proteina nascente e ne rallenta il ripiegamento aiutandola a raggiungere il complesso di membrana; 
    \item SecA lega SecB e porta la proteina a prossimit\`a di SecYEG; 
    \item SecB rilascia la proteina; il legame dell'ATP a SecA ne cambia la conformazione, iniziando il processo di traslocazione; 
    \item L'idrolisi dell'ATP in ADP + P$_i$ fornisce l'energia necessaria alla traslocazione;
    \item La sequenza segnale viene rimossa da una peptidasi segnale.
\end{enumerate}
Alcune proteine non devono essere portate all'esterno, ma devono essere inserite nella membrana come canali o strutture per il riconoscimento. Per fare questo viene utilizzato il sistema SRP, che \`e costituito da una proteina e da una molecola di RNA.
\begin{enumerate}
    \item SRP lega una specifica sequenza segnale della proteina nascente; 
    \item FtsY lega il complesso SRPr-ribosoma e lo indirizza alla membrana, verso il complesso SecY o verso un'altra proteina di membrana, YidC;
    \item La proteina si ripiega all'interno della membrana nella sua conformazione funzionale.
\end{enumerate}
\subsection{Sistema di secrezione di tipo II}
\`E un sistema altamente specifico e viene utilizzato per: 
\begin{itemize}
    \item Produzione di tossine, cellulasi, proteasi, lipasi in molti batteri patogeni; 
    \item assemblaggio di strutture superficiali come i pili.
\end{itemize}
Funzionamento:
\begin{enumerate}
    \item Attraversamento della membrana con il sistema Sec; 
    \item Taglio della sequenza segnale nel periplasma e ripiegameto della proteina;
    \item Secrezione attraverso il complesso di Golgi che fornisce anche l'energia necessaria tramite l'idrolisi di ATP o GTP.
\end{enumerate}
\subsection{Sistema di secrezione a 2 partner (TPS, Two Partner Secretion)}
\`E un sistema costituito da una singola proteina di trasporto che funge da canale. Infatti questo tipo di secrezione non ha un canale dedicato gi\`a presente.
\\La proteina trasportata (TpsA) possiede, oltre alla sequenza segnale, un dominio TPS importante per il riconoscimento della proteina canale (TpsB) che ne media il trasporto. 
\\Una volta secreta, la proteina pu\`o rimanere associata alla superficie del batterio o rilasciata nell'ambiente extracellulare. 
\subsection{Sistema dell'autotrasporto, tipo V}
In questa via la proteina non necessita di altri fattori: il dominio C-terminale della proteina ne media la secrezione. 
\\La sequenza segnale N-terminale viene riconosciuta e traslocata attraverso Sec nel periplasma; il dominio interno passa nel periplasma (passenger domain, dominio funzionale della membrana). Poi il dominio C-terminale va a formare il canale di trasporto. Avviene il taglio della proteina funzionale tramite autoproteolisi o proteolisi mediata da una proteasi specifica nella membrana esterna.
\subsection{Secrezione attraverso la via "chaperon/usher"}
Questa via viene utilizzata per la secrezione e l'assemblaggio di strutture della superficie cellulare, come alcuni tipi di pili, strutture di adesione. Viene richiesta la presenza di 2 proteine:
\begin{itemize}
    \item Chaperonina periplasmatica; 
    \item Proteina di membrana esterna (usciere).
\end{itemize}
Per prima cosa il sistema Sec trasporta le subunit\`a di pilina nel periplasma. Dopo viene rimossa la sequenza segnale e si instaura un legame tra chaperonina PapD ad una regione C-terminale conservata. Questo evita l'aggregamento prematuro delle subunit\`a di pilina. PapD porta le subunit\`a alla proteina usher che procede con la secrezione e l'assemblaggio.
\section{Vie delle secrezioni indipendenti da Sec}
\subsection{Il sistema Tat: Twin-arginine translocation system}
Questo sistema \`e costituito da 3 proteine di membrana: TatA, TatB, TatC, con la presenza di arginina. 
\begin{enumerate}
    \item Viene riconosciuta la sequenza N-terminale; 
    \item Produzione e ripiegamento della proteina con l'intervento di chaperonine e aggiunta di eventuali co-fattori; 
    \item Indirizzamento al complesso di membrana TatBC; 
    \item Il canale di trasporto  (TatA) si associa al complesso TatBC e ne consente il passaggio. La sequenza segnale viene rimossa nel periplasma. 
\end{enumerate}
L'energia viene fornita dalla forza proton-motrice. 
\subsection{I trasportatori ABC: ATP-Binding Cassette}
In questa via non vi \`e un passaggio da un intermedio periplasmatico. \`E costituita da 3 componenti:
\begin{itemize}
    \item Proteina associata alla membrana interna per idrolizzare ATP (dominio ABC); 
    \item Proteina che si estende nel periplasma (MFP); 
    \item Proteina associata alla membrana esterna (OMP).
\end{itemize}
Non \`e presente la sequenza segnale N-terminale, ma una sequenza di riconoscimento C-terminale che alla fine non viene rimossa. 
\subsection{Il sistema di secrezione di tipo III}
\`E un sistema che viene annoverato come fattore di virulenza e deve attraversare 3 membrane: quella interna, quella esterna e quella della cellula bersaglio.
\\Il batterio secerne proteine tossiche direttamente nel citoplasma di una cellula eucariota bersaglio e in questo modo ne causa la morte. Il sistema viene attivato dal contatto con la cellula ospite e presenta un struttura complessa composta d pi\`u di 20 proteine diverse. Questo sistema presenta una certa omologia con le proteine che costituiscono il corpo basale del flagello. 
\\La secrezione necessita di energia ed \`e assistita dalle chaperonina (Syc) che hanno il compito di portare la proteina alla base del canale di trasporto. 
\\La proteina YopN funge da apertura/chiusura del canale. 
\\Dopo si ha la formazione di un poro (canale) nella cellula bersaglio (YopB e YopD). 
\subsection{Il sistema di secrezione di tipo IV}
Questo \`e un processo molto versatile che secreta DNA (coniugazione) o proteine. 
\\Avviene una traslocazione direttamente nelle cellule bersaglio. Sono presenti almeno 12 proteine diverse che formano una struttura che abbraccia gli involucri del batterio. Questo sistema porta all'assemblaggio di monomeri che formano il pilo IV a spirale. 
