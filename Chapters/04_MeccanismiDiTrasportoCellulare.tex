\chapter{Meccanismi di trasporto cellulare}
La membrana citoplasmatica svolge numerose e fondamentali funzioni: 
\begin{itemize}
    \item \textbf{Barriera di permeabilit\`a}: previene la dispersione di sostanze e funziona come una porta di controllo per il trasporto di nutrienti verso l'interno e di sostanze di scarto verso l'esterno della cellula; 
    \item \textbf{Sito di ancoraggio}: sito di ancoraggio di molte proteine coinvolte nel trasporto, nella bioenergetica e nella chemiotassi; 
    \item \textbf{Conservazione dell'energia}: sito di generazione e di utilizzazione della forza proton-motrice.
\end{itemize}

\section{Tipologie di trasporto}
	
	\subsection{Diffusione passiva}
	Nella diffusione passiva le molecole possono passare liberamente e secondo gradiente.
	Le molecole si muovono dalla regione a pi\`u alta concentrazione a quella a pi\`u bassa per agitazione termica. 

	Si possono diffondere liberamente molecole di \emph{H$_2$O}, \emph{O$_2$} e \emph{CO$_2$}, pertanto molecole piccole e apolari.
	Pi\`u alto il gradiente pi\`u il flusso \`e evvicace e veloce.	


	\subsection{Diffusione facilitata}
	Nella diffusione facilitata le molecole si muovono secondo gradiente ma necessitano di canali particolari che ne permettono il passaggio.
	Come nella diffusione passiva il movimento non richiede energia e il valore del gradiente incide sul tasso di assorbimento.
	Questo movimento richiede per\`o uso di molecole trasportatici come carrier o permeasi, che rendono necessario un minore gradiente di concentrazione affinch\`e avvenga un assorbimento significativo della molecola.
	Questo processo viene utilizzato per il trasporto di glicerolo, zuccheri e amminoacidi.

		\subsubsection{Tipologie di diffusione facilitata}

			\paragraph{Non specifica}
			La diffusione facilitata non specifica avviene tramite proteine che permettono il passaggio di molecole di una certa dimensione o carica elettrica.

			\paragraph{Specifica}
			La diffusione facilitata specifica avviene tramite permeasi, proteine specifiche che possiedono un sito di legame per il substrato da trasportare.

			\paragraph{Osmosi}
			L'osmosi \`e la diffusione di acqua attraverso una membrana selettivamente permeabile.
			Nella cellula la maggior parte del trasporto dell'acqua viene svolto da porine, proteine che formano canali con molti amminoacidi idrofilici.

			Il movimento avviene dal compartimento con minor concentrazione di soluti vero quello con la maggior concentrazione di soluti. 

			La membrana separa due soluzioni ed \`e permeabile all'acqua ma non ai soluti, pertanto questa si sposta verso il compartimento con maggiore concentrazione dei secondi.
			La pressione osmotica \`e in grado di contrastare la forza di gravit\`a.

				\subparagraph{Ambiente isotonico}
				Nell'ambiente isotonico la concentrazione di soluti nella cellula \`e uguale a quella dell'ambiente esterno, pertanto non avviene nessun movimento netto di acqua.

				\subparagraph{Ambiente ipertonico}
				Nell'ambiente ipertonico la concentrazione di soluti nella cellula \`e minore maggiore a quella dell'ambiente esterno, pertanto l'acqua esce causando la diminuzione del volume cellulare.

				\subparagraph{Ambiente ipotonico}
				Nell'ambiente ipotonico la concentrazione di soluti nella cellula \`e maggiore rispetto a quella dell'ambiente, pertanto l'acqua entra causando la lisi delle membrane non protette da una parete.

				\subparagraph{Resistenza dei batteri}
				La parete cellulare dei batteri conferisce loro resistenza ai vari stress osmotici.

	\subsection{Trasporto attivo}
	Il trasporto attivo richiede energia per movimentare le sostanze contro il proprio gradiente di concentrazione. 

	Questa energia viene formata dall'idrolisi dell'ATP o dalla forza proton-motrice. 

		
		\subsubsection{Meccanismi di trasporto}

			\paragraph{Uniporto}
			Durante l'uniporto un solo tipo di sostanza viene portato in una sola direzione.

			\paragraph{Antiporto}
			Durante l'antiporto due tipi di sostanza vengono trasportate in direzioni opposte.
			Una delle due molecole viaggia secondo gradiente e fornisce l'energia necessaria al trasporto dell'altra.

			\paragraph{Simporto}
			Durante il simporto due molecole vengono portate nella stessa direzione.
			Una delle due molecole viaggia secondo gradiente e fornisce l'energia necessaria al trasporto dell'altra.

			\paragraph{Esempi}

				\subparagraph{Assunzione del lattosio in E. coli}
				La Lac permeasi \`e la molecola responsabile del trasporto di lattosio in E. coli.
				Il trasporto avviene attraverso un meccanismo di simporto: l'energia necessaria al trasporto del lattosio \`e fornita dal trasporto di protoni all'interno della cellula.
				Con il passare del tempo l'energia si riduce, rallentando il trasporto.

		\subsubsection{Tipologie di trasporto attivo}
		Le proteine che permettono il trasporto sono proteine transmembrana con struttura tubulare e costituite da molte $\alpha$-eliche.

			\paragraph{Trasporto semplice}
			Il trasporto semplice \`e il trasporto di molecole guidato dall'energia associata alla forza proton-motrice.

			\paragraph{Traslocazione di gruppo}
			Nella traslocazione di gruppo avviene una modifica chimica della sostanza trasportata.
			In questo modo diventa impermeabile alla membrana della cellula e non ne pu\`o pi\`u uscire.

				\subparagraph{Trasporto del glucosio}
				Il glucosio viene trasportato attraverso un canale e fosforilato.
				La sua fosforilazione in glucosio $6-P$ \`e il primo stadio della glicolisi: il sistema lo prepara in modo che possa essere immediatamente assunto in un pathway metabolico.
				La fosforilazione viene guidata dal sistema della fosfotransferasi: $5$ proteine tra cui \emph{EnzI} e \emph{HPr} non specifiche e \emph{$ENZII_{a,b,c}$} specifiche per ogni zucchero.
				Un fosfoenolpiruvato \emph{PEP} cede un gruppo fosfato a un enzima \emph{EnzI} che lo cede ad altri enzimi in una catena, mentre l'ultimo lo cede al glucosio mentre sta attraversando il doppio strato lipidico.
				\[EnzI\rightarrow HPr\rightarrow EnzII_a\rightarrow EnzII_b\rightarrow EnzII_c\rightarrow Glu\]
				Ogni passaggio in questo pathway richiede energia.

			\paragraph{Sistema \emph{ABC} - \emph{ATP} - binding cassette}
			Il sistema \emph{ABC} coinvolge proteine periplasmatiche di legame e energia proveniente dall'idrolisi di \emph{ATP}.
			Questo sistema \`e spesso coinvolto nel trasporto di ioni \emph{$K^+$, $Na^+$, $Ca^{2+}$, $Cl^-$, $H^+$}.
			
				\subparagraph{Componenti}
				Le componenti coinvolte nel processo sono:
				\begin{multicols}{2}
					\begin{itemize}
						\item Proteine di legame periplasmatiche: riconoscono e legano le molecole da trasportare in modo specifico.
						\item Proteine integrali di membrana: fungono da canale.
						\item Proteine per l'idrolisi del \emph{ATP}: forniscono l'energia necessaria.
					\end{itemize}
				\end{multicols}

				\subparagraph{Procedimento}
				Le proteine di legame periplasmatiche mostrano un'elevata affinit\`a per il loro substrato e lo sequestrano.
				Si forma cos\`i un complesso che interagisce con una proteina integrale di membrana che causa il suo trasporto attraverso l'idrolisi di \emph{ATP}.

			\paragraph{Gram$\mathbf{-}$}
			I Gram$-$ richiedono un sistema di trasporto complesso in quanto deve attraversare due membrane.

				\subparagraph{Trasporto del ferro}
				Il trasporto del ferro nei Gram$-$ richiede un sistema \emph{ABC} per l'attraversamento della membrana interna e forza proton-motrice per l'attraversamento della membrana esterna.
				Il passaggio nella membrana esterna avviene tramite un recettore \emph{TonB}-dipendente.
				Si nota pertanto che viene utilizzato un sistema combinato di pi\`u tipologie di trasporto, dove:
				\begin{enumerate}
					\item \emph{ABC} permette l'attraversamento della membrana interna; 
					\item \emph{Enterochelina} riconosce il ferro e lo passa alla proteina sottostante permettendogli l'attraversamento della membrana interna.
				\end{enumerate}
\section{Secrezione, traslocazione ed esportazione}
In molti processi a livello microscopico la fuoriuscita di molecole \`e di fondamentale importanza.

	\subsection{Tipologie di trasporto}

		\subsubsection{Traslocazione}
		La traslocazione \`e il trasferimento di una proteina da un compartimento ad un altro. 
		Molte proteine necessitano di essere trasportate fuori dalla cellula o di essere inserite in modo specifico nella membrana. 
		Nei procarioti l'esportazione delle proteine avviene attraverso l'attivit\`a di proteine chiamate traslocasi. 
		Le traslocasi riconoscono sequenze specifiche sulle proteine che determinano il luogo di destinazione.

		\subsubsection{Secrezione}
		Nella secrezione la proteina viene secreta nell'ambiente esterno.

		\subsubsection{Esportazione}
		Nell'esportazione la proteina viene inserita e rimane attaccata alla membrana.

	\subsection{Il sistema \emph{Sec}}

		\subsubsection{Riconoscimento della sequenza leader}
		Il riconoscimento di una proteina da parte del sistema Sec avviene prima che venga ultimata la sintesi del ribosoma. 
		La coda $N$-terminale infatti \`e un sito di riconoscimento detto sequenza segnale.

			\paragraph{La sequenza segnale}
			La sequenza segnale \`e formata da 15-30 aminoacidi:
			\begin{multicols}{2}	
				\begin{itemize}
	 				\item Porzione idrofila basica; 
	    				\item Regione idrofoba; 
	    				\item Porzione polare.
				\end{itemize}
			\end{multicols}
			Dopo la traslocazione la sequenza segnale viene rimossa dopo un'alanina specifica.
	
		\subsubsection{Via di secrezione dipendente da \emph{Sec}}
			
			\paragraph{Componenti}
			La via di secrezione dipendente da \emph{Sec} \`e composta da tre componenti:
			\begin{multicols}{2}
				\begin{itemize}
			    		\item Complesso transmembrana; 
	    				\item ``Motore'' citoplasmatico che fornisce l'energia; 
	    				\item Sistema citoplasmatico che riconosce le proteine da trasportare.
				\end{itemize}
			\end{multicols}

			\paragraph{Processo di traslocazione}
			\begin{multicols}{2}
				\begin{enumerate}
	    				\item Il ribosoma traduce l'mRNA per la proteina da trasportare.
					\item \emph{SecB} lega la proteina nascente e ne rallenta il ripiegamento aiutandola a raggiungere il complesso di membrana.
					\item \emph{SecA} lega \emph{SecB} e porta la proteina a prossimit\`a di \emph{SecYEG}.
					\item \emph{SecB} rilascia la proteina; il legame dell'ATP a \emph{SecA} ne cambia la conformazione, iniziando il processo di traslocazione; 
	    				\item L'idrolisi dell'ATP in ADP + P$_i$ fornisce l'energia necessaria alla traslocazione;
	    				\item La sequenza segnale viene rimossa da una peptidasi segnale.
				\end{enumerate}
			\end{multicols}
			
			\paragraph{Inserimento in membrana - sistema \emph{SRP}}
			Alcune proteine non devono essere portate all'esterno, ma devono essere inserite nella membrana come canali o strutture per il riconoscimento. 
			Per fare questo viene utilizzato il sistema \emph{SRP}, che \`e costituito da una proteina e da una molecola di RNA.
			\begin{enumerate}
				\item \emph{SRP} lega una specifica sequenza segnale della proteina nascente; 
				\item \emph{FtsY} lega il complesso \emph{SRP}-ribosoma e lo indirizza alla membrana, verso il complesso \emph{SecY} o verso un'altra proteina di membrana, \emph{YidC};
	    			\item La proteina si ripiega all'interno della membrana nella sua conformazione funzionale.
			\end{enumerate}
	
			\paragraph{Sistema di secrezione di tipo $\mathbf{II}$}
			\`E un sistema altamente specifico e viene utilizzato per: 
			\begin{multicols}{2}
				\begin{itemize}
	    				\item Produzione di tossine, cellulasi, proteasi, lipasi in molti batteri patogeni; 
	    				\item assemblaggio di strutture superficiali come i pili.
				\end{itemize}
			\end{multicols}

				\subparagraph{Processo}
				\begin{enumerate}
					\item Attraversamento della membrana con il sistema \emph{Sec}; 
	    				\item Taglio della sequenza segnale nel periplasma e ripiegameto della proteina;
	    				\item Secrezione attraverso il complesso di Golgi che fornisce anche l'energia necessaria tramite l'idrolisi di ATP o GTP.
				\end{enumerate}

			\paragraph{Sistema di secrezione a $\mathbf{2}$ partner (\emph{TPS}, Two Partner Secretion)}
			\`E un sistema costituito da una singola proteina di trasporto che funge da canale. 
			Questo tipo di secrezione non ha un canale dedicato gi\`a presente.
			La proteina trasportata (\emph{TpsA}) possiede, oltre alla sequenza segnale, un dominio \emph{TPS}.
			Il dominio \emph{TPS} riconosce la proteina canale \emph{TpsB} che ne media il trasporto.
			Una volta secreta, la proteina pu\`o rimanere associata alla superficie del batterio o rilasciata nell'ambiente extracellulare. 

			\paragraph{Sistema dell'autotrasporto, tipo $\mathbf{V}$}
			Il dominio $C$-terminale della proteina stessa media la sua secrezione.
			La sequenza segnale N-terminale viene riconosciuta e traslocata attraverso \emph{Sec} nel periplasma.
			Il dominio interno funzionale o passenger domain passa nel periplasma.
			Il dominio $C$-terminale va a formare il canale di trasporto. 
			La proteina funzionale viene tagliata tramite autoproteolisi o proteolisi mediata da una proteasi specifica nella membrana esterna.

			\paragraph{Secrezione attraverso la via ``chaperon/usher''}
			Questa via viene utilizzata per la secrezione e l'assemblaggio di strutture della superficie cellulare, come alcuni tipi di pili e strutture di adesione. 
			
				\subparagraph{Componenti}
				\begin{itemize}
					\item Chaperonina periplasmatica; 
	    				\item Proteina di membrana esterna (usciere).
				\end{itemize}

				\subparagraph{Processo}
				Per prima cosa il sistema \emph{Sec} trasporta le subunit\`a di pilina nel periplasma. 
				Dopo viene rimossa la sequenza segnale e si instaura un legame tra chaperonina \emph{PapD} ad una regione C-terminale conservata. 
				Questo evita l'aggregamento prematuro delle subunit\`a di pilina. 
				\emph{PapD} porta le subunit\`a alla proteina usher che procede con la secrezione e l'assemblaggio.

	\subsection{Vie delle secrezioni indipendenti da Sec}

		\subsubsection{Il sistema Tat: Twin-arginine translocation system}
		Questo sistema \`e costituito da 3 proteine di membrana: \emph{TatA}, \emph{TatB}, \emph{TatC}, con la presenza di arginina. 

			\paragraph{Processo}
			\begin{multicols}{2}
				\begin{enumerate}
    					\item Viene riconosciuta la sequenza N-terminale; 
    					\item Produzione e ripiegamento della proteina con l'intervento di chaperonine e aggiunta di eventuali co-fattori; 
    					\item Indirizzamento al complesso di membrana TatBC; 
    					\item Il canale di trasporto  (TatA) si associa al complesso TatBC e ne consente il passaggio. 
						La sequenza segnale viene rimossa nel periplasma. 
				\end{enumerate}
			\end{multicols}
			L'energia viene fornita dalla forza proton-motrice. 

		\subsubsection{I trasportatori ABC: ATP-Binding Cassette}
		In questa via non vi \`e un passaggio da un intermedio periplasmatico. 
		
			\paragraph{Componenti}
			\begin{multicols}{2}
				\begin{itemize}
					\item Proteina associata alla membrana interna per idrolizzare ATP (dominio \emph{ABC}); 
					\item Proteina che si estende nel periplasma (\emph{MFP}); 
					\item Proteina associata alla membrana esterna (\emph{OMP}).
				\end{itemize}
			\end{multicols}
			Non \`e presente la sequenza segnale N-terminale, ma una sequenza di riconoscimento C-terminale che alla fine non viene rimossa. 

			\subsubsection{Il sistema di secrezione di tipo $\mathbf{III}$}
			\`E un sistema che viene annoverato come fattore di virulenza.
			Deve attraversare 3 membrane: quella interna, quella esterna e quella della cellula bersaglio.
			Il batterio secerne proteine tossiche direttamente nel citoplasma di una cellula eucariote bersaglio e in questo modo ne causa la morte. 
			Il sistema viene attivato dal contatto con la cellula ospite e presenta un struttura complessa composta d pi\`u di 20 proteine diverse. 
			Questo sistema presenta una certa omologia con le proteine che costituiscono il corpo basale del flagello. 
			La secrezione necessita di energia ed \`e assistita dalle chaperonine \emph{Syc} che hanno il compito di portare la proteina alla base del canale di trasporto. 
			La proteina \emph{YopN} apre e chiude il canale.
			Il canale viene formato nella cellula bersaglio da \emph{YopB} e \emph{YopD}.

			\subsubsection{Il sistema di secrezione di tipo $\mathbf{IV}$}
			Questo \`e un processo molto versatile che secreta DNA (coniugazione) o proteine. 
			Avviene una traslocazione direttamente nelle cellule bersaglio. 
			Sono presenti almeno $12$ proteine diverse che formano una struttura che abbraccia gli involucri del batterio. 
			Questo sistema porta all'assemblaggio di monomeri che formano il pilo $IV$ a spirale. 

