\chapter{Meccanismi di trasporto cellulare}
La membrana media un'ampia varietà di funzioni. Una delle funzioni pi\`u importanti \`e il controllo di ci\`o  che entra ed esce dalla cellula.La membrana durante questo processo è molto selettiva: permette ad alcune membrane di oltrepassarla, mentre lo nega ad altre. Esistono due tipi diversi di diffusione: uno che usufruisce di energia e un altro che non ne fa uso.
\subsection{Processi passivi}
In questo tipo di processi non c'\`e utilizzo di energia. Le molecole passsano da un ambiente dove la loro concentrazione \`e maggiore a una dove la loro concentrazione \`e minore.
\begin{itemize}
        \item Diffusione passiva
        \item Diffusione facilitata
        \item Passaggio specifico
        \item Osmosi: \`e la diffusione di acqua attraverso una membrana selettivamente permeabile, dal compartimento con la minor concentrazione di soluti verso quello con la maggior concentrazione di soluti. \`E un feomeno fondamentale di regolazione nella cellule: essa tende a rendere la conentrazione di soluti presenti all'interno della cellula pari a quella presenta nell'ambiente extracellulare. In base alle soluzioni in cui la cellula \`e immersa ne esistono tre tip: isotonica (nessun movimento netto di acqua); ipertonica (usicta di acqua causando la diminuzione del volume cellulare); ipotonica (ingresso di acqua portando alla lisi delle cellule prive di parete).
\end {itemize}
\subsection{Trasporto attivo}
Il trasporto attivo richiede energia per movimentare sostanze contro il proprio gradiente di concentrazione. L'energia viene fornita dall'idrolisi dell'ATP o dalla forza proton-motrice.
I tre principali tipi di trasporto attivo sono: 
\begin{itemize}
        \item Uniporto: permette il passaggio di una molecola specifica solamente in una direzione, anche se questo significa andare contro gradiente. 
        \item Antiporto: 
        \item Simporto:
\end{itemize}
