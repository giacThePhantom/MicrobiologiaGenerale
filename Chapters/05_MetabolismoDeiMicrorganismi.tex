\chapter{Metabolismo dei microrganismi}

\section{Introduzione}
Il metabolismo \`e un insieme di reazioni biochimiche controllate. 

	\subsection{Elementi fondamentali}
	Alcuni degli elementi fondamentali sono:
	\begin{multicols}{2}
		\begin{itemize}
		    \item nutrienti: elementi chimici essenziali; 
		    \item energia: ricavata dalla luce o dalla degradazione dei nutrienti;
		    \item enzimi: catabolizzano e anabolizzano i nutrienti;
		    \item macromolecole: assemblaggio e polimerizzazione partendo da monomeri; 
		    \item struttura cellulare: assemblaggio di pi\`u macromolecole. 
		\end{itemize}
	\end{multicols}

	\subsection{Tipi di metabolismo}
	Il metabolismo viene diviso in due principali tipi di reazioni: cataboliche e anaboliche. 

		\subsubsection{Catabolismo}
		Il catabolismo \`e la parte del metabolismo che libera energia, grazie alla scomposizione di molecole organiche. 
		Una parte dell'energia viene conservata sotto forma di legami nell'ATP, mentre altra viene dispersa come calore. 
		\`E una via esoergonica.

		\subsubsection{Anabolismo}
		L'anabolismo utilizza l'energia che viene liberata dal catabolismo per creare molecole pi\`u grandi. 
		Anche in questo tipo di processi viene persa dell'energia sotto forma di calore. 
		\`E una via endoergonica.
		Una via metabolica non \`e costituita da una sola reazione, ma da una serie complessa di reazioni. 

		\subsubsection{Energia libera}
		Durante una reazione chimica l'energia libera $\Delta G^{0'}$ \`e definita come energia rilasciata disponibile per compiere un lavoro utile. 
		Se $\Delta G^{0'}$ \`e negativo significa che essa proceder\`a con liberazione di energia libera.
		Se $\Delta G^{0'}$ \`e positivo la reazione per aver luogo richiede energia.
		L'energia libera non fornisce informazioni riguardo la velocit\`a di reazione: determina unicamente se una reazione libera o richiede energia.
		Reazioni energeticamente favorevoli come la formazione di acqua ($\Delta G^{0'}=-237\si{kJ}$) non avvengono mescolando i reagenti: \`e necessario prima fornire energia per rompere i legami dei reagenti.
		Questa energia viene detta energia di attivazione e pu\`o essere abbassata da catalizzatori, che aumenteranno quindi la velocit\`a di reazione.

\section{Catalisi ed enzimi}

	\subsection{Enzimi}
	Gli enzimi sono i catalizzatori biologici. 
	
		\subsubsection{Composizione}
		Gli enzimi sono proteine, o raramente RNA, altamente specifici per la reazione da essi catalizzata.
		Sono tipicamente pi\`u grandi del substrato: il dominio che lo lega viene detto sito attivo.
		Possono contenere piccole molecole non proteiche che partecipano alla funzione catalitica.
		Si nota pertanto come molti enzimi sono composti da pi\`u elementi organici e inorganici.

			\paragraph{Apoenzima}
			Si dice apoenzima la porzione proteica degli enzimi che viene attivata dal legame dei suoi cofattori.

			\paragraph{Cofattori}
			I cofattori possono essere molecole inorganiche come ioni metallici (ferro, zinco o rame) o molecole organiche (coenzimi).
			I coenzimi tipicamente sono vitamine o le contengono.

			\paragraph{Oloenzima}
			Si dice oloenzima la forma completa e attiva dell'enzima.
		
		\subsubsection{Reazioni catalizzate da enzimi}
		Ciascun enzima catalizza un solo tipo di reazione o una classe di reazioni strettamente affini.
		La specificit\`a dipende dalla struttura tridimensionale particolare che assume l'enzima.
		\[E + S \leftrightarrow S-E \leftrightarrow E + P \]
		Dove:
		\begin{multicols}{2}
			\begin{itemize}
				\item $E$: enzima.
				\item $S$: substrato.
				\item $P$: prodotto.
			\end{itemize}
		\end{multicols}
		Si nota come l'enzima non viene consumato durante la reazione.

		\subsubsection{Tipologie di enzimi}
		Esistono diversi tipi di enzimi e la maggior parte dei loro nomi contiene il suffisso "-asi" e spesso fa riferimento al tipo di substrato e di reazione biochimica mediata:
		\begin{multicols}{2}
			\begin{itemize}
	    			\item Idrolasi: catalizzano la rottura di un legame chimico con l'intervento di una molecola d'acqua (catabolismo);
	    			\item Isomerasi: catalizzano l'interconversione tra due isomeri (n\`e catabolismo, n\`e anabolismo);
	    			\item Ligasi e polimerasi: assemblano molecole della stessa natura chimica (anabolismo);
	    			\item Liasi:  catalizzano la rottura di diversi legami chimici attraverso processi differenti dall'idrolisi e dalla ossidazione (catabolismo);
	    			\item Ossidoriduttasi: catalizzano il trasferimento di elettroni da una molecola (donatrice di elettroni) ad un'altra (accettore di elettroni) (catabolismo o anabolismo);
	    			\item Trasferasi: spostano gruppi funzionali (amino, fosfato, acetile, etc.) da una molecola all'altra (anabolismo).
			\end{itemize}
		\end{multicols}

	\subsection{Parametri esterni che alterano l'azione degli enzimi}
	Elementi esterni possono avere effetto sulla capacit\`a degli enzimi di svolgere reazioni.

		\subsubsection{Temperatura}
		L'alzarsi della temperatura tende ad incrementare la velocit\`a delle reazioni biochimiche. 
		Tuttavia, le reazioni enzimatiche hanno un range di temperatura in cui possono svolgersi. 
		Sopra una determinata temperatura i legami non-covalenti dell'enzima si rompono ed esso si denatura, portando alla perdita della struttura tridimensionale e della funzionalit\`a. 
		La denaturazione pu\`o essere permanente o reversibile a seconda degli enzimi.

		\subsubsection{$\mathbf{pH}$}
		Valori estremi di $pH$ portano alla denaturazione.
		Gli ioni rilasciati da acidi o basi interferiscono con i legami idrogeno che assicurano la struttura tridimensionale dell'enzima.
	
		\subsubsection{Concentrazioni relative di substrato ed enzima}
		L'attivit\`a enzimatica aumenta in funzione della concentrazione di substrato.
		L'attivit\`a si stabilizza quando si raggiunge il punto di saturazione: tutti i siti attivi sono legati al substrato.

		\subsubsection{Inibitori}
		Gli inbitori regolano l'attvit\`a enzimatica. 
		
			\paragraph{Tipologie di inibitori}

				\subparagraph{Competitivi}
				Gli inibitori competitivi sono in grado di legare il sito attivo.
				Presentano una forma e struttura chimica simile a quella del substrato. 
				Competono con il substrato per il sito attivo dell'enzima. 
				In linea generale, la loro inibizione \`e reversibile e pu\`o essere superata con l'aumento della concentrazione del substrato. 
				La sulfanilamide presenta una forte affinit\`a per il sito attivo dell'enzima che catalizza la conversione del \emph{PABA} in acido folico, un precursore dei nucleotidi fondamentale per la sintesi del DNA.

				\subparagraph{Non-competitivi}
				Gli inibitori non-competitivi non legano il sito attivo ma un'altra regione detta sito allosterico.
				Portano ad un cambiamento conformazionale del sito attivo. 
				Esistono due forme di inibitori non competitivi: inibitori ed eccitatori.
				Alcuni enzimi possiedono multipli siti allosterici per una fine regolazione.

				\subparagraph{Feedback-negativo}
				Gli inibitori a feedback negativo regolano la quantit\`a di una certa sostanza in base alla sua concentrazione.
				Il prodotto finale della via metabolica \`e un inibitore allesterico di un enzima che interviene pi\`u a monte nel pathway. 
				Dato che il prodotto di ogni reazione \`e anche il substrato della successiva, l'intero pathway viene disattivato quando il prodotto finale \`e presente in concentrazione sufficiente. 
				Nella via ramiificata per la produzione di tirosina, fenilanina e triptofano ciascuno dei tre prodotti finali inibisce uno dei tre enzimi sintetasi. 
				Solo quando tutti i tre prodotti finali sono presenti in concentrazione adeguata la sintesi del loro precursore, \emph{DHAP}, viene interrotta. 
 
\section{Redox}
Molte reazioni metaboliche prevedono il trasferimento di un elettrone da una molecola \emph{$e^{-}$ donor} ad un'altra \emph{$e^{-}$ acceptor}.

	\subsection{Potenziale di riduzione standard}
	Con potenziale di riduzione standard $E_0$ si intende la costante di equilibrio per una reazione di ossido-riduzione.
	La capacit\`a di una molecola di scambiare elettroni \`e espressa in Volt e prende come riferimento una sostanza standard \emph{$H_2$}.
	Pi\`u negativo il valore di $E_0$ pi\`u la molecola tende a donare elettroni.
	Pi\`u positivo il valore di $E_0$ pi\`u la molecola tende ad accettare elettroni.

	\subsection{Torre degli elettroni}
	La torre degli elettroni rappresenta il campo dei potenziali di riduzione possibili per le coppie redox in natura.
	Le reazioni con $E^{I}_0$ pi\`u negativo si trovano in cima alla torre, mentre quelle con $E^{I}_O$ pi\`u positivo alla sua base.
	La sostanza ridotta nella coppia posta in cima alla torre ha la massima tendenza a donare elettroni.
	La sostanza ossidata nella coppia posta in fondo alla torre ha la massima tendenza ad accettare elettroni.

\section{Trasportatori di elettroni}
Le reazioni redox sono mediate da piccole molecole. 

	\subsection{\emph{NADH} e \emph{NADPH}}
	Un intermedio redox molto comune \`e il \emph{$NAD^+$} (nicotinammide adenina dinucleotide), ha la funzione di coenzima e trasportatore di elettroni.
	La sua forma ridotta \`e \emph{NADH}. 
	Sono coinvolti nelle reazioni cataboliche che generano energia.
	\emph{$NADP^+$} e \emph{$NADPH$} sono assolutamente analoghi a \emph{$NAD^+$} e a \emph{$NADH$}, l'unica differenza \`e il gruppo fosfato.
	Entrambe le forme ridotte sono buoni donatori di elettroni con $E'_0=-0.32$.
	Sono coinvolti nelle reazioni anaboliche di biosintesi.
	
	\subsection{Ciclo \emph{$NAD^+/NADH$}}
	\emph{$NAD^+$} e \emph{$NADH$} sono coenzimi e attivano enzimi senza essere consumati.
	Gli enzimi si legano a \emph{$NAD^+$} e quando incontrano un substrato adatto lo legano in prossimit\`a del coenzima non ridotto.
	\emph{$NAD^+$} a questo punto si riduce: riceve due elettroni e un protone \emph{$H^+$} e passa in forma \emph{$NADH$}.
	Si noti come in questo passaggio perde l'anello aromatico.
	In questo modo cambia la conformazione dell'enzima che svolge la sua funzione.
	Se invece all'enzima si lega \emph{$NADH$} avviene il contrario: si ossida e \emph{$H^+$} e passa in forma \emph{$NAD^+$}.

\section{\emph{ATP}}

	\subsection{Struttura}
	L'adenosina trifosfato \emph{ATP} \`e il pi\`u importante composto fosforilato. 
	\`E costituito dal ribonucleoside adenosina a cui sono legati in serie tre molecole di fosfato. 
	I suoi legami fosfoanidride sono molto energetici.
	Viene generato durante le reazioni esoergoniche e consumato nelle reazioni endoergoniche. 

	\subsection{Funzione}
	Nel catabolismo l'energia rilasciata dalla degradazione dei nutrienti viene concentrata e stoccata nei legami ad alta energia tra i gruppi fosfato della molecola di \emph{ATP}.

	\subsection{Sintesi}
	Il \emph{ATP} si forma dalla fosforilazione dell'ADP.

		\subsubsection{Metodi di fosforilazione}
	
			\paragraph{Fosforilazione a livello di substrato}
			La fosforilazione a livello di substrato prevede il trasferimento del fosfato da una molecola organica a \emph{ADP} per formare \emph{ATP}.

			\paragraph{Fosforilazione ossidativa}
			Nella fosforilazione ossidativa l'energia derivata da reazioni redox dalla respirazione cellulare viene utilizzata per aggiungere un fosfato inoragnico $P_i$ a \emph{ADP}.

			\paragraph{Fotofosforilazione}
			Nella fotofosforilazione l'energia luminosa viene utilizzata per fosforilare \emph{ADP} con fosfato inorganico.

\section{Catabolismo dei carboidrati}
Il glucosio ed altri zuccheri vengono catabolizzati dai microrganismi tramite due processi: 
\begin{multicols}{2}
	\begin{itemize}
    		\item Respirazione cellulare: completa demolizione del glucosio per formare anidride carbonica ed acqua;
    		\item Fermentazione: produce molecole organiche di scarto.
	\end{itemize}
\end{multicols}
Entrambi i processi iniziano con la glicolisi nel quale ogni molecola viene catabolizzata in due molecole di piruvato con la produzione netta di $2$ molecole di \emph{ATP}.
La respirazione cellulare, poi, prosegue con il ciclo di Krebs e la catena di trasporto elettronico con sostanziale produzione di \emph{ATP}.
La fermentazione invece converte l'acido piruvico in altre molecole organiche senza produzione di \emph{ATP}. 
Nello stesso organismo \`e possibile utilizzare sia la respirazione cellulare che la fermentazione.

		\subsection{Glicolisi}
		La glicolisi, anche detta Via di Embden-Meyerhof-Parnas, \`e il primo passo per la metabolizzazione del glucosio. 
		Questo processo scinde il glucosio (6 Carboni) in piruvato (3 Carboni).

			\subsubsection{Descrizione}
			La glicolisi di divide in tre parti fondamentali, per un totale di $10$ reazioni enzimatiche.
			\begin{enumerate}
			    \item Investimento energetico (1-3);
			    \item Rottura della molecola (4-5);
			    \item Conservazione dell'energia (6-10).
			\end{enumerate}

			\subsubsection{Prodotti}
			Nella glicolisi vengono formati $4$ \emph{ATP} e consumati $2$ \emph{ATP}, per un bilancio netto di $2$ \emph{ATP}. 
			Due molecole di \emph{$NAD^{+}$}, invece, vengono ridotte a \emph{NADH}.

			\subsubsection{Processo}
			I passaggi della glicolisi della glicolisi sono:
			\begin{multicols}{2}
				\begin{itemize}
					\item[1.] Fosforilazione del glucosio (\emph{$ATP\rightarrow ADP$}) per formare \emph{glucosio-6-fosfato}.
						L'enzima utilizzato \`e l'esochinasi.
					\item[2.] Fosforilazione (\emph{$ATP\rightarrow ADP$}). 
						L'enzima utilizzato \`e l'isomerasi.
					\item[3.] Isomerizzazione per formare fruttosio \emph{1,6-bifosfato}.
						L'enzima utilizzato \`e il fosfofruttachinasi.
					\item[4.] Il fruttosio \emph{1,6-bifosfato} viene tagliato per formare \emph{gliceraldeide 3-fosfato} \emph{G3P} e \emph{diidrossiacetone fosfato} \emph{DHAP}. 
						Sono delle molecole con $3$ Carboni. 
						L'enzima utilizzato \`e l'aldolasi.
					\item[5.] Il \emph{DHAP} viene isomerizzato a \emph{G3P}, grazie all'enzima triosofosfato isomerasi.
					\item[6.] Dopo l'aggiunta di 2 fosfati c'è la formazione di 2 NADH da 2 NAD\ap{+}. 
						Si forma il \emph{1,3-difosfoglicerico}.
						L'enzima impiegato \`e il \emph{G3P deidrogenasi}.
					\item[7.] Formazione di $2$ \emph{ATP} da $2$ \emph{ADP}.
					\item[8-9.] Rilascio di $2$ molecole di \emph{$H_2O$} e isomerizzazione con conseguente produzione di \emph{fosfoenolpiruvato} \emph{PEP}.
					\item[10.] Viene tolto l'ultimo fosfato dalle $2$ \emph{PEP} per formare $2$ \emph{ATP} e formazione di $2$ piruvati. 
						In questo step avviene un passaggio diretto di un fosfato dal \emph{PEP} a una molecola di \emph{ADP}. 
						Questo \`e mediato da un enzima, con attaccato \emph{$Mg^{2+}$}.
						Il complesso \`e quindi un oloenzima.
				\end{itemize}
			\end{multicols}

		\subsection{Respirazione cellulare}
		Durante il processo della respirazione cellulare \`e prevista la degradazione completa della molecola, in seguito ad una serie di reazioni redox. 
		
			\subsubsection{Fasi della respirazione cellulare}	

				\paragraph{Sintesi acetilCoA}
				L'enzima decarbossilasi rimuove un atomo di carbonio dall'acido piruvato sotto forma di $CO_2$, poi media l'attacco con l'acetato al \emph{coenzima-A} con un legame ad alta energia. 
				In questo ultimo processo una molecola di \emph{$NAD^{+}$} \`e ridotta a \emph{NADH}.

				\paragraph{Ciclo di Krebs}
				Il ciclo di Krebs o ciclo dell'acido citrico \`e composto da $8$ reazioni enzimatiche che trasferiscono l'energia contenuta nei legami dell'\emph{acetyl-CoA} ai coenzimi \emph{NAD} e \emph{FAD}, riducendoli.

					\subparagraph{Fasi del ciclo di Krebs}
					Si pu\`o dividere nei seguenti passaggi:
					\begin{multicols}{2}
						\begin{itemize}
							\item[1.] \emph{Acetyl-CoA} entra nel ciclo unendosi all'\emph{acido ossoaloacetico} per formare \emph{acido citrico}.
							\item[2-4.] due ossidazioni e decarbossilazioni e l'aggiunta di \emph{CoA} formano il \emph{succinyl-CoA}.
							\item[5.] Fosforilazione a livello di substrato, rilascio di \emph{CoA} e produzione di \emph{ATP}.
							\item[6-8.] Ulteriori ossidazioni rigenerano l'\emph{acido ossaloacetico}.
								All'ultimo passaggio viene ridotto \emph{$NAD^+$} in \emph{NADH}.
						\end{itemize}
					\end{multicols}
					
					\subparagraph{Generazione di energia}
					Durante il passaggio $5$, una piccola parte dell'energia del processo viene stoccata in $1$ molecola di \emph{ATP} grazie alla fosforilazione.
					Il \emph{GTP} viene utilizzato come prodotto intermedio.
					La maggior parte dell'energia viene conservata sotto forma di elettroni in reazioni redox, nei trasportatori di elettroni \emph{NADH} (step $3$, $4$, $8$) e \emph{FADH} (step $6$). 
					L'energia di questi elettroni servir\`a a valle per produrre \emph{ATP}.
					Vengono inoltre prodotte anche $3$ molecole di $CO_2$. 
		
					\subparagraph{Ruolo biosintetico}
					Il ciclo di Krebs, come la glicolisi, ha anche un ruolo biosintetico: molti dei suoi intermedi possono essere usati come precursori per la costruzione di altre molecole come amminoacidi o \emph{anelli porfirinici} dei \emph{citocromi}. 

				\paragraph{Catena di trasporto degli elettroni}
				La catena di trasporto degli elettroni \`e la fase in cui viene prodotta la maggior parte di \emph{ATP}.
				Il processo avviene indirettamente: il movimento degli elettroni genera un gradiente che viene utilizzato per fosforilare \emph{ADP}.
				
					\subparagraph{Composizione}	
					La catena di trasporto elettronico consiste in una serie di molecole associate alla membrana che a turno ricevono e cedono elettroni.
					Alla fine si trova un accettore finale di elettroni.
					Molecole carrier:
					\begin{multicols}{2}
						\begin{itemize}
							\item \emph{Flavoproteine}: proteine integrali di membrana che contengono il coenzima \emph{flavina}, molecola derivata dalla vitamina $B_2$. 
								La \emph{flavina mononucleotide} (\emph{FMN}) \`e l'accettrice iniziale di elettroni. 
								Accettano \emph{$2H^{+}$} e $2$ elettroni, ma donano soltanto elettroni. 
								Come tutti gli altri componenti della catena alternano tra stato ridotto e ossidato;
							\item Proteine ferro-zolfo: sono un gruppo di proteine di membrana che contiene ioni metallici (\emph{Fe} e \emph{S}) i quali si ossidano e riducono durante il passaggio di elettroni. 
								Come i citocromi trasportano solamente elettroni.
								Il ferro \`e legato a cisteine. 
								Possono alternare tra stato ridotto e ossidato.
							\item \emph{Ubichinone}: carrier non proteici derivati dalla vitamina $K$.
								Sono altamente idrofobici e vengono chiamati  \emph{Coenzima Q}.
								Accettano \emph{$2H^{+}$} e $2$ elettroni.
							\item \emph{Citocromi}: proteine integrali di membrana che legano un gruppo \emph{eme}, formato da un \emph{anello porfirinico} e un atomo di ferro. 
								Il ferro alterna tra stato ridotto e ossidato: $Fe^{2+}\leftrightarrow Fe^{3+}$. 
								Il citocromo $c$ serve da intermedio fra il citocromo \emph{bc} e il \emph{aa}. 
								Subiscono delle ossidazioni e riduzioni mediante la perdita o l'acquisto di un singolo elettrone da parte del ferro. 
								Sono diversificati e possono formare complessi fra loro, come il citocromo \emph{$bc_1$}.
						\end{itemize}
					\end{multicols}

					\subparagraph{Utilizzo dell'energia}
					L'energia prodotta viene utilizzata per pompare elettroni attraverso la membrana, creando la forza proton-motrice. 
					Mentre gli elettroni si spostano lungo la catena il livello dell'energia si abbassa e vengono pompati ioni \emph{$H^{+}$} all'esterno.
					Gli ioni utilizzati derivano dai trasportatori \emph{NADH} e \emph{FADH} o dall'idrolisi dell'acqua nei citocromi.
					I protoni poi, passando attraverso \emph{ATP sintasi} permettono la generazione di \emph{ATP}.
					Ogni $2$ \emph{$H^{+}$} viene, approssimativamente, generato un \emph{ATP}.

					\subparagraph{Accettore finale}
					Alla fine gli elettroni vengono accettati dall'ossigeno. 
					La reazione tra elettroni, ossigeno e \emph{$H^{+}$} forma l'acqua. 
					Questa funzione determina l'importanza dell'ossigeno per un organismo.

					\subparagraph{\emph{ATPsintasi}}
					L'ATPsintasi o ATPasi \`e un grande complesso enzimatico di membrana che serve da catalizzatore per la conversione della forza proton-motrice in \emph{ATP}.
					\`E composto da due domini principali:
					\begin{multicols}{2}
						\begin{itemize}
							\item Una testa con subunit\`a multiple $F_1$ collocata nella faccia citoplasmatica della membrana.
							\item Un canale conduttore di protoni $F_0$ che attraversa la membrana. 
						\end{itemize}
					\end{multicols}
					Il complesso $F_1/F_0$ catalizza la reazione:
					\[ADP+P_i \rightarrow ATP\]
					Il dominio $F_0$ \`e composto da due subunit\`a principali:
					\begin{multicols}{2}
						\begin{itemize}
		    					\item \emph{a}: il canale attraverso il quale passano gli ioni \emph{$H^{+}$}, provoca il movimento delle subunit\`a $c$.
		    					\item $c$: il rotore composto da $12-15$ subunit\`a singole. 
								La sua torsione che provoca dei movimenti e dei cambiamenti conformazionali che permettono di generare \emph{ATP}.
						\end{itemize}
					\end{multicols}
					Il dominio $F_1$ \`e composto da sei subunit\`a principali:
					\begin{multicols}{2}
						\begin{itemize}
							\item $\varepsilon$ e $\gamma$: connettono il rotore alla parte pi\`u massiccia di $F_1$. 
								La torsione della subunit\`a $c$ genera la rotazione accoppiata dalle due subunit\`a;
							\item $\alpha$: con funzione strutturale.
								Sono $3$.
							\item $\beta$: con ruolo di sintesi, si trovano alternate alla subunit\`a $\alpha$.
								Sono $3$.
								Si alterano in tre conformazioni:
								\begin{itemize}
									\item Vuota.
									\item Con \emph{ADP+P}.
									\item Con \emph{ATP}.
								\end{itemize}
								In ogni momento ogni subunit\`a ha una conformazione diversa.
								\emph{ATP} viene generato quando la subunit\`a lo rilascia e torna vuota.
							\item Subunit\`a $b_2$ e $\delta$: connettono le subunit\`a $\alpha$ e $\beta$ impedendo la rotazione di $F_1$ impartendo stabilit\`a.
						\end{itemize}
					\end{multicols}
					L'\emph{ATPasi} pu\`o funzionare anche al contrario: idrolizza \emph{ATP} per pompare all'esterno \emph{$H^+$}.

	\subsection{Bilancio globale}
	\begin{center}
		\begin{tabular}{|c|c|c|}
			\hline
			Processo & Elementi consumati & Elementi prodotti \\
			\hline
			Glicolisi & Glucosio, \emph{$NAD^+$}, \emph{2ATP} & $2$ piruvato, $4$ \emph{ATP}, $2$ \emph{NADH} \\
			\hline
			Ciclo di Krebs & Piruvato, $4$ \emph{$NAD^+$}, \emph{FAD} & $4$ \emph{NADH}, $1$ \emph{FADH}, $1$ \emph{GTP}, $3$ \emph{$CO_2$} \\
			\hline
		\end{tabular}
	\end{center}
	Durante la glicolisi inoltre essendo che ogni \emph{NADH} genera \emph{3 ATP}, vengono prodotti $2$ \emph{ATP} dalla fosforilazione a livello del substrato e $6$ dalla respirazione cellulare del \emph{NADH}.
	Durante il ciclo di Krebs inoltre essendo che ogni \emph{FADH} genera \emph{2 ATP}, vengono prodotti \emph{1 ATP} dalla fosforilazione a livello di substrato e \emph{14 ATP} dalla respirazione
	dei $4$ \emph{NADH} e del \emph{FADH}.
	Essendo che per ogni circolo entrano $2$ molecole di piruvato si ottengono in totale $30$ \emph{ATP}.
	In totale:
	\begin{center}
		\begin{tabular}{|c|c|}
			\hline
			$1$ molecola di glucosio & $38$ \emph{ATP}\\
			\hline
		\end{tabular}
	\end{center}

	\subsection{Le alternative cataboliche}
	I microrganismi anaerobi non utilizzano l'ossigeno come accettore finale di elettroni.
		
		\subsubsection{Alternative all'ossigeno}
		\begin{multicols}{3}
			\begin{itemize}
				\item \emph{$SO_4^{2-}\rightarrow H_2S$}.
				\item \emph{$CO_3^- \rightarrow CH_4$}.
				\item \emph{$NO_3^-\rightarrow \{N_2, N_2\}$}.
			\end{itemize}
		\end{multicols}
		Si nota come questi accettori possiedono un $E_0$ meno positivo e pertanto il processo \`e meno efficiente.

		\subsubsection{Anaerobi e aerobi facoltativi}
		Anaerobi e aerobi facoltativi sono in grado di utilizzare sia ossigeno che altre sostanze come accettore finale di elettroni.

		\subsubsection{Organismi che non utilizzano ossigeno come accettore finale di elettroni}
		La diversit\`a metabolica tra respirazione e fotosintesi si trova nei metodi di generazione della fotosintesi.
		
			\paragraph{Chemiolitotrofia}
			La chemiolitotrofia prevede l'utilizzo di sostanze inorganiche come donatori di elettroni (\emph{FAD} o \emph{NAD}).
			I chemioorganotrofi usano come unica fonte il carbonio per produrre energia e composti organici.

			\paragraph{Fotoautotrofia}
			La fotoautotrofia utilizza come fonte di energia la luce, mentre \emph{ATP} viene generato tramite il processo di fosforilazione.
			I fotoautotrofi assimilano \emph{$CO_2$} come fonte di carbonio, mentre i fotoeterotrofi composti inorganici.
			Esiste una fotosintesi ossigenica che nei cianobatteri produce \emph{$CO_2$} e anossigenica.

		\subsubsection{La fermentazione}
		\`E una via metabolica alternativa in caso di mancanza di un accettore finale di elettroni nel processo di respirazione cellulare. 
		Se manca un accettore finale di elettroni tutta la via respiratoria si blocca. 
		La fosforilazione avviene solamente a livello del substrato e dipende unicamente dalla forza proton-motrice.
		Essendo che glicolisi e ciclo di Krebs richiedono \emph{$NAD^+$} non possono produrre \emph{ATP}.
		Permette la produzione di \emph{ATP} durante il ciclo di Krebs.
		Nonostante sia meno efficiente della respirazione cellulare, permette di produrre \emph{ATP} senza un accettore di elettroni.
		Consiste nella parziale ossidazione di zuccheri o metaboliti per la produzione di energia utilizzando una molecola organica come accettore di elettroni.
		Ossidano \emph{NADH} riducendo molecole organiche endogene.
			
			\paragraph{Fermentazione lattica}
			Nel processo della fermentazione lattica due atomi di idrogeno vengono trasferiti sul carbonio in posizione $2$ dell'acido piruvico, producendo l'acido lattico.
			\[CH_3COCOOH^- + NADH + H^+ \rightarrow CH_3HCOHCOOH + NAD^+\]
			\`E un processo che viene attuato da alcuni batteri, come i lattobacilli, e dalle cellule del corpo umano in condizioni di anaerobiosi, come i muscoli. 

			\paragraph{Fermentazione alcolica}
			Nella prima reazione l'acido piruvico forma acetaldeide e anidride carbonica:
			\[C_3H_4O_3 \rightarrow C_2H_4O + CO_2\]
			Nella seconda reazione l'acetaldeide con \emph{NADH} e \emph{$H^+$} produce l'alcol etilico e \emph{$NAD^+$}.
			\[C_2H_4O + (NADH + H^+) \rightarrow C_2H_6O + NAD^+\]
			
			\paragraph{Prodotti della glicolisi}
			Dalla fermentazione del glucosio si possono avere vari prodotti. 
			La glicolisi produce piruvato, che pu\`o essere convertito ad acido lattico attraverso la fermentazione lattica o etanolo attraverso la fermentazione alcolica. 
			La fermentazione acido-mista produce una miscela di etanolo, acido lattico, succinico, formico e acetico.

			\paragraph{Prodotti alimentari o industriali derivati da processi di fermentazione}
				
				\subparagraph{Pane}
				Durante la panificazione il lievito fermenta gli oligosaccaridi che si staccano dall'amido durante la fase di impasto e di riposo della massa in lavorazione. 
				I prodotti della fermentazione alcolica (alcol etilico ed anidride carbonica) passano in fase gassosa formando le caratteristiche bolle durante la lievitazione e la cottura.	
				\subparagraph{Vino}
				Il vino viene prodotto a partire da soluzioni zuccherine ottenute dallo schiacciamento del grappolo d'uva asciate a fermentare con i lieviti del genere Saccharomyces presenti sulla buccia dell'acino o provenienti da colture selezionate. 
				A seconda delle condizioni di fermentazione, si differenziano le qualit\`a organolettiche (colore, sapori, aromi) del vino.

				\subparagraph{Birra}
				La birra si ottiene per l'azione di lieviti su un mosto contenente malto di orzo e quantit\`a variabili di altri cereali. 
				La lavorazione \`e tale da conservare nel prodotto anche l'anidride carbonica.

				\subparagraph{Yogurt}
				Lo yogurt \`e il risultato della fermentazione lattica operata da ceppi selezionati di lattobacilli sul latte. 
				L'abbassamento del $pH$ dovuto all'accumulo dell'acido lattico determina la denaturazione della caseina che coagula conferendo al prodotto la caratteristica consistenza.

\section{Altre vie cataboliche}
Lipidi e proteine contengono una grande quantit\`a di energia nei loro legami. 
Perch\`e questa energia venga utilizzata dalla cellula, essi devono essere scomposti nei loro monomeri e entrare come substrati nella glicolisi e nel ciclo di Krebs.
	
	\subsection{Lipidi}
	I lipidi pi\`u utilizzati per la produzione di \emph{ATP} sono i grassi composti da glicerolo e code di acidi grassi. 
	Lipasi idrolizzano i lipidi producendo glicerolo e tre catene di acidi grassi. 
	Il glicerolo viene convertito in \emph{DHAP} che integra la via metabolica della glicolisi.
	
		\subsubsection{$\mathbf{\beta}$-ossidazione}
		Gli acidi grassi sono degradati in un processo chiamato $\beta$-ossidazione. 
		Durante questo processo degli enzimi tagliano ripetutamente $2$ carboni idrogenati nelle code di acidi grassi, e li uniscono ad una molecola di coenzima. 
		Vengono prodotte cos\`i delle molecole di acetyl-CoA, che possono entrare all'interno del ciclo di Krebs. 
		Il proccesso prosegue finch\`e tutti gli acidi grassi non vengono convertiti in acetyl-CoA. 
		Vengono generate delle grandi quantit\`a di trasportatori di elettroni \emph{NADH} e \emph{$FADH_2$}, usate per la catena di trasporto degli elettroni.

		
	\subsection{Proteine}
	Alcuni microbi catabolizzano \textbf{proteine} come un importante fonte di energia. 
	La maggior parte delle cellule le catabolizza solo quando non ci sono a disposizioni fonti di carbonio come il glucosio. 
	Dato che le proteine sono troppo grosse per superare la membrana plasmatica, procarioti iniziano a catabolizzarli all'esterno. 
	Gli enzimi protesi degradano le proteine in amminoacidi idrolizzando i legami peptidici. 
	Vengono quindi portati all'interno della cellula e subiscono modificazioni chimiche (deamminazione). 
	La molecola risultante pu\`o entrare nel ciclo di Krebs.
\section{La fotosintesi}
Gli organismi fotosintetici catturano l'energia luminosa e l'utilizzano per la sintesi di carboidrati a partire da \emph{$CO_2$} e \emph{$H_2O$}. 
I cianobatteri sono stati i primi organismi fotosintetici. 
Ora anche molte alghe, batteri verdi solfurei e non, piante e alcuni protozoi fanno parte di questo gruppo. 
Essi riescono a catturare l'energia della luce solare grazie a delle piccole molecole, la pi\`u importante delle quali \`e la clorofilla. 

	\subsection{Classificazione degli organismi}

		\subsubsection{Fotoautotrofi}
		I fotoautotrofi assimilano \emph{$CO_2$} come fonte di carbonio.

		\subsubsection{Fotoeterotrofi}
		I fotoetertrofi utilizzano composti organici fome fonte di carbonio.

	\subsection{Tipi di fotosintesi}

		\subsubsection{Fotosintesi ossigenica}
		Nella fotosintesi ossigenica si produce ossigeno come materiale di scarto.

		\subsubsection{Fotosintesi anossigenica}
		Nella fotosintesi anossigenica non viene utilizzata acqua e non viene prodotto scarto.

	\subsection{Clorofilla}
	La clorofilla \`e un pigmento, una molecola organica formata da una coda idrocarburica idrofobica attaccata a un centro che assorbe luce composto anche da uno ione \emph{$Mg^{2+}$}. 
	La clorofilla assomiglia ai citocromi, ma al posto del magnesio contengono il ferro al centro dell'anello. 
	Si nota una serie di legami con una delocalizzazione di elettroni.
	La coda si trova inserita nella membrana cellulare, mentre il sito attivo al di sopra.
	
		\subsubsection{Tipi di clorofilla}
		\begin{multicols}{2}
			\begin{itemize}
 	   			\item Clorofille che si trovano nelle piante, nelle alghe e nei cianobatteri; 
    				\item Batterioclorofille che si trovano nei batteri verdi e porpora, e negli eliobatteri.
			\end{itemize}
		\end{multicols}
		Le clorofille si differenziano per la lunghezza d'onda a cui assorbono.
		Questa determina anche diversi habitat di insediamento per gli organismo.

		\subsubsection{Sito di fotosintesi}
		La fotosintesi avviene nella membrana citoplasmatica ricca di clorofille raggruppate in tilacoidi.
		Quando la luce colpisce i fotosistemi le molecole di clorofilla attivano elettroni che si muovono fino ad arrivare al reaction center.

		\subsubsection{Ritorno allo stato iniziale}
		La clorofilla, dopo che \`e stata eccitata torna allo stato iniziale per:
		\begin{multicols}{2}
			\begin{itemize}
				\item Decadimento per cessione di luce e calore.
				\item Decadimento per risonanza di trasferimento di energia: passaggio dell'energia alla molecola adiacente.
				\item Un elettrone passa da una molecola all'altra attraverso riduzione dell'accettore ed ossidazione del donatore.
			\end{itemize}
		\end{multicols}

	\subsection{Fotosistemi}
	I fotosistemi sono formati dalle proteine di membrana e dalle clorofille.
	I tilacoidi dei fotosistemi dei procarioti sono invaginazioni della membrana citoplasmatica.
	Le invaginazioni permettono una massimizzazione della superficie della membrana dove pu\`o avvenire la fotosintesi.

		\subsubsection{Tipologie di fotosistemi}

			\paragraph{Fotosistemi \emph{$PSI$}}
			I fotosistemi \emph{$PSI$} svolgono le reazioni dipendenti dalla luce.

			\paragraph{Fotosistemi \emph{$PSII$}}
			I fotosistemi \emph{$PSII$} svolgono le reazioni indipendenti dalla luce.

		\subsubsection{Funzione dei fotosistemi}
		I fotosisemi assorbono la luce solare e conservano l'energia in molecole di \emph{ATP} e \emph{NADH} grazie a reazioni redox.
		Nelle reazioni indipendenti dalla luce viene sintetizzato il glucosio a partire da \emph{$CO_2$} e \emph{$H_2O$}.

	\subsection{Reazioni dipendenti dalla luce}
	Nei fotosistemi dipendenti dalla luce le centinaia di clorofille in essi si passano l'energia da uno all'altro grazie all'eccitamento degli elettroni provocato dalla luce. 
	I pigmenti del fotosistema assorbono l'energia della luce e la trasferiscono a molecole adiacenti per indirizzarla presso una molecola di clorofilla detta centro di reazione. 

		\subsubsection{Fotofosforilazione ciclica}
		La fosforilzione ciclica avviene in tutti gli organismi fotosintetici. 
		Gli elettroni vengono eccitati nel fotosistema, passano dal centro di reazione a una molecola di \emph{Fe}, e da qui vanno ai citocromi. 
		In questi il livello di energia scende e questo permette il passaggio degli ioni \emph{$H^+$} contro gradiente. 
		Gli elettroni che non sono pi\`u eccitati, tornano al fotosistema $I$ che funge da accettore finale di elettroni e il ciclo ricomincia. 
		Il gradiente di protoni creato viene utilizzato per la sintesi di \emph{ATP}.

		\subsubsection{Fotofosforilazione non ciclica}
		Questo tipo di fosforilazione \`e utilizzata da alcuni  batteri fotosintetici e da tutte le piante, alghe e protisti fotosintetici. 
		Richiede l'utilizzo di due fotosistemi, $PSI$ e $PSII$.
		Produce \emph{ATP} e potere riducente sotto forma di NADPH. 
		Gli elettroni vengono eccitati nel $PSII$, trasmessi in sequenza ad accettori di elettroni ed ulteriormente energizzati nel $PSI$. 
		L'accettore finale di elettroni \`e il \emph{$NADP^+$} che viene ridotto \emph{NADPH} e viene ulteriormente utilizzato nelle reazioni luce-dipendenti. 
		Il $PSII$ deve essere continuamente rifornito di elettroni; nella fotosintesi ossigenica essi provengono dalla dissociazione di \emph{$H_2O$}. 
		Questa reazione produce $2$ elettroni, $2$ protoni ed ossigeno molecolare \emph{$O_2$} come prodotto di scarto. 
		Nella fotosintesi anossigenica i batteri ottengono elettroni da altri donatori inorganici come \emph{$H_2S$}.

	\subsection{Reazioni non dipendenti dalla luce}
	Questi sistemi utilizzano \emph{ATP} ed \emph{NADPH} prodotti dalle reazioni luce-dipendenti. 
	La loro funzione principale \`e la fissazione del carbonio e la formazione di molecole di glucosio. 
	Tutto questo avviene durante il ciclo di Calvin-Benson. 

		\subsubsection{Ciclo di Calvin-Benson}
		Il ciclo di Calvin-Benson si compone di tre fasi principali.

			\paragraph{Fissazione}
			Durante la fissazione $3$ molecole di \emph{$CO_2$} legano $3$ molecole di \emph{RuBP} (ribulosio bifosfato) a $5$ atomi di $C$, che vengono scisse per formare $6$ molecole di \emph{acido fosfoglicerico} a $3$ atomi di $C$.

			\paragraph{Riduzione}
			Durante la riduzione $6$ molecole di \emph{NADH} riducono $6$ molecole di \emph{acido fosfoglicerico} per formare $6$ molecole di \emph{G3P}. 
			Questa fase consuma $6$ molecole di \emph{ATP}.

			\paragraph{Rigenerazione di \emph{RuBP}}
			Durante la rigenerazione di \emph{RuBP} $3$ molecole di \emph{RuBP} vengono prodotte da $5$ molecole di \emph{G3P}. 
			La molecola di \emph{G3P} rimanente \`e utilizzata per sintetizzare glucosio attraverso una serie di reazioni inverse a quelle della glicolisi.
	
			\paragraph{Prodotti}
			Due giri di ciclo producono $2$ molecole di \emph{G3P}. 
			Queste vengono polimerizzate e defosforilate per produrre glucosio. 

	\subsection{Confronto con respirazione aerobica}
	La fotosintesi processa gli elettroni in una direzione opposta rispetto alla respirazione aerobica. 
	Gli elettroni vengono donati dall'acqua per produrre ossigeno, e ceduti (tramite \emph{NADPH}) all'anidride carbonica per sintetizzare glucosio.
	\begin{center}
		\begin{tabular}{|c|c|c|}
			\hline
			Processo & Consumo & Produzione \\
			\hline
			Fotosintesi ossigenica & \emph{$H_2O$}, \emph{$CO_2$} & Glucosio, \emph{$O_2$}\\
			\hline
			Respirazione aerobica & Glucosio, \emph{$O_2$} & \emph{$H_2O$}, \emph{$CO_2$}\\
			\hline
		\end{tabular}
	\end{center}
