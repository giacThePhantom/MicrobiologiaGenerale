\chapter{Genetica batterica}
	
\section{Introduzione}
La genetica batterica tratta: 
\begin{multicols}{2}
	\begin{itemize}
	    \item Mutazioni; 
	    \item Ricombinazione genetica; 
	    \item Scambio di materiale genetico; 
	    \item Clonaggio genico.
	\end{itemize}
\end{multicols}

	\subsection{Dogma centrale della biologia}
	\begin{center}
		\emph{DNA} $\rightarrow$ \emph{RNA} $\rightarrow$ \emph{proteine}
	\end{center}
	Il dogma centrale della biologia enuncia che: il DNA esprime le informazione racchiuse nei geni attraverso la trascrizione in mRNA. 
	Il mRNA \`e composto da codoni, triplette di nucleotidi che codificano le proteine sintetizzate durante la traduzione.
	Nei procarioti traduzione e trascrizione sono meccanismi semplici non compartimentati che avvengono quasi simultaneamente.
	Negli eucarioti il processo \`e complesso: nei geni sono presenti zone non codificanti, gli introni che devono venire eliminate tramite splicing.
	Questa e altre modifiche nucleari producono un mRNA maturo in grado di spostarsi nel citoplasma per essere tradotto.
	Negli eucarioti pertanto i processi sono compartimentati e separati temporalmente.
	Inoltre gli mRNa dei procarioti sono policistronici: portano l'informazione per pi\`u geni.

\section{Mutazioni}
La mutazione è un cambiamento ereditabile nella sequenza delle basi nucleotidiche del genoma. 
Un ceppo che porta un cambiamento viene detto mutante, mentre quello non mutato viene detto selvatico o wild type. 

	\subsection{Fenotipo}
	Le proprietà osservabili del mutante rappresentano il suo fenotipo. 
	Viene indicato da tre lettere con un apice  più o meno, che indica la presenza o meno di una certa proprietà. 
	Per esempio \emph{$HisC^{-}$} indica un ceppo che non è in grado di produrre istidina.

	\subsection{Genotipo}
	Il genotipo viene indicato con tre lettere minuscole seguite da una lettera maiuscola, tutte in corsivo. 
	Per esempio il gene \emph{hisC} di \textit{E. coli} codifica una proteina denominata \emph{HisC}, coinvolta nella sintesi dell'aminoacido istidina. 
	Le mutazioni del gene \emph{hisC} sono indicate come \emph{hisC1}, \emph{hisC2} e via dicendo. 

	\subsection{Le basi molecolari delle mutazioni}
		
		\subsubsection{Codice genetico degenerato}
		Si nota come ci sono pi\`u codoni ($64$) rispetto ad amminoacidi ($20$) da sintetizzare.
		Questo permette alla cellula un sistema di sicurezza contro le mutazioni, permettendo quelle silenti.
		Tipicamente \`e la terza base del codone la pi\`u flessibile: codoni codificanti per un particolare amminoacido presentano le prime due conservate e la terza variabile.
		si notano in particolare i codoni $UAA$, $UAG$ e $UGA$ di stop e $AUG$ di inizio.
		Si noti inoltre come non tutti gli amminoacidi sono critici per il funzionamento di una proteina, pertanto non tutte le mutazioni causano gravi danni ad essa.
		Gli amminoacidi critici si trovano tipicamente nel sito attivo o sono coinvolti nella determinazione della struttura.

		\subsubsection{Mutazione puntiforme}
		Si dice mutazione puntiforme la sostituzione di una coppia di basi.
		L'errore viene poi trascritto nel mRNA e nella proteina.
		Se questo errore avviene in una regione codificante la mutazione risulta nell'alterazione della sequenza degli amminoacidi.
		\begin{multicols}{2}
			\begin{itemize}
				\item Mutazioni missenso: viene alterata la sequenza della proteina.
				\item Mutazioni nonsenso: si forma un codone di stop prematuro e una proteina tronca.
				\item Mutazioni silenti: mutazioni di una singola base possono non causare cambi nella sequenza.
					Avvengono tipicamente nella terza base del codone.
			\end{itemize}
		\end{multicols}
	
		\subsubsection{Mutazioni frameshift}
		Nello scivolamento dello schema di lettura o mutazioni frameshift perdite o integrazioni di basi portano a uno scivolamento del frame e a una modifica dell'intera sequenza della proteina.
		Porta spesso alla sintesi di proteine non funzionanti, ma in caso di inserimento o delezione di sequenze lunghe $3nt$ o suoi multipli vengono unicamente acquisiti o persi amminoacidi.
	
	\subsection{Retromutazioni o reversioni}
	Le mutazioni puntiformi sono generalmente reversibili attraverso le reversioni. 
	Un revertante viene definito come un ceppo in cui il fenotipo selvatico, che era stato perso nel mutante, viene ripristinato. 

		\subsubsection{Revertanti dello stesso tipo}
		Nei revertanti dello stesso tipo la mutazione che ripristina l'attivit\`a si verifica nel medesimo sito in cui \`e avvenuta la mutazione originale.

		\subsubsection{Revertanti di secondo sito}
		Nei revertanti di secondo sito la mutazione avviene in un sito diverso del DNA.
		Queste mutazioni, dette soppressive, compensano l'effetto della mutazione originale ripristinando il fenotipo selvatico.
		Possono essere:
		\begin{multicols}{2}
	    		\begin{itemize}
	        		\item Mutazioni nello stesso gene che ristabiliscono il frameshift originale;
	        		\item Mutazioni in altri geni che possono ripristinare la funzione del gene originale mutato; 
	        		\item Mutazioni in altri geni che determinano la produzione di un enzima che può sostituire quello mutante. 
	    		\end{itemize}
		\end{multicols}

	\subsection{Frequenza di mutazione}
	Gli errori nella replicazione del DNA ricorrono con una frequenza di circa $10^{-7}$-$10^{-11}$ per coppia di basi durante un ciclo di replicazione. 
	In quanto un gene tipo possiede circa $1000$ paia di basi, la frequenza degli errori in esso \`e di $10^{-4}$-$10^{-8}$ per generazione. 
	Si nota pertanto come in una coltura batterica, avente circa $10^{8}\frac{cellule}{\si{mL}}$ esiste la probabilità che in ciascun $\si{ml}$ di coltura, per un dato gene, ci sia almeno un mutante. 
	Non tutte le mutazioni hanno la stessa probabilit\`a di avvenire: si trovano nell'ordine $nonsenso<silenti<missenso$

		\subsubsection{Esempio}
		Si prenda in considerazione un codone $GGG$ che codifica per la glicina.
		Le possibili mutazioni sono:
		\begin{multicols}{3}
			\begin{itemize}
				\item $GGU$:  \emph{Gly} (silent);
				\item $GGC$: \emph{Gly} (silent);
				\item $GGA$:  \emph{Gly} (silent);
				\item $GUG$:  \emph{Val} (missense);
				\item $GCG$:  \emph{Ala} (missense);
				\item $GAG$:  \emph{Glu} (missense);
				\item $UGG$:  \emph{Trp} (missense);
				\item $CGG$:  \emph{Arg} (missense);
				\item $AGG$:  \emph{Arg} (missense).
			\end{itemize}
		\end{multicols}
		Se a cambiare è solo l'ultimo nucleotide di un codone otteniamo una mutazione silente. 
		Se cambia il primo o il secondo nucleotide sicuramente abbiamo una mutazione missenso. 
		Le mutazioni nonsenso avvengono per un raro caso di coincidenza. 

	\subsection{Mutagenesi}
	La frequenza di mutazione può essere aumentata da vari agenti chimici, fisici o biologici. 
  
		\subsubsection{Mutageni chimici}
		I mutageni chimici portano ad analoghi delle basi nucleotidiche. 
		Questi sono simili nella struttura alle basi del DNA ma si dimostrano difettosi nell'appaiamento. 
		Aumentano con gli errori nella replicazione del DNA con l'incorporazione di una base sbagliata nell'elica di DNA completa. 
		Due esempi di analoghi dei nucleotidi che causano sostituzioni da $AT$ e $GC$ sono:
		\begin{multicols}{2}
			\begin{itemize}
				\item \emph{5-Bromouracil} si associa con $G$ anzichè $A$; 
				\item \emph{2-Aminopurine} si associa con $C$ anzichè $T$. 
			\end{itemize}
		\end{multicols}
		Per esempio la timina e il suo analogo \emph{bU}: l'inserimento di \emph{bU} al posto di $T$ causa una prima mutazione ($G$ anzichè $A$) dopo il primo ciclo replicativo. 
		Dopo il secondo ciclo replicativo avviene la sostituzione completa di un paio di basi $AT$ in $GC$. 

			\paragraph{Tipologie di mutageni chimici}

				\subparagraph{Agenti alchilanti}
				Gli agenti alchilanti interagiscono con il DNA, creando legami crociati tra le sue eliche.
				Sono in grado di indurre cambiamenti anche in assenza di replicazione.

				\subparagraph{Agenti intecalanti}
				Gli agenti intercalanti si inseriscono tra due coppie di basi del DNA separandole.
				Portano a microinserzioni o microdelezioni inducendo mutazioni frameshift.\\
				Un esempio di agente intercalante o frameshift mutagen \`e l'acridina, che si inserisce nella doppia elica del DNA, causando inserimento o delezione di un nucleotide da parte della DNA polimerasi.
				Queste sostanze vengono chiamate anche teratogene o cancerogene.
		
		\subsubsection{Agenti fisici}
		Le mutazioni possono essere indotte anche da agenti fisici come le radiazioni.

			\paragraph{Radiazioni}

				\subparagraph{Raggi \emph{UV}}
				I raggi \emph{UV} inducono la formazione di dimeri di pirimidine ($C$ o $T$), uno strato in cui le basi vengono legate covalentemente durante la replicazione del DNA.
				Questo legame impedisce la formazione di legami idrogeno con le basi del filamento complementare, aumentando la probabilit\`a che la DNA polimerasi inserisca un nucleotide sbagliato.
				\subparagraph{Radiazioni ionizzanti}
				Le radiazioni ionizzanti causano effetti mutageni indiretti tramite ionizzazione dell'acqua e formazione del radicale libero \emph{$OH^-$}, creando stress ossidativo in grado di danneggiare le molecole di DNA.

\section{Isolamento dei mutanti}

	\subsection{Mutazioni selezionabili}
	Si dice mutazione selezionabile un tipo di mutazione che permette l'isolamento del ceppo mutante in laboratorio.
	Le mutazioni non selezionabili possono comunque portare a un profondo cambiamento del fenotipo dell'organismo.

		\subsubsection{Resistenza agli antibiotici}
		Un esempio di mutazione selezionabile \`e la resistenza agli antibiotici:
		un mutante antibiotico pu\`o crescere in presenza di una concentrazione di farmaco in grado di inibire il tipo selvatico.

			\paragraph{Isolare un mutante per resistenza agli antibiotici}
			\begin{multicols}{2}
				\begin{enumerate}
					\item Si consideri una piastra di coltura con tappeto uniforme di batteri.
    					\item Si posiziona in essa un disco contenente antibiotico che diffonde radialmente. 
						Più ci si allontana dall'anello, minore è la concentrazione del farmaco.
    					\item La distanza dal disco in cui si trovano colonie indica il grado di sensibilit\`a di un batterio o la sua resistenza.
				\end{enumerate}
			\end{multicols}
			La selezione è quindi uno strumento estremamente potente che permette l'isolamento di un singolo mutante all'interno di una popolazione. 

	\subsection{Isolamento di mutanti nutrizionali per selezione indiretta}

		\subsubsection{Replica plating}
		Con la tecnica di ``replica plating'' possono essere identificati mutanti nutrizionali direttivi. 

			\paragraph{Procedimento}\mbox{}\\
			\begin{multicols}{2}
				\begin{enumerate}
					\item Si crea una coltura con diluizione corretta in modo da far crescere colonie individuabili e non sovrapposte su un terreno di coltura ricco.
					\item Si ottiene una stampa delle colonie attraverso stampa con velluto sterile dalla piastra madre.
					\item Si passano le colonie su piastre mancanti nutrienti specifici.
				\end{enumerate}
			\end{multicols}
	
			\paragraph{Risultati}\mbox{}\\
			\begin{multicols}{2}
				\begin{itemize}
	    				\item L'incapacità di una colonia a crescere sulla piastra replicata la segnala come mutante: non \`e in grado di sintetizzare il nutriente mancante.
						Quella colonia si dice auxotrofa per quella richiesta nutrizionale.
	    				\item La capacit\`a di una colonia di crescere sulla piastra replicata la segnala come wild type per il nutriente mancante.
						Si dice pertanto prototrofa per quella richiesta nutrizionale.
				\end{itemize}
			\end{multicols}
	
	\subsection{Saggi di laboratorio per l'identificazione dei mutageni}

		\subsubsection{Identificazioni di mutazione per seleziona positiva}
		Il test di identificazione di mutazione per selezione positiva consiste di prendere una coltura batterica e versarla su una piastra contenente un antibiotico come penicillina.
		Solo i batteri resistenti ad esso resistono e vengono pertanto eliminati i fenotipi wild-type.
		Come test di conferma si versa la stessa coltura batterica su piastra senza penicillina e si osserva quante colonie in pi\`u crescono.
			
			\paragraph{Aggiunta di mutageno}
			Dopo il test si aggiunge alla coltura la sostanza che si vuole testare come mutagena.
			Se la sostanza \`e mutagena si nota come dopo aver fatto crescere la coltura sulla piastra con penicillina il numero di colonie aumenta: \`e aumentata la frequenza di mutazione.

				\subparagraph{Calcolo della frequenza di mutazione}
				\[\dfrac{N^{o}_{\text{colonie con mutageno}} - N^{o}_{\text{colonie senza mutageno}}}{N^{o}_{\text{colonie senza mutageno}}} \times 100\]

		\subsubsection{Test di Ames}
		Il test di Ames utilizza mutanti del genere \textit{Salmonella} con una mutazione puntiforme che impedisce la biosintesi dell'aminoacido istidina (\emph{$His^{-}$}). 
		Se il terreno non presenta istidina, allora la colonia non cresce per mancanza di questo aminoacido. 
		Ad una sospensione di mutanti viene aggiunto estratto di fegato, che simula le condizioni fisiologiche nelle quali gli enzimi epatici possono trasformare sostanze  in agenti mutageni. 
		I batteri vengono piastrati su un terreno senza istidina: la comparsa di colonie revertanti (\emph{$His^{+}$}) è indicativa della mutagenicità della sostanza testata.
		Se i revertanti compaiono sia nella piastra di controllo sia nella piastra di testa, con l'aggiunta della sostanza si ha un netto aumento del numero di revertanti attorno al disco. 
		Questo dimostra la sua azione mutagena. 
		In prossimità della sostanza non sono presenti batteri. 
		Questo accade perchè la concentrazione della sostanza mutagena è troppo alta e causa una quantità eccessiva di mutazioni all'interno del batterio che portano alla morte dell'individuo.

\section{Ricombinazione genetica omologa}
Si intende per ricombinazione genetica omologa uno scambio fisico di materiale genetico. 
Consiste nello scambio genetico tra sequenza omologhe di DNA. 
Le sequenze omologhe di DNA non sono complementari, ma presentano un alto tasso di identità che permette l'appaiamento. 

	\subsection{Processo}

		\subsubsection{Taglio}
		Il processo inizia con un taglio o nick prodotto da una endonucleasi (enzima che taglia il DNA) e che spesso presenta anche un'attvità elicasica per srotolare il DNA. 
		A questo taglio si legano le proteina \textbf{SSB + RecA} che formano un complesso che facilita il riappaiamento con la sequenza complementare del DNA recipiente e avviene la strand invasion.
		Lo spostamento del filamento residente avviene spontaneamente.
		Dopo l'appaiamento può avvenire uno scambio di molecole omologhe di DNA, che porta alla formazione di intermedi di ricombinazione. 
		Questi contengono delle regioni eteroduplici (heteroduplex) dove ciascun frammento è originato da cromosomi differenti. 

		\subsubsection{Risoluzione}
		Si intende per risoluzione la liberazione delle molecole dell'ibrido.
		Pu\`o avvenire in due modi.
			
			\paragraph{Patches}
			Nel metodo patches solo un filamento presenta lo scambio e diventa ibrido.

			\paragraph{Splices}
			Nel metodo splices entrambi i filamenti sono ibridi.

	\subsection{Identificazione dei ricombinanti}
	Per identificare i ricombinanti vengono utilizzati dei ceppi riceventi che mancano di alcune caratteristiche selezionabili che i ricombinanti dovranno possedere. 
	Per la vera identificazione di un'avvenuta ricombinazione è importante che il tasso di retromutazione per il carattere studiato sia basso dato che oltre ai ricombinanti anche i revertanti potranno formare colonie. 
	Vengono utilizzati spesso doppio mutanti, cioè ceppi che presentano mutazioni diverse, perché è poco probabile che possano avvenire retromutazioni nella stessa cellula. 

		\subsubsection{Processo}
		Viene preso del DNA libero estratto da delle cellule \emph{$Trp^{+}$} e lo si mette nella stessa provetta della coltura batterica. 
		Una volta che vengono rimesse sulla piastra, solo le cellule che hanno integrato il nuovo estratto sono in grado di crescere e formare delle colonie.
		Nei procarioti è possibile osservare la ricombinazione genetica quando i frammenti di DNA omologo vengono trasferiti da una cellula donatrice (donor) ad una cellula ricevente (recipient) tramite trasferimento genico orizzontale.

	\subsection{Trasferimento genico orizzontale}

		\subsubsection{Trasformazione}
		Durante la trasformazione il DNA di una cellula viene assimilato da un'altra senza contatto diretto.
		\`E un processo mediante il quale una molecola di DNA libero viene incorporata in una cellula ricevente e determina un cambiamento genetico o ricombinazione.

			\paragraph{Scoperta}
			La trasformazione viene osservata da Griffith nel $1928$.
			In questo esperimento viene preso in considerazione il batterio Streptococcus pneumoniae, che presenta due ceppi principali:
			\begin{multicols}{2}
				\begin{itemize}
					\item $S$ smooth patogeno con una capsula polisaccaride esterna.
					\item $R$ rough non patogeno senza capsula e incapace di causare infezione.
				\end{itemize}
			\end{multicols}
				
				\subparagraph{Prove sperimentali}\mbox{}\\
				\begin{multicols}{2}
					\begin{enumerate}
    						\item Il ceppo $S$ vivo iniettato nel topo: questo muore e si trovano cellule vive del ceppo $S$ nel cuore; 
    						\item Il ceppo $R$ vivo iniettato nel topo: questo è sano e non viene trovata nessuna cellula batterica nel cuore;
    						\item il ceppo $S$ inattivato dal calore (lisi): il topo è sano e non viene trovata nessuna cellula batterica nel cuore; 
    						\item Il ceppo $R$ vivo insieme al ceppo $S$ inattivato dal calore: il topo muore e vengono trovate delle cellule vive del ceppo S nel cuore.
					\end{enumerate}
				\end{multicols}

				\subparagraph{Risultati}
				Questo esperimento oltre a mettere in luce il ruolo centrale della capsula batterica, mostra come una cellula batterica è capace di acquisire determinate caratteristiche da un altra. 
				Questo avviene in quanto il materiale genetico che viene rilasciato nell'ambiente esterno mantiene la capacità di codificare informazioni.
				La trasformazione pu\`o avvenire anche in vitro.

			\paragraph{Lunghezza delle molecole trasformate}
			A causa della sua estrema lunghezza (ad esempio in $1700\si{\micro\metre}$ in \textit{Bacillus}) la molecola di DNA si può rompere facilmente. 
			Anche dopo un'estrazione delicata un cromosoma batterico si riduce in frammenti di circa $15kb$. 
			Questa dimessone rappresenta un tipico frammento trasformabile. 

			\paragraph{Competenza}
			Si dicono competenti le cellule con l'abilit\`a di acquisire DNA dall'ambiente.
			La competenza è il risultato delle alterazioni degli involucri cellulari (membrane e parete).
			Alcune cellule presentano competenza naturale, mentre ad altre deve essere indotta.

				\subparagraph{Competenza indotta}
				La competenza viene indotta attraverso shock elettrici o esposizione a cloruro di calcio in modo da rendere la membrana pi\`u permeabile al DNA.

			\paragraph{Trasformazione con DNA plasmidico} 
			Il DNA plasmidico \`e un ottimo vettore di trasformazione in quanto:
			\begin{multicols}{2}
				\begin{itemize}
    					\item I plasmidi si degradano meno facilmente dei frammenti lineari.
    					\item Non richiedono necessariamente integrazione nel cromosoma batterico tramite ricombinazione omologa; 
    					\item possono replicarsi all'interno della cellula ospite.
				\end{itemize}
			\end{multicols}

			\paragraph{Processo}

				\subparagraph{Introduzione nella cellula e ricombinazione}
    				Il DNA trasformante (lineare) si lega alla superficie della cellula mediante una proteina legante il DNA. 
				Successivamente pu\`o penetrare la membrana o l'intero doppio filamento.
				In alcuni casi un'endonucleasi degrada un filamento e ne viene acquisito solo uno.
				Questo si associa a una proteina specifica che lo protegge dalle nucleasi.
				Il frammento viene integrato nel genoma attraverso ricombinazione omologa non reciproca grazie alla proteina \emph{RecA}.
				Durante la replicazione del DNA eteroduplice si forma una molecola parentale e una di DNA ricombinante.

				\subparagraph{Ricombinazione omologa non reciproca}
				Si associano i segmenti omologhi e si apre la doppia elica del DNA del ricevente.
				Questo permette l'appaiamento con la sequenza omologa sul DNA donatore.
				L'endonucleasi taglia parte del filamento donatore e crea fratture sull'ospite in cui si posiziona il nuovo DNA.
				Vengono poi riparati i gap sul filamento.
				Il DNA ospite \`e eteroduplice.

			\paragraph{Introduzione del DNA trasformante nella cellula}\mbox{}\\
			\begin{multicols}{2}
				
				\subparagraph{Gram negativi}
				\begin{itemize}
					\item \emph{PilQ}: causa il movimento attraverso la membrana esterna.
					\item \emph{PilE}: trasferisce il DNA attraverso parete e spazio periplasmatico.
					\item \emph{ComE}: proteina di legame al DNA.
					\item $N$: nucleasi che degrada un filamento di DNA.
					\item \emph{ComA}: canale che consente il passaggio di DNA nel citoplasma.
				\end{itemize}
				\columnbreak
				
				\subparagraph{Gram positivi}
    				\begin{itemize}
        				\item ComGC = PilE;
        				\item ComEA = ComE;
        				\item Nuclasi (N);
        				\item ComEC = ComA;
        				\item ComFA è un DNA traslocasi in grado di trasferire il DNA nel citoplasma (nessun analogo nei Gram$-$).
				\end{itemize}
			\end{multicols}
		
		\subsubsection{Trasduzione}
		La trasduzione implica il trasferimento di DNA da una cellule donatrice ad una cellula ricevente tramite un virus. 
		Può avvenire sia in cellule eucariote che in procariote, ed è limitata dalla specificità di infezione del virus stesso. 

			\paragraph{Fagi}
			I virus che infettano i batteri vengono chiamati batteriofagi o fagi.
			I fagi che infettano un batterio possono compiere due cicli.

				\subparagraph{Struttura}
				I fagi sono composti da una testa o capside proteico che contiene RNA o DNA.
				La testa presenta un collo seguito da collare e guaina della coda.
				All'estremità si trova una piastra basale con attaccata la coda di filamenti proteici. 

				\subparagraph{Ciclo litico}
				\begin{enumerate}
    					\item Il batteriofago si fissa alla superficie batterica e inietta il suo acido nucleico.
    					\item Il genoma del virus si chiude ad anello, si replica e sfruttando gli organuli  dell'ospite presiede alla sintesi di nuove particelle virali.
    					\item Le nuove particelle virali si assemblano e formano nuovi virus che degradano la cellula ospite. 
				\end{enumerate}

				\subparagraph{Ciclo lisogenico}
				Il ciclo lisogenico \`e un evento reversibile
				\begin{enumerate}
    					\item Il batteriofago si fissa alla superficie batterica e inietta il suo acido nucleico.
    					\item Il genoma del virus si chiude ad anello e si integra con quello della cellula ospite.
						Diventa in questo modo un profago.
    					\item La cellula ospite si divide mantenendo il genoma virale integrato nel proprio DNA.
				\end{enumerate}

				\subparagraph{Classificazione dei fagi in base al ciclo}
				La capacit\`a di svolgere ciclo litico o lisogenico \`e caratteristica per una specie di fagi.
				\begin{multicols}{2}
					\begin{itemize}
						\item I fagi temperati sono in grado di compiere a seconda delle condizioni uno dei due cicli.
						\item I fagi virulenti possono compiere solo il ciclo litico e per questo sono immediatamente patogeni.
					\end{itemize}
				\end{multicols}
			
			\paragraph{Meccanismo della trasduzione}\mbox{}\\
			\begin{multicols}{2}
				\begin{enumerate}
		    			\item Il fago entra in contatto con la cellula ospite, inietta il proprio materiale genetico e devia il metabolismo cellulare verso la sintesi di nuove particelle virali o virioni. 
    					\item Durante l'assemblaggio dei virioni, i frammenti di DNA della cellula ospite possono essere incapsulati e trasferiti a un'altra cellula ospite.
				\end{enumerate}
			\end{multicols}

			\paragraph{Trasferimento dei geni}
			Il trasferimento dei geni dell'ospite utilizzando il virus come vettore pu\`o avvenire in due modi.

				\subparagraph{Trasduzione generalizzata}
				Si intende per trasduzione generalizzata un processo in cui qualunque frammento di DNA derivante dal genoma dell'ospite pu\`o diventare la componente di DNA dei nuovi virus.
				Durante il ciclo litico gli enzimi responsabili per l'impacchettamento del DNA virale nella testa del fago impacchettano anche DNA dell'ospite.
				Queste particelle vengono rilasciate durante la lisi della cellula.
				Le particelle trasducenti non creano una normale infezione virale e vengono dette pertanto difettive.
				Questo avviene in quanto i geni batterici hanno sostituito alcuni geni virali indispensabili.
				Il lisato, formato da particelle e virioni normali viene usato per infettare una popolazione di cellule riceventi.
				Una parte di esse entrano in contatto con le particelle traducenti e acquisiscono il DNA del precedente batterio ospite.
				Il DNA di queste particelle non pu\`o replicarsi ma pu\`o subire una ricombinazione genetica con il DNA del nuovo ospite.

				\subparagraph{Trasduzione specializzata}
				Si intende per trasduzione specializzata un processo in cui il DNA di una specifica regione cromosomica dell'ospite viene integrato con il genoma del virus.
				Permette un trasferimento efficiente e garantisce a una piccola regione del cromosoma batterico di venire trasdotta indipendentemente dal resto.
    
				\subparagraph{Esempio - trasduzione dei geni del galattosio}
				La trasduzione dei geni del galattosio a opera del fago $\lambda$ di E. coli \`e un esempio di trasduzione specializzata.
				La regione in cui $\lambda$ si integra nel cromosoma \`e adiacente ai geni che controllano gli enzimi coinvolti nella catabolisi del galattosio.
				Durante la fase di induzione il DNA del profago viene escisso come un'unit\`a e si riproduce.
				In casi rari il genoma fagico viene escisso in modo non corretto e alcuni geni del cromosoma batterico sono erroneamente escissi insieme al DNA fagico.
				Questa particella \emph{$\lambda$ dgal} \`e difettiva a causa della perdita dei geni fagici e non pu\`o generare fagi maturi in un'infezione successiva.
				Cellule coinfettate da \emph{$\lambda$ dgal} e da un fago helper ($\lambda$ non difettivo) si ottiene un lisato con alcune particelle \emph{dgal} e un gran numero di virioni normali.
				Se una coltura batterica auxotrofa per galattosio \emph{$Gal^-$} viene infettata con il lisato misto si possono selezionare i batteri trasducenti \emph{$Gal^+$}.
				Inducendo i doppi lisogeni con \emph{$\lambda$ dgal} e fago helper si ottiene un lisato ricco di fagi \emph{dgal} in modo da trasdurre con alta efficienza.

		\subsubsection{Coniugazione}
		La coniugazione è il principale meccanismo di trasferimento dei plasmidi da cellulla a cellula. 
		Questa funzione viene codificata dagli stessi plasmidi. 
		Si tratta di un processo replicativo alla fine del quale entrambe le cellule conterranno una copia del plasmide. 
		La trasmissibilità mediante coniugazione viene controllata da una serie di geni localizzati nella regione \textit{tra} del plasmide. 
		Alcuni plasmidi hanno la capacità di trasferirsi tra organismi molto diversi fra loro: tra gram-negativi e gram-positivi, tra batteri e cellule vegetali, tra batteri e funghi.

			\paragraph{I plasmidi}
			I plasmidi sono degli elementi genetici, solitamente di forma circolare, in grado di replicarsi indipendentemente dal cromosoma dell'ospite.
			Contengono geni non essenziali, ma che in certe condizioni possono diventare utili alla vita dell'ospite. 
			La loro dimensione varia da $1Kb$ a $1 Mb$ e sono presenti da $1$ ad oltre $100$ copie in una cellula ospite. 
			Sono conosciuti migliaia di tipi differenti e sono presenti oltre $300$ tipi in \textit{E. coli}. 
			Gli enzimi che vengono coinvolti nella replicazione dei plasmidi sono gli stessi utilizzati per la replicazione del genoma della cellula ospite.

				\subparagraph{Incompatibilit\`a}
				Due plasmidi incompatibili non possono essere mantenuti assieme nella cellula ospite in quanto competono l'un l'altro per l'inizio della replicazione. 
				Si possono formare gruppi di incompatibilit\`a: plasmidi appartenenti a un gruppo condividono un meccanismo comune di regolazione della replicazione e sono correlati.
				Una coltura batterica pu\`o contenere diversi tipi di plasmidi se questi non sono geneticamente correlati: appartengono a diversi gruppi di incompatibilit\`a.

				\subparagraph{Plasmidi episomi}
				I plasmidi episomi sono in grado di integrarsi nel cromosoma.
				Quando lo fanno la loro replicazione procede sotto il controllo del cromosoma stesso.

				\subparagraph{Curing}
				Si dice curing la perdita di un plasmide da parte di una cellula.
				Pu\`o avvenire spontaneamente in popolazioni in cui non vi \`e pressione selettiva per il mantenimento del plasmide.

			\paragraph{Dimostrazione della coniugazione batterica}
			Per dimostrare che avviene coniugazione batterica due ceppi con doppia o tripla auxotrofia vengono mescolati. 
			Se avviene crescita su terreno minimo questo indica eventi di ricombinazione tra i $2$ cromosomi batterici.
			Questo esperimento non dà alcuna informazione sulla direzione del trasferimento del materiale genetico.

				\subparagraph{Esperimento del tubo a $\mathbf{U}$}
				L'esperimento del tubo a $U$ dimostra che la ricombinazione genetica nel processo di coniugazione avviene per contatto diretto tra cellule batteriche.
				Si prende un tubo di coltura e a una sua estremit\`a viene collegata una pompa per far circolare il terreno di coltura liquido.
				Nel mezzo si pone un filtro di vetro poroso che consente il passaggio dei nutriente ma non delle cellule batteriche.
				Entrambe le popolazioni possiedono due auxotrofie opposte.
				Se avvenisse coniugazione queste verrebbero trasformate in prototrofie.
				Piastrando poi le due colonie su terreno minimo non si osserva nessuna crescita in quanto il filtro ha impedito il contatto diretto e la ricombinazione delle popolazioni.

			\paragraph{Plasmide \emph{F} (fertility) di \textit{E. coli}}
			Il plasmide \emph{F} di E. coli \`e formato da $99159 bp$.
			\begin{multicols}{2}
				\begin{itemize}
    					\item Una regione contiene geni coinvolti nella replicazione del plasmide.
    					\item La regione \emph{tra} contiene geni coinvolti nel trasferimento coniugativo come quelli per la sintesi del pilo $IV$.
					\item La regione \emph{oriT} è l'origine del trasferimento durante la coniugazione. 
    					\item Due regioni trasponibili (mobili) consentono l'integrazione del plasmide nel cromosoma batterico. 
				\end{itemize}
			\end{multicols}

			\paragraph{Coniugazione tra una cellula \emph{$F^{+}$} e una \emph{$F^{-}$}}
			
				\subparagraph{Processo di coniugazione}\mbox{}\\
				\begin{multicols}{2}
					\begin{enumerate}
    						\item Si crea la struttura coniugativa; 
    						\item Il pilo si depolimerizza portando le due strutture in contatto.
						\item Un filamento del DNA del fattore \emph{F} viene tagliato da una endonucleasi e si muove attraverso il ponte coniugativo.
    						\item Il DNA complementare viene sintetizzato su entrambi i singoli filamenti.
    						\item Si completa il movimento attraverso il ponte di coniugazione e la sintesi del DNA.
    						\item La ligasi circolarizza la molecola di DNA. 
						\item I batteri coniugati si separano. 
					\end{enumerate}
				\end{multicols}
				In questo modo il batterio exconiugante ha acquisito un plasmide e quello donatore non l'ha perso.

				\subparagraph{Replicazione a ciclo rotante}
				La replicazione a ciclo rotante \`e una replicazione asimmetrica e procede in una sola direzione: solo una delle eliche parentali viene replicata.
				L'endonucleasi taglia il plasmide a \emph{oriT}.
				Dopo un giro completo il filamento comincia il processo di trasferimento durante cui avviene la replicazione di un filamento del plasmide.
				In corrispondenza del poro coniugativo si trova un enzima bifunzionale \emph{rilassosoma} o \emph{Tra1}.
				Questo enzima \`e codificato dalla cellula donatrice e ha attivit\`a nucleasica ed elicasica.
				La sintesi del nuovo filamento avviene in maniera continua nella cellula donatrice e in maniera discontinua nella cellula ricevente.

				\subparagraph{Replicazione e trasferimento del plasmide}\mbox{}\\
				\begin{multicols}{2}
					\begin{enumerate}
    						\item L'enzima \emph{Tra1} taglia il DNA in \emph{oriT} e porta a un rilassamento della molecola stessa (srotolamento). 
						\item L'estremità $5'$ del filamento che deve essere trasferito o external strand si lega covalentemente all'aminoacido tirosina dell'enzima. 
							Il primo tratto del filamento codifica per le proteine \emph{SSB} che proteggono il singolo filamento di DNA trasferito dall'azione delle nucleasi. 
    						\item Una volta avvenuto il legame dell'enzima al poro coniugativo inizia la replicazione del retained strand, filamento stampo che non viene trasferito.
						\item Si sintetizza il DNA donatore mediante il meccanismo a ciclo rotante.
						\item Il filamento di DNA viene spinto nel batterio ricevente dove viene convertito in DNA a doppio filamento.
					\end{enumerate}
				\end{multicols}
				La cellula ricevente presenta sulla superficie delle proteine che riconoscono il sito di attacco. 

			\paragraph{Formazione di ceppi \emph{Hfr} e mobilizzazione del cromosoma}
			Il plasmide F è un episoma e quindi può integrarsi nel cromosoma dell'ospite e trasferire il cromosoma alla cellula ricevente. 
			Le cellule che possiedono questo plasmide non integrato si indicano \emph{$F^{+}$}, mentre quelle che hanno il plasmide integrato si indicano \emph{Hfr} (High frequency of recombination). 
			Entrambe agiscono come donatori ma non sono in grado di acquisire stabilmente una seconda copia del plasmide F o di plasmidi ad esso correlati.

				\subparagraph{Effetti del plasmide \emph{F} nella cellula}
				Il plasmide \emph{F} nella cellula:
				\begin{multicols}{2}
					\begin{itemize}
    						\item Conferisce la capacit\`a di sintetizzare il pilo; 
    						\item Mobilizza il DNA cromosomico per il suo trasferimento in un'altra cellula quando integrato.
    						\item Altera i recettori di superficie, in modo che la cellula non sia più in grado di comportarsi come un ricevente nella coniugazione. 
					\end{itemize}
				\end{multicols}

				\subparagraph{Coniugazione di cellule \emph{Hfr}}
				Le cellule \emph{Hfr} trasferiscono una porzione del loro DNA e del plasmide \emph{F} ad una cellula ricevente.
				La cellula ricevente viene ricombinata ma rimane \emph{$F^{-}$} in quanto il plasmide non viene passato completamente.
				La struttura rimane instabile fino a che viene integrata nel genoma.

				\subparagraph{Integrazione del plasmide F}
				L'inserzione può avvenire in varie regioni del cromosoma in corrispondenza di siti specifici \emph{IS} (insertion sequence) che mostrano omologia con la sequenza del plasmide. 
				Una volta che il plasmide è stato integrato, non è più in grado di controllare la propria replicazione ma rimane capace di sintetizzare il \emph{pilus}. 
       				Il processo di coniugazione rimane analogo rispetto al \emph{F} non integrato, con l'eccezione che vengono trasferiti anche geni cromosomali.

				\subparagraph{Uso dei ceppi \emph{Hfr} negli incroci genetici}
				Durante la coniugazione sia le cellule del donatore sia quelle del ricevente sono vitali. 
				Si devono pertanto selezionare i ricombinanti desiderati in modo che possano crescere senza che i ceppi parentali formino colonie. 
				Normalmente si utilizza:
				\begin{multicols}{2}
					\begin{itemize}
    						\item Un ricevente resistente a un antibiotico ma auxotrofo per qualche sostanza.
    						\item Un donatore sensibile all'antibiotico ma prototrfo per la stessa sostanza.
					\end{itemize}
				\end{multicols}
				Per esempio si ha:
				\begin{multicols}{2}
					\begin{itemize}
    						\item Donatore \emph{Hfr} sensibile alla streptomicina e prototrofo per il lattosio.
    						\item Ricevente \emph{$F^{-}$} resistente alla streptomicina e auxotrofo per il lattosio.
    						\item Un terreno selettivo con streptomicina e privo di lattosio.
					\end{itemize}
				\end{multicols}
				Le colonie che crescono nel terreno sono quelle in cui \`e avvenuta coniugazione.
				
				\subparagraph{Processo di coniugazione tra \emph{Hfr} e \emph{$\mathbf{F^-}$}}\mbox{}\\
				\begin{multicols}{2}
					\begin{enumerate}
						\item Il fattore \emph{F} viene integrato nel cromosoma batterico e la cellula diventa una cellula \emph{Hfr}; 
						\item Avviene la coniugazione tra una cellula \emph{Hfr} e una \emph{$F^{-}$}. 
						\item Il fattore F viene tagliato da un enzima creando l'origine di trasferimento del cromosoma; 
    						\item Inizia il trasferimento del cromosoma attraverso il ponte di coniugazione; 
    						\item Inizia la replicazione su entrambi i frammenti mentre continua il trasferimento del cromosoma. 
							\emph{F} si trova alla fine del cromosoma adiacente all'origine.
					\end{enumerate}
				\end{multicols}
				Tipicamente la coniugazione si interrompe prima del trasferimento completo del cromosoma: solo alcuni geni sono trasferiti e vengono ricombinati.

				\subparagraph{Coniugazione interrotta}
				L'ordine in cui i geni sono presenti sul cromosoma donatore può essere determinato dalla cinetica di trasferimento dei geni individuali. 
				Le due cellule possono essere separate per agitazione a un dato tempo per controllare i ricombinanti sul terreno selettivo.
				I geni più vicini all'origine di trasferimento (\emph{oriT}) sono quelli che entrano per primi nelle cellule riceventi e sono quindi presenti in una percentuale più alta dei ricombinanti rispetto ai geni che poi entreranno. 

				\subparagraph{Trasferimenti di geni cromosomali al plasmide \emph{F}}
				I plasmidi \emph{F} integrati possono occasionalmente separarsi dal cromosoma incorporando geni cromosomici.
				In questo caso vengono detti plasmidi \emph{$F'$}. 
				Contengono stabilmente dei geni cromosomali normalmente espressi che possono essere trasmessi ad altre cellule. 
				Trasferendo un plasmide \emph{$F'$} in una cellula ricevente si possono creare delle cellule diploidi che possono contenere due copie dello stesso gene. 
				La coniugazione con \emph{$F'$} deriva da un'escissione scorretta del fattore \emph{F} nel cromosoma ospite. 
				Alcuni geni dell'ospite vengono prelevati da \emph{F} e possono essere trasferiti ad un'altra cellula mediante la coniugazione. 
				Alle volte l'escissione avviene in modo impreciso e il plasmide \emph{F} porterà con se una sequenza adiacente del cromosoma batterico. 
				Si ottiene un plasmide \emph{$F'$} portatore del marcatore \emph{$lac^{+}$}. 
				Se \emph{$F'lac^{+}$} viene trasferito per coniugazione in una cellula \emph{$lac^{-}$}, si ottiene un diploide parziale (merodiploide) \emph{$lac^{+}/lac^{-}$}.
				Questo processo viene anche detto di complementazione.

				\subparagraph{Formazione di un fattore \emph{$\mathbf{F'}$} e coniugazione con un ceppo \emph{$F^-$}}\mbox{}\\
				\begin{multicols}{2}
					\begin{enumerate}
    						\item Dopo l'escissione illegittima \emph{F} può acquisire una porzione di cromosoma batterico e diventare \emph{$F'$}.
						\item La cellula portatrice di \emph{$F'$} può coniugare e trasferire il plasmide a una cellula \emph{$F^-$}.
						\item Si forma la coppia coniugativa.
    						\item Il fattore \emph{$F'$} replica mentre il filamento viene trasferito.
    						\item La cellula \emph{$F'$} ricevente diventa parzialmente diploide ed è chiamata merozigote. 
					\end{enumerate}
				\end{multicols}
\section{I trasposoni e la trasposizione}
	
Alcuni geni o gruppi di geni hanno la capacità di muoversi da una posizione ad un'altra nel genoma e sono detti elementi trasponibili o trasposoni. 

	\subsection{Introduzione}
	I trasposoni contengono i geni che codificano una trasposasi e brevi ripetizioni terminali invertite \emph{IR} alle estremità del loro DNA. 
	La trasposasi riconosce le proprie sequenze \emph{IR} nel genoma, taglia il DNA del sito bersaglio ed inserisce il trasposone o una sua copia. 
	Le sequenze di inserzione \emph{IS}  sono i tipi più semplici di elementi trasponibili e non trasportano informazioni geniche oltre a quelle necessaria per muoversi in nuovi siti.
	I trasposoni complessi contengono uno o più geni non coinvolti nel meccanismo di trasposizione stesso, come ad esempio geni di antibiotico-resistenza.

	\subsection{Trasposizione}
	La trasposizione è un evento di ricombinazione che avviene tra sequenze non omologhe.
	Non richiede l'uso del sistema di ricombinazione della cellula. 
	Richiede il riconoscimento di una specifica sequenza di basi e viene pertanto detta ricombinazione sito specifica.
	
		\subsubsection{Meccanismo di trasposizione}
		La trasposasi riconosce, taglia e lega il DNA durante il processo di trasposizione. 
		Una breve sequenza di basi del DNA bersaglio viene duplicata nel sito di integrazione. 

		\subsubsection{Duplicazione delle sequenze bersaglio}
		La duplicazione avviene in quanto il taglio della trasposasi \`e asimmetrico e crea filamenti sfalsati.
		Dopo l'inserimento da parte della trasposasi enzimi coinvolti nella riparazione del DNA riempiono i buchi, sintetizzando una giunzione a livello della catena.


	\subsection{Tipologie di meccanismi di trasposizione}
	Sono conosciuti due meccanismi di trasposizione.
	Entrambi iniziano allo stesso modo: la trasposasi riconosce i punti \emph{IR}, taglia e inserisce il trasposone all'interno del DNA bersaglio.

		\subsubsection{Meccanismo conservativo}
		Nel meccanismo conservativo l'elemento viene escisso da un sito del cromosoma e reinserito in un secondo.
		Il numero di trasposoni rimane invariato.
		Vengono creati dei tagli in modo da espellere il trasposone dal DNA donatore e la riparazione duplica le sequenze target e completa l'evento.
		La trasposasi agisce come dimero o tetramero e l'operazione di taglio e riunione avvengono in contemporanea sul trasposone e sul sito bersaglio.

		\subsubsection{Meccanismo replicativo}
		Nel meccanismo replicativo viene prodotta e inserita in un nuovo sito una nuova copia del trasposone.
		La replicazione avviene senza il taglio completo dei trasposoni dal sito donatore.
		Si forma un co-integrato intermedio, dove le due molecole si fondono per un breve periodo.
		La replicazione del trasposone attraverso DNA polimerasi avviene mentre l'elemento mobile \`e ancora attaccato al sito originale e a quello bersaglio.
		Una resolvasi scompone il co-integrato in due strutture separate, ognuna con una copia del trasposone.

		
	\subsection{Verifica di un evento di trasposizione}
	Si considerano due cellule batteriche: \emph{$F^+$} e \emph{$F^-$}. 
	Si vuole dimostrare la mobilità di un trasposone \emph{Tn3} dal suo sito donatore sul plasmide \emph{X} a quello ricevente sul plasmide \emph{F}.

		\subsubsection{Ceppo \emph{$\mathbf{F^+}$}}
		Il ceppo \emph{$F^+$} possiede:
		\begin{multicols}{2}
			\begin{itemize}
				\item Resistenza all'ampicillina codificata da \emph{Tn3} sul plasmide \emph{X}.
				\item Resistenza alla kanamicina codificata da un gene sul plasmide \emph{F}.
			\end{itemize}
		\end{multicols}
		Inoltre il fattore \emph{F} viene modificato in modo che non si integri nel cromosoma.

		\subsubsection{Ceppo \emph{$\mathbf{F^-}$}}
		Il ceppo F\ap{-} presenta:
		\begin{multicols}{2}
			\begin{itemize}
    				\item Mutazione \emph{polA1} che impedisce al plasmide di replicarsi; 
    				\item Resistenza a \emph{Nal} (acido nalidixico).
			\end{itemize}
		\end{multicols}

		\subsubsection{Movimento del trasposone}
		Se il trasposone salta da \emph{X} a \emph{F} si forma una struttura \emph{F-Tn3} trasferibile per coniugazione al batterio \emph{$F^-$}. 
		Dopo l'incrocio i batteri vengono selezionati per piastramento su terreni che presentano i tre antibiotici. 
		\emph{Nal} serve per selezionare il ricevente, mentre gli altri due per verificare il corretto trasferimento di informazioni geniche. 

		\subsubsection{Cointegrato}
		Il cointegrato, lo stadio intermedio di questo processo, è formato da due plasmidi \emph{F} e \emph{X} e due copie di \emph{Tn3}.
		In questo stadio si prevede la risoluzione del cointegrato in due plasmidi \emph{F} e \emph{X} ognuno portatore di \emph{Tn3}.
		Si parla quindi di trasposizione replicativa che si svolge in due fasi principali:
		\begin{multicols}{2}
			\begin{itemize}
    				\item Produzione del cointegrato; 
    				\item Risoluzione del cointegrato.
			\end{itemize}
		\end{multicols}
		
	\subsection{Mutagenesi con elementi trasponibili}
	Se il sito di inserzione per un elemento trasponibile è all'interno di un gene l'inserzione del trasposone porterà alla disattivazione (disruption) del gene considerato. 
	Si ha una variazione del ceppo e questa mutagenesi viene indotta con elementi trasponibili. 

		\subsubsection{Random Mutagenesis Protocol}
		I batteri che contengono il trasposone possono essere selezionati attraverso l'isolamento di colonie su un terreno ricco contenente antibiotico kanamicina. 
		Questo tipo di approccio viene utilizzato per identificare i geni essenziali di un certo batterio. 
		Nella prima piastra viene utilizzato il \emph{DAP} per permettere la crescita del batterio donatore (\textit{E. coli}). 
		In questo modo sia il donatore sia il ricevente possono crescere. 
		Nella seconda piastra viene eliminato il \emph{DAP} per isolare il batterio ricevente in quanto E. coli non \`e in grado di crescere senza \emph{DAP}.
		Viene infine aggiunto l'antibiotico per selezionare il nuovo batterio che avrebbe dovuto ricevere la resistenza per esso.

	\subsection{Integroni}
	Gli integroni sono elementi genetici mobili di DNA con la capacit\`a di catturare geni di varia origine e farli esprimere.
	Sovraesprimono un dato gene.
	Contengono gli elementi di ricombinazione sito-specifica e sono in grado di riconoscere e catturare cassette geniche e trasferirle orizzontalmente.
	Si possono trovare in trasposoni, plasmidi o cromosoma batterico.
	Non vengono inseriti casualmente ma contengono un gene che codifica un integrasi, richiesta nella ricombinazione sito specifica e una sequenza \emph{attI} che determina il sito di inserzione.
	Il gene catturato si trova in cassette geniche insieme a un promotore.
 
\section{Clonaggio genico}
Il clonaggio ha lo scopo di isolare una grande quantità di geni specifici in forma pura. 
La strategia è quella di spostare il gene o la regione di interesse da un genoma grande e complesso a uno piccolo e più semplice. 

	\subsection{Scopi del clonaggio genico}
	\begin{multicols}{2}
		\begin{itemize}
   			\item Isolare un gene specifico e replicarlo, definire la struttura del gene clonato;
    			\item Definire la modalità di espressione (spaziale, temporale, induzione ambientale); 
    			\item Esprimere il gene in altri organismi per studiarne la funzione;
    			\item Produrre una grande quantità della proteina codificata.
		\end{itemize}
	\end{multicols}

	\subsection{Enzimi di restrizione}
	Gli enzimi di restrizione sono enzimi specifici che riconoscono sequenze palindromiche sul DNA e le tagliano, permettendo l'isolamento di un gene specifico.
	Il taglio pu\`o essere simmetrico con formazione di blunt ends come avviene per \emph{AIuI}, \emph{HaeIII} o asimmetrico con formazione di sticky ends come avviene per \emph{BamHI}, \emph{HindIII} e \emph{EcoRI}.

	\subsection{Fasi del clonaggio genico}
	\begin{multicols}{2}
		\begin{enumerate}
    			\item Isolamento e frammentazione del DNA originario, anche da specie diverse da quelle batteriche, con enzimi di restrizione che viene poi usato per la restrizione; 
    			\item Giunzione dei frammenti ad un vettore di clonaggio, derivato da un plasmide o da un virus; 
    			\item Introduzione, mantenimento e moltiplicazione del DNA clonato nell'organismo ospite. 
		\end{enumerate}
	\end{multicols}
	I batteri ospite sono resi competenti e vengono indotti ad esprimere il gene.

	\subsection{I plasmidi come vettori di clonaggio}

		\subsubsection{Caratteristiche fondamentali dei plasmidi}
		\begin{multicols}{2}
			\begin{itemize}
    				\item Dimensione ridotta: facilita l'isolamento e la manipolazione del DNA.
    				\item Origine di replicazione: permette una replicazione indipendente del DNA batterico.
    				\item Numero elevato di copie: facilita l'amplificazione del DNA;
    				\item Presenza di marcatori selezionabili: facilitano il riconoscimento e la selezione dei cloni. 
			\end{itemize}
		\end{multicols}
		I plasmidi utilizzati sono modificati in modo da prevenire il trasferimento coniugativo e sono \emph{$F^-$} in modo da mantenere la loro integrit\`a.
		L'inserzione avviene per trasformazione.
		
		\subsubsection{Vettori plasmidici di prima generazione}
		Nei vettori plasmidici di prima generazione l'inserimento di DNA provoca la disattivazione del gene di resistenza alla tetraciclina. 
		Si ha una selezione pertanto indiretta dei plasmidi con DNA esogeno.
		Si ipotizza di avere un batterio con due resistenze agli antibiotici, tra cui la tetramicina. 
		L'inserimento di DNA provoca la disattivazione del gene di resistenza per la tetramicina, rendendo il batterio sensibile ad essa all'inserimento del plasmide.
		Il DNA viene tagliato con gli stessi enzimi che poi vengono utilizzati per il taglio del plasmide in modo che le estremità del gene e del plasmide siano complementari mantenendo l'integrità del plasmide. 
		Se non si riesce a inserire nulla nel plasmide, si aggiunge la ligasi per chiudere la struttura, quello che si ottiene è un plasmide vuoto (senza modifiche). 
		Per distinguere i plasmidi modificati dagli altri devo usare la tecnica di replica plating, quindi li sottopongo a entrambi gli antibiotici, in modo da eliminare una popolazione di batteri, quella in cui c'è stato l'inserimento nel gene, e poi ad uno solo.
		Se confronto le due piastre riesco a selezionare i primi che sono stati utilizzati. 

		\subsubsection{Vettori plasmidici di seconda generazione}
		I vettori plasmidici di seconda generazione richiedono meno tempo rispetto a quello visto in precedenza. 
		Con l'inserzione del DNA da clonare viene inattivato il gene \emph{lacZ} che codifica la \emph{$\beta$-galattosidasi}, che esegue la scissione del galattosio. 
		L'\emph{X-Gal}, un omologo del galattosio che produce un colore blu intenso quando viene scisso viene aggiunto al terreno di coltura. 
		Le colonie blu hanno quindi un gene \emph{lacZ} funzionale e non contengono un plasmide clonato.
		Se la $\beta$-galattosidasi non funziona \emph{X-Gal} non viene scisso e non diventa blu.
		Se il gene è stato inserito in mezzo al sito della $\beta$-galattosidasi la colonia diventa bianca. 

	\subsection{Regioni polilinker}
	Le regioni polilinker sono siti di restrizione riconosciuti da diversi enzimi e pi\`u facili da aprire.
	Si nota come un gene che viene selezionato non deve contenere sequenze interne riconosciute dall'enzima di restrizione altrimenti verrebbe tagliato.

	\subsection{Il batteriofago $\mathbf{\lambda}$ come vettore di clonaggio}
	Il fago lambda può essere usato come vettore di clonaggio. 
	La regione tra i geni $J$ e $N$ del genoma virale non è essenziale e può essere sostituita con un DNA esogeno. 
	Le regioni non essenziali permettono l'entrata del batterio nel ciclo lisogenico.
	Si possono usare come luoghi per inserire geni d'interesse rendendo $\lambda$ capace di compiere solo il ciclo litico.
	Con questa tecnica è possibile fare clonaggio di frammenti anche piuttosto grandi.

	
		\subsubsection{Fasi di cloanggio con il vettore lambda}
		\begin{multicols}{2}
			\begin{enumerate}
    				\item Isolamento del DNA del fago e digestione con un enzima di restrizione; 
    				\item Ligazione dei due frammenti di lambda ai frammenti del DNA esogeno. 
					Vengono selezionati  frammenti di DNA esogeno della lunghezza appropriata, circa 20 kb); 
    				\item Impacchettamento del DNA per aggiunta di estratti cellulari contenenti le proteine della testa e della coda con formazione spontanea di particelle fagiche (incapsulamento); 
    				\item Infezione di \textit{E. coli} con sospensione di fagi che contengono il gene preso come riferimento. 
				\item Si piastra il tutto e creano dei buchi trasparenti nel tappeto batterico. 
				\item Si procede con l'isolamento dei cloni fagici tramite analisi delle placche formatesi su una coltura del ceppo ospite; 
    				\item Analisi dei fagi ricombinanti. 

			\end{enumerate}
		\end{multicols}

	\subsection{La mutagenesi sito-diretta}
	La mutagenesi sito-diretta consente di causare mutazioni all'interno di uno specifico gene.
	Viene utilizzata per studiare l'azione di proteine che contengono specifiche sostituzioni aminoacidiche. 
	Il vettore contenente il DNA mutato viene inserito in un ceppo batterico mutante incapace di produrre la proteina in questione. 
	Viene introdotto il gene di interesse in un vettore a singolo filamento.
	Un frammento di DNA sintetico con le mutazioni (oligonucleotide sintetico) si può legare al vettore per complementarietà e viene esteso attraverso la DNA polimerasi. 
	Poi seguono le fasi di clonaggio e selezione. 
	
	\subsection{Mutagenesi a cassetta e inattivazione genica}
	\begin{enumerate}
		\item Un plasmide contenente il gene $X$ viene tagliato con l'enzima di restrizione \emph{EcoR1} per introdurre una cassetta di resistenza alla kanamicina.
	    	\item Dopo la ligazione abbiamo un plasmide che contiene la cassetta di kanamicina come mutazione di inserzione nel gene X. 
			Il plasmide viene linearizzato con un altro enzima di restrizione \emph{BamH1}; 
	    	\item Vengono trasformate le cellule batteriche che contengono una versione wild type del gene $X$.
			Avviene ricombinazione omologa tra il gene $X$ del vettore e quello wild type disattivandolo.
	    	\item Dopo la ricombinazione di possono selezionare i batteri mutanti (gene $X$ inattivato) su terreno contenente kanamicina.
	\end{enumerate}
