\chapter{Genomica microbica}
Il termine genomica fa riferimento alla mappatura, al sequenziamento e all'analisi dei genomi. 
La conoscenza della sequenza di un genoma non rivela solamente i geni dell'organismo ma fornisce anche informazioni sulle sue funzionalit\`a e sulla sua storia evolutiva. 

\section{Branche della genomica}
	
	\subsection{Genomica strutturale}
	La genomica strutturale esamina la struttura fisica dei genomi con l'obiettivo di determinare e analizzare la sequenza di DNA del genoma annotando i geni.

	\subsection{Genomica funzionale}
	La genomica funzionale indaga i meccanismi di funzionamento del genoma, dei trascritti e delle proteine codificate da essi.

	\subsection{Genomica comparata}
	La genomica comparata pone a confronto diversi genomi per risalire alle affinit\`a e alle differenze tra essi, identificando porzioni conservate che codificano per le proteine essenziali e determinano i modelli di funzione e regolazione.
	I dati servono anche per lo studio dell'evoluzione microbica e del trasferimento orizzontale di geni.
	Si nota come le differenze di genoma portano a differenze fenotipiche.

\section{Genomi}

	\subsection{Genoma procariote}
	I procarioti presentano un genoma compatto fatto quasi unicamente da geni.
	Si misura un elevato polimorfismo anche nella stessa specie.
	Sono molto variabili.

	\subsection{Componenti del genoma}

		\subsubsection{Core genome}
		Il core genome \`e la parte in comune tra le componenti di una stessa specie.

		\subsubsection{Genoma accessorio}
		Il genoma accessorio \`e composto dai geni che distinguono un individuo da un altro.
		Sono sequenze specifiche che possono occupare fino al $26\%$ di un genoma.
		In queste regioni si trova spesso fattori di patogenicit\`a di un batterio.

		\subsubsection{Genoma minimo}
		Il genoma minimo \`e il numero di geni essenziali alla vita di uno specifico organismo. 

			\paragraph{Core genome}
			Il core genome o \emph{COG}, cluster of orthologous genes.

			\paragraph{\emph{NOGD}}
			Non orthologous gene displacement \emph{NODG} sono geni con stessa funzione e stessa sequenza con antenati diversi.

\section{Sequenziamento genomico di prima generazione (1995)}

	\subsection{Clonaggio dei frammenti}
	Il primo passo per sequenziare un genoma \`e il clonaggio dei suoi frammenti.

		\subsubsection{Vettori di clonaggio}
		\begin{multicols}{2}
			\begin{itemize}
    				\item Clonaggio con vettori plasmidici (tipo puC19), frammenti di circa $2kb$.
    				\item Clonaggio con batteriofagi $\lambda$, frammenti di circa $20 kb$.
				\item Clonaggio con vettori \emph{BAC} (Bacterial Artificial Chromosome), frammenti di circa $300 kb$. 
				\item Clonaggio con vettori \emph{YAC} (Yeast Artificial Chromosome), frammenti di circa $800 kb$.
			\end{itemize}
		\end{multicols}


	\subsection{Processo}
	\begin{enumerate}
    		\item frammentando l'estratto di DNA (enzima di restrizione); 
    		\item mescolando i frammenti con il plasmide; 
    		\item introducendo i plasmidi nel batterio.
	\end{enumerate}

	\subsection{Cromosomi artificiali batterici \emph{BAC}}
	I \emph{BAC} sono derivati dai plasmidi \emph{F}. 

		\subsubsection{Composizione}
		\begin{multicols}{2}
			\begin{itemize}
    				\item Closing region sequenze dove si inserisce il frammento che deve essere clonato.
				\item \emph{oriS} e \emph{repE} necessari per la replicazione.
				\item \emph{sopA} e \emph{sopB} mantengono il numero di copie/cellula basso.
			\end{itemize}
		\end{multicols}
		I ceppi di batteri utilizzati per il clonaggio con vettori \emph{BAC} sono difettivi dei normali sistemi di restrizione in modo prevenire la degradazione di \emph{BAC} e dei sistemi di ricombinazione in modo da prevenire il riarrangiamento del DNA clonato nei \emph{BAC} al cromosoma dell'ospite.

	\subsection{Cromosomi artificiali di lievito \emph{YAC}}
	Gli \emph{YAC} sono vettori che si replicano nel lievito come cromosomi normali ma presentano dei siti dove pu\`o essere inserito del DNA esogeno di grandi dimensioni (200-800 kb).
	
		\subsubsection{Composizione}
		\begin{multicols}{2}
			\begin{itemize}
				\item Origine di replicazione.
				\item Telomeri.
				\item Centromero.
				\item Sito di clonaggio.
				\item Gene di selezione.
			\end{itemize}
		\end{multicols}
		Presentano problemi notevoli di ricombinazione e riarrangiamento del DNA clonato rispetto al DNA clonato nei batteri.

\section{Sequenziamento del DNA}
Il sequenziamento del DNA pu\`o avvenire per mezzo di due tecniche principali: metodo Sanger o shotgun.

	\subsection{Metodo Sanger}
	Il metodo Sanger \`e il metodo di sequenziamento classico ed \`e stato inventato alla fine degli anni $'70$. 
	Viene chiamato anche metodo dei ``dideossinucleotidi terminali''. 
	Consente di generare frammenti di DNA che terminano in corrispondenza di ognuno delle 4 basi marcare con isotopi radioattivi o marcatori fluorescenti. 
	Il principio \`e basato sulla sintesi di un filamento copia del DNA da studiare con la DNA polimerasi.

		\subsubsection{Miscela di incubazione}
		Per permettere la sintesi della DNA polimerasi marcata con isotopi radioattivi si mette nella miscela di incubaizone analoghi dei deossiribonulceosidi trifosfati \emph{dNTP}.
		$4$ nel caso di isotopi radioattivi o $1$ per i fluorescenti.
		Questi mancano del gruppo ossidrile e impediscono pertanto alla DNA polimerasi di continuare con la sintesi dopo che ha aggiunto un monomero anormale.

		\subsubsection{Processo}
		\begin{multicols}{2}
			\begin{enumerate}
    				\item Fase di clonaggio, seguita dal taglio e l'isolamento di una specifica sequenza.
    				\item Denaturazione a $95\si{\degree C}$, causa l'apertura della doppia elica.
    				\item Attacco di un singolo primer sul sito di taglio specifico.
    				\item La DNA polimerasi si attacca al filamento e comincia la sintesi di uno nuovo. 
					Quando i dideossinucleotidi entrano nel meccanismo, la replicazione si ferma in corrispondenza della base modificata e opportunamente marcata con isotopi radioattivi.
    				\item Si procede con la loro separazione su gel di poliacrilamide, isotopi radioattivi, o in tubo capillare, con i marcatori fluorescenti. 
				\item Elettroforesi: i frammenti si spostano verso il polo positivo: i pi\`u piccoli in modo pi\`u veloce, mentre i pi\`u grandi pi\`u lentamente. 
					Si ricava il sequenziamento andando in ordine e aggiungendo la base marcata con il fuoroforo o il radioattivo.
			\end{enumerate}
		\end{multicols}
		Se viene utilizzato l'isotopo radioattivo \`e necessario avere una piastra per ciascun nucleotidice per capire quale dei quattro \`e stato aggiunto progressivamente. 
		Se vengono utilizzati i marcatori fluorescenti, ogni base verr\`a marcata con un diverso colore e quindi sar\`a necessaria solamente una pista per frammento. 
		In questo secondo caso il sequenziaento pu\`o essere fatto anche automaticamente. 

	\subsection{Sequenziamento shotgun}
	La tecnica shotgun applicata al genoma prevede il clonaggio dell'intero genoma e il sequenziamento casuale dei cloni risultanti. 
	Questa tecnica genera molti frammenti ridondanti o che si sovrappongono parzialmente. 
	L'ordinamento dei frammenti \`e detto assemblaggio. 

		\subsubsection{Assemblaggio}
		Durante l'assemblaggio si collocano tutti i frammenti nel corretto ordine, si eliminano le sovrapposizioni e si genera un genoma utilizzabile per l'annotazione.

		\subsubsection{Annotazione}
		Durante l'annotazione vengono riconosciuti gli \emph{ORF} (Open Reading Frame), sequenze di DNA con schema di lettura aperto, attraverso identificazione di un codone di inizio ($AUG$) o terminazione ($UAA$, $UGA$ o $UAG$). 
		Queste sequenze appaiono anche casualmente ed \`e quindi necessario prendere in considerazione anche la dimensione degli \emph{ORF}

			\paragraph{\emph{ORF}}
			\begin{itemize}
    				\item La maggior parte delle proteine contiene $100$ aminoacidi; 
    				\item Tra start e stop devo avere multipli di tre;
    				\item Si possono ricercare informazioni aggiuntive in geni non codificanti (promotori e terminazione di trascrizione) oppure sequenze di legame al ribosoma.
			\end{itemize}
			Nel DNA che viene annotato:
			\begin{itemize}
    				\item Si considerano i 6 quadri possibili di lettura: 1 per ogni aminoacido delle sequenze ATG e sui due filamenti; 
    				\item la versione che presenta meno sequenze di stop \`e da preferire. 
			\end{itemize}
			Si trovano poi delle regioni pi\`u o meno conservate nei vari individui. 
			Un esempio sono le proteine di membrana che si trovano in tutti i batteri e che presentano particolari domini idrofobici per inserirsi nelle membrane. 

\section{Mappe genomiche}
Le mappe genomiche forniscono informazioni riguardo:
\begin{multicols}{2}
	\begin{itemize}
		\item Ordine: i geni vengono espressi in pacchetti, solitamente $8$-$10$ geni sono espressi insieme in proteine che lavorano nello stesso processo.
		\item Categorie funzionali: vengono presentate con lo stesso colore.
    		\item Lunghezza.
    		\item Orientamento: il gene trascritto \`e espresso solo su un filamento rispetto all'altro.
	\end{itemize}
\end{multicols}

	\subsection{Genoma di Haemophilus influenzae}
	Nella mappa del genoma di Haemophilus influenzae si ha:
	\begin{multicols}{2}
	\begin{itemize}
    		\item cerchio esterno: regioni codificanti.
    		\item primo cerchio interno: regioni ad elevato contenuto di $GC$ o di $AT$.
    		\item secondo cerchio interno: copertura dei cloni usati per il sequenziamento.
    		\item terzo cerchio interno: profagi, tRNA, rRNA.
    		\item quarto cerchio interno: sequenze ripetute (funzione regolatoria) e origine di replicazione.
	\end{itemize}
	\end{multicols}
	Tuttavia la mappa genomica non dice quali e quanti geni sono espressi in un solo momento. 
	L'analisi complessiva dell'insieme dei trascritti (RNA) viene chiamata trascrittomica; in alcuni casi questo porta ad avere genomi uguali ma trascrittomi diversi. 

	\subsection{Contenuto genico e stile di vita}
	Il contenuto genico riflette lo stile di vita dell'organismo. 
	Per esempio:
	\begin{itemize}
	    	\item un parassita obbligatorio come \textit{Treponema pallidum} non possiede geni per la sintesi degli aminoacidi perch\`e vengono tutti forniti dal suo ospite;
	    	\item \textit{E. coli} possiede 131 geni coinvolti nella biosintesi degli aminoacidi;
    		\item \textbf{Bacillus subtilis}, un organismo che vive nel suolo, ha 200 geni coinvolti.
	\end{itemize}

	\subsection{Geni identificati}
	Il numero di geni identificati in un dato genoma, per confronto con altri genomi, corrisponde a circa il 50-60$\%$ delle ORFs individuate.
	Infatti, ci sonno delle ORFs che non vengono identificate come proteine ipotetiche e che probabilmente esistono ma di cui non \`e nota la funzione. 
	Per esempio, in E. coli le funzioni assegnate sono relative a 2700 geni sun un totale di 4300 (63$\%$). 
	Si prevede che la maggior parte delle funzioni codificate dalle ORFs non identificate non siano essenziali e coinvolte in attivit\`a di regolazione, catabolismo di substrati inusuali, proteine ridondanti utilizzate come sistemi di riserva o ridondanza
	Dall'analisi del genoma si possono derivare molte capacit\`a metaboliche: trasportatori ABC per zuccheri, peptidi, fosfato, ferro, zinco; principali rami del metabolismo energetico; sintesi del flaggello; ATP sintasi.

	\subsection{Categorie geniche}
	La percentuale dei geni dedicata a una data funzione cellulare \`e in rapporto alle dimensioni del genoma. 
	La percentuale dei geni dedicati alla replicazione del DNA e alla sintesi proteica \`e alta nei genomi di piccola dimensione, come i parassiti. 
	La percetuale dei geni dedicata al metabolismo e alla regolazione \`e alta nei genomi di grandi dimensioni. 
 	Gli organismi con grandi genomi vivono per la maggior parte nel suolo (quelli con piccoli genomi sono normalmente dei parassiti). 
	Il suolo \`e un habitat nel quale le fonti di carbonio e energia sono scarse, disponibili in una grande variet\`a di tipi differenti e spesso fruibili in maniera intermittente. 

\section{Genomica comparativa}

	\subsection{Differenziazione nei genomi}
	\begin{itemize}
	    \item Genoma core \`e esterno alla membrana ed \`e uguale per tutti i ceppi;
	    \item Buchi sono presenti nello strato del core e rappresentano il genoma accessorio, che varia in un ciascuno dei ceppi di un determinato batterio.
	\end{itemize}
	I vari ceppi hanno quindi un diverso appesto clinico, in cui le regioni intersecanti, dato che comuni a tutti, hanno delle funzioni importanti. 
	
	\subsection{Tipologie di mappatura}

		\subsubsection{Genoma}
		Il genoma \`e la mappatura a livello dell'individuo.

		\subsubsection{Pangenoma}
		Il pangenoma \`e la mappatura a livello della specie.

		\subsubsection{Metagenoma}
		Il metagenoma \`e la mappatura a livello della comunit\`a

	\subsection{Compattezza del genoma procariote}
	Nei procarioti all'aumento delle dimensioni del genoma corrisponde un conseguente aumento del numero dei geni: dimensioni del genoma e il totale di ORF sono quindi direttamente proporzionali. 
	Questo \`e quello che testimonia la compattezza del genoma procariote.
	Il pi\`u piccolo genoma procariotico noto \`e quello di specie del genere \textit{Mycoplasma}: 470 ORFs.
	Confrontando i genomi di due specie di \textit{Mycoplasma}, \textit{M. genitalium} e \textit{M. pneumoniae}, e portando avanti studi di mutagenesi con trasposoni, si \`e concluso che sono necessari circa 300 geni codificanti proteine per stabilire la minima funzionalit\`a cellulare. 
	I pi\`u grandi genomi procariotici sono di oltre 8 Mb, come ad esempio quello di \textit{Bradyrhizobium japonicum}, responsabile della fissazione dell'azoto nei noduli delle radici delle piante di soia, e contiene 8846 ORFs (2800 in pi\`u rispetto a quello del lievito \textit{S. cerevisiae}).

\section{esempi di genomi batterici}
\begin{multicols}{2}
\textit{Mycoplasma genitalium}:
\begin{itemize}
    \item patogeno umano (vie respiratorie, sistema immunitatio), 580 Kb, corredo genico minimo di 517 geni;
    \item 90 coinvolti nella sintesi delle proteine;
    \item 29 nella replicazione del DNA; 
    \item 140 codificano per proteine di membrana; 
    \item 5 geni implicati nei meccanismi di regolazione.
\end{itemize}
\textit{Haemophilus influenzae}:
\begin{itemize}
    \item patogeno umano (vie respiratorie superiori), 1.8
 Mb, 1743 geni;
    \item 40$\%$ con funzione sconosciuta;
    \item 64 geni di regolazione; 
    \item sprovvisto di 3 geni del ciclo di Krebs; 
    \item il genoma contiene 1465 copie della sequenza di riconoscimento usata nell'uptake di DNA durante la trasformazione.
\end{itemize}
\textit{Methanococcus jannaschii}: 
\begin{itemize}
    \item Archaea, 1.6
6 Mb, 1738 geni; 
    \item soltanto il 44$\%$ dei geni corrispondono a quelli degli altri organismi; 
    \item geni per funzioni  essenziali (replicazione, trascrizione, traduzione);
    \item simili a quelli degli eucarioti.
\end{itemize}
\textit{Escherichia coli}: 
\begin{itemize}
    \item 4.6
 Mb, 4288 geni; molto simile a H. 
influenzae;
    \item 5$\%$ dei geni per proteine di membrana, 13$\%$ trasporto, 10$\%$ metabolismo, 4$\%$ regolazione, 8$\%$ per replicazione, trascrizione, traduzione;
    \item 2500 geni dissimili da geni noti.
\end{itemize}
\textit{Deinococcus radiodurans}:
\begin{itemize}
    \item batteri del suolo. 
Sono in grado di ricongiungere frammenti di DNA generati dall'esposizione a forti radiazioni. 
2 cromosomi, 2.6
 Mb e 0.4
 Mb;
    \item un megaplasmide 177 Kb, un plasmide 45 Kb;
    \item il batterio dispone di maggior quantit\`a di geni impegnati in processi di riparazione del DNA. 
Esempio MmutT (eliminazione dei nucleotidi ossidati) \`e presente in 20 versioni (1 sola nella maggior parte dei microrganismi).
\end{itemize}
\textit{Rickettsia prowazekii}:
\begin{itemize}
    \item parassita endocellulare obbligato dei pidocchi e dell'uomo, agente del tifo epidemico; \item 1.1
 Mb (25$\%$ non codificante), geni con affinit\`a a quelli mitocondriali. 
Processo di sintesi dell'ATP simile a quello osservato dal mitocondrio. 
Mancanza di geni dedicati alla sintesi di diversi aa (come nel mitocondrio).
\end{itemize}
\textit{}{Chamydia trachomatis}:
\begin{itemize}
    \item batteri privi di motilit\`a, parassiti intracellulari. 
Privo di peptidoglicano, ma possiede tutti i geni per costruirlo;
    \item non ha il gene FtsZ (formazione del setto divisorio), meccanismo molecolare di dividione cellulare sconosciuto;
    \item contiene pi\`u di 20 geni di origine eucariotica di cui alcuni provenienti da piante.
\end{itemize}
\textit{Treponema pallidum}:
\begin{itemize}
    \item agente della sifilide. 
\`E metabolicamente deficitario: manca del ciclo di Krebs e della fosforilazione ossidativa e di diverse vie di biosintesi; 
    \item 5$\%$ dei geni codificano per proteine di trasporto; 
    \item funzione do 40$\%$ dei geni sconosciuta. 

\end{itemize}
\textit{Mycobacterium tuberculosis}:
\begin{itemize}
    \item agente della tuberculosi, 4.4
 Mb; 
    \item 4000 geni, 60$\%$ sconosciuti. 
250 geni per il metabolismo dei lipidi (50 in E. 
coli). 
Il batterio ottiene molta energia dalla degradazione dei lipidi dell'ospite; 
    \item 10$\%$ del genoma formato da 2 famiglie di proteine che potrebbero conferire variabilit\`a antigenica e quindi un meccanismo di difesa contro il sistema immunitario dell'ospite.
\end{itemize}
\textit{Mycobacterium leprae}:
\begin{itemize}
    \item agente della lebbra, genoma molto diverso di quello di \textit{M. 
tubercolosis};
    \item 50$\%$ del genoma da geni non funzionali. 
Privo di enzimi coinvolti nella produzione di energia e nella replicazione del DNA (tempo di replicazione nel topo, circa 2 settimane).
\end{itemize}
\textit{Staphylococcus aereus}:
\begin{itemize}
    \item agente di varie infezioni come le intossicazioni alimentari o infezioni nosocomiali; 
    \item 2.6
 Mb, 2600 geni. 
Possiede molti geni di resistenza; 
    \item agli antibiotici, alcuni collocati su plasmidi o trasposoni.
\end{itemize}
\textit{Streptococcus pyogenes}:
\begin{itemize}
    \item tre principali ceppi in grado di causare diversi tipi di infezioni; 
    \item i tre ceppi differiscono principalmente per il contenuto dei profagi, in cui sono ospitati i geni che codifcano per fattori di virulenza.
\end{itemize}
\end{multicols}
