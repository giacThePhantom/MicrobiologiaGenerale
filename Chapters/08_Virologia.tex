\chapter{Virologia}
Un virione \`e una particella virale completa, funzionale e in grado di infettare. Consiste di almeno una molecola di DNA o RNA racchiusa in un rivestimento proteico, detto capside. Pu\`o presentare anche degli strati addizionali. \`E un parassita endosimbionte obbligato che non pu\`o riprodursi indipendentemente dalle cellule viventi e che non pu\`o duplicarsi in assenza di un organismo ospite come fanno eucarioti e procarioti. 
\\Quindi, i virus sono degli elementi genetici che si replicano indipendentemente dai cromosomi cellulari, ma non dalle cellule stesse. Al contrario di alcuni elementi genetici, come i plasmidi, possiedono una forma extracellulare che ne permette la persistenza al di fuori dell'ospite.
\section{Struttura dei virus}
La dimensione del virione \`e compresa fra 10-400 nm in diametro, ma la maggior parte del virus \`e troppo piccola per essere visibile al microscopio ottico. Tutti i virioni: 
\begin{itemize}
    \item nucleocapside composto da acido nucleico (DNA o RNA);
    \item rivestimento proteico (capside);
    \item componenti addizionali, come per esempio l'envelope (rivestimento lipidico).
\end{itemize}
Si distinguono due tipi di virus: 
\begin{itemize}
    \item nudi $\xrightarrow{}$ presentano solo l'involucro proteico del capside;
    \item rivestiti $\xrightarrow{}$ envelope o ivolucro percapsidico, con membrana formata da un doppio strato lipidico al quale sono asscoiate delle proteine. La componente lipidica deriva dalla membrana dell'ospite, mentre la componente proteica \`e codificata dak virus.
\end{itemize}
\subsection{Capside e simmetria}
Una prima classificazione viene fatta sulla struttura base del capside:
\begin{itemize}
    \item Isocaedrico;
    \item Isocaedrico con envelope;
    \item Elicoidale senza envelope;
    \item Elicoidale con envelope.
\end{itemize}
La capside \`e una struttura macromolecolare che serve da rivestimento proteico del virus. Protegge il materiale genetico virale e aiuta il suo trasferimento fra le cellule ospiti.
\\\`E formato da delle unit\`a morfologiche dette capsomeri. Questi sono formati a loro volta da delle subunit\`a strutturali proteiche dette protomeri. I protomeri hanno la capacit\`a di autoassemblarsi spontaneamente, combinandosi in strutture elicoidali o icosaedriche. I capsidi elicoidali hanno forma di un tubo vuoto con parete proteica.
\\L'acido nucleico si trova all'interno della particella, circondato dal capside. 
\\Si riconoscono due differenti simmetrie a cui corrispondono le due forme principali:
\begin{enumerate}
    \item bastoncellari a simmetria elicoidale, come per esempio il \textit{Virus del Mosaico del Tabacco}. Il capside a elica forma un tubo vuoto con parete proteica. 
    \item sferoidali a simmetria isocaedrica, composti da 20 facce o capsomeri. Un esempio \`e il virus dell'influenza che \`e rivestito da un nucleocapside a elica. \`E poliequivalente perch\`e il suo genoma \`e presente su pi\`u nucleocapsidi diversi. Oltre a questo le proteine sono distribuite a intervalli regolari sula suerficie. ($\xrightarrow{}$ nucleocapside+involucro lipidico$\xrightarrow{}$ simmetria marcata).
\end{enumerate}
La simmetria isocaedrica rappresenta la disposizione pi\`u efficiente delle subunit\`a nella formazione della capside. Il capside asssomiglia ad una sfera ed \`e un solido in cui la superficie \`e minima con von volume massimo ($\xrightarrow{}$ il rapporto tra volume e superficie \`e massimo). In questo modo si minimizza il numero di subunit\`a necessarie alla sua costruzione.
\\La disposizione pi\`u semplice \`e di 3 protomeri per capsomero, per un totale di 60 protomeri per virione. Le altre configurazioni conosciute sono 180, 240, 360 e 420 unit\`a.  
\\Il capside isocaedrico \`e normalmente composto da: \begin{itemize}
    \item P = pentoni, in corrispondenza dei vertici
    circondati da 5 esoni;
    \item H = esoni, formano i lati e le facce dell'icosaedrico. 
\end{itemize}
In totale, quindi, si hanno 42 capsomeri con un solo tipo di protomeri.
\\Questi tipi di capsidi hanno la capacit\`a di autoassemblaggio: si assemblano in maniera autonoma, non hanno bisogno della presenza di fattori cellulari che controllino o regolino la loro formazione. Proprio per questo motivo \`e possibile realizzarli in vitro.
\subsection{Virus con capsidi a simmetria complessa}
Alcuni virus non ricadono nelle categoria con capsidi ad elica o isocaedrici. Degli esempi sono i poxivirus ed i grandi batteriofagi. 
\begin{itemize}
    \item i Poxvirus sono formati da un core centrale biconcavo che contiene i genomi e numerosi enzimi virali, due corpi laterali e un doppio involucro di membrana; 
    \item il Batteriofago T4 presenta una simmetria binaria, con combinazione di simmetria isocaerica (testa, capside con acido nucleico) ed elicoidale (guaina). Presenta anche una piastra basale esagonale e delle fibre caudali, con cui interagisce con la cellula ospite e inietta il suo genoma.
\end{itemize}
Alcuni virus presentano anche l'envelope: un doppio strato fosfolipidico esterno. Questa membrana viene acquisita dal virus durante il processo di uscita della cellula ospite, detto budding (gemmazione). Mentre i virus senza envelope abbandonano la cellula lisandola, il processo di budding non altera l'integrit\`a cellulare.
\begin{itemize}
    \item Il genoma viene replicato nella cellula ospite; 
    \item Vengono create le proteine strutturali; 
    \item Le proteine virali con parte idrofobica vengono espresse. Questo permette di inserirsi nella membrana della cellula ospite.
    \item Le glicoproteien sulla membrana dell'ospite crescono di numero man mano che aumentano le proteine virali della matrice;
    \item Dal nucleocapside intracellulare si forma un virione libero. La membrana si richiude intorno al nucleo del capside.
\end{itemize}
\subsection{Genomi virali}
I genomi virali hanno dimensioni ridotte e codificano quelle funzioni che non possono essere fornite dai loro ospiti. Il virus riprogramma le funzioni metaboliche e biosintetiche dell'ospite finalizzandole alla sua replicazione e all'assemblaggio di nuovi virioni. Le particelle virali hanno dimensioni da 20 a 300 nm; mentre i genomi virali sono compresi tra 500 e 5000 kb (a volte possono essere pi\`u grandi di alcuni batteri).
\section{Tassonomia virale}
La mancanza d'informazione sull'origine e sulla storia evolutiva rende la classificazione virale difficile. Mentre per gli eucarioti si \`e riuscito a costituire un sistema tassonomico su specie, genere, famiglia, ordine, classe e fila, e pi\`u recentemente anche su sinapormofie e simplesiomorfie, per i virus questo non \`e possibile.
\\Nel 1971 l'International Committee for Taxonomy of Viruses (ICTV) ha sviluppato un sistema di classificazione uniforme che descrive circa 2000 virus. Questa classificazione si basa sulla natura del genoma, sulla simmetria del capside, sulla presenza o assenza di envelope e sulle dimensioni del virione e del capside. Per la classificazione dei virus esiste un database chiamato ICTVdB.
\\I virus dimostrano un'assenza di autonomia replicativa e da questo si deduce che siano comparsi dopo gli altri gruppi. Infatti essi non presentano una replicazione sessuata o asessuata e per questo motivo non danno luogo a una linea evolutiva classica. Non esiste, quindi, un albero filogenetico dei virus, come invece c'\`e per gli altri esseri viventi. 
\\Si \`e deciso cos\`i di stilare la classificazione di Baltimore che si occupa dell'espressione dell'informazione genetica. 
\begin{enumerate}
    \item \textbf{Classe I} $\xrightarrow{}$ \`e la classe di virus pi\`u convenzionale perch\`e le sue caratteristiche sono simili a quelle dell'ospite: presentano un DNA a doppio filamento. 
    \\Si ha la trascrizione del filamento negativo per produrre un mRNA positivo, che porta quindi alla traduzione grazie a ribosomi e alla produzione di proteine funzionali. Dato che il DNA \`e standard (a doppio filamento e con direzioni opposte, pu\`o replicarsi grazie ai meccanismi della cellula ospite. Questo tipo di classe non richiede alcun tipo di modifica.
    \\Degli esempi sono: \textit{Adenoviridae, Herpesviridae, Papillomaviridae, Polyomaviridae, Poxviridae}.
    \item \textbf{Classe II} $\xrightarrow{}$ questi virus presentano un DNA a singolo filamento. C'\`e la necessit\`a di un filamento complementare, costruzione di un intermedio di DNA a doppio filamento, prima della trascrizione. \\Per prima cosa avviene la sintesi di un filamento di DNA complementare, poi si ottiene un DNA a doppio filamento standard. 
    \\Si pu\`o procedere con la trascrizione come avviene nella Classe 1. Si utilizza uno dei due filamenti stampo per produrre un mRNA che possa poi essere tradotto in proteine strutturali grazie a enzimi. Per la generazione di nuove unit\`a genomiche l'acido nucleico viene processato in modo da ottenere la separazione della doppia elica che porta a un unico filamento (DNA+ o DNA-).
    \\Per esempio: \textit{Circoviridae, Parvoviridae}.
    \item \textbf{Classe III} $\xrightarrow{}$ questo tipo di virus presenta un RNA a doppio filamento. La cellula ospite non sa produrre altro RNA da RNA a doppio filamento. 
    \\Il filamento negativo viene trascritto e porta a sintesi di proteine funzionali con l'aiuto degli enzimi. Vengono introdotti nuovi enzimi virali, che non sono presenti nella cellula ospite: RNA polimerasi RNA-dipendente (= RNA replicasi) codificata dal genoma virale. Questa ha la capacit\`a di leggere il filamento di RNA ed \`e in grado di ricostruire il genoma originale.
    \\Per esempio: \textit{Reoviridae}.
    \item \textbf{Classe IV} $\xrightarrow{}$ questo tipo presenta un RNA a singolo filamento positivo che pu\`o essere usato direttamente come mRNA per la sintesi di proteine strutturali dato che la cellula ospite \`e in grado di riconoscerlo. Per la codifica del nuovo virale si necessita di RNA polimerasi RNA-dipendente (= RNA replicasi). Con questo metodo si pu\`o produrre RNA positivo a partire da uno stampo, dato che non \`e possibile replicarlo direttamente. 
    \\Per esempio: \textit{Picornaviridae, Togaviridae, Flaviviridae, Coronaviridae}.
    \item \textbf{Classe V} $\xrightarrow{}$ questi virus presentano un RNA a singolo filamento negativo, che viene trascritto. Si necessita di un enzima virale RNA replicasi sia: 
    \begin{itemize}
        \item per la trascrizione perch\`e non c'\`e la polarit\`a giusta. Una volta che il lo stampo di RNA positivo viene prodotto si possono codificare nuove proteine; 
        \item per la replicazione non \`e possile produrre un nuovo filamento direttamente ma ho bisogno di sintetizzare uno stamp di RNA positivo per produrne delle copie. 
        \\Per esempio: \textit{Orthomyxoviridae, Paramyxoviridae, Rhabdoviridae}
        \item \textbf{Classe VI} $\xrightarrow{}$ sono dei retrovirus a RNA. Presentano un singolo filamento di RNA positivo (simile alle classe 4). Necessit\`a di una trascrittasi inversa codificata dal genoma virale. In questo modo si ha: 
        \begin{itemize}
            \item RNA positivo; 
            \item Intermedio a DNA a singolo filamento (DNA-); 
            \item Intermedio a DNA a doppio filamento; 
            \item Trascrizione del filamento DNA-, in modo da ottenere mRNA+ per sintesi di proteine strutturali e RNA+ per la replicazione del genoma virale per virioni successivi.
        \end{itemize}
        Per esempio: \textit{HIV, Retroviridae}.
    \end{itemize}
    \item \textbf{Classe 7} $\xrightarrow{}$ presenta un DNA a doppio filamento.
    \begin{itemize}
        \item Costruzione di un intermedio a RNA+; 
        \item Attraverso la trascrittasi inversa ottengo DNA- come versione complementare; 
        \item Cos\`i pu\`o essere generato un DNA a doppio filamento come genoma per le popolazioni successive.
    \end{itemize}
    Per esempio: \textit{Hepadnaviridae}.
\end{enumerate}
\section{Ciclo replicativo di un virus animale}
In una cellula osite animale il virus deve adottare delle strategie per produrre proteine e nuove copie del genoma. Il ciclo replicativo virale si pu\`o dividere in 6 fasi:
\begin{enumerate}
    \item \textbf{Adsorbimento o attacco} del virione ad una cellula ospite. 
    \\Avviene il contatto tra il virione e la membrana. La specificit\`a tra il virus e l'ospite \`e molto forte ed \`e mediata dal contatto tra superficie nuda o envelope e recettori della membrana citoplasmatica. Se il virus non trova determinati recettori, il ciclo replicativo non avviene.
    \item \textbf{Penetrazione e decapsidazione}. L'attacco di un virus alla cellula opsite provoca delle modificazioni nel virus e nella superficie cellulare che portano alla penetrazione. Questa entrata assomiglia ad una endocitosi. 
    \\La cellula che permette lo solgimento di un intero ciclo replicativo di un virus \`e definita permissiva per quel determi virus. Durante la fase di penetrazione avviene anche la spoilazione, cio\`e il processo attraverso cui i virioni perdono il loro rivestimento esterno e il genoma virale viene cos\`i esposto all'ambiente cellulare.
    \item \textbf{Espressione genica e sintesi delle proteine vitali}. Per i virus, soprattutto per quelli che entrano in cellule non permissive, \`e fondamentale l'espressione genica per produrre le proteine. Queste possono essere: 
    \begin{itemize}
        \item Strutturali, cio\`e vanno a formare il capside per consentire la formazione di nuovi virioni; 
        \item Non strutturali, che sono quindi necessari per la replicazione del genoma. 
    \end{itemize}
    \item \textbf{Replicazione del genoma virale}.
    \item \textbf{Assemblaggio e maturazione dei virioni}. In particolar modo nei virus con envelope, il processo di maturazione comprende anche il passaggio attraverso organuli come il reticolo endoplasmatico e l'apparato di Gogli per produrre delle glicoproteine presenti sull'envelope stesso.
    \item \textbf{Rilascio e fuoriuscita della cellula}. Se il virus presenta l'envelope si parla di gemmazione. Viene cos\`i chiamato perch\` nel mommento dell'uscita del virus viene circondato da parte della membrana doppio strato fosfolipidico della cellula ospite.
\end{enumerate}
\subsection{Herpes simplex, classe I}
La replicazione del genoma virale avviene nel nucleo. Il genoma viene trascritto come vari mRNA, che vengono poi esportati nel citoplasma per essere tradotti. Questa espressione genica virale avviene in tre fasi: 
\begin{enumerate}
    \item Immediate-Early (Immediata precoce): porta alla regolazione delle fasi successive; 
    \item Early (precoce): produzione degli enzimi necessari alla replicazione del genoma virale. Ci\`o implica che la proteina venga prodotta nel citoplasma, ma rientra in seguito nel nucleo per controllare la trascrizione e la replicazione;
    \item Late (tardiva): sintesi delle proteine strutturali.
\end{enumerate}
Dopo queste fasi possono seguire dei processi di concatenamento in cui molte unit\`a di genoma vengono messe insieme in un'unica molecola. La maturazione del virione continua con il passaggio nel reticolo endoplasmatico e nel Golgi, per portare all'aggiunta di numerosi altre proteine sulla superficie (es. envelope). Infine si arriva all'espulsione del virione dalla cellula ospite (gemmazione).
\subsection{Poxvirus (vaiolo), classe I}
La maggior parte dei processi avviene nel citoplasma: sia il rilascio del nucleocapside e sia la replicazione del genoma virale. 
\\Il virus deve fornire gli enzimi necessari per la replicazione del DNA e la trascrizione dell'RNA visto che la cellula ospite non \`e in grado di soddisfare questi compiti al di fuori del nucleo. Poi c'\`e un passaggio per il reticolo endoplasmatico e per l'apparato di Golgi che porta ad aggiungere dei nuovi strati endoplasmatici e proteine al virione maturo (MV). Alla fine si arriva allo stato di virione extracellulare (EV) che si distingue dal precedente MV per la tipologia dell'involucro esterno, che questa volta deriva dalla membrana plasmatica.
\subsection{Picornavirus (poliovirus), classe IV}
Questo virus presenta un genoma ad RNA positivo, che funge da mRNA e quinsi pu\`o venire tradotto direttamente. Non \`e protetto da envelope e il suo nucleocapside non entra nella cellula durante la penetrazione.
\\Viene prodotta una poliproteina che viene poi scissa da delle protesasi per la produzione delle proteine strutturali. 
\\La RNA polimerasi RNA-dipendente sintetizza l'RNA a polarit\`a negativa che viene utilizzato da stampo per il nuovo RNA genomico positivo. 
\subsection{Virus dell'influenza umana di tipo A, classe V}
Presenta un genoma a RNA negativo e segmentato. Questo significa che \`e composto da un insieme di frammenti con le informazioni necessarie per compiere un ciclo replicativo. 
\\La proteina HA (emoagglutinina) lega recettori sulla superficie cellulare (contatto con la membrana):
\begin{itemize}
    \item Introduzione e rilascio del nucleocapside; 
    \item Trascrizione e replicazione del genoma all'interno del nucleo; 
    \item Traduzione nel citoplasma.
\end{itemize}
Alcune proteine tornano nel nucleo per l'assemblaggio del nucleocapside. Poi le proteine dell'enevelope inserite nella membrana del reticolo endoplasmatico e trasportate poi dal Golgi. 
\\Infine avviene la gemmazione.
\\Per il virus dell'influenza non esiste un vaccino che rimane attivo e valido per molti anni. Questo avviene perche\`e:
\begin{itemize}
    \item a causa del suo genoma segmetato muta molto velocemente; 
    \item non presenta una specificit\`a particolarmente elevata per l'ospite;
    \item va incontro alla deriva antigenica, piccole variazioni (mutazioni puntiformi) che determinano epidemie influenzali ogni 2-3 anni; 
    \item riassortimento antigenico o shift. Questo pu\`o essere alla base di pandemie, ogni 10-40 anni, per comparsa di nuovi tipi di HA e NA. Il riassortimento \`e dovuto all'infezione spontanea in un ospite permissivo di ceppi virali provenienti da ospiti diversi. In questo modo si ottengono strutture ibride; 
    \item saltano da una specie all'altra. 
\end{itemize}
\subsection{Retrovirus animale, classe VI}
L'HIV \`e il retrovirus pi\`u conosiuto globalmente. Questo virus colpisce il sistema immunitario dell'ospite fino a portarlo alla morte. \`E composto da un genoma a RNA negativo a singolo filamento. Per questo motivo necessita di alcuni enzimi propri per poter portare a compimento il suo ciclo replicativo: 
\begin{itemize}
    \item Proteasi PR; 
    \item Integrasi IN; 
    \item Trascrittasi inversa RT.
\end{itemize}
Dopo l'attacco avviene la fusione dell'envelope con la membrana cellulare, che viene seguita dal rilascio immediato del nucleocapside. Il suo genoma \`e organizzato in tre regioni principali:
\begin{itemize}
    \item Gag che codifica per proteine strutturali; 
    \item Pol che codifica per enzimi virali; 
    \item Env che codifica per le glicoproteine del capside. 
\end{itemize}
Per prima cosa si ha la sintesi di DNA/RNA e quindi una doppia elica di DNA da parte della trascrittasi inversa. Una copia di questo doppio filamento viene integrato nel cromosoma dell'ospite e viene chiamato provirus. In questo modo il DNA virale diventa leggibile dall'apparato di lettura delle cellula ospite. Ora avvengono l'espressione del genoma virale e la sintesi delle proteine virali. Le proteine che vengono tradotte possono essere: dell'enevelope, funzionali o strutturali. 
\\Grazie alla proteasi si ha la maturazione del genoma, l'assemblaggio dei virioni, la traslocazione della proteina di superficie del reticolo endoplasmatico al Golgi e per finire alla membrana plasmatica. Alla fine avviene la gemmazione. 
\section{I Batteriofagi}
Per la quantificazione della crescita dei batteriofagi si può eseguire il seguente esperimento:
\begin{itemize}
    \item Diluzione di una sospesione contenente materiale virale viene mescolato in una piccola mistura di agar con un ospite suscettibile. Questa viene messe su una piastra di nutrient agar; 
    \item Il batterio ospite inizia a crescere sulla superficie della piastra di agar e, dopo un'incubazione overnight, forma un tappeto di crescita batterica; 
    \item Allo strato precedente viene aggiunta la componente virale. Ogni particella virale che si attacca a una cellula e si riproduce, causa la lisi della cellula stessa e le particelle virali rilasciate possono diffondere alle cellule adicenti, infettarle, portare di nuovo alla lisi con ciclo litico e venire rilasciate nuovamente;
    \item Si formano dei buchi nella piastra in corrispondenza delle colonie lisate, chiamate placche. La dimensione delle placche dipende dal virus, dall'ospite e dalle condizioni della coltura.
\end{itemize}
Contando il numero di placche si può calcolare il numero delle unità infettive virali presenti nel campione di partenza. 
\\Il titolo, ossia la concentrazione del virus, viene espresso in PFU (plaque forming unit) per ml. Questo si calcola:
\begin{center}
    $\frac{n. placche}{V x diluizione}$
\end{center}
\section{Replicazione dei batteriofagi}
Nel ciclo replicativo dei batteriofagi si distinguono cinque fasi principali:
\begin{enumerate}
    \item Adsorbimento o fase di attacco; 
    \item Penetrazione, in cui avviene l'iniziezione del materiale genetico; 
    \item Sintesi dell'acido nucleico e di proteine strutturali; 
    \item Autoassemblaggio dei capsomeri e impacchettamento; 
    \item Rilascio di virioni maturi. 
\end{enumerate}
Il tempo totale di replicazione varia dai 20 ai 60 minuti per i batteriofagi (T4, 20-22 minuti), fino ad arrivare dalle 8 alle 40 ore per i virus animali. 
\\Si possono così tracciare dei graifici in particolare sul ciclo litico che riportano sull'asse delle ascisse il tempo, mentre sull'asse delle ordinate le unità formanti placca (il numero di placche che posso vedere) e l'OD della popolazione batterica sottoposta ad attacco da parte dei batteri.
\\Avvenuto il contatto, il numero di cellule virali cala di un ordine di grandezza. Questo numero rappresenta i fagi che riescono a infettare le cellule batteriche: 
\begin{itemize}
    \item \textbf{Fase di latenza} $\xrightarrow{}$ i fagi entrano nelle cellule e le infettano, ma non accade nulla, All'inzio di questa fase si ha il periodo di eclissi. Il genoma virale è separato dal capside pproteico. Durante questo stadio il virione non viene considerato come una particella infettiva.
    \item \textbf{Maturazione} $\xrightarrow{}$ ha inizion non appena i genomi neo-sintetizzati vengono impacchettati all'iterno dei capsidi. Nella fase di scoppio si ha un brusco aumento nel numero di batteriofagi (2 ordini di grandezza), associato con una drastica diminuzione dei batteri presenti. Questo corrisponde al momento di lisi delle cellule e al rilascio dei fagi dell'ambiente extracellulare.
\end{itemize}
\subsection{Batteriofago T4}
Il ciclo del batteriofago T4 è molto veloce:
\begin{itemize}
    \item 1 minuto dopo l'attacco la sintesi del DNA e del RNA dell'ospite cessano e inizia la trascrizione di alcuni geni fagici;
    \item dopo 2 minuti vengono sintetizzare le prime proteine virali;
    \item dopo 4 minuti inizia la replicazione del DNA virale;
    \item l'assemblaggio di nuovi virioni iniza dopo 20 minuti; 
\end{itemize}
T4 utilizza la RNA polimerasi dell'ospite. Questo enzima riconosce preferenzialmente i promotori virali. La trscrizione nel batterio viene soppressa da una proteina virale che lega il fattore sigma dell'ospite. 
\\Ulteriori modifiche della RNA polimerasi batterica portano alla trascrizione dei "middle" e "late" RNA. Le ultime proteine ad essere sintetizzate sono quelle strutturali. 
\\La fuoriuscita avviene grazie alla lisi della cellula provocata dalla produzione di un enzima virale che attacca il peptidoglicano dell'ospite. 
\subsubsection{Fase di adsorbimento}
L'inteazione virus-ospite è altamente specifica. Questa specificità viene data dall'interazione delle proteine di superficie del virus con componenti della membrana dell'sopite (recettori come proteine, carboidrati, lipidi). I recettori svolgono delle funzioni normali per la cellula. Per esempio il recettore per il batteriofago T1 è una proteina per l'assorbimento del ferro. 
\\In assenza di uno specifico recettore, il virus non è in grado di adsorbirsi; possono però insorgere forme mutanti del virus i grado di infettare un ospite resistente.
\subsubsection{Fase di iniezione}
Una cellula che permette la penetrazione e la replicazione di un virus viene detta permissiva. Le particelle lvirali si attaccano alla cellule attraverso le fibre caudali che interagiscono in maniera specifica con i polisaccaridi della membrana esterna. Queste fibre si contraggono e la parte centrale della coda entra in contatto con la parete del batterio attraverso una serie di spine situate all'estremità della coda. 
\\L'azione di un enzima virale determina la formazione di un poro di peptidoglicano. Dopo la contrazione della guaina caudale, il DNA virale viene spinto all'interno della cellula.
\subsubsection{Restrizione e modificazione del virus da parte dell'ospite}
I procarioti si possono difendere dall'infezione virale con un meccanismo basato sulla distruzione del DNA genomico virale a doppio filamento, operata da endonucleasi di restrizione. Questo fenomeno, detto restrizione, non è efficiente contro i virus a RNA o a DNA a singolo filamento. 
\\Il DNA dell'ospite deve essere protetto dall'azione dei propri enzimi di restrizione attraverso un processo di modificazione che coinvolge la metilazione di basi specifiche sulla sequenza di riconoscimento. 
\\Alcuni virus a DNA possono superare i meccanismi di restrizione modificando il proprio genoma, metilazioni e glicosilazioni, e rendendolo resistente all'attacco enzimatico.
\\Questo fenomeno di restrizione non è efficace contro i virus a RNA o DNA a singolo filamento. 
\\\\Il DNA di T4 presenta una base insolita, la 5-idrossimetilcitosina al posto della citosina. I gruppi ossidrilici sono modificati dall'aggiunta di residui di glucosio rendendo il DNA resistente agli enzimi di restrizione della cellula ospite. 
\\Il batteriofago T4 presenta:
\begin{itemize}
    \item testa isocaedrca, 85 x 110 nm; 
    \item coda con struttura tubolare elicoidale 25 x 110 nm; 
    \item più di 25 tipi di proteine strutturali; 
    \item DNA lineare a doppio filamento, 168903 bp, che codifica più di 250 proteine diverse; 
    \item DNA premutato circolarmente; 
    \item DNA con ripetizioni terminali di 3-6 kb.
\end{itemize}
\subsection{Permutazione circolare}
Il DNA di T4 viene replicato, prima come singole unità. Poi le diverse unità genomiche vengono legate tra di loro per formare un'unica molecola denominata concatenamero. 
\\L'impacchettamento del DNA nella testa dei virioni necessita il taglio della molecole, indipendentemente dalla sequenza nucleotidica. La testa di T4 può ospitare una molecola di DNA leggermente più lunga di un intero genoma e si ha la comparsa di terminazioni ridondanti. 
\\Il genoma viene letto e replicato più volte e le estremità si possono combinare tra loro per omologia delle basi. Poi viene tagliato in più punti per produrre il genoma virale. I frammenti tagliati hanno alle estremità sequenze ripetute ma diverse tra loro.
\subsection{Batteriofagi temperati}
Nel ciclo lisogenico un gene codificato dal profago gioca il ruolo di repressore dell'espressione del genoma virale. Questo repressore previene anche l'espressione di qualsiasi altro genoma virale introdotto nella stessa cellul: conferisce ai batteri losogenizzati uno stato di immunità.
\subsection{Il batteriofago lambda}
Il batteriofago lambda può intraprendere sia la via litica che la via lisogenica. Non possiede fibre caudali e presenta un genoma di DNA lineare a doppio filamento con una coda di 12 nucleotidi a singola elica (regione cos). 
\\All'interno della cellula batterica queste sequenza terminali complementari si associano e il DNA è rilegato in una forma circolare contenente 48502 bp. Il genoma di questo fago presenta:
\begin{itemize}
    \item Regioni che codificano per le varie funzioni; 
    \item Possibilità di essere letto in entrambe le direzioni:
    \begin{itemize}
        \item Lettura in senso orarrio (Right T) controlla le funzioni del ciclo litico; 
        \item Lettura in senso antiorario (Left T) controlla le funzioni del ciclo litogenico.
    \end{itemize}
\end{itemize}
Sono presenti due proteine che hanno un ruolo chiave:    
\begin{itemize}
    \item \textbf{Repressore di lambda} $\xrightarrow{}$ è prodotto dal gene \textit{cI} e ha il compito di bloccare la trascrizione del gene \textit{cro} e degli altri geni necessari per il ciclo litico; 
    \item \textbf{Proteina Cro} $\xrightarrow{}$ prodotta dal gene \textit{cro} e serve per inibire la trascrizione del gene del repressore lambda. 
\end{itemize}
Il processo tramite il quale avviene l'infezione con il fago $\lambda$:
\begin{itemize}
    \item il virione di lambda si attacca a uno specifico recettore della membrana esterna di \textit{E. coli} e inietta il proprio DNA all'interno della cellula;
    \item il DNA circolarizza e ha inizio l'espressione del genoma virale; 
    \item la RNA polimerasi dell'ospite comincia a trascrivere a partire da 2 promotori P$_L$ e P$_R$. Verranno prodotti i trascritti L1 e R1. Questo è dovuto al fatto che le sequenza non sono palidromiche: si può trascrivere in una sola direzione, o su un filamento o sull'altro;
    \item Vengono prodotti 2 mRNA tradotti nelle proteine Cro e N, coinvolte in processi di regolazione:
    \begin{itemize}
        \item Cro definisce l'entrata nella via litica o lisogenica;
        \item N è un antiterminatore, consente alla RNA polimerasi di continuare la trascrizione oltre ai terminatori, allungando i trascitti a partire da P$_L$ e P$_R$, in particolare i geni impegnati nella replicazione del DNA.
    \end{itemize}
    \item i geni tardivi vengono attivati dopo la sintesi della proteina Q che attiverà un ultriore evento di trascrizione dei geni codificanti le proteine strutturali (trascritto R2); 
    \item una volta che la proteina Cro raggiunge una certa concentrazione lega gli operatori O$_R$ e O$_L$, bloccando la trascrizione guidata dai promotori P$_R$ e P$_L$. Ha la funzione di repressore insieme al prodotto del gene \textit{cI}.
\end{itemize}
I prodotti dei geni \textit{cro} e \textit{cI} determinano l'entrata nella via litica o lisogenica:
\begin{itemize}
    \item Ordine di legame cI: sito 1, , 3 $\xrightarrow{}$ via lisogenica (geni \textit{cII}, \textit{cIII} attivi);
    \item Ordine di legame di Cro: sito 3, 2, 1 $\xrightarrow{}$ via litica (geni \textit{cII}, \textit{cIII} inattivi).
\end{itemize}
\subsubsection{Via litica o lisogenica: l'interruttore genetico}
Per entrare nella via lisogenica devono verificarsi due condizioni:
\begin{enumerate}
    \item deve essere inibita la produzione di proteine tardive;
    \item una copia del genoma virale deve essere inserita nel cromosoma dell'ospite.
\end{enumerate}
L'inibizione delle proteine tardive avviene dopo l'espressione del gene \textit{Cl}, il repressore di lambda. \textit{Cl} viene trascritto con l'attivazione di un promotore P$_E$, che viene attivato dal gene \textit{cII}. Anche se la proteina viene sintetizza precocemente dopo l'infezione diventa poco stabile e quindi necessita della presenza di \textit{cIII}.
Entrambi \textit{cI} e Cr controllano l'espressione del proprio gene reprimendo o attivando i promotori P$_M$ e P$_R$ rispettivamente. 
\\Cro inibisce la trascrizione di \textit{cI} e stimola la trascrizione rightward; mentre \textit{cI} inibisce la trascrizione di cro e stimola la trascrizione leftward. 
\subsubsection{Schema riassuntivo delle fasi dell'infezione di lambda}
\begin{itemize}
    \item Attivazione dei promotori P$_L$ e P$_R$;
    \item Trascrizione di L$_1$ e R$_1$; 
    \item L'antiterminatore N consente di allungare i trascritti oltre ai terminatori. Inizia la replicazione del DNA virale;
    \item Trascrizione di Q. Attivazione della trascrizione dei geni tardivi R2 e inizio della via litica; 
    \item Trascrizione di cII e cIII, che sono entrambi necessari alla trascrizione di cI. cIII stabillizza cII, che a sua volta attiva P$_E$ che controlla la trascrizione di cI. 
\end{itemize}
\subsubsection{Integrazione di lambda}
Avviene in un singolo sito specifico del cromosoma batterico. Necessita l'attivazione dei geni cI e int (integrasi, ossia nucleasi sito-specifica) per catalizzare il taglio sito-specifico e la ricombinazione tra il sito att del virus e quello del batterio. 
\\In questo modo avviene un crossing over che permette la ricombinazione cromosomica e di conseguenza la duplicazione del genoma. 
\\Se il sistema di repressione di lambda (controllato dal gene cI) viene interrotto, il virus riprende il ciclo litico, liberandosi del cromosoma batterico. Questa escissione richiede il prodotto del gene xis e int. 
\subsection{Crescita litica di lambda}
Gli agenti che inducono l'induzione del ciclo litico dono quelli che danneggiano il DNA: radiazioni UV, raggi X, mutageni chimici. 
\\Durante la risposta SOS, l'attività proteasi della proteina RecA distrugge il repressore di lambda e possono avere inizio nuovi eventi litici. 
\section{Genomi dei batteriofagi}
I genomi dei batteriofagi hanno una struttura a mosaico e sono composti da blocchi di sequenze correlate che sono condivise in diverse combinazioni. Questo ci suggerisce che il trasferimento genico orizzontale e la ricombinazione non omologa hanno avuto un ruolo importante nell'evoluzione fagica. 
\section{Coltivazione dei virus animali}
La coltivazione dei virus animali necessita dell'inoculo in un ospite permissivo. L'uovo embrionato offre una varietà di tessuti differenziati o in via di differenziamento che fungono da ospite per la crescita virale.
\\Le colture cellulari è il sistema ospite più utilizzato. Lo sviluppo della replicazione virale si manifesta con la comparsa di effetti citopatici. Sono delle alterazioni morfologiche pronunciate:
\begin{itemize}
    \item Aumento di rifrangenza e dimensioni;
    \item Comparsa di vacuolizzazione; 
    \item Comparsa di inclusioni cellulari; 
    \item Distacco dalla superficie di crescita;
    \item Necrosi; 
    \item Fusione tra più cellule; 
    \item etc.
\end{itemize}
Normalmente si lavora in vitro con delle cellule suscettibili all'infezione dei virus. 
\\Si sa che il ciclo di replicazioe è attivo ma non si sa quale due due e non si hanno nemmeno maggiori informazioni riguardo al numero.
\subsubsection{Saggio delle placche}
Il saggio delle placche si utilizza per condurre un'analisi quantitativa. 
\\Il titolo virale, espresso in plaque-forming unit/ml (PFU/ml), può essere determinato con metodologie che determinano l'infettività, ossia la capacità di iniziare e concludere un ciclo replicativo. Questo saggio consiste nell'infezione di colture cellulari in monostrato con diluzioni scalari/seriali della preparazione di particelle virali. Dopo l'adsorbimento si rimuove l'inoculo virale e le cellule vengono ricoperte da terreno di coltura con agar in modo da mantenere la vitalità cellulare e limitare la diffusione delle particelle virali al resto della coltura. 
\\Alcuni virus animali non producono effetti riconoscibili si colture cellulari e per questo motivo si procede con delle diluizioni seriali fino a quando si ottiene un fenomeno contabile:
\begin{itemize}
    \item la stima quantitativa viene fatta con l'iniezione di una diluizione seriale in un certo numero di animali; 
    \item si determina una diluizione al punto finale, alla quale muore la metà degli animali inoculati (LD$_50$).
\end{itemize}
\subsubsection{Conta delle particelle virali}
\`E un tipo di conta diretta che viene praticata con l'ausilio del microscopio elettronico. Il campione virale viene miscelato ad una concentrazione nota di "beads" (biglie). I volumi del campione virale e di beads sono uguali. Dal rapporto che si presenta tra beads e virus riesco a trovare la concentrazione. 
\section{Purificazione di virus}
Si hanno diverse tipologie per la purificazione dei virus:
\begin{itemize}
    \item Centrifugazione differenziale;
    \item Centrifugazione in gradiente di densità;
    \item Precipitazione differenziale dei virus;
    \item Denaturazione dei contaminanti; 
    \item Digestione enzimatica dei costituenti delle cellule ospiti.
\end{itemize}
\subsection{Centrifugazione differenziale}
\begin{enumerate}
    \item Si ha una prima centrifugazione per separare le particelle virali e gli organelli cellulari dalle molecole più piccole; 
    \item Si toglie il surnattante; 
    \item Si risciacqua quello che si è ottenuto e si risospende il tutto;
    \item Si ha una centrifugazione più leggera; \item Sedimentazione di organelli seguita da separazione tra surnattante con virus e pellet di batteri;
    \item Seconda ultra centrifugazione per ottenere la precipitazione dei virus. 
\end{enumerate}
\subsection{Centrifugazione in gradiente di densità}
In questo tipo di centrifugazione si ottengono una migliore precisione e migliore purificazione. Per prima cosa viene preparata una provetta con un certo gradiente, che viene dato dalla presenza di certi zuccheri. Poi vengono aggiunti i componenti che si devono separare in queste soluzione e vengono successivamente recuperati per frazioni. Ci sono 2 tipi principali: 
\begin{itemize}
    \item Nella \textbf{centrifugazione isoponica} il fondo del gradiente è più denso che qualsiasi particella. Queste particelle si posizionano nella porzione in cui la densità di gradiente è uguale alla propria densità.
    \item Nella \textbf{centrifugazione zonale} il fondo del gradiente è meno denso dalle particelle, ed esse si separano secondo il proprio coefficiente di sedimentazione. Questo coefficiente tiene conto della forma, della densità e della struttura della particella. 
\end{itemize}
\subsection{Precipitazione differenziale dei virus}
Viene utilizzato il solfato di ammonio o glicol polietilenico, che è una precipitazione di proteine. Viene aggiunto il solfato di ammonio fino a portare la sua concentrazione a un livello appena inferiore al punto di precipitazione del virus. Dopo aver rimosso i contenuti precipitati si aumenta la concentrazione del solfato d'ammonio per precipitare i virus stessi. 
\subsection{Denaturazione dei contaminanti}
Vengono utilizzati calore, pH e solventi organici. Alcuni virus tollerano il trattamento con solventi organici, per esempio il cloroformio. Il trattament con solventi viene usato per denaturare i contaminanti proteici e per estrarre i lipidi. Il virus rimane nella fase acquosa, i lipidi si dissolvono nella fase organica mentre le sostanze denaturate dai solventi si raccolgono in corrispondenza dell'interfaccia tra le due fasi. 
\subsection{Digestione enzimatica dei costituenti delle cellule ospiti}
Si usano le proteasi (es. tripsina) e nucleasi (es. ribonucleasi) per rimuovere proteine e acidi nucleici cellulari. 
\section{Particelle sub-virali}
Non sono dei virus, ma hanno delle proprietà simili ai virus e ad organismi che causano malattie e patologie. 
\subsection{Viroidi}
Sono delle piccole molecole di RNA circolare a singolo filamento (250-400 bp) e sono la causa di malattie delle piante. L'RNA non contiene geni che codificano proteine e dipendono totalmente dalle funzioni dell'ospite per la loro replicazione. 
\\Formano delle strutture secondarie che assomigliano a molecole a doppio filamento. I viroidi entrano nell'ospite attraverso una ferita e si replicano nel nucleo della cellula ospite con l'aiuto della RNA polimerasi dell'ospite. Hanno il ruolo di RNA regolatori  che interferiscono con le normali funzioni dell'ospite. 
\subsection{Prioni}
I prioni hanno una struttura interamente proteica. Il prione è responsavile della BSE (encefalopatia sponfigorme bovina, anche chiamata mucca pazza). Questa malattia può infettare l'uomo e causa una variante della malattia di Kreutzfeldt-Jakob, caratterizzata da aggregati della proteina prionica del cervello causata alla perdita di solubilità delle proteine stesse.
\\La forma normale del prione PrPc è repressa nelle cellule nervose e viene modificata nella forma anomala PrPsc. Quest'ultima mostra una certa resistenza alla proteasi ed è insolubile, portando così alla formazione di aggregati. 
\\Il prione non si replica autonomamente. Infatti, coonverta la proteina normale cellulare in una froma patogena inducendo uno stato conformazionale auto-propagante.
