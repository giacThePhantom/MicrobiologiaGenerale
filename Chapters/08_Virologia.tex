\chapter{Virologia}
Un virione \`e una particella virale completa, funzionale e in grado di infettare. Consiste di almeno una molecola di DNA o RNA racchiusa in un rivestimento proteico, detto capside. Pu\`o presentare anche degli strati addizionali. \`E un parassita endosimbionte obbligato che non pu\`o riprodursi indipendentemente dalle cellule viventi e che non pu\`o duplicarsi in assenza di un organismo ospite come fanno eucarioti e procarioti. 
\\Quindi, i virus sono degli elementi genetici che si replicano indipendentemente dai cromosomi cellulari, ma non dalle cellule stesse. Al contrario di alcuni elementi genetici, come i plasmidi, possiedono una forma extracellulare che ne permette la persistenza al di fuori dell'ospite.
\section{Struttura dei virus}
La dimensione del virione \`e compresa fra 10-400 nm in diametro, ma la maggior parte del virus \`e troppo piccola per essere visibile al microscopio ottico. Tutti i virioni: 
\begin{itemize}
    \item nucleocapside composto da acido nucleico (DNA o RNA);
    \item rivestimento proteico (capside);
    \item componenti addizionali, come per esempio l'envelope (rivestimento lipidico).
\end{itemize}
Si distinguono due tipi di virus: 
\begin{itemize}
    \item nudi $\xrightarrow{}$ presentano solo l'involucro proteico del capside;
    \item rivestiti $\xrightarrow{}$ envelope o ivolucro percapsidico, con membrana formata da un doppio strato lipidico al quale sono asscoiate delle proteine. La componente lipidica deriva dalla membrana dell'ospite, mentre la componente proteica \`e codificata dak virus.
\end{itemize}
\subsection{Capside e simmetria}
Una prima classificazione viene fatta sulla struttura base del capside:
\begin{itemize}
    \item Isocaedrico;
    \item Isocaedrico con envelope;
    \item Elicoidale senza envelope;
    \item Elicoidale con envelope.
\end{itemize}
La capside \`e una struttura macromolecolare che serve da rivestimento proteico del virus. Protegge il materiale genetico virale e aiuta il suo trasferimento fra le cellule ospiti.
\\\`E formato da delle unit\`a morfologiche dette capsomeri. Questi sono formati a loro volta da delle subunit\`a strutturali proteiche dette protomeri. I protomeri hanno la capacit\`a di autoassemblarsi spontaneamente, combinandosi in strutture elicoidali o icosaedriche. I capsidi elicoidali hanno forma di un tubo vuoto con parete proteica.
\\L'acido nucleico si trova all'interno della particella, circondato dal capside. 
\\Si riconoscono due differenti simmetrie a cui corrispondono le due forme principali:
\begin{enumerate}
    \item bastoncellari a simmetria elicoidale, come per esempio il \textit{Virus del Mosaico del Tabacco}. Il capside a elica forma un tubo vuoto con parete proteica. 
    \item sferoidali a simmetria isocaedrica, composti da 20 facce o capsomeri. Un esempio \`e il virus dell'influenza che \`e rivestito da un nucleocapside a elica. \`E poliequivalente perch\`e il suo genoma \`e presente su pi\`u nucleocapsidi diversi. Oltre a questo le proteine sono distribuite a intervalli regolari sula suerficie. ($\xrightarrow{}$ nucleocapside+involucro lipidico$\xrightarrow{}$ simmetria marcata).
\end{enumerate}
La simmetria isocaedrica rappresenta la disposizione pi\`u efficiente delle subunit\`a nella formazione della capside. Il capside asssomiglia ad una sfera ed \`e un solido in cui la superficie \`e minima con von volume massimo ($\xrightarrow{}$ il rapporto tra volume e superficie \`e massimo). In questo modo si minimizza il numero di subunit\`a necessarie alla sua costruzione.
\\La disposizione pi\`u semplice \`e di 3 protomeri per capsomero, per un totale di 60 protomeri per virione. Le altre configurazioni conosciute sono 180, 240, 360 e 420 unit\`a.  
\\Il capside isocaedrico \`e normalmente composto da: \begin{itemize}
    \item P = pentoni, in corrispondenza dei vertici
    circondati da 5 esoni;
    \item H = esoni, formano i lati e le facce dell'icosaedrico. 
\end{itemize}
In totale, quindi, si hanno 42 capsomeri con un solo tipo di protomeri.
\\Questi tipi di capsidi hanno la capacit\`a di autoassemblaggio: si assemblano in maniera autonoma, non hanno bisogno della presenza di fattori cellulari che controllino o regolino la loro formazione. Proprio per questo motivo \`e possibile realizzarli in vitro.
\subsection{Virus con capsidi a simmetria complessa}
Alcuni virus non ricadono nelle categoria con capsidi ad elica o isocaedrici. Degli esempi sono i poxivirus ed i grandi batteriofagi. 
\begin{itemize}
    \item i Poxvirus sono formati da un core centrale biconcavo che contiene i genomi e numerosi enzimi virali, due corpi laterali e un doppio involucro di membrana; 
    \item il Batteriofago T4 presenta una simmetria binaria, con combinazione di simmetria isocaerica (testa, capside con acido nucleico) ed elicoidale (guaina). Presenta anche una piastra basale esagonale e delle fibre caudali, con cui interagisce con la cellula ospite e inietta il suo genoma.
\end{itemize}
Alcuni virus presentano anche l'envelope: un doppio strato fosfolipidico esterno. Questa membrana viene acquisita dal virus durante il processo di uscita della cellula ospite, detto budding (gemmazione). Mentre i virus senza envelope abbandonano la cellula lisandola, il processo di budding non altera l'integrit\`a cellulare.
\begin{itemize}
    \item Il genoma viene replicato nella cellula ospite; 
    \item Vengono create le proteine strutturali; 
    \item Le proteine virali con parte idrofobica vengono espresse. Questo permette di inserirsi nella membrana della cellula ospite.
    \item Le glicoproteien sulla membrana dell'ospite crescono di numero man mano che aumentano le proteine virali della matrice;
    \item Dal nucleocapside intracellulare si forma un virione libero. La membrana si richiude intorno al nucleo del capside.
\end{itemize}
\subsection{Genomi virali}
I genomi virali hanno dimensioni ridotte e codificano quelle funzioni che non possono essere fornite dai loro ospiti. Il virus riprogramma le funzioni metaboliche e biosintetiche dell'ospite finalizzandole alla sua replicazione e all'assemblaggio di nuovi virioni. Le particelle virali hanno dimensioni da 20 a 300 nm; mentre i genomi virali sono compresi tra 500 e 5000 kb (a volte possono essere pi\`u grandi di alcuni batteri).
\section{Tassonomia virale}
La mancanza d'informazione sull'origine e sulla storia evolutiva rende la classificazione virale difficile. Mentre per gli eucarioti si \`e riuscito a costituire un sistema tassonomico su specie, genere, famiglia, ordine, classe e fila, e pi\`u recentemente anche su sinapormofie e simplesiomorfie, per i virus questo non \`e possibile.
\\Nel 1971 l'International Committee for Taxonomy of Viruses (ICTV) ha sviluppato un sistema di classificazione uniforme che descrive circa 2000 virus. Questa classificazione si basa sulla natura del genoma, sulla simmetria del capside, sulla presenza o assenza di envelope e sulle dimensioni del virione e del capside. Per la classificazione dei virus esiste un database chiamato ICTVdB.
\\I virus dimostrano un'assenza di autonomia replicativa e da questo si deduce che siano comparsi dopo gli altri gruppi. Infatti essi non presentano una replicazione sessuata o asessuata e per questo motivo non danno luogo a una linea evolutiva classica. Non esiste, quindi, un albero filogenetico dei virus, come invece c'\`e per gli altri esseri viventi. 
\\Si \`e deciso cos\`i di stilare la classificazione di Baltimore che si occupa dell'espressione dell'informazione genetica. 
\begin{enumerate}
    \item \textbf{Classe I} $\xrightarrow{}$ \`e la classe di virus pi\`u convenzionale perch\`e le sue caratteristiche sono simili a quelle dell'ospite: presentano un DNA a doppio filamento. 
    \\Si ha la trascrizione del filamento negativo per produrre un mRNA positivo, che porta quindi alla traduzione grazie a ribosomi e alla produzione di proteine funzionali. Dato che il DNA \`e standard (a doppio filamento e con direzioni opposte, pu\`o replicarsi grazie ai meccanismi della cellula ospite. Questo tipo di classe non richiede alcun tipo di modifica.
    \\Degli esempi sono: \textit{Adenoviridae, Herpesviridae, Papillomaviridae, Polyomaviridae, Poxviridae}.
    \item \textbf{Classe II} $\xrightarrow{}$ questi virus presentano un DNA a singolo filamento. C'\`e la necessit\`a di un filamento complementare, costruzione di un intermedio di DNA a doppio filamento, prima della trascrizione. \\Per prima cosa avviene la sintesi di un filamento di DNA complementare, poi si ottiene un DNA a doppio filamento standard. 
    \\Si pu\`o procedere con la trascrizione come avviene nella Classe 1. Si utilizza uno dei due filamenti stampo per produrre un mRNA che possa poi essere tradotto in proteine strutturali grazie a enzimi. Per la generazione di nuove unit\`a genomiche l'acido nucleico viene processato in modo da ottenere la separazione della doppia elica che porta a un unico filamento (DNA+ o DNA-).
    \\Per esempio: \textit{Circoviridae, Parvoviridae}.
    \item \textbf{Classe III} $\xrightarrow{}$ questo tipo di virus presenta un RNA a doppio filamento. La cellula ospite non sa produrre altro RNA da RNA a doppio filamento. 
    \\Il filamento negativo viene trascritto e porta a sintesi di proteine funzionali con l'aiuto degli enzimi. Vengono introdotti nuovi enzimi virali, che non sono presenti nella cellula ospite: RNA polimerasi RNA-dipendente (= RNA replicasi) codificata dal genoma virale. Questa ha la capacit\`a di leggere il filamento di RNA ed \`e in grado di ricostruire il genoma originale.
    \\Per esempio: \textit{Reoviridae}.
    \item \textbf{Classe IV} $\xrightarrow{}$ questo tipo presenta un RNA a singolo filamento positivo che pu\`o essere usato direttamente come mRNA per la sintesi di proteine strutturali dato che la cellula ospite \`e in grado di riconoscerlo. Per la codifica del nuovo virale si necessita di RNA polimerasi RNA-dipendente (= RNA replicasi). Con questo metodo si pu\`o produrre RNA positivo a partire da uno stampo, dato che non \`e possibile replicarlo direttamente. 
    \\Per esempio: \textit{Picornaviridae, Togaviridae, Flaviviridae, Coronaviridae}.
    \item \textbf{Classe V} $\xrightarrow{}$ questi virus presentano un RNA a singolo filamento negativo, che viene trascritto. Si necessita di un enzima virale RNA replicasi sia: 
    \begin{itemize}
        \item per la trascrizione perch\`e non c'\`e la polarit\`a giusta. Una volta che il lo stampo di RNA positivo viene prodotto si possono codificare nuove proteine; 
        \item per la replicazione non \`e possile produrre un nuovo filamento direttamente ma ho bisogno di sintetizzare uno stamp di RNA positivo per produrne delle copie. 
        \\Per esempio: \textit{Orthomyxoviridae, Paramyxoviridae, Rhabdoviridae}
        \item \textbf{Classe VI} $\xrightarrow{}$ sono dei retrovirus a RNA. Presentano un singolo filamento di RNA positivo (simile alle classe 4). Necessit\`a di una trascrittasi inversa codificata dal genoma virale. In questo modo si ha: 
        \begin{itemize}
            \item RNA positivo; 
            \item Intermedio a DNA a singolo filamento (DNA-); 
            \item Intermedio a DNA a doppio filamento; 
            \item Trascrizione del filamento DNA-, in modo da ottenere mRNA+ per sintesi di proteine strutturali e RNA+ per la replicazione del genoma virale per virioni successivi.
        \end{itemize}
        Per esempio: \textit{HIV, Retroviridae}.
    \end{itemize}
    \item \textbf{Classe 7} $\xrightarrow{}$ presenta un DNA a doppio filamento.
    \begin{itemize}
        \item Costruzione di un intermedio a RNA+; 
        \item Attraverso la trascrittasi inversa ottengo DNA- come versione complementare; 
        \item Cos\`i pu\`o essere generato un DNA a doppio filamento come genoma per le popolazioni successive.
    \end{itemize}
    Per esempio: \textit{Hepadnaviridae}.
\end{enumerate}
\section{Ciclo replicativo di un virus animale}
In una cellula osite animale il virus deve adottare delle strategie per produrre proteine e nuove copie del genoma. Il ciclo replicativo virale si pu\`o dividere in 6 fasi:
\begin{enumerate}
    \item \textbf{Adsorbimento o attacco} del virione ad una cellula ospite. 
    \\Avviene il contatto tra il virione e la membrana. La specificit\`a tra il virus e l'ospite \`e molto forte ed \`e mediata dal contatto tra superficie nuda o envelope e recettori della membrana citoplasmatica. Se il virus non trova determinati recettori, il ciclo replicativo non avviene.
    \item \textbf{Penetrazione e decapsidazione}. L'attacco di un virus alla cellula opsite provoca delle modificazioni nel virus e nella superficie cellulare che portano alla penetrazione. Questa entrata assomiglia ad una endocitosi. 
    \\La cellula che permette lo solgimento di un intero ciclo replicativo di un virus \`e definita permissiva per quel determi virus. Durante la fase di penetrazione avviene anche la spoilazione, cio\`e il processo attraverso cui i virioni perdono il loro rivestimento esterno e il genoma virale viene cos\`i esposto all'ambiente cellulare.
    \item \textbf{Espressione genica e sintesi delle proteine vitali}. Per i virus, soprattutto per quelli che entrano in cellule non permissive, \`e fondamentale l'espressione genica per produrre le proteine. Queste possono essere: 
    \begin{itemize}
        \item Strutturali, cio\`e vanno a formare il capside per consentire la formazione di nuovi virioni; 
        \item Non strutturali, che sono quindi necessari per la replicazione del genoma. 
    \end{itemize}
    \item \textbf{Replicazione del genoma virale}.
    \item \textbf{Assemblaggio e maturazione dei virioni}. In particolar modo nei virus con envelope, il processo di maturazione comprende anche il passaggio attraverso organuli come il reticolo endoplasmatico e l'apparato di Gogli per produrre delle glicoproteine presenti sull'envelope stesso.
    \item \textbf{Rilascio e fuoriuscita della cellula}. Se il virus presenta l'envelope si parla di gemmazione. Viene cos\`i chiamato perch\` nel mommento dell'uscita del virus viene circondato da parte della membrana doppio strato fosfolipidico della cellula ospite.
\end{enumerate}
\subsection{Herpes simplex, classe I}
La replicazione del genoma virale avviene nel nucleo. Il genoma viene trascritto come vari mRNA, che vengono poi esportati nel citoplasma per essere tradotti. Questa espressione genica virale avviene in tre fasi: 
\begin{enumerate}
    \item Immediate-Early (Immediata precoce): porta alla regolazione delle fasi successive; 
    \item Early (precoce): produzione degli enzimi necessari alla replicazione del genoma virale. Ci\`o implica che la proteina venga prodotta nel citoplasma, ma rientra in seguito nel nucleo per controllare la trascrizione e la replicazione;
    \item Late (tardiva): sintesi delle proteine strutturali.
\end{enumerate}
Dopo queste fasi possono seguire dei processi di concatenamento in cui molte unit\`a di genoma vengono messe insieme in un'unica molecola. La maturazione del virione continua con il passaggio nel reticolo endoplasmatico e nel Golgi, per portare all'aggiunta di numerosi altre proteine sulla superficie (es. envelope). Infine si arriva all'espulsione del virione dalla cellula ospite (gemmazione).
\subsection{Poxvirus (vaiolo), classe I}
La maggior parte dei processi avviene nel citoplasma: sia il rilascio del nucleocapside e sia la replicazione del genoma virale. 
\\Il virus deve fornire gli enzimi necessari per la replicazione del DNA e la trascrizione dell'RNA visto che la cellula ospite non \`e in grado di soddisfare questi compiti al di fuori del nucleo. Poi c'\`e un passaggio per il reticolo endoplasmatico e per l'apparato di Golgi che porta ad aggiungere dei nuovi strati endoplasmatici e proteine al virione maturo (MV). Alla fine si arriva allo stato di virione extracellulare (EV) che si distingue dal precedente MV per la tipologia dell'involucro esterno, che questa volta deriva dalla membrana plasmatica.
\subsection{Picornavirus (poliovirus), classe IV}
Questo virus presenta un genoma ad RNA positivo, che funge da mRNA e quinsi pu\`o venire tradotto direttamente. Non \`e protetto da envelope e il suo nucleocapside non entra nella cellula durante la penetrazione.
\\Viene prodotta una poliproteina che viene poi scissa da delle protesasi per la produzione delle proteine strutturali. 
\\La RNA polimerasi RNA-dipendente sintetizza l'RNA a polarit\`a negativa che viene utilizzato da stampo per il nuovo RNA genomico positivo. 
\subsection{Virus dell'influenza umana di tipo A, classe V}
Presenta un genoma a RNA negativo e segmentato. Questo significa che \`e composto da un insieme di frammenti con le informazioni necessarie per compiere un ciclo replicativo. 
\\La proteina HA (emoagglutinina) lega recettori sulla superficie cellulare (contatto con la membrana):
\begin{itemize}
    \item Introduzione e rilascio del nucleocapside; 
    \item Trascrizione e replicazione del genoma all'interno del nucleo; 
    \item Traduzione nel citoplasma.
\end{itemize}
Alcune proteine tornano nel nucleo per l'assemblaggio del nucleocapside. Poi le proteine dell'enevelope inserite nella membrana del reticolo endoplasmatico e trasportate poi dal Golgi. 
\\Infine avviene la gemmazione.
\\Per il virus dell'influenza non esiste un vaccino che rimane attivo e valido per molti anni. Questo avviene perche\`e:
\begin{itemize}
    \item a causa del suo genoma segmetato muta molto velocemente; 
    \item non presenta una specificit\`a particolarmente elevata per l'ospite;
    \item va incontro alla deriva antigenica, piccole variazioni (mutazioni puntiformi) che determinano epidemie influenzali ogni 2-3 anni; 
    \item riassortimento antigenico o shift. Questo pu\`o essere alla base di pandemie, ogni 10-40 anni, per comparsa di nuovi tipi di HA e NA. Il riassortimento \`e dovuto all'infezione spontanea in un ospite permissivo di ceppi virali provenienti da ospiti diversi. In questo modo si ottengono strutture ibride; 
    \item saltano da una specie all'altra. 
\end{itemize}
\subsection{Retrovirus animale, classe VI}
L'HIV \`e il retrovirus pi\`u conosiuto globalmente. Questo virus colpisce il sistema immunitario dell'ospite fino a portarlo alla morte. \`E composto da un genoma a RNA negativo a singolo filamento. Per questo motivo necessita di alcuni enzimi propri per poter portare a compimento il suo ciclo replicativo: 
\begin{itemize}
    \item Proteasi PR; 
    \item Integrasi IN; 
    \item Trascrittasi inversa RT.
\end{itemize}
Dopo l'attacco avviene la fusione dell'envelope con la membrana cellulare, che viene seguita dal rilascio immediato del nucleocapside. Il suo genoma \`e organizzato in tre regioni principali:
\begin{itemize}
    \item Gag che codifica per proteine strutturali; 
    \item Pol che codifica per enzimi virali; 
    \item Env che codifica per le glicoproteine del capside. 
\end{itemize}
Per prima cosa si ha la sintesi di DNA/RNA e quindi una doppia elica di DNA da parte della trascrittasi inversa. Una copia di questo doppio filamento viene integrato nel cromosoma dell'ospite e viene chiamato provirus. In questo modo il DNA virale diventa leggibile dall'apparato di lettura delle cellula ospite. Ora avvengono l'espressione del genoma virale e la sintesi delle proteine virali. Le proteine che vengono tradotte possono essere: dell'enevelope, funzionali o strutturali. 
\\Grazie alla proteasi si ha la maturazione del genoma, l'assemblaggio dei virioni, la traslocazione della proteina di superficie del reticolo endoplasmatico al Golgi e per finire alla membrana plasmatica. Alla fine avviene la gemmazione. 
