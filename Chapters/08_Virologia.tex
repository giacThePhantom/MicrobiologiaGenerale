\chapter{Virologia}

\section{Introduzione}
Si intende per virus un elemento genetico che si replica indipendentemente dai cromosomi cellulari ma non dalle cellule stesse.
Possiedono un forma extracellulare che ne permette la presenza al di fuori dell'ospite.
Sono parassiti endosimbionti obbligati che non possono riprodursi indipendentemente dalle cellule viventi e che non possono duplicarsi in assenza di un organismo ospite.

	\subsection{Virioni}
	Si intende per virione una particella virale completa, funzionale e in grado di infettare nel suo stato extracellulare e metaboliticamente inerte.
	Consiste di almeno una molecola di DNA o RNA racchiusa in un rivestimento proteico, detto capside. 
	Pu\`o presentare anche degli strati addizionali. 

\section{Struttura dei virus}
La dimensione del virione \`e compresa fra $10$ e $400\si{nm}$ in diametro, ma la maggior parte del virus \`e troppo piccola per essere visibile al microscopio ottico.

	\subsection{Composizione dei virioni}
	Tutti i virioni contengono:
	\begin{multicols}{2}
		\begin{itemize}
    			\item Un nucleocapside composto da acido nucleico (DNA o RNA);
    			\item Un rivestimento proteico (capside);
    			\item Componenti addizionali, come per esempio l'envelope (rivestimento lipidico).
		\end{itemize}
	\end{multicols}	

	\subsection{Tipologie di virus}
	\begin{multicols}{2}
		\begin{itemize}
    			\item Nudi: presentano solo l'involucro proteico del capside.
			\item Rivestiti: comprendono un envelope o involucro percapsidico, con una membrana formata da un doppio strato lipidico al quale sono associate delle proteine. 
		\end{itemize}
	\end{multicols}
	La componente lipidica dei virus rivestiti deriva dalla membrana dell'ospite, mentre la componente proteica \`e codificata dal virus.

	\subsection{Capside e simmetria}
	Una prima classificazione dei virus viene fatta sulla struttura base del capside:
	\begin{multicols}{2}
		\begin{itemize}
		    \item Isocaedrico;
		    \item Isocaedrico con envelope;
		    \item Elicoidale senza envelope;
		    \item Elicoidale con envelope.
		\end{itemize}
	\end{multicols}
	La capside \`e una struttura macromolecolare che serve da rivestimento proteico del virus. 
	Protegge il materiale genetico virale e aiuta il suo trasferimento fra le cellule ospiti.
	Le molecole di DNA o RNA si trovano pertanto all'interno della particella circondate dal capside.

		\subsubsection{Capsomeri}
		I capsomeri sono le unit\`a morfologiche del capside.
		Sono formati da subunit\`a strutturali proteiche dette protomeri.

			\paragraph{Protomeri}
			I protomeri sono proteine con la capacit\`a di assemblarsi spontaneamente combinandosi in strutture elicoidali o icosaedriche.
			La mancanza della necessit\`a di meccanismi di regolazione permette il loro assemblaggio anche in vitro.

		
		\subsubsection{Simmetrie del capside}
		I capsidi si dispongono secondo due tipi principali di simmetrie.

			\paragraph{Bastoncellari}
			I virus bastoncellari hanno una simmetria elicoidale.
			Il capside a elica forma un tubo vuoto con parete proteica.
			Un esempio \`e il Virus del Mosaico del Tabacco.

			\paragraph{Sferoidali}
			I virus sferoidali hanno una simmetria icosaedrica.
			Sono composti da $20$ facce o capsomeri.
			La simmetria icosaedrica rappresenta la disposizione pi\`u efficiente delle subunit\`a.
			Il capside assomiglia ad una sfera solida e massimizza il volume minimizzando la superficie.
			La disposizione pi\`u semplice \`e di $3$ protomeri per capsomero, per un totale di $60$ protomeri per virione.
			Altre configurazioni sono a $180$, $240$, $360$ e $420$ unit\`a.
			\`E tipicamente composto da:
			\begin{multicols}{2}
				\begin{itemize}
					\item Pentoni $P$ in corrispondenza dei vertici.
					\item Esoni $H$, formano i lati e le facce dell'icosaedro, circondano i pentoni.
				\end{itemize}
			\end{multicols}
			Si ha pertanto un totale di $42$ capsomeri.

				\subparagraph{Virus dell'influenza}
				Il virus dell'influenza presenta un capside sferoidale circondato da un nucleocapside a elica.
				Il genoma \`e presente in pi\`u nucleocapsidi diversi e viene detto poli equivalente.
				Le proteine sono distribuite a intervalli regolari sulla sua superficie.
			
			\paragraph{Virus con capsidi a simmetria complessa}
			Alcuni virus non ricadono nelle categoria con capsidi ad elica o isocaedrici. 
	
				\subparagraph{Poxvirus}
				I poxvirus sono formati da un core centrale biconcavo che contiene i genomi e numerosi enzimi virali, due corpi laterali e un doppio involucro di membrana.

				\subparagraph{Grandi batteriofagi}
				Il batteriofago \emph{T4} presenta una simmetria binaria: composta da simmetria icosaedrica per testa e capside ed elicoidale per la guaina.
				Presenta inoltre una piastra basale esagonale e fibre caudali con cui interagisce con la cellula ospite e inietta il genoma.

	\subsection{Envelope}
	Alcuni virus possiedono un envelope, un doppio strato fosfolipidico esterno.
	Questo viene acquisito durante il processo di gemmazione o budding e di uscita dalla cellula ospite.
	Questo processo non altera l'integrit\`a cellulare.
		
		\subsubsection{Budding}
		\begin{multicols}{2}
			\begin{enumerate}
   					\item Il genoma viene replicato nella cellula ospite.
   					\item Vengono create le proteine strutturali.
   					\item Le proteine virali con parte idrofobica vengono espresse. 
					In questo modo il virus si inserisce sulla membrana della cellula ospite: le gicoproteine sulla membrana dell'ospite crescono di numero e dal nucleocapside intracellulare si forma un virione libero.
				\item La membrana si richiude intorno al nucleo del capside.
			\end{enumerate}
		\end{multicols}
		
	\subsection{Genomi virali}
	I genomi virali hanno dimensioni ridotte e codificano quelle funzioni che non possono essere fornite dai loro ospiti. 
	Riprogrammano le funzioni metaboliche e biosintetiche dell'ospite finalizzandole alla sua replicazione e all'assemblaggio di nuovi virioni. 
	Le particelle virali hanno dimensioni da $20$ a $300 \si{nm}$.
	I genomi virali sono compresi tra $500$ e $5000 kb$ (a volte possono essere pi\`u grandi di alcuni batteri).

\section{Tassonomia virale}
La classificazione virale \`e resa difficile dalla mancanza di informazioni su origine e storia evolutiva.
Non \`e possibile costruire un sistema tassonomico basato su specie, genere, famiglia, ordine, classe, fila, sinapomorfie e simplesiomorife come per gli eucarioti.
In quanto i virus non presentano autonomia replicativa devono essere emersi dopo le altre forme di vita.
Non svolgono n\`e replicazione sessuata n\`e asessuata e pertanto non danno luogo a una linea evolutiva classica. 
Si nota quindi come non esista un albero filogenetico dei virus. 

	\subsection{International Committee for Taxonomy for Viruses}
	Nel 1971 l'International Committee for Taxonomy of Viruses (\emph{ICTV}) ha sviluppato un sistema di classificazione uniforme che descrive circa $2000$ virus. 
	Questa classificazione si basa sulla natura del genoma, sulla simmetria del capside, sulla presenza o assenza di envelope e sulle dimensioni del virione e del capside. 
	Per la classificazione dei virus esiste un database chiamato \emph{ICTVdB}.

	\subsection{Classificazione di Baltimore}
	La classificazione di Baltimore classifica i virus in base all'espressione della loro informazione genetica.
	
		\subsubsection{Classe $\mathbf{I}$}
		La classe $I$ \`e la classe di virus pi\`u convenzionale: le sue caratteristiche sono simili a quelle dell'ospite.
		Questi virus trasportano materiale genetico come DNA a doppio filamento.
		Viene trascritto il filamento negativo per produrre un mRNA positivo che viene tradotto grazie ai ribosomi e produce le proteine funzionali al virus.
		Essendo il DNA analogo a quello della cellula ospite pu\`o replicarsi grazie ai suoi meccanismi, anche se la trascrizione avviene a tassi pi\`u elevati.

			\paragraph{Esempi}\mbox{}\\
			Esempi di virus di classe $I$ sono:
			\begin{multicols}{3}
				\begin{itemize}
					\item Adenoviridae.
					\item Herpesviridae.
					\item Papillomaviridae.
					\item Polymaviridae.
					\item Poxviridae.
				\end{itemize}
			\end{multicols}

		\subsubsection{Classe $\mathbf{II}$}
		I virus di classe $II$ trasportano le loro informazioni come DNA a filamento singolo.
		Prima della trascrizione si deve sintetizzare un intermedio di DNA a doppio filamento.
		Il primo passo dopo l'infezione \`e pertanto la sintesi di un filamento di DNA complementare da cui si ottiene un DNA a doppio filamento standard.
		Uno dei due filamenti viene usato come stampo per la produzione di mRNA che viene tradotto dai ribosomi in proteine.
		Per la generazione di nuove unit\`a l'acido nucleico viene processato in modo da ottenere la separazione della doppia elica creando di nuovo DNA a filamento singolo.

			\paragraph{Esempi}\mbox{}\\
			Esempi di virus di classe $II$ sono:
			\begin{multicols}{2}
				\begin{itemize}
					\item Circoviridae.
					\item Parvoviridae.
				\end{itemize}
			\end{multicols}

		\subsubsection{Classe $\mathbf{III}$}
		I virus di classe $III$ trasportano le loro informazioni come RNA a doppio filamento.
		Il filamento negativo viene trascritto e permette la sintesi delle proteine funzionali.
		Queste sono anche enzimi virali non presenti nella cellula ospite come RNA polimerasi RNA-dipendente o RNA replicasi che lega il filamento di RNA e ricostruisce il genoma originale.

			\paragraph{Esempi}\mbox{}\\
			\begin{center}
				\begin{itemize}
					\item Reoviridae.
				\end{itemize}
			\end{center}

		\subsubsection{Classe $\mathbf{IV}$}
		I virus di classe $IV$ trasportano le loro informazioni come RNA a singolo filamento positivo.
		Questo pu\`o essere usato direttamente come mRNA per la sintesi di proteine strutturali in quanto la cellula ospite lo riconosce.
		Per la codifica del nuovo genoma virale si necessita di RNA polimerasi RNA-dipendente o RNA replicasi.
		Si produce RNA positivo a partire da uno stampo a RNA negativo.
		
			\paragraph{Esempi}\mbox{}\\
			\begin{multicols}{2}
				\begin{itemize}
					\item Picornaviridae.
					\item Togaviridae.
					\item Flaviviridae.
					\item Coronaviridae.
				\end{itemize}
			\end{multicols}

		\subsubsection{Classe $\mathbf{V}$}
		I virus di classe $V$ trasportano le loro informazioni come RNA a singolo filamento negativo.
		Si necessita pertanto di una RNA replicasi virale in modo da ottenere RNA con la giusta polarit\`a da cui vengono tradotte le proteine.
		Lo RNA positivo viene anche usato per la replicazione.

			\paragraph{Esempi}\mbox{}\\
			\begin{multicols}{3}
				\begin{itemize}
					\item Orthomyxoviridae.
					\item Paramyxoviridae.
					\item Rhabdoviridae.
				\end{itemize}
			\end{multicols}

		\subsubsection{Classe $\mathbf{VI}$}
		I virus di classe $VI$ sono retrovirus a RNA.
		Trasportano le loro informazioni come un singolo filamento di RNA Positivo simile alla classe $IV$.
		Una trascrittasi inversa viene codificata dal genoma virale.
		Dal RNA positivo si produce un intermedio a DNA a singolo filamento grazie alla trascrittasi inversa.
		Dal DNA a singolo filamento si produce un intermedio a filamento doppio.
		Viene trascritto il DNA negativo in modo da ottenere mRNA positivo per la sintesi delle proteine strutturali e per la replicazione del genoma virale per i virioni successivi.
			
			\paragraph{Esempi}\mbox{}\\
			\begin{multicols}{2}
				\begin{itemize}
					\item HIV.
					\item Retroviridae.
				\end{itemize}
			\end{multicols}

		\subsubsection{Classe $\mathbf{VII}$}
		I virus di classe $VII$ trasportano le loro informazioni come DNA a doppio filamento.
		Viene costruito un intermedio a RNA positivo da cui attraverso la trascrittasi inversa si ottiene un filamento a DNA negativo.
		In questo modo viene generato il DNA a doppio filamento come genoma per le popolazioni successive.

\section{Ciclo replicativo di un virus animale}
In una cellula ospite animale il virus deve adottare delle strategie per produrre proteine e nuove copie del genoma.

	\subsection{Fasi}

		\subsubsection{Assorbimento o attacco}
		Nella fase di assorbimento o attacco avviene il contatto tra il virione e la membrana;
		La specificit\`a del contatto \`e mediata dai recettori della membrana citoplasmatica su cui il virus si attacca.
		L'assenza di questi recettori impedisce il ciclo replicativo.

		\subsubsection{Penetrazione e decapsidazione}
		L'attacco del virus alla cellula ospite causa modifiche in entrambi che portano alla penetrazione del primo.
		Il processo \`e simile a un'endocitosi.
		La cellula che permette lo svolgimento di un intero ciclo replicativo viene detta permissiva per quel virus.
		Avviene qua la spoilazione, il processo con cui i virioni perdono il rivestimento esterno ed espongono il genoma all'ambiente cellulare.

		\subsubsection{Espressione genica e sintesi delle proteine vitali}
		I virus una volta entrati nelle cellule devono produrre le proteine strutturali che formano il capside consentendo la formazione di nuovi virioni e le proteine non strutturali, necessarie per la replicazione del genoma.

		\subsubsection{Replicazione del genoma virale}
		Durante la replicazione del genoma il virus ha prodotto le proteine funzionali necessarie per la sua replicazione o sfrutta quelle dell'ospite per replicare le proprie informazioni geniche.

		\subsubsection{Assemblaggio e maturazione dei virioni}
		In particolare nei virus con envelope la maturazione comprende il passaggio attraverso organuli come ER e Golgi per produrre glicoproteine che saranno presenti sull'envelope.

		\subsubsection{Rilascio e fuoriuscita dalla cellula}
		La fuoriuscita di un virus con envelope viene detta gemmazione in quanto al momento dell'uscita questo viene circondato da un doppio strato fosfolipidico che prima faceva parte della membrana della cellula ospite.
		In caso di virus senza envelope il rilascio avviene attraverso lisi dell'ambiente cellulare.

	\subsection{Conseguenze dell'infezione virale}

		\subsubsection{Trasformazione}
		La trasformazione consiste nell'interferenza del virus con i circuiti di regolazione della cellula.
		Viene pertanto trasformato il fenotipo cellulare.
		Pu\`o causare tumori.

		\subsubsection{Infezione litica}
		L'infezione litica pu\`o causare la distruzione della cellula ospite.

		\subsubsection{Infezione persistente}
		L'infezione persistente pu\`o avvenire unicamente con virus con envelope: il rilascio del virus avviene per gemmazione senza rilascio della cellula.

		\subsubsection{Infezione latente}
		L'infezione latente avviene in caso di ritardo tra il momento di infezione e gli eventi di lisi: il virus si trova nel citoplasma e pu\`o riprendere la sua attivit\`a dopo anni.

	\subsection{Esempi}

		\subsubsection{Herpes simplex, classe $\mathbf{I}$}
		La replicazione del genoma virale avviene nel nucleo. 
		Il genoma viene trascritto come vari mRNA che vengono poi esportati nel citoplasma per essere tradotti. 

			\paragraph{Espressione genica}
			L'espressione dei geni del Herpes simplex avviene in tre fasi:
			\begin{multicols}{2}
				\begin{enumerate}
    					\item Immediate-Early (Immediata precoce): porta alla regolazione delle fasi successive; 
    					\item Early (precoce): produzione degli enzimi necessari alla replicazione del genoma virale. 
						Ci\`o implica che la proteina venga prodotta nel citoplasma, ma rientra in seguito nel nucleo per controllare la trascrizione e la replicazione;
    					\item Late (tardiva): sintesi delle proteine strutturali.
				\end{enumerate}
			\end{multicols}
			Dopo queste fasi possono seguire dei processi di concatenamento in cui molte unit\`a di genoma vengono messe insieme in un'unica molecola. 
			La maturazione del virione continua con il passaggio nel reticolo endoplasmatico e nel Golgi, per portare all'aggiunta di numerosi altre proteine sulla superficie come l'envelope. 
			Infine si arriva all'espulsione del virione dalla cellula ospite per gemmazione.

		\subsubsection{Poxvirus vaiolo, classe $\mathbf{I}$}
		Il rilascio del nucleocapside e la replicazione del genoma virale avvengono nel citoplasma della cellula ospite.
		Il virus fornisce gli enzimi necessari per la replicazione del DNA e la trascrizione dell'RNA in quanto la cellula non ne \`e capace al di fuori del nucleo.
		Avviene un passaggio per il reticolo endoplasmatico e per l'apparato di Golgi che porta ad aggiungere nuovi strati endoplasmatici e proteine al virione maturo \emph{MV}. 
		Alla fine si arriva allo stato di virione extracellulare \emph{EV} che si distingue dal precedente \emph{MV} per la tipologia dell'involucro esterno che in questo caso deriva dalla membrana plasmatica.

		\subsubsection{Picornavirus poliovirus, classe $\mathbf{IV}$}
		Questo virus presenta un genoma ad RNA positivo, che funge da mRNA e quindi pu\`o venire tradotto direttamente. 
		Non \`e protetto da envelope e il suo nucleocapside non entra nella cellula durante la penetrazione.
		Viene prodotta una poliproteina che viene poi scissa da delle protesasi per la produzione delle proteine strutturali. 
		La RNA polimerasi RNA-dipendente sintetizza l'RNA a polarit\`a negativa che viene utilizzato da stampo per il nuovo RNA genomico positivo. 

		\subsubsection{Virus dell'influenza umana di tipo A, classe $\mathbf{V}$}
		Il virus dell'influenza umana presenta un genoma a RNA negativo e segmentato, ovvero \`e composto da un insieme di frammenti che contengono le informazioni necessarie per un ciclo replicativo.
		
			\paragraph{Contatto con la membrana}
			Il contatto con la membrana \`e mediato dalla proteina \emph{HA} o emoagglutinina.
			\begin{multicols}{2}
				\begin{itemize}
					\item Viene introdotto e rilasciato il nucleocapside.
					\item Si trascrive e replica il genoma all'interno del nucleo.
					\item La traduzione avviene nel citoplasma.
				\end{itemize}
			\end{multicols}
			Alcune proteine tornano nel nucleo per l'assemblaggio del nucleocapside.
			Le proteine dell'envelope inserite in membrana di ER vengono trasportate nel Golgi.
			Il virus esce dalla cellula per gemmazione.

			\paragraph{Vaccino}
			Per il virus dell'influenza non esiste un vaccino che rimane attivo e valido per molti anni. 
			Questo avviene:
			\begin{multicols}{2}
				\begin{itemize}
    					\item A causa del genoma segmentato muta molto velocemente.
    					\item Non presenta una specificit\`a particolarmente elevata per l'ospite.
    					\item Subisce deriva antigenica, piccole variazioni (mutazioni puntiformi) che determinano epidemie influenzali ogni 2-3 anni; 
    					\item Avviene riassortimento antigenico o shift. 
						Questo pu\`o essere alla base di pandemie, ogni 10-40 anni, per comparsa di nuovi tipi di \emph{HA} e \emph{NA}. 
					\item Saltano da una specie all'altra.
				\end{itemize}
			\end{multicols}
			Il riassortimento \`e dovuto all'infezione spontanea in un ospite permissivo di ceppi virali provenienti da ospiti diversi con la nascita di strutture ibride.

			
		\subsubsection{Retrovirus animale, classe $\mathbf{VI}$}
		L'HIV \`e il retrovirus pi\`u conosciuto globalmente. 
		Colpisce il sistema immunitario dell'ospite fino a portarlo alla morte. 
		\`E composto da un genoma a RNA negativo a singolo filamento. 

			\paragraph{Enzimi coinvolti}
			HIV necessita di enzimi propri per poter portare a compimento il ciclo replicativo:
			\begin{multicols}{2}
				\begin{itemize}
					\item Proteasi \emph{PR}; 
					\item Integrasi \emph{IN}; 
					\item Trascrittasi inversa \emph{RT}.
				\end{itemize}
			\end{multicols}

			\paragraph{Genoma}
			Il genoma del HIV \`e organizzato in tre regioni principali:
			\begin{multicols}{2}
				\begin{itemize}
    					\item \emph{Gag}: codifica per proteine strutturali.
					\item \emph{Pol}: codifica per enzimi virali.
					\item \emph{Env}: codifica per le glicoproteine del capside. 
				\end{itemize}
			\end{multicols}

			\paragraph{Ciclo replicativo}
			Dopo l'attacco avviene la fusione dell'envelope con la membrana cellulare, che viene seguita dal rilascio immediato del nucleocapside. 
			Viene sintetizzato DNA/RNA e si forma una doppia elica di DNA da parte della trascrittasi inversa. 
			Una copia di questo doppio filamento viene integrato nel cromosoma dell'ospite e viene chiamato provirus. 
			In questo modo il DNA virale diventa leggibile dall'apparato di lettura delle cellula ospite. 
			Avviene l'espressione del genoma virale e la sintesi delle proteine virali. 
			Le proteine che vengono tradotte possono essere: dell'enevelope, funzionali o strutturali. 
			Grazie alla proteasi si ha la maturazione del genoma, l'assemblaggio dei virioni, la traslocazione della proteina di superficie del reticolo endoplasmatico al Golgi e per finire alla membrana plasmatica. 
			Alla fine avviene la gemmazione. 

\section{I Batteriofagi}
	
	\subsection{Genomi}
	I genomi dei batteriofagi hanno una struttura a mosaico e sono composti da blocchi di sequenze correlate che sono condivise in diverse combinazioni. 
	Questo suggerisce come il trasferimento genico orizzontale e la ricombinazione non omologa abbiano avuto un ruolo importante nell'evoluzione fagica.
	\subsection{Quantificazione della crescita}
	\begin{multicols}{2}
		\begin{itemize}
    			\item Una diluizione di una sospensione contenente materiale virale viene mescolato in una piccola mistura di agar con un ospite suscettibile. 
			\item Piastrazione su agar.
    			\item Il batterio ospite inizia a crescere sulla superficie della piastra di agar e, dopo un'incubazione overnight, forma un tappeto di crescita batterica.
    			\item Allo strato precedente viene aggiunta la componente virale.
				Ogni particella virale che si attacca a una cellula e si riproduce, causa la lisi della cellula stessa e le particelle virali rilasciate possono diffondere alle cellule adiacenti, infettarle, portare di nuovo alla lisi con ciclo litico e venire rilasciate nuovamente.
    			\item Si formano dei buchi nella piastra in corrispondenza delle colonie lisate, chiamate placche. 
				La dimensione delle placche dipende dal virus, dall'ospite e dalle condizioni della coltura.
		\end{itemize}
	\end{multicols}
	Contando il numero di placche si può calcolare il numero delle unità infettive virali presenti nel campione di partenza. 

		\subsubsection{Titolo}
		Si intende per titolo la concentrazione del virus.
		Questo viene espresso in \emph{PFU} (plaque foring unit) per millilitro.

			\paragraph{Calcolo}
			\[\dfrac{n\ placche}{volume\ di\ diluizione}\]
	
	\subsection{Replicazione dei batteriofagi}

		\subsubsection{Fasi della replicazione}
		Nel ciclo replicativo dei batteriofagi si distinguono cinque fasi principali:
		\begin{multicols}{2}
			\begin{enumerate}
    				\item Assorbimento o fase di attacco.
    				\item Penetrazione, in cui avviene l'iniziezione del materiale genetico.
    				\item Sintesi dell'acido nucleico e di proteine strutturali.
    				\item Autoassemblaggio dei capsomeri e impacchettamento.
    				\item Rilascio di virioni maturi. 
			\end{enumerate}
		\end{multicols}

		\subsubsection{Tempo di replicazione}
		Il tempo totale di replicazione varia dai $20$ ai $60$ minuti per i batteriofagi.
		Per \emph{T4} \`e di $20$-$22$ minuti.
		I virus animali possono arrivare a impiegare un tempo tra le $8$ e le $24$ ore.

			\paragraph{Grafici}
			I grafici del ciclo litico riportano sull'asse delle ascisse il tempo e su quello delle ordinate le unit\`a formanti placche e \emph{OD} della popolazione batterica sottoposta ad attacco.
			Il numero di cellule virali dopo il contatto cala di un'ordine di grandezza: sono i fagi in grado di infettare le cellule batteriche.
			Questi grafici individuano nell'infezione due fasi.

				\subparagraph{Fase di latenza}
				Nella fase di latenza i fagi entrano nelle cellule e le infettano ma non accade nulla.
				All'inizio di questa fase si ha il periodo di eclissi in cui diminuisce la popolazione virale.
				Durante questa fase il genoma \`e separato dal capside proteico e il virione non viene considerato come particella invettiva.

				\subparagraph{Fase di Maturazione}
				La fase di maturazione maturazione ha inizio appena i genomi neo-sintetizzati vengono impacchettati all'interno dei capsidi.

				\subparagraph{Fase di scoppio}
				Nella fase di scoppio si ha un brusco aumento nel numero di batteriofagi associato con una drastica diminuzione dei batteri presenti.
				Avviene infatti la lisi delle cellule e vengono rilasciati i fagi nell'ambiente extracellulare.

	\subsection{Batteriofagi virulenti}

		\subsubsection{Batteriofago \emph{T4}}

			\paragraph{Ciclo replicativo}
			Il ciclo replicativo di \emph{T4} \`e molto veloce:
			\begin{multicols}{2}
				\begin{itemize}
    					\item $1$ minuto dopo l'attacco la sintesi del DNA e del RNA dell'ospite cessano e inizia la trascrizione di alcuni geni fagici.
    					\item Dopo $2$ minuti vengono sintetizzate le prime proteine virali.
    					\item Dopo $4$ minuti inizia la replicazione del DNA virale.
    					\item L'assemblaggio di nuovi virioni iniza dopo $20$ minuti. 
				\end{itemize}
			\end{multicols}

			\paragraph{RNA polimerasi}
			\emph{T4} utilizza la RNA polimerasi dell'ospite che riconosce preferenzialmente i promotori virali. 
			La trascrizione nel batterio viene soppressa da una proteina virale che lega il fattore $\sigma$ dell'ospite. 
			Ulteriori modifiche della RNA polimerasi batterica portano alla trascrizione dei ``middle'' e ``late'' RNA. 
			Le ultime proteine ad essere sintetizzate sono quelle strutturali. 
			La fuoriuscita avviene grazie alla lisi della cellula provocata dalla produzione di un enzima virale che attacca il peptidoglicano dell'ospite. 

			\paragraph{Fase di assorbimento}
			L'interazione virus-ospite è altamente specifica. 
			Questa specificità viene data dall'interazione delle proteine di superficie del virus con componenti della membrana dell'ospite (recettori come proteine, carboidrati, lipidi). 
			I recettori svolgono delle funzioni normali per la cellula. 
			Per esempio il recettore per il batteriofago \emph{T1} è una proteina per l'assorbimento del ferro. 
			In assenza di uno specifico recettore, il virus non è in grado di assorbirsi.
			Questa condizione pu\`o essere rapidamente superata attraverso mutazioni.

			\paragraph{Fase di iniezione}
			Una cellula che permette la penetrazione e la replicazione di un virus viene detta permissiva. 
			Le particelle virali si attaccano alla cellule attraverso le fibre caudali che interagiscono in maniera specifica con i polisaccaridi della membrana esterna. 
			Queste fibre si contraggono e la parte centrale della coda entra in contatto con la parete del batterio attraverso una serie di spine situate all'estremità della coda. 
			L'azione di un enzima virale determina la formazione di un poro di peptidoglicano. 
			Dopo la contrazione della guaina caudale, il DNA virale viene spinto all'interno della cellula.

			\paragraph{Restrizione e modificazione del virus da parte dell'ospite}
			I procarioti si possono difendere dall'infezione virale con un meccanismo basato sulla distruzione del DNA genomico virale a doppio filamento, operata da endonucleasi di restrizione. 
			Questo fenomeno, detto restrizione, non è efficiente contro i virus a RNA o a DNA a singolo filamento. 
			Il DNA dell'ospite deve essere protetto dall'azione dei propri enzimi di restrizione attraverso un processo di modificazione che coinvolge la metilazione di basi specifiche sulla sequenza di riconoscimento. 
			Alcuni virus a DNA possono superare i meccanismi di restrizione modificando il proprio genoma attraverso metilazioni e glicosilazioni, che lo rendono resistente all'attacco enzimatico.

			\paragraph{Struttura}
			Il DNA di \emph{T4} presenta una base insolita, la \emph{5-idrossimetilcitosina} al posto della citosina. 
			I gruppi ossidrilici sono modificati dall'aggiunta di residui di glucosio rendendo il DNA resistente agli enzimi di restrizione della cellula ospite. 
			Il batteriofago \emph{T4} presenta:
			\begin{multicols}{2}
				\begin{itemize}
    					\item Testa isocaedrca, $85\times 110\si{nm}$. 
    					\item Coda con struttura tubolare elicoidale $25\times 110\si{nm}$.
    					\item Più di $25$ tipi di proteine strutturali.
    					\item DNA lineare a doppio filamento, $168903 bp$, che codifica più di $250$ proteine diverse.
    					\item DNA permutato circolarmente.
    					\item DNA con ripetizioni terminali di $3$-$6 kb$.
				\end{itemize}
			\end{multicols}

			\paragraph{Permutazione circolare}
			Il DNA di \emph{T4} viene replicato come singole unità che vengono legate tra di loro per formare un'unica molecola detta concatenamero.
			L'impacchettamento del DNA necessita il taglio del concatenamero, che avviene indipendentemente dalla sequenza nucleotidica.
			La testa pu\`o alla fine ospitare una molecola leggermente pi\`u lunga con la comparsa di terminazioni ridondanti.
			Questo avviene in quanto il DNA letto e replicato pi\`u volte contiene estremit\`a che si possono combinare tra loro per omologia della basi.
			Dopo il taglio pertanto i frammenti possono presentare sequenze ripetute e diverse tra loro.

	\subsection{Batteriofagi temperati}
	Nel ciclo lisogenico un gene codificato dal profago gioca il ruolo di repressore dell'espressione del genoma virale. 
	Questo repressore previene anche l'espressione di qualsiasi altro genoma virale introdotto nella stessa cellula: conferisce ai batteri lisogenizzati uno stato di immunità.

		\subsubsection{Il batteriofago $\mathbf{\lambda}$}
		Il batteriofago $\lambda$ può intraprendere sia la via litica che la via lisogenica. 
		Non possiede fibre caudali e presenta un genoma di DNA lineare a doppio filamento con una coda di $12$ nucleotidi a singola elica detta regione \emph{cos}. 
		All'interno della cellula batterica queste sequenza terminali complementari si associano e il DNA è rilegato in una forma circolare contenente $48502 bp$.

			\paragraph{Genoma}
			Il genoma del fago $\lambda$ presenta regioni che codificano per le varie funzioni.
			Pu\`o essere inoltre letto in entrambe le direzioni.
			\begin{multicols}{2}
    				\begin{itemize}
        				\item Lettura in senso orario (Right $R$) controlla le funzioni del ciclo litico.
        				\item Lettura in senso antiorario (Left $L$) controlla le funzioni del ciclo lisogenico.
    				\end{itemize}
			\end{multicols}

			\paragraph{Proteine fondamentali}

				\subparagraph{Repressore di $\mathbf{\lambda}$}
				Il repressore di $\lambda$ viene prodotto dal gene \emph{cI} e ha il compito di bloccare la trascrizione del gene \emph{cro} e degli altri geni necessari al ciclo litico.

				\subparagraph{Proteina \emph{Cro}}
				La proteina \emph{Cro} prodotta dal gene \emph{cro} serve a inibire la trascrizione del gene del repressore $\lambda$.
			
			\paragraph{Infezione del fago $\mathbf{\lambda}$}
			\begin{multicols}{2}
				\begin{enumerate}
    					\item Il virione di $\lambda$ si attacca a uno specifico recettore della membrana esterna di \textit{E. coli} e inietta il proprio DNA all'interno della cellula.
    					\item Il DNA circolarizza e ha inizio l'espressione del genoma virale.
					\item La RNA polimerasi dell'ospite comincia a trascrivere a partire da 2 promotori \emph{$P_L$} e \emph{$P_R$}, producendo \emph{L1} e \emph{R1}.
						Questo avviene in quanto le sequenze non sono palindromiche e permettono la trascrizione in una sola direzione su un filamento.
					\item Vengono prodotte le proteine \emph{Cro} e \emph{N}.
						La prima determina l'entrata nella via litica o lisogenica.
						La seconda \`e un antiterminatore e consente alla RNA polimerasi di continuare la trascrizione oltre ai terminatori a partire di \emph{$P_L$} e \emph{$P_R$}, creando trascritti per i geni impegnati nella replicazione del DNA.
					\item I geni tardivi vengono attivati dopo la sintesi di \emph{Q}, che attiva il trascritto \emph{R2} per le proteine strutturali.
					\item La proteina \emph{Cro} raggiunge una certa concentrazione e lega \emph{$O_R$} e \emph{$O_L$}, bloccando la trascrizione guidata da \emph{$P_R$} e \emph{$P_L$}.
				\end{enumerate}
			\end{multicols}
			
			\paragraph{Avvio della via litica o lisogenica, gli interruttori molecolari}
			Se \emph{cI} viene legato nei siti $1$, $2$, $3$ questo attiva i geni \emph{cII} e \emph{cIII} causando la via lisogenica.
			Se invece \emph{Cro} lega \emph{cI} nei siti $3$, $2$, $1$ questo disattiva i geni \emph{CII} e \emph{CIII} causando la via litica.

				\subparagraph{Via lisogenica}
				Affinch\`e avvenga la via lisogenica:
				\begin{multicols}{2}
					\begin{itemize}
						\item Deve essere inibita la traduzione di proteine tardive.
						\item Una copia del genoma virale deve essere inserita nel cromosoma dell'ospite.
					\end{itemize}
				\end{multicols}
				L'inibizione delle proteine tardive avviene ad opera di \emph{CI}, il repressore di $\lambda$.
				Viene trascritto da \emph{$P_E$}, attivato da \emph{cII}.
				La proteina sintetizzata \`e poco stabile e deve pertanto essere stabilizzata da \emph{cIII}.
				\emph{cI} e \emph{Cro} controllano l'espressione del proprio gene reprimendo e attivando i promotori \emph{$P_M$} e \emph{$P_R$} rispettivamente.
				\emph{Cro} inibisce la trascrizione di \emph{cI} e stimola la trascrizione rightward, mentre \emph{CI} inibisce la trascrizione di \emph{cro} e stimola la trascrizione leftward.
				
			\paragraph{Integrazione di $\mathbf{\lambda}$}
			L'integrazione di $\lambda$ avviene in un singolo sito specifico del cromosoma batterico. 
			Necessita l'attivazione dei geni \emph{cI} e \emph{int}, un integrasi, ovvero nucleasi sito-specifica, per catalizzare il taglio sito-specifico e la ricombinazione tra i siti \emph{att} del virus e quello del batterio. 
			In questo modo avviene un crossing over che permette la ricombinazione cromosomica e di conseguenza la duplicazione del genoma. 
			Se il sistema di repressione di $\lambda$ (controllato dal gene \emph{cI}) viene interrotto, il virus riprende il ciclo litico, liberandosi del cromosoma batterico. 
			Questa escissione richiede il prodotto del gene \emph{xis} e \emph{int}.

			\paragraph{Crescita litica di $\mathbf{\lambda}$}
			Gli agenti attivatori del ciclo litico sono quelli che provocano un danno al DNA del genoma.
			Durante la risposta SOS, l'attività proteasi della proteina \emph{RecA} distrugge il repressore di $\lambda$ e possono avere inizio nuovi eventi litici. 


\section{Coltivazione dei virus animali}
La coltivazione dei virus animali necessita dell'inoculo in un ospite permissivo. 
Un possibile ospite \`e l'uovo embrionato in quanto offre una varietà di tessuti differenziati o in via di differenziamento che fungono da ospite per la crescita virale.
Il sistema pi\`u utilizzato sono comunque le colture cellulari.
Normalmente si lavora in vitro con delle cellule suscettibili all'infezione dei virus. 

	\subsection{Effetti citopatici}
	Gli effetti citopatici sono la manifestazione dello sviluppo della replicazione virale.
	Sono alterazioni morfologiche pronunciate:
	\begin{multicols}{2}
		\begin{itemize}
    			\item Aumento di rifrangenza e dimensioni,
    			\item Comparsa di vacuolizzazione.
    			\item Comparsa di inclusioni cellulari. 
    			\item Distacco dalla superficie di crescita.
    			\item Necrosi.
    			\item Fusione tra più cellule;.
		\end{itemize}
	\end{multicols}

	\subsection{Saggio delle placche}
	Il saggio delle placche permette un'analisi quantitativa.
	Il titolo virale ($\frac{PFU}{\si{mL}}$) viene determinato.
	Questo fornisce informazioni riguardo l'infettivit\`a, ovvero la capacit\`a di iniziare e concludere un ciclo replicativo.
	Vengono infettate colture cellulari in monostrato con diluizioni seriali della preparazione di particelle virali.
	Dopo l'assorbimento si rimuove l'inoculo virale e le cellule vengono ricoperte con terreno di coltura contenente agar.
	Si mantiene cos\`i la vitalit\`a cellulare e si limita la diffusione delle particelle virali al resto della coltura.
	Si procede con diluizioni seriali fino a quando si producono effetti riconoscibili sulle colture cellulari.
	La stima quantitativa viene fatta attraverso iniezione di una diluizione seriale in un certo numero di animali e si determina una diluizione al punto in cui muore la met\`a degli animali inoculati o $LD_{50}$.

	\subsection{Conta delle particelle virali}
	La conta delle particelle virali \`e un tipo di conta diretta praticata attraverso microscopio elettronico
	Il campione virale viene miscelato ad una concentrazione nota di "beads" (biglie). 
	I volumi del campione virale e di beads sono uguali. 
	Dal rapporto che si presenta tra beads e virus si ricava la concentrazione. 

	\subsection{Purificazione di virus}
	Si intende per purificazione di virus il processo con cui si ottiene una soluzione contenente unicamente particelle virali senza colture cellulari in cui sono state inoculate.

		\subsubsection{Centrifugazione differenziale}
		Nella centrifugazione differenziale:
		\begin{multicols}{2}
			\begin{enumerate}
				\item Si svolge una prima centrifugazione per separare le particelle virali e gli organelli cellulari dalle molecole più piccole.
    				\item Si toglie il surnatante.
    				\item Si risciacqua quello che si è ottenuto e lo si risospende.
    				\item Si svolge una centrifugazione pi\`u leggera che causa la sedimentazione di organelli.
				\item Viene separato il surnatante contenente virus e i pellet di batteri.
    				\item Si svolge una seconda ultra centrifugazione per ottenere la precipitazione dei virus. 
			\end{enumerate}
		\end{multicols}
		
	\subsubsection{Centrifugazione in gradiente di densità}
	Nella centrifugazione in gradiente di densit\`a si ottengono una migliore precisione e migliore purificazione. 
	Viene preparata una provetta con un certo gradiente.
	Il gradiente viene determinato da concentrazioni diverse di zuccheri. 
	Vengono aggiunti i componenti che si devono separare.
	I componenti vengono recuperati per frazioni.

		\paragraph{Centrifugazione isopicnica}
		Nella centrifugazione isopicnica il fondo del gradiente \`e pi\`u denso rispetto a qualsiasi particella: si posizionano nella porzione in cui la densit\`a di gradiente \`e uguale alla loro.

		\paragraph{Centrifugazione zonale}
		Nella centrifugazione zonale il fondo del gradiente \`e meno denso delle particelle.
		Le particelle pi\`u pesanti andranno a depositarsi sul fondo secondo il coefficiente di sedimentazione.
		Il coefficiente di sedimentazione tiene conto di forma, densit\`a e struttura della particella.

	\subsubsection{Precipitazione differenziale dei virus}
	Nella precipitazione differenziale dei virus vengono utilizzati il solfato di ammonio o glicol polietilenico, precipitazioni di proteine.
	Viene aggiunto il solfato di ammonio fino a portare la sua concentrazione a un livello appena inferiore al punto di precipitazione del virus. 
	Dopo aver rimosso i contenuti precipitati si aumenta la concentrazione del solfato d'ammonio per precipitare i virus stessi. 

	\subsubsection{Denaturazione dei contaminanti}
	Nella denaturazione dei contaminanti vengono utilizzati calore, $pH$ e solventi organici. 
	Questo viene fatto in quanto alcuni virus tollerano il trattamento con solventi organici come il cloroformio.
	Il trattamento con solventi viene usato per denaturare i contaminanti proteici e per estrarre i lipidi. 
	Il virus rimane nella fase acquosa, i lipidi si dissolvono nella fase organica mentre le sostanze denaturate dai solventi si raccolgono in corrispondenza dell'interfaccia tra le due fasi. 

	\subsubsection{Digestione enzimatica dei costituenti delle cellule ospiti}
	Si usano le proteasi come tripsina e nucleasi come ribonucleasi per rimuovere proteine e acidi nucleici cellulari. 

\section{Particelle sub-virali}
Si dicono particelle sub-virali particelle con propriet\`a simili a virus e ad altri organismi patogeni.

	\subsection{Viroidi}
	I viroidi sono delle piccole molecole di RNA circolare a singolo filamento ($250$-$400 bp$) e sono spesso causa di malattie delle piante. 
	L'RNA non contiene geni che codificano proteine e dipendono totalmente dalle funzioni dell'ospite per la loro replicazione. 
	Formano delle strutture secondarie che assomigliano a molecole a doppio filamento. 
	I viroidi entrano nell'ospite attraverso una ferita e si replicano nel nucleo della cellula ospite con l'aiuto della RNA polimerasi dell'ospite. 
	Hanno il ruolo di RNA regolatori che interferiscono con le normali funzioni dell'ospite. 

	\subsection{Prioni}
	I prioni hanno una struttura interamente proteica. 
	Un prione \`e responsabile della \emph{BSE} (encefalopatia sponfigorme bovina, anche chiamata mucca pazza). 
	Questa malattia può infettare l'uomo e causa una variante della malattia di Kreutzfeldt-Jakob, caratterizzata da aggregati della proteina prionica del cervello causata alla perdita di solubilità delle proteine.
	La forma normale del prione \emph{PrPc} è repressa nelle cellule nervose e viene modificata nella forma anomala \emph{PrPsc}. 
	Quest'ultima mostra una certa resistenza alla proteasi ed è insolubile, portando così alla formazione di aggregati. 
	Il prione non si replica autonomamente. 
	Infatti, converte la proteina normale cellulare in una forma patogena inducendo uno stato conformazionale auto-propagante.
