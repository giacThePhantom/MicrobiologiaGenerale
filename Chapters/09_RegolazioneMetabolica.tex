\chapter{Regolazione metabolica}
Per poter sopravvivere in natura i microrganismi devono essere in grado di rispondere ai cambiamenti delle condizioni ambientali. Grazie all'aiuto delle mappe metaboliche si possono notare strutture complesse dove i punti e le linee rappresentano rispettivamente i composti e le reazioni enzimatiche che portano alla loro formazione. I due meccanismi centrali sono la glicolisi e il ciclo di Krebs. 
\\Per coordinare in modo efficiente le loro reazioni chimiche, le cellule devono regolare il tipo e la quantità di macromolecole che sintetizzano. La maggior parte dei microrganismi possiede geni che codificano molte più proteine di quante ne siano presenti all'interno della cellula in qualunque condizione di crescita. 
\\Le strategie di regolazione sono: 
\begin{itemize}
    \item Controllo dell'attività enzimatica, infatti bisogna evitare di produrre proteine e quindi bisogna esercitare un tipo di regolazione a livello post-traduzionale; 
    \item Controllo della quantità sintetizzata:
    \begin{itemize}
        \item Controllo a livello trascrizionale $\xrightarrow{}$ viene determinato se il gene viene trascritto oppure no controllando che i fattori che interagiscono con il DNA permettono o meno la trascrizione; 
        \item Controllo a livello trasduzionale $\xrightarrow{}$ sono presenti i trascritti ma non vengono presi in cosiderazione.
    \end{itemize}
\end{itemize}
\section{Regolazione dell'attività enzimatica}
\subsection{Inibizione da feedback}
In questo tipo di regolazione è il prodotto finale di una data via biosintetica a inibire l'attivazione del primo enzima della via. Questo tipo di inibizione è reversibile perchè non appena il prodotto finale si esaurisce la sintesi può riprendere. 
\\L'inibizione è detta non covalente dato che avviene tramite il legame, reversibile, del prodotto finale all'enzima ad un sito detto "allosterico". La conformazione dell'enzima cambia in modo che il substrato non riesce più a legarsi al sito attivo. 
\\Esempi:
\begin{itemize}
    \item In una via ramificata entrambi i prodotti finali possono controllare l'enzima specifico per la propria sintesi; 
    \item Gli isoenzimi, che sono gli enzimi isofunzionali, catabolizzano la stessa reazione ma che sono soggette a sistemi di regolazione indipendenti. Ognuno dei tre prodotti della reazione regola il proprio enzima. La quantità totale d'attività enzimatica si azzera soltano quando sono presenti in quantità adeguata tutti i tre prodotti di reazione.
\end{itemize}
\subsection{Modificazione covalente degli enzimi}
Questo tipo di modificazione avviene generalmente per addizione o delezione di piccole molecole alla proteina (ad esempio AMPc, ADP, PO$_4$\ap{2-}, CH$_3$).
\\Il legame covalente del gruppo modificante provoca un cambiamento conformazionale della proteina alternandone l'attività enzimatica. 
\\Per esempio, l'enzima glutammina sintetasi viene modificato con l'aggiunta di gruppi AMP. Ha 12 siti dove può accogliere questi gruppi e la sua attività enzimatica diminuisce gradualmente man mano che aumentano gli AMP legati. 
\\Viene chiamata covalente per via della natura del legame che si forma tra il gruppo e l'enzima. Nel caso dell'inibizione da feedback, invece, si tratta di un legame tra domini, ma non covalente.
\subsection{Processamento delle proteine}
Alcune proteine batteriche vengono attivate da dei meccanismi post-traduzionali come il "proteine splicing". Sono presenti dei segmenti simili agli introni del RNA, ma sono chiamati inteine. Questi vengono eliminati per rendere funzionale la proteina. Esempi:
\begin{itemize}
    \item Girasi $\xrightarrow{}$ taglio proteolitico che le permette il funzionamento; 
    \item Proteasi per degradazione di peptidi $\xrightarrow{}$ la proteina viene attivata solo quando esce dalla cellula.
\end{itemize}
\section{Regolazione a livello trascrizionale}
\subsection{DNA binding proteins (DBP)}
Sono le proteine che legano il DNA. Sono fattori di trascrizione che possono agire positivamente attivando la trascrizione o negativamente non facendola avvenire. Sono composte da dimeri o tetrameri che si legano al DNA in determinati punti specifici. Ne esistono di vari tipi:
\begin{enumerate}
    \item \textbf{Elica-giro-elica} $\xrightarrow{}$ è costituito da aminoacidi in grado di formare una struttura secondaria ad alpha-elica composta da un elica di riconoscimento del DNA e da un'elica stabilizzant. Il riconoscimento di sequenze specifiche del DNA avviene tramite interazioni non covalenti; 
    \item \textbf{Zinc finger} $\xrightarrow{}$ proteina che lega uno ione zinco. Sono presenti almeno due strutture "a dito" coinvolte in questo tipo di legame. Zn è legato, a sua volta, a due cisteine e due istidine; 
    \item \textbf{Leucine zipper} $\xrightarrow{}$ proteina che contiene regioni con residui di leucina regolarmente ripetuti che formano una specie di cerniera. Non interagisce direttamente con il DNA ma serve a tenere unite due altre alpha eliche nella posizione corretta. 
\end{enumerate}
\subsection{Il controllo negativo della trascrizione}
Si intende un controllo effettuato tramite l'azione di un repressore. Induttori e compressori agiscono in modo indiretto, combinandosi con le proteine di legame DAN che influenzano la sintesi dell'mRNA. Ne esistono due tipi:
\begin{enumerate}
    \item \textbf{Repressione} $\xrightarrow{}$ gli enzimi che catalizzano la sintesi di uno specifico prodotto non vengono sintetizzati se questo è presente nel mezzo in quantità sufficiente. Un esempio è l'arginina; 
    \item \textbf{Induzione} $\xrightarrow{}$ la sintesi ha luogo solo quando è presente il suo substrato. Il fenomeno riguarda spesso gli enzimi catabolici. Un esempio è il lattosio. 
\end{enumerate}
Per prima cosa, la RNA polimerasi necessita del fattore sigma per riconoscere le sequenza promotrici. 
\begin{itemize}
    \item \textbf{Meccanismo di repressione} $\xrightarrow{}$ descrive il comportamento della cellula quando reagisce soprattutto alla presenza di aminoacidi. Se il corepressore non è presente nell'ambiente cellulare, la cellula prototrofa deve adoperarsi per produrlo. Nel caso in cui sia presente in giuste quantità, la sintesi dell'mRNA viene bloccata per evitare di produrlo e di andare così a spendere enegergia inutilmente. In questo caso il repressore è attivato in presenza di corepressore: legandosi, infatti, l'aminoacido induce un cambio conformazionale nel repressore che porta a un forte aumento di affinità per il DNA. 
    \\Il repressore dunque si attacca al filamento di DNA ingombrando così lo spazio e impedendo alla RNA polimerasi di associarsi al DNA per cominciare la trascrizione. 
    \\Un esempio può essere l'arginina.
    \\Si tratta sempre di reazioni reversibili dato che, in mancanza dell'aminoacido, la cellula deve essere in grado di rimuovere il repressore dal DNA per cominciare la trascrizione e successiva traduzione. 
    \item \textbf{Meccanismo di induzione}$\xrightarrow{}$ è l'esatto contrario del meccanismo precedente. Il repressore è sempre attaccato al DNA. In presenza dell'induttore (es. lattosio) viene attaccato da questa molecola che porta a un cambio conformazionale interno che modifica la sua affinità per il DNA. Così facendo il sito per l'attacco della RNA polimerasi viene lasciato libero e si può procedere con la trascrizione e traduzione dell'operone per il metabolismo del lattosio. 
    \\Se il lattosio non è presente, non vi è nemmeno la necessità di produrre enzimi per processarlo e dunque il repressore  rimane attccato al suo sito. 
\end{itemize}
\subsection{Controllo positivo della trascrizione}
Questo tipo di controllo coinvolge proteine che promuovono il legame dell'RNA polimerasi al promotore ed all'aumento della sintesi di mRNA. Per esempio, la proteina attivatrice del maltosio non può legarsi al DNA se prima non è legata al suo effettore, cioè il maltosio. 
\\I promotosi degli operoni controllati positivamente hanno una sequenza che non assomiglia fedelmente alla sequenza consenso: anche in presenza del fattore sigma corretto l'RNA polimerasi ha difficoltà a riconoscerli. La proteina attivatrice aiuta l'RNA polimerasi a riconoscere il promotore. Per fare questo il complesso della RNA polimerasi viene scortata dalla proteina che presenta una particolare affinità con il DNA se legata all'effettore.
\\Non sempre il sito di legame dell'attivatore e il promotore sono vicini. In tal caso si rende necessaria la formazione di strutture a loop.
\subsection{Sistemi di controllo globale}
Sono dei meccanismi di regolazione che rispondono a dei segnali ambientali regolando l'espressione di molti geni diversi. 
\subsubsection{Effetto glucosio}
Questo sistema viene chiamato anche repressione da catabolita.
\\In queste ccndizioni si può osservare una curva di crescita con 2 fasi esponenziali. Questo tipo di crescita è detta crescita diauxica. 
\\Quando \textit{E. coli} si trova in presenza di varie fonti di carbonio, il glucosio viene sempre utilizzato per primo dato che è facilmente metabolizzabile. Così facendo gli enzimi che catabolizzano altri substrati vengono repressi. 
\\Quando non è più presente abbastanza glucosio la crescita si ferma per poi ricominciare utilizzando il lattosio o un'altra fonte di carbonio. Questa curva è meno veloce perchè il processamento è meno veloce. 
\\Nel caso di enzimi reprimibili da catabolita, il legame della RNA polimerasi al DNA che li codifica avviene solo quando vi si è legata un'altra proteina chiamata attivatore proteico del catabolismo (CAP). Quest'ultima è una proteina allosterica, che si può legare al DNA soltanto se prima si è legata all'AMP ciclico. 
\\In presenza di glucosio la sintesi dell'AMPc a cura dell dell'adenilato ciclasi viene inibita. In questo modo il livello di AMPc diminuisce. La RNA polimerasi non si lega più ai promotori dei geni regolati da catabolita. 
Si possono presentare varie condizioni:
\begin{enumerate}
    \item \textbf{Assenza di glucosio e lattosio} $\xrightarrow{}$ livelli di AMPc molto alti. CAP si lega sul sito a monte del promotore. Dato che il lattosio non è presente e quindi c'è il repressore che inattiva l'operone Lac. Anche se sono presenti alte concentrazioni di AMPc, a trascrizione viene repressa. 
    \item \textbf{Presenza di glucosio ma assenza di  lattosio} $\xrightarrow{}$ è presente il repressore che inattiva l'operone. CAP non si lega al sito dato che AMPc è a bassi livelli. Anche in questo caso la tracrizione viene repressa.
    \item \textbf{Presenza di glucosio e lattosio} $\xrightarrow{}$ la presenza del lattosio provoca la rimozione del repressore. Ha luogo la trascrizione, ma sarà basale a causa dei bassi livelli AMPc. Questi livelli sono causati dalla presenza del glucosio. 
    \item \textbf{Assenza di glucosio ma presenza di lattosio} $\xrightarrow{}$ in questo caso la cellula ha un assoluto bisogno di glucosio. Il lattosio rimuove il repressor, CAP si ega al sito di attacco (dati gli alti livelli di AMPc). La trascrizione si attiva in modo efficiente.
\end{enumerate}
\subsubsection{Risposta stringente}
In seguito alla carenza di aminoacidi ottiene un blocco della sintesi degli RNA ribosomiali e dei tRNA con conseguente mancanza di assemblaggio di nuovi ribosomi. 
\\La risposta stringente viene indotta da 2 nucleotidi du guanina modificati con l'aggiunta di gruppi fosfato: 
\begin{itemize}
    \item ppGpp; 
    \item pppGpp.
\end{itemize}
In carenza di aminoacidi un tRNA scarico può legarsi al ribosoma che funge da segnale per la proteina RelA. Questa proteinza gruppo fosfato da GTP e ATP per produrre ppGpp e pppGpp. Questi ultimi svolgono un ruolo di regolatori globali: 
\begin{itemize}
    \item Inibiscono la sintesi degli rRNA e dei tRNA, interferendo direttamente con la RNA polimerasi per l'inizio della trascrizione in corrispondenza dei geni relativi a questi RNA;
    \item Attivano operoni deputati alla sintesi degli aminoacidi mancanti integrando cofattori di trascrizione. 
\end{itemize}
\section{Altri sistemi di controllo globale}
Aggiungi tabella
\subsection{Fattori sigma alternativi}
Il fattore sigma è una proteina che si attacca alla RNA polimerasi per facilitare l'attacco al DNA dato che riconosce particolare sequenze specifiche. Molti dei geni impegnati nei processi di controllo globale utilizzano fattori sigma alternativi. La regolazione è determinata dalla quantità (concentrazione) o dall'attività (fattori antagonisti antisigma) dei diversi fattori sigma in quanto ogni fattore riconosce nel genoma soltanto un certo gruppo di promotori. Ci sono 7 fattori sigma in \textit{E. coli}, 14 in \textit{B. subtilis} dei quali 4 sono specifici per i geni necessari alla formazione dell'endospora. 
\\$\sigma$\ap{70} è  il fattore più diffuso. 
\subsection{Risposta allo shock termico}
Nel caso di uno shock termico la degradazione del fattore $\sigma$\ap{32} viene inibito. Questo dà luogo a una massiccia sintesi di proteine codificate da geni sotto il controllo di $\sigma$\ap{32}, in particole le proteine heat shock (Hsp).
\begin{itemize}
    \item La proteina \textbf{Hsp70} (DnaK) previene l'aggregazione delle proteine neosintetizzate e stabilizza quelle ancora non ripiegata.
    \item Le proteine \textbf{Hsp60} (GroEL) e \textbf{Hsp10} (GroEs) catalizzano il corretto ripiegamento di proteine ripiegate erroneamente. 
\end{itemize}
\subsection{Quorum sensing}
Sono sistemmi di regolazione dipendenti dalla percezione della densità delle cellule della stessa specie presenti nella popolazione. Permette alle cellule di attivare una particolare risposta biologica soltanto quando un numero sufficiente di cellule sono presenti nell'ambiente circostante.
\\L'omoserina lactone (AHL) viene sintetizzata dalle specie dotate di quorum sensing e diffonde all'esterno della cellula.  AHL è un induttore. Quando raggiunge una data concentrazione, si combina ad una proteina attivatrice permettnedo l'avvio della trascrizione di geni specifici. 
\\Esempi: 
\begin{itemize}
    \item Regolazione della luminescenza in Vibrio fischeri: induzione dell'operone lux, sotto il controllo della proteina attivatrice LuxR + AHL; 
    \item Formazione di biofilms da \textit{Pseudomonas aeruginosa}; 
    \item Produzione di fattori di virulenza in \textit{Straphylococcus aureus}.
\end{itemize}
Sistema per la sintesi di triptofano basato sulla presenza di un peptide leader (L) a monte dell'operone deputato alla biosintesi di uno specifico aminoacido. Questa sequenza leader è ricca di residui dell'aminoacido in questione. 
\\Questo fenomeno, nei procarioti, può avvenire perchè i processi di trascrizione e traduzione avvengono simultaneamente. 
\\L'attenuazione ha luogo perchè una parte dell'mRNA si ripiega a formare un'ansa a doppia elica che determina il blocco dell'attività della RNA polimerasi. 
\begin{itemize}
    \item Se il Trp è abbondante $\rightarrow$ il ribosoma potrà tradurre rapidamente la sequenza del peptide leader. In questo modo la regione 2 dell'mRNA non si può appaiare con la regione 3. Si forma una struttura a loop nelle regioni 3 e 4 che blocca l'attvità della RNA polimerasi. 
    \item In caso di carenza di Trp $\rightarrow{}$ il peptide leader verrà sintetizzato più lentamente. Questo consente la formazione di una struttura ansa-stelo alternativa (tra le regioni 2 e 3) che previene la formazione dell'ansa-stelo di terminazione (tra le regioni 3 e 4). La trascrizione e la successiva traduzione procedono normalmente, portando così alla biosintesi del triptofano. 
\end{itemize}
Tutto dipende dalla velocità con cui i ribosomi possono tradurre. 
\subsection{Trasduzione di segnale e sistemi di regolazione a 2 componenti}
Il sistema è composto da due componenti principali:
\begin{enumerate}
    \item una proteina di membrana, che ha il ruolo di sensore;
    \item una proteina regolatrice della risposta. 
\end{enumerate}
La proteina sensore ha un'attività di tipo chinasi ed è in grado di autofosforilarsi. Il gruppo fosforico viene poi trasmesso alla proteina regolatrice della risposta. Questa proteina è generalmente una proteina di legame al DNA che regola la tracrizione. 
\\Questo è un tipo di sistema dinamico che per funzionare necessita di avere la possibilità di ritornare sempre al punto di partenza. Questo è reso necessario dalla fosfatasi. La fosfatasi è un enzima che rimuove il gruppo fosforixo della proteina regolatrice per ripristinare il sistema.
\\Un esempio di questo meccanismo è la regolazione della chemiotassi.
\begin{itemize}
    \item \textbf{MCP} $\xrightarrow{}$ può rilevare una varietà di composti attraenti e/o repellenti. Svolge quindi il ruolo di sensore. Le MCP si possono presentare libere oppure legate agli attraenti/repellenti direttamente o indirettamente tramite interazioni con proteine di legame periplasmatiche. 
    \item \textbf{CheW} e \textbf{CheA} $\xrightarrow{}$ sono delle chinasi sensore. Si autofosforilano quando MCP si lega ad una determinata sostanza. 
    \item \textbf{CheY} e \textbf{CheB} $\xrightarrow{}$ regolatori della risposta. A loro viene ceduto il gruppo fosfato portato da CheA. 
\end{itemize}
\subsubsection{Controllo della rotazione del flagello}
Il controllo della rotazione del flagello avviene tramite \textbf{CheY-P}. Esso induce il motore del flagello a ruotare in senso orario. Se \textbf{CheY} non è fosforilato o viene defosforilato da \textbf{CheZ}, non può legarsi al motore del flagello che continua a ruotare in senso antiorario. 
\\Gli attraenti diminuiscono la frequenza di autofosforilazione e quindi consentono alla cellula di proseguire nel suo moto di avanzamento regolare; mentre i repellenti aumentano la frequenza e questo permette al flagello di cambiare frequentemente la direzione della corsa per allontanarsi dalla sostanza. 
\subsubsection{Il processo di adattamento}
Questo processo consente di ripristinare il sistema di regolazione. \textbf{CheR} aggiunge continuamente gruppi metilici alle MCP, mentre \textbf{CheB-P} le rimuove. Anche \textbf{CheB} presenta una fosforilata e una defosforilata. \\Il livello di metilazione delle MCP influenza la loro conformazione e controlla l'adattamento a un dato segnale. 
\\Il fattore decisivo per questo tipo di regolazione viene dato dal cambiamento di concentrzione di attraente o repellente nel tempo, non della concentrazione assoluta. 
