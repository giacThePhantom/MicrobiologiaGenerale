\chapter{Regolazione metabolica}

\section{Introduzione}
Per poter sopravvivere in natura i microrganismi devono essere in grado di rispondere ai cambiamenti delle condizioni ambientali. 
Grazie all'aiuto delle mappe metaboliche si possono notare strutture complesse dove i punti e le linee rappresentano rispettivamente i composti e le reazioni enzimatiche che portano alla loro formazione. 
I due meccanismi centrali sono la glicolisi e il ciclo di Krebs. 
Per coordinare in modo efficiente le loro reazioni chimiche, le cellule devono regolare il tipo e la quantità di macromolecole che sintetizzano. 
La maggior parte dei microrganismi possiede geni che codificano molte più proteine di quante ne siano presenti all'interno della cellula in qualunque condizione di crescita. 
	
	\subsection{Strategie di regolazione}

		\subsubsection{Controllo dell'attivit\`a enzimatica}
		Il controllo dell'attivit\`a enzimatica \`e necessario per evitare di produrre proteine inutili.
		La regolazoine avviene a livello post-traduzionale.

		\subsubsection{Controllo della quantit\`a sintetizzata}
		
			\paragraph{Controllo a livello trascrizionale}
			Il controllo a livello trascrizionale determina la trascrizione di un gene utilizzando fattori che interagiscono con il DNA favorendo o bloccando la trascrizione.

			\paragraph{Controllo a livello traduzionale}
			Il controllo a livello traduzionale determina la traduzione dei trascritti in proteine.

\section{Regolazione dell'attività enzimatica}
	
	\subsection{Inibizione da feedback}
	In questo tipo di regolazione il prodotto finale di una data via biosintetica a inibire l'attivit\`a di un enzima a monte di un pathway.
	L'inibizione \`e reversibile: non appena il prodotto finale si esaurisce la sintesi pu\`o riprendere.
	L'inibizione avviene tramite il legame non covalente del prodotto in un sito allosterico dell'enzima.
	Viene pertanto detta non covalente.
	Il legame della molecola nel sito allosterico causa un cambio conformazionale dell'enzima impedendogli il legame nel sito attivo.

		\subsubsection{Esempi}

			\paragraph{Pathway ramificato}
			In un pathway ramificato i prodotti finali possono controllare l'enzima specifico della loro sintesi.
			Quando tutti i prodotti finali sono presenti in concentrazione sufficiente il pathway viene disattivato.

			\paragraph{Isoenzimi}
			Gli esoenzimi sono enzimi isofunzionali che catabolizzano la stessa reazione ma sono soggetti a sistemi di regolazione indipendenti.
	
	\subsection{Modifica covalente degli enzimi}
	La modifica covalente degli enzimi generalmente per addizione o delezione di piccole molecole alla proteina come \emph{cAMP}, \emph{ADP}, \emph{$PO_4^{2-}$} o \emph{$CH_3$}.
	Il legame covalente del gruppo modificante provoca un cambiamento conformazionale della proteina modificandone l'attività enzimatica. 
	Viene detta covalente a causa della natura del legame che si forma tra enzima e molecola.

		\subsubsection{Esempi}

			\paragraph{Glutammina sintetiasi}
			L'enzima glutammina sintetasi viene modificato con l'aggiunta di gruppo \emph{AMP}.
			Possiede $12$ siti di legami per questa molecola e l'attivit\`a enzimatica diminuisce con l'aumentare dei gruppi legati.

	\subsection{Processamento delle proteine}
	Alcune proteine batteriche vengono attivate da dei meccanismi post-traduzionali come il ``protein splicing''.
	Queste proteine contengono segmenti simili agli introni detti inteine che devono essere eliminati per rendere la proteina funzionale.

		\subsubsection{Esempi}

			\paragraph{Girasi}
			Le girasi sono attivate da un taglio proteolitico.

			\paragraph{Proteasi per la degradazione dei peptidi}
			Le proteasi per la degradazione dei peptidi sono attivate solo quando escono dalla cellula.

\section{Regolazione a livello trascrizionale}
	
	\subsection{DNA binding proteins \emph{DBP}}
	Le DNA binding proteins sono proteine che legano il DNA. 
	Sono tipicamente fattori di trascrizione positivi e negativi che possono attivare o inattivare la trascrizione.
	Sono composte da dimeri o tetrameri che si legano al DNA in determinati punti specifici.
	Il riconoscimento della basi avviene tramite interazioni non covalenti.

		\subsubsection{Domini di  \emph{DBP}}

			\paragraph{Elica-giro-elica}
			Il dominio elica-giro-elica \`e costituito da amminoacidi in grado di creare una struttura ad $\alpha$-elica composta da un'elica di riconoscimento del DNA e una stabilizzante.

			\paragraph{Zing finger}
			Il dominio zinc finger lega uno ione zinco e presenta due strutture a dito che riconoscono il DNA.
			Lo zinco \`e legato a due cisteine e due istidine.

			\paragraph{Leucine zipper}
			Il dominio leucine zipper contiene regioni con residui di leucina regolarmente ripetuti che formano una cerniera.
			Non interagisce con il DNA direttamente ma tiene unite due $\alpha$-elica nella posizione corretta.
			Le $\alpha$-eliche interagiscono con il DNA riconoscendolo.

	\subsection{Controllo negativo della trascrizione}
	Si intende per controllo negativo un controllo effettuato tramite l'azione di un repressore. 
	Induttori e repressori agiscono in modo indiretto: si combinano con le \emph{DBP} che influenzano la sintesi del mRNA.

		\subsubsection{Repressione}
		Nella repressione gli enzimi che catalizzano la sintesi di uno specifico prodotto non vengono sintetizzati se questo \`e presente nel mezzo in quantit\`a sufficiente.
		Un esempio \`e la produzione di arginina.

			\paragraph{Meccanismo di repressione}
			Il meccanismo di repressione utilizza una molecola co-repressore, tipicamente il prodotto del pathway che si vuole controllare.
			Quando questo \`e presente in quantit\`a sufficiente la sintesi del mRNA viene bloccata per risparmiare energia.
			Il repressore viene pertanto attivato in presenza del co-repressore: il suo legame con il repressore causa un suo cambio conformazionale che porta a un forte aumento di affinit\`a per il DNA.
			Il repressore si attacca al filamento e causa un ingombro sterico che impedisce alla RNA polimerasi di associarsi al DNA per cominciare la trascrizione.
			Queste reazioni sono reversibili: in mancanza del co-repressore la cellula deve essere in grado di ricominciare la traduzione.

		\subsubsection{Induzione}
		Nell'induzione la sintesi ha luogo solo quando \`e presente il substrato del pathway regolato.
		Un esempio \`e la catabolisi del lattosio.

			\paragraph{Meccanismo di induzione}
			Il meccanismo di induzione utilizza anch'esso una molecola secondaria detta induttore.
			Il repressore si trova legato stabilmente al DNA.
			L'induttore, quando presente, si lega al repressore causando ad esso un cambio conformazionale che riduce la sua affinit\`a per il DNA: il repressore si stacca da esso.
			In questo modo l RNA polimerasi pu\`o trascrivere l'operone necessario.

	\subsection{Controllo positivo della trascrizione}
	Il controllo positivo della trascrizione coinvolge proteine che promuovono il legame della RNA polimerasi al promotore ed all'aumento della sintesi di mRNA. 
	Un esempio \`e la proteina attivatrice del pathway del maltosio, che non pu\`o legarsi al DNA se questo non \`e presente.
	I promotori degli operoni controllati positivamente hanno una sequenza che non assomiglia fedelmente alla sequenza consenso per la RNA polimerasi: anche in presenza del fattore $\sigma$ corretto ha difficoltà a riconoscerli. 
	La proteina attivatrice aiuta la RNA polimerasi a riconoscere il promotore. 
	La RNA polimerasi viene infatti scortata dalla proteina attivatrice attivata verso il DNA.
	Se il sito di legame dell'attivatore e il promotore non sono vicini si formano nel DNA strutture a loop che li avvicinano.

	\subsection{Sistemi di controllo globale}
	I sistemi di controllo globale sono meccanismi di regolazione che rispondono a dei segnali ambientali regolando l'espressione di molti geni diversi. 

		\subsubsection{Effetto glucosio}
		Questo sistema viene chiamato anche repressione da catabolita.
		In queste condizioni si può osservare una curva di crescita con $2$ fasi esponenziali. 
		Questo tipo di crescita è detta crescita diauxica. 
		Quando \textit{E. coli} si trova in presenza di varie fonti di carbonio, il glucosio viene sempre utilizzato per primo dato che è quello pi\`u facilmente metabolizzabile. 
		Così facendo gli enzimi che catabolizzano altri substrati vengono repressi. 
		Quando non è più presente abbastanza glucosio la crescita si ferma per poi ricominciare utilizzando il lattosio o un'altra fonte di carbonio. 
		Questa curva è meno veloce perchè il processamento è meno veloce. 
		
			\paragraph{Attivatore proteico del catabolismo}
			L'attivatore proteico del catabolismo \emph{CAP} \`e una proteina che permette il legame della RNA polimerasi al DNA per gli enzimi reprimibili da catabolita.
			\`E una proteina allosterica che si pu\`o legare al DNA solo se prima viene legata da \emph{cAMP}.
			In presenza di glucosio la sintesi di \emph{cAMP} da parte dell'adenilato ciclasi viene inibita, causando una diminuizione dei livelli di \emph{cAMP}.
			La RNA polimerasi non \`e in gradi di legarsi ai promotori dei geni regolati da catabolita.

			\paragraph{Assenza di glucosio e lattosio}
			In assenza di glucosio e lattosio i livelli di \emph{cAMP} sono molto alti.
			\emph{CAP} si lega sul sito a monte del promotore.
			Il lattosio non presente attiva il repressore dell'operone \emph{lac}.
			Nonostante le alte concentrazioni di \emph{cAMP} la trascrizione viene repressa.

			\paragraph{Presenza di glucosio ma assenza di lattosio}
			In presenza di glucosio e assenza di lattosio \`e presente il repressore che inattiva \emph{lac}.
			Inoltre i livelli di \emph{cAMP} sono bassi e \emph{CAP} non si lega al sito sul DNA.
			Viene repressa la trascrizione dei geni.

			\paragraph{Presenza di glucosio e lattosio}
			In presenza di glucosio e lattosio la presenza del lattosio provoca la rimozione del repressore, ma la trascrizione sar\`a a livelli basali a causa dei bassi livelli di \emph{cAMP} dovuti agli alti livelli di glucosio.

			\paragraph{Assenza di glucosio ma presenza di lattosio}
			In presenza di lattosio ma assenza di glucosio la cellula rimuove il repressore su \emph{lac}, \emph{CAP} si lega al sito di attacco grazie agli alti livelli di \emph{cAMP} e la trascrizione viene attivata in modo efficiente.

		\subsubsection{Risposta stringente}
		La carenza di amminoacidi causa un blocco della sintesi degli rRNA e dei tRNA.
		In questo modo non vengono pi\`u assemblati nuovi ribosomi.

			\paragraph{Induzione della risposta stringente}
			La risposta stringente viene indotta da $2$ nucleotidi di guanina modificati con gruppi fosfato:
			\begin{multicols}{2}
				\begin{itemize}
					\item \emph{ppGpp}.
					\item \emph{pppGpp}.
				\end{itemize}
			\end{multicols}
			In carenza di aminoacidi un tRNA scarico può legarsi al ribosoma.
			Questo attiva la proteina \emph{RelA} che idrolizza gruppi fosfato da \emph{GTP} e \emph{ATP} produgendo \emph{ppGpp} e \emph{pppGppp}.

				\subparagraph{Effetto dei nucleotidi modificati}
				\emph{ppGpp} e \emph{pppGppp} inibiscono la sintesi di rRNA e tRNA interferendo con la RNA polimerasi per l'inizio della trascrizione in corrispondenza dei geni corrispondenti e attiva operoni deputati alla sintesi degli amminoacidi mancanti integrando cofattori di trascrizione.

\section{Altri sistemi di controllo globale}

	\subsection{Fattori $\mathbf{\sigma}$ alternativi}
	Il fattore $\sigma$ è una proteina che si attacca alla RNA polimerasi per facilitare l'attacco al DNA dato che riconosce particolare sequenze specifiche. 
	Molti dei geni impegnati nei processi di controllo globale utilizzano fattori $\sigma$ alternativi. 
	La regolazione è determinata dalla concentrazione o dall'attività di fattori antagonisti \emph{ant $\sigma$} dei diversi fattori $\sigma$ in quanto ogni fattore riconosce nel genoma soltanto un certo gruppo di promotori. 
	Ci sono $7$ fattori $\sigma$ in \textit{E. coli}, $14$ in \textit{B.subtilis} dei quali $4$ sono specifici per i geni necessari alla formazione dell'endospora. 
	Il fattore $\sigma^{70}$ è il fattore più diffuso. 


	\subsection{Risposta allo shock termico}
	Nel caso di uno shock termico viene inibita la degradazione del fattore $\sigma^{32}$. 
	Questo dà luogo a una massiccia sintesi di proteine codificate da geni sotto il controllo di $\sigma^{32}$, in particole le proteine heat shock \emph{Hsp}.
	\begin{multicols}{2}
	\begin{itemize}
    		\item La proteina \emph{Hsp70} \emph{DnaK} previene l'aggregazione delle proteine neosintetizzate e stabilizza quelle ancora non ripiegate.
    		\item Le proteine \emph{Hsp60} \emph{GroEL} e \emph{Hsp10} \emph{GroEs} catalizzano il corretto ripiegamento di proteine ripiegate erroneamente. 
	\end{itemize}
	\end{multicols}

	\subsection{Quorum sensing}
	Il quorum sensing \`e un sistema di regolazione dipendente dalla percezione della densità delle cellule della stessa specie presenti nella popolazione. 
	Permette alle cellule di attivare una particolare risposta biologica soltanto quando un numero sufficiente di cellule sono presenti nell'ambiente circostante.

		\subsubsection{Omoserina lactone}
		L'omoserina lactone \emph{AHL} viene sintetizzata dalle specie dotate di quorum sensing e diffonde all'esterno della cellula.
		\`E un induttore che quando raggiunge una data concentrazione, si combina ad una proteina attivatrice permettendo l'avvio della trascrizione di geni specifici. 

			\paragraph{esempi}

				\subparagraph{Regolazione della luminescenza in Vibrio fischeri}
				La luminescenza in Vibrio fischeri \`e sotto il controllo dell'operone \emph{lux} che viene indotto dalla proteina \emph{LuxR} in complesso con \emph{AHL}.

				\subparagraph{Formazione di biofilms di Pseudomonas aeruginosa}
				Pseudomonas aeruginosa crea biofilm in presenza di \emph{AHL}.

				\subparagraph{Produzione di fattori di virulenza in Staphylococcus aureus}
				Staphylococcus aureus produce fattori di virulenza in presenza di \emph{AHL}.

	\subsection{Sistema per la sintesi di triptofano}
	Il sistema per la sintesi di triptofano si basa sulla presenza di un peptide leader (L) a monte dell'operone deputato alla biosintesi di uno specifico aminoacido. 
	Il peptide leader \`e ricco di triptofano.
	Questo fenomeno avviene nei procarioti in quanto trascrizione e traduzione avvengono simultaneamente.
	Il processo dipende dalla velocit\`a con cui traducono i ribosomi.

		\subsubsection{Triptofano abbondante}
		In condizioni di triptofano abbondante il ribosoma riesce a tradurre rapidamente la sequenza del peptide leader, in questo modo la regione $2$ non si pu\`o appaiare con la regione $3$.
		Si forma pertanto una struttura a loop nelle regioni $3$ e $4$ che blocca l'attivit\`a della RNA poliemrasi.

		\subsubsection{Carenza di triptofano}
		In caso di carenza di triptofano il peptide leader viene sintetizzato lentamente e nasce una struttura ansa-stelo alternativa nella regione $2$-$3$ che impedisce la formazione dell'ansa nel sito di terminzione $3$-$4$: la trascrizione e traduzione procedono normalmente portando alla sintesi di triptofano.


	\subsection{Trasduzione di segnale e sistemi di regolazione a due componenti}

		\subsubsection{Componenti}

			\paragraph{Proteina di membrana}
			La proteina di membrana, con il ruolo di sensore ha un'attivit\`a chinasica ed \`e in grado di autofosforilarsi.
			Il gruppo fofsato viene trasmesso alla proteina regolatrice della risposta.

			\paragraph{Proteina regolatrice della risposta}
			La proteina regolatrice della risposta \`e una proteina che lega il DNA e regola la trascrizione.
			La sua attivit\`a \`e regolata da gruppi fosfato.

			\paragraph{Fosfatasi}
			Le fosfatasi sono enzimi che rimuovono il gruppo fosfato dalla proteina regolatrice.
			Riportano pertanto il sistema allo stato di riposo.

			
		\subsubsection{Chemiotassi}
		La chemiotassi \`e un esempio di sistema di regolazione a due componenti.
		\emph{MCP} rileva una variet\`a di composti attraenti o repellenti e svolge il ruolo di sensore.
		Possono legare direttamente le sostanze attraenti o repellenti o interagire con proteine di legame periplasmatiche.
		\emph{CheW} e \emph{CheA} sono chinasi sensore che si autofosforilano quando \emph{MCP} si lega a una sostanza.
		\emph{CheY} e \emph{CheB} sono regolatori della risposta a cui viene ceduto il gruppo fosfato di \emph{CheA}.

			\paragraph{Controllo della regolazione del flagello}
			La rotazione del flagello viene controllata da \emph{CheY-P}.
			Questa proteina induce al motore del flagello di ruotare in senso orario.
			\emph{CheY} non fosforilato o defosforilato da \emph{CheZ} non pu\`o legarsi al motore del flagello che continua a ruotare in senso antiorario.

			\paragraph{Attraenti}
			Gli attraenti diminuiscono la frequenza di autofosforilazione e consentono alla cellula di proseguire nel suo moto di avanzamento regolare.

			\paragraph{Repellenti}
			I repellenti aumentano la frequenza di autofosforilazione permettendo al flagello di cambiare frequentemente la direzione della corsa per allontanarsi dalla sostanza.

		\subsubsection{Il processo di adattamento}
		Il processo di adattamento consente alla cellula di ripristinare il sistema di regolazione.
		\emph{CherR} aggiunge gruppi metilici a \emph{MCP}, mentre \emph{CheB-P} li rimuove.
		Il livello di metilazione di \emph{MCP} influenza la conformazione e controlla l'adattamento a un dato segnale: concentrazioni persistenti di una sostanza causano una riduzione della risposta.
		Il fattore decisivo per questo tipo di regolazione \`e pertanto il cambiamento di concentrazione di attraente o repellente nel tempo.
