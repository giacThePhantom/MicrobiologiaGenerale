\chapter{Controllo della crescita microbica}

\section{Introduzione}
Il controllo della crescita microbica tiene conto delle caratteristiche biologiche dei microrganismi da selezionare e non dell'ambiente in cui vivono. 
	
	\subsection{Metodi di controllo}

		\subsubsection{Sterilizzazione}
		La sterilizzazione consiste nell'uccisione o rimozione di tutti gli organismi vivi all'interno di un terreno di crescita.

		\subsubsection{Inibizione}
		L'inibizione consiste nella riduzione della crescita microbica causata da diminuzione del numero di organismi presenti o da alterazioni nell'ambiente microbico.

		\subsubsection{Decontaminazione}
		La decontaminazione consiste nel trattamento di oggetti o superfici che permette il loro utilizzo senza rischio di contaminazione.

		\subsubsection{Disinfezione}
		La disinfezione \`e un processo che colpisce direttamente i microrganismi, uccidendone o inibendone la crescita.

		\subsubsection{Pastorizzazione}
		La pastorizzazione consiste nella riduzione della carica microbica nei liquidi sensibili al calore con lo scopo di distruggere tutti i microrganismi patogeni e ridurre il numero dei microrganismi responsabili del deterioramento degli alimenti.

	\subsection{Categorizzazione di agenti o trattamenti antimicrobici in base al loro effetto sul batterio}
	Si possono dividere gli agenti o i trattamenti antimicrobici in base all'effetto che hanno sul batterio.

		\subsubsection{Battericida}
		Un agente battericida \`e un agente antimicrobico che uccide i microrganismi. 
		La conta vitale subisce una deflessione, mentre la conta totale rimane costante. 
		Le cellule morte non perdono integrit\`a.

		\subsubsection{Batteriolitico}
		Un agente batteriolitico \`e un agente antimicrobico che uccide i microrganismi provocandone la lisi. 
		Avviene una totale distruzione della cellula: si nota sia un calo della conta vitale sia nella conta totale.

		\subsubsection{Batteriostatico}
		Un agente batteriostatico \`e un agente antimicrobico che inibisce la crescita dei microrganismi. 
		Dopo la sua aggiunta, la conta totale e la conta vitale rimangono costanti.

\section{Metodi fisici}
\`E possibile controllare la crescita microbica attraverso trattamenti fisici solo se vengono applicati a oggetti o terreni ci coltura.
Non si pu\`o applicare a sistemi in vivo in quanto portano alla morte dell'organismo.

	\subsection{Calore}

		\subsubsection{Caldo umido}
		Esempi di controllo della crescita attraverso caldo umido sono bollitura, autoclave, pastorizzazione e sterilizzazione.

			\paragraph{Sterilizzazione mediante calore}
			La letalit\`a dovuta all'incremento della temperatura \`e una funzione esponenziale: il tempo necessario per uccidere una determinata frazione di cellule \`e indipendente dal numero di cellule iniziale.
			
				\subparagraph{Tempo di riduzione decimale}
				Viene definito il tempo di riduzione decimale come il tempo necessario per ridurre di dieci volte a una data temperatura la densit\`a della popolazione.

				\subparagraph{Grafici}
				Nei grafici viene utilizzato sull'asse delle ascisse il tempo in scala lineare.
				Sull'asse delle ordinate si trova il numero di microbi in scala logaritmica.
				Si nota come le sostanze e i trattamenti battericidi uccidono una percentuale costante di cellule per ogni intervallo di tempo.
				Questo si nota tracciando una retta che rappresenta la decrescita esponenziale e indica un tasso di morte costante.
				Una diminuzione di un ordine di grandezza corrisponde a una diminuzione del $90\%$.
				Si nota come gli organismi che resistono meglio al calore sono quelli che formano endospore.

				\subparagraph{Tempo di inattivazione termica}
				Il tempo di inattivazione termica indica il tempo necessario affinch\`e tutte le cellule di una coltura siano morte a causa del calore.
				Dipende dal numero di cellule iniziale.

			\paragraph{Autoclave}

				\subparagraph{Composizione}
				L'autoclave \`e un macchinario composto da una camera a chiusura ermetica che permette l'immissione di vapore sotto pressione.
				\`E presente inoltre un tubo per l'immissione del vapore e un tappo ermetico.
				Un sistema di raffreddamento velocizza il processo a $20$ minuti.

				\subparagraph{Procedura}
				La procedura di utilizzo di un'autoclave prevede il riscaldamento a pressione di $1.1\frac{\si{kg}}{\si{cm^2}}$ che consente di raggiungere una temperatura di $121\si{\degree}$.
				La morte dei microorganismi viene provocata dall'alta temperatura che viene raggiunta in condizioni di vapore.
				Se non venisse aumentata la pressione la temperatura di ebollizione sarebbe $100\si{\degree}$, che non permetterebbe l'uccisione delle endospore.
				La temperatura dell'oggetto aumenta pi\`u lentamente rispetto a quella dell'autoclave.
				Per verificare la sterilizzazione si pu\`o utilizzare un nastro che diventa di colore nero se la sterilizzazione \`e avvenuta nel modo corretto.
				Un altro metodo \`e un'ampolla con un terreno di crescita con indicatore di $pH$.
				Durante l'autoclavaggio si rompe l'ampolla e l'indicatore viene esposto al terreno di coltura.
				Se dopo il ciclo l'indicatore resta rosso le endospore sono morte.
			
			\paragraph{Pastorizzazione}
			La pastorizzazione è un processo che utilizza una temperatura controllata per ridurre la carica microbica nel latte e in altri alimenti particolarmente sensibili al calore. 
			Lo scopo di questo processo \`e prevenire la diffusione di patogeni e procrastinare della crescita di microrganismi responsabili del deterioramento degli alimenti. 
			
				\subparagraph{Batteri uccisi dalla pastorizzazione}\mbox{}\\
				\begin{multicols}{2}
					\begin{itemize}
						\item \textit{Listeria monocytogenes}.
						\item \textit{Escherichia Coli} O15:H7.
						\item Specie di \textit{Campylobacter}.
						\item \textit{Salmonella}.
					\end{itemize}
				\end{multicols}

				\subparagraph{Pastorizzazione istantanea}
				Nella pastorizzazione istantanea piccoli volumi vengono riscaldati a $72\si{\degree}$ per $15$ secondi.

				\subparagraph{Pastorizzazione di massa}
				Nella pastorizzazione di massa grandi volumi vengono riscaldati a $63$-$66\si{\degree}$ per $30$ minuti.

				\subparagraph{Pastorizzazione \emph{UHT}}
				Nella pastorizzazione \emph{UHT} o flash heating i volumi vengono colpiti da un getto di vapore a $134\si{\degree}$ per $1$ secondo.

				\subparagraph{Sterilizzazione \emph{UHT}}
				Nella sterilizzazione \emph{UHT} i volumi vengono colpiti da un getto di vapore a $140\si{\degree}$ per $1$-$3$ secondi.
				Il processo viene seguito da un rapido raffreddamento.
		
		\subsubsection{Caldo secco}
		Esempi di controllo della crescita attraverso caldo secco sono aria calda o incenerizzazione.
		\`E meno efficace rispetto al caldo umido in quanto l'acqua ha una capacit\`a calorifica maggiore dell'aria ed \`e pertanto un veicolo pi\`u efficiente per trasmettere la temperatura.
		Viene utilizzato per sostanze che tollerano umidit\`a come le polveri.
		Avviene tipicamente a $170\si{\degree}$ per un'ora.

		\subsubsection{Refrigerazione e congelamento}
		Il freddo rallenta o ferma il metabolismo microbico.
			
			\paragraph{Refrigerazione}
			La refrigerazione $0$-$8\si{\degree}$ inibisce la crescita della maggioranza dei patogeni umani.

			\paragraph{Congelamento}
			Il congelamento $<0\si{\degree}$ lento consente la formazione di cristalli di ghiaccio che danneggiano le membrane dei microorganismi.
			\`E per questo motivo migliore rispetto ad un congelamento rapido.

		\subsubsection{Essiccazione e liofilizzazione}
		I processi di liofilizzazione ed essiccazione inibiscono la crescita microbica in quanto le reazioni metaboliche si svolgono in soluzione acquosa.
		La liofilizzazione combina il congelamento rapido con azoto liquido seguito dalla rimozione dell'acqua tramite sublimazione.

	\subsection{Trattamenti fisici alternativi}

		\subsubsection{Sterilizzazione con radiazioni}

			\paragraph{Sterilizzazione mediante radiazioni non ionizzanti}
			Le radiazioni non-ionizzanti hanno una lunghezza d'onda maggiore di $1 nm$.
			Vengono utilizzati raggi \emph{UV} in quanto sufficientemente energetici.
			Inducono la formazione di dimeri di pirimidina nel DNA.
			Non penetrano nella materia e sono usate per sterilizzare le superfici, acqua e liquidi trasparenti.

			\paragraph{Sterilizzazione mediante radiazioni ionizzanti}
			Le radiazioni ionizzanti hanno una lunghezza d'onda minore di $1 nm$, possono produrre ioni che interagiscono con le macromolecole biologiche rompendo legami idrogeno o producendo ioni \emph{$OH^+$}.
			L'irradiazione con fascio di elettroni \`e molto efficace ma non penetra nella materia.
			Lo standard per la sterilizzazione \`e dato dalla dose di radiazione assorbita.
			Viene espressa in $\si{rad}$ o in $\si{Gy}$ (Gray).
			La relazione tra frazione di sopravvivenza microbica su scala logaritmica e dose di radiazioni \`e lineare.
			La dose letale standard per una sterilizzazione completa \`e $12$ volte \emph{D10} ed \`e:
			\begin{multicols}{2}
    				\begin{itemize}
        				\item $39600 Gy$ per \textit{Clostridium botulinum}. 
        				\item $2400 Gy$ per \textit{Salmonella typhimurium}.
        				\item $10 Gy$ per l'uomo.
    				\end{itemize}
			\end{multicols}

		\subsubsection{Sterilizzazione per filtrazione}
		La sterilizzazione per filtrazione viene utilizzata per sterilizzare soluzioni sensibili al calore.
		Necessita di usare un dispositivo in grado di trattenere microorganismi di dimensioni comprese tra $0.3$ e $10\si{\micro\metre}$.

			\paragraph{Filtri a spessore}
			I filtri a spessore sono costituiti da strati fibrosi di carta, amianto o lana di vetro.
			Vengono usati come pre-filtri per la rimozione delle particelle di maggiore dimensione che potrebbero intasare quelli utilizzati nel processo di sterilizzazione.
			Viene usata nei sistemi di condizionamento, filtri \emph{HEPA} delle cappe biologiche.
			
				\subparagraph{Cappe biologiche}
				Nelle cappe biologiche si crea un flusso di aria laminare che evita che l'aria contaminata e quella di laboratorio si mescolino.
				Il flusso passa attraverso i filtri \emph{HEPA} che trattengono i microbi permettendo un sistema di ricircolo dell'aria.

			\paragraph{Membrane filtranti}
			Le membrane filtranti vengono usate per la sterilizzazione di liquidi.
			Sono costituite da dischetti di acetato di cellulosa o nitrocellulosa.
			Modificando le condizioni di polimerizzazione pu\`o essere controllata la dimensione dei pori, che va da $0.1$ a $10\si{\micro\metre}$.
			Esistono diverse membrane filtranti in base a se si vogliono filtrare batteri di grandi dimensioni o tutti i batteri e i virus pi\`u grandi.

			\paragraph{Filtro nucleopore}
			Il filtro nucleopore viene prodotto trattando un sottile strato di policarbonato con un componente chimico corrosivo.
			Vengono usati per la preparazione di campioni per la microscopia elettronica in quanto l'organismo viene facilmente rimosso dalla fase liquida e viene distribuito su un unico piano della superficie del filtro.
\section{Metodi chimici}
	
	\subsection{Agenti antimicrobici}
	Un agente antimicrobico \`e un composto chimico che uccide i microbi o ne inibisce la crescita.
	Il suffisso ``-cida'' indica un agente antimicrobico che uccide i microorganismi, mentre il suffisso ``-statico'' indica un agente che inibisce la crescita dei microorganismi.
	
		\subsubsection{Fenolo}
		Il fenolo e i derivati fenolici hanno attivit\`a antisettica di medio-basso livello: denaturano le proteine e distruggono la membrana.

		\subsubsection{Alcol}
		Gli alcol sono battericidi, funghicidi e virucidi.
		Non hanno effetto contro le endospore.
		Sono disinfettanti di livello intermedio come isopropanolo e etanolo.
		Distruggono le membrane cellulari e denaturano le proteine.

		\subsubsection{Alogeni}
		Gli alogeni sono disinfettanti di livello intermedio e sono attivi contro batteri vegetali, funghi, loro spore, alcune endospore, cisti dei protozoi e molti virus.
		Denaturano le proteine.

		\subsubsection{Agenti ossidanti}
		I perossidi, ozono e peracidi uccidono i microbi ossidando i loro enzimi e rendendoli indisponibili per il metabolismo.

		\subsubsection{Surfattanti}
		I surfattanti sono prodotti chimici che riducono la tensione superficiale dell'acqua riducendo l'attrazione tra le molecole.
		Si possono aggregare in micelle come saponi e detergenti.
		Risciacquano i batteri ma non li uccidono.

		\subsubsection{Metalli pesanti}
		Gli ioni di metalli pesanti sono antimicrobici in quanto si combinano con lo zolfo nella cisteina alterando la tridimensionalit\`a della proteina e la sua funzione.

		\subsubsection{Aldeidi}
		Le aldeidi sono composti organici che denaturano le proteine e inattivano gli acidi nucleici.

		\subsubsection{Agenti gassosi}
		Alcuni gas che denaturano le proteine e DNA sono ossido di etilene, ossido di propilene e il $\beta$-propriolattone.

	\subsection{Controllo della crescita in vivo}

		\subsubsection{Scoperta della penicillina}
		Nel 1929 Alexander Fleming riporta per la prima volta l'azione antibatterica della penicillina, prodotta da \textit{Penicillium chrysogenum} un fungo.
		
		\subsubsection{Antibiotici}
		Gli agenti che vengono utilizzare per controllare la crescita batterica in vivo, che sia per uso clinico o veterinario, sono detti antibiotici. 
		Esistono antibiotici naturali, semisintetici e sintetici. 
		Gli antibiotici agiscono solamente contro i batteri.
		Vengono prodotti da funghi o batteri.

			\paragraph{Caratteristiche degli antibiotici}\mbox{}\\
			\begin{multicols}{2}
				\begin{enumerate}
    					\item Sono farmaci salva-vita.
    					\item Trattano solamente infezioni batteriche. 
    					\item Ci sono molti rischi se vengono assunti senza prescrizione medica.
				\end{enumerate}
			\end{multicols}

			\paragraph{Terapia antimicrobica in vivo}
			La terapia antimicrobica in vivo è molto complesssa: è difficile trovare delle che molecole che rispettino tutte le caratteristiche necessarie. 
				
				\subparagraph{Caratteristiche}\mbox{}\\
				\begin{multicols}{2}
					\begin{itemize}
						\item Tossicit\`a selettiva: uccisione di microbi e preservazione dell'organismo ospite.
						\item Drug delivery: deve raggiungere la regione bersaglio senza essere alterata.
						\item Deve arrivare al sito target con la giusta concentrazione.
					\end{itemize}
				\end{multicols}

				\subparagraph{Complicazioni}\mbox{}\\
				\begin{multicols}{2}
					\begin{itemize}
    						\item Ritezione del farmaco: l'ospite può degradare o disattivare l'antibiotico. 
    						\item Utilizzo di un farmaco errato: bisogna conoscere lo spettro di resistenza prima di utilizzarlo. 
    						\item Problemi di delivery. 
    						\item Tossicità sull'ospite, reazioni allergiche, alterazioni del microbioma interno.
					\end{itemize}
				\end{multicols}

				\subparagraph{Problematiche}\mbox{}\\
				\begin{multicols}{2}
					\begin{itemize}
    						\item Pericolo di sviluppo di resistenza.
    						\item Pericolo di rilascio di tossine dopo la lisi. 
							Esotossine e endotossine provenienti dalla parte lipidica che portano ad un peggioramento nell'ospite dopo il trattamento.
							In particolare da Gram$-$.
					\end{itemize}
				\end{multicols}

				\subparagraph{Accorgimenti}\mbox{}\\
				\begin{multicols}{2}
					\begin{itemize}
						\item Uso corretto e per il giusto periodo.
    						\item Possibilità di utilizzare varie tipologie di antibiotici in cicli sequenziali.
					\end{itemize}
				\end{multicols}

				\subparagraph{Caratteristiche dell'antibiotico ideale}\mbox{}\\
				\begin{multicols}{2}
					\begin{itemize}
    						\item Disponibilità.
    						\item Basso costo: deve essere facile da preparare, poco dispendioso e molto vendibile.
    						\item Chimicamente stabile per trasporto e stoccaggio.
    						\item Semplice da somministrare.
    						\item Non allergenico.
    						\item Tossicità selettiva.
					\end{itemize}
				\end{multicols}
				Nessuna molecola possiede contemporaneamente queste caratteristiche.

				\subparagraph{Super infections}
				Si intende per super infections infezioni dovute a microbi alterati in modo da essere resistenti agli antibiotici o che possono causare infezioni secondarie.

\section{Misurazione dell'attività antimicrobica}

	\subsection{Test di Kirby-Bauer}
	Il test di Kirby-Bauer viene utilizzato per determinare la suscettibilità o la resistenza di un determinato batterio a una cura con un certo antibiotico. 
	Viene formato un tappetto uniforme di batteri in cui vengono posti dei dischetti imbevuti di antibiotici diversi. 
	Poi si misurano gli aloni di inibizione, che sono le zone che si trovano intorno al disco in cui i batteri non sono stati in grado di crescere. 
	Va comunque indagata anche la sopportazione dell'organismo all'agente usato.

	\subsection{Test di Minimum Inhibitory Concentration \emph{MIC}}
	Si intende per \emph{MIC} la minore concentrazione di un antibiotico in grado di inibire la crescita batterica. 
	Vengono poste uguali quantità di batteri in una serie di tubi in cui vengono aggiunte delle soluzioni con differenti concentrazioni di antibiotici diversi. 
	Una volta avvenuta l'incubazione, la torbidità indica la crescita batterica e quindi la sua assenza individua i tubi dove i batteri o sono stati uccisi o la replicazione è stata bloccata. 
	Per misurare la MIC viene utilizzato il metodo dell'assorbanza. 

	\subsection{Etest}
	L'Etest associa le caratteristiche del test di Kirby-Bauer con il test MIC. 
	Si basa su una coltura in terreno solido dove viene applicato uno striscio che contiene concentrazioni differenti di antibiotico. 
	Data la concentrazione diversa si ottiene un alone di inibizione a forma di goccia. 
	Questo permette di avere una lettura immediata per trovare il MIC dove comincia la goccia di inibizione.

	\subsection{Minimum bactericidal concentration (MBC)}
	Il test \emph{MBC} \`e un estensione del test MIC e determina la concentrazione minima necessaria di antibiotico per uccidere i batteri presenti. 
	Si prendono le provette utilizzate per MIC e le si piastra su terreni di coltura senza antibiotico. 
	Questo assicura di trovarsi in presenza di antibiotici battericidi che portano alla morte di tutti i batteri e non solo di quello batteriostatici, che vanno ad inibire la crescita dei microrganismi. 

\section{Meccanismi di azione degli antibiotici}

	\subsection{Tipologie di meccanismi di azione}
	\begin{multicols}{2}
		\begin{itemize}
    			\item Inibizione della sintesi della parete cellulare: $\beta$ lattamici e glicopeptidi.
    			\item Inibizione della sintesi delle proteine: amminoglicosidi.
    			\item Inibizione delle vie metaboliche.
    			\item Inibizione della sintesi di acidi nucleici.
    			\item Inibizione del riconoscimento o attacco del patogeno al suo ospite, colpiscono fattori di virulenza e le patogenicità frequenti. 
		\end{itemize}
	\end{multicols}

	\subsection{Classificazione degli antibiotici}
	\begin{multicols}{2}
		\begin{itemize}
    			\item Aminoglicosidi e tetracicline: inibiscono la sintesi proteica legando la subunità $30S$ dei ribosomi batterici. 
    			\item $\beta$-lattamici e glicopeptidi:  interferiscono con la sintesi della parete batterica.
    			\item Fluoroquinolones: inibiscono l'attività della DNA girasi o topoisomerasi coinvolti nella replicazione del DNA.
    			\item Macrolidi: inibiscono la sintesi proteica legando la subunità $50S$ dei ribosomi batterici.
    			\item Sulfonamidi: inibiscono la biosintesi dell'acido folico.
		\end{itemize}
	\end{multicols}

	\subsection{Inibizione della sintesi della parete cellulare}
	La parete cellulare dei batteri è composta da macromolecole di peptidoglicano formato a sua volta da catene di \emph{NAM-NAG} che sono legate da ponti peptidici tra le subunità di \emph{NAM}.
	Al processo di inibizione della sintesi della parete prendono parte due classi di antibiotici.
	
		\subsubsection{$\mathbf{\beta}$ lattamici}
		Gli antibiotici $\beta$-lattamici come cephalosporine e penicillina rappresentano quasi la metà delle $500$ tonnellate di antibiotici che vengono utilizzati ogni anno. 
		Agiscono solamente sui batteri in fase di divisione. 
		Sono presenti dei composti con gruppi funzionali differenti tra loro. 
		Tutti presentano l'anello $\beta$-lattamico che impedisce la formazione dei legami crociati peptidici tra subunità di \emph{NAM} del peptidoglicano. 
		Si legano con gli enzimi transpeptidasi indebolendo progressivamente la parete in quanto le catene di peptidoglicano non sono più unite tra loro.
		La cellula non è in grado di resistere alla pressione osmotica e va incontro alla lisi.

			\paragraph{Tipologie}

				\subparagraph{Penicilline naturali}
				Un esempio di penicillina naturale \`e la benzilpenicillina.

				\subparagraph{Penicilline semisintetiche}
				Le penicilline semisintetiche non sono presenti in natura e hanno un'attivit\`a pi\`u efficiente rispetto alla penicillina naturale.
				Esempi sono Meticillina, l'Oxacillina, l'Ampicillina e la Carbenicillina.  
				Vantaggi:
				\begin{multicols}{2}
    					\begin{itemize}
        					\item Sono attive anche contro i Gram$-$. 
        					\item Sono resistenti alla $\beta$-lattamasi. 
        					\item Sono più stabili in ambiente acido, come lo stomaco. 
        					\item Vengono assorbite più facilmente dall'epitelio intestinale.
    					\end{itemize}
				\end{multicols}
		\subsubsection{Glicopeptide}
		I glicopeptidi hanno il compito di legare il ponte peptidico in corrispondenza di ogni filamento prima che la transpeptidasi possa agire. 
		Questa classe di antibiotici riconosce i ponti peptidici e si legano ad essi prima che gli enzimi deputati alla loro fusione possano essere in grado di assemblarli. 
		Wi ottiene una cellula debole, che non ha la capacità di far fronte alla forza osmotica.

			\paragraph{Esempi}

				\subparagraph{Vancomicina}
				La vancomicina interferisce con la formazione dei ponti \emph{Ala-Ala} dei legami crociati nei Gram$+$. 

				\subparagraph{Bacitracina}
				La bacitracina blocca la defosforilazione del bactoprenolo, inibendo la secrezione delle subunità \emph{NAM} e \emph{NAG} del citoplasma. 

			\paragraph{Efficacia}
			Questi antibiotici hanno efficacia solo nei confronti di cellule in divisione attiva.
			Un batterio ha la possibilità di diventare resistente a questa classe modificando l'ultimo aminoacido della sequenza del ponte. 
			In questo modo l'antibiotico non è capace di riconoscere più le catene visto che si viene a creare un legame imperfetto.

		\subsubsection{Isoniziade ed etambutolo}
		I batteri del genere \textit{Mycobacterium}, agenti del morbo di Hansen o lebbra e della tuberculosi, presentano una parete cellulare peculiare contenente uno strato di acidi micolici e arabinogalactan. 
		L'isoniziade e l'etambutolo inibiscono la formazione di questo involucro. 

	\subsection{Inibizione della sintesi proteica}
	La tossicità selettiva di alcuni antibiotici si basa sulle lievi differenze strutturali tra ribosomi procariotici ed eucariotici. 
	Metodi innovativi sono acidi nucleici antisenso, DNA o RNA complementari a mRNA codificanti per proteine essenziali.

		\subsubsection{Aminoglicosidi}
		Gli aminoglicosidi alterano la struttura della subunit\`a $30S$ portando alla lettura erronea dei codoni e all'incorporazione di aminoacidi scorretti.
		
		\subsubsection{Tetracicline}
		Le tetracicline bloccano il sito di legame del tRNA impedendo l'allungamento della proteina.

		\subsubsection{Cloramfenicolo}
		Il cloramfenicolo blocca l'attività enzimatica della subunità 50S impedendo la formazione dei legami peptidici. 

		\subsubsection{Macrolidi}
		Le macrolidi come l'eritromicina legano la subunità $50S$ interferendo con il movimento del ribosoma lungo la molecola di mRNA. 

	\subsection{Distruzione della membrana citoplasmatica}
	Alcuni antibiotici agiscono sulla membrana citoplasmatica danneggiandola o distruggendola.
	Sono detti polieni.

		\subsubsection{Anfotericina}
		L'anfotericina \`e un poliene funghicida: si lega a steroidi specifici dei funghi come l'ergosterolo formando pori e causando la lisi cellulare.

		\subsubsection{Azoli e allilamine}
		Azoli e allilamine sono due classi di antibiotici funghicidi.
		Agiscono inibendo la sintesi dell'ergosterolo.
		La membrana cellulare non rimane intatta e la cellula muore.

		\subsubsection{Polioxina}
		La polioxina \`e un antibiotico che si attacca agli \emph{LPS} dei Gram$-$, ma ha effetti tossici sulle cellule dell'ospite.

	\subsection{Inibizione delle vie metaboliche}
	Se esistono differenze metaboliche tra un patogeno e l'ospite possono essere usati antibiotici per colpirle.

		\subsubsection{Sulfamidici}
		I sulfamidici agiscono come antibiotici antimetabolici in quanto sono analoghi strutturalmente al \emph{PABA}, un acido cruciale per la sintesi dei nucleotidi di DNA ed RNA.
		Questo viene trasformato in acido diidrofolico e in acido tetraidrofolico usato per la sintesi dei nucleotidi.
		La sulfanilammide e i suoi derivati competono con il \emph{PABA} per il sito attivo dell'enzima coinvolto e diminuisce il numero di nucleotidi disponibili fino alla morte della cellula.

	\subsection{Inibizione degli acidi nucleici}
	Alcuni antibiotici bloccano la replicazione del DNA o la sua trascrizione.
	Esistendo poche differenze in questo processo sono colpite sia le cellule del batterio che quelle dell'ospite.

		\subsubsection{Nucleotidi analoghi}
		I nucleotidi analoghi sono simili ai nucleotidi dei patogeni, ma quando assemblati causano mutazioni e disagi.
		
		\subsubsection{Chinoloni e fluorochinoloni}
		I chinoloni e i fluorochinoloni colpiscono la DNA girasi che aiuta la replicazione del DNA nei procarioti.

		\subsubsection{Rifampina}
		La rifampina agisce legandosi alla subunit\`a $\beta$ della polimerasi legandosi pi\`u specificatamente a quella dei procarioti.

\section{Farmaci antivirali}
La maggior parte dei farmaci attivi sui virus agiscono anche sulle strutture cellulari dell'ospite risultando tossici.

	\subsection{Nucleoside reverse transcriptase}
	Quelli pi\`u utilizzati sono gli analoghi dei nucleotidi o \emph{NRTI}, nucleoside reverse transciptase inhibitors, che inibiscono l'allungamento nucleico virale controllato da una polimerasi.

	\subsection{Non-\emph{NRTI}}
	Altri agenti non-\emph{NRTI} inibiscono direttamente la trascrittasi inversa dei retrovirus.

	\subsection{Inibitori di proteasi}
	Gli inibitori di proteasi inibiscono la replicazione virale inibendo il processamento dei polipeptidi virali.

	\subsection{Inibitori di fusione}
	Gli inibitori di fusione legano proteine di membrana e prevengono le modifiche conformazionali necessarie per la fusione della membrana virale con quella delle cellule ospite.

	\subsection{Interferoni}
	Gli interferoni sono glicoproteine prodotte dal sistema immunitario in risposta all'infezione di alcuni virus.
	Si legano alla membrana delle cellule e ne stimolano la produzione di enzimi antivirali.


\section{Progettazione di nuovi farmaci}

	\subsection{Progettazione computerizzata di farmaci virali}
	Nuove tecnologie hanno permesso scoperte importanti nello studio di macromolecole:
	Modelli 3D fedeli permettono lo studio e creazione di farmaci in base a predizioni computazionali, creando molecole che possono interagire specificatamente con il target.
	Esempi sono il saquinavir a l'indinavir.
	Una proteasi del HIV \`e capace di smagliare una proteina precursore inibendo la lavorazione dei precursori e la maturazione del HIV.

	\subsection{Chimica combinatoriale}
	La chimica combinatoriale \`e utile per la ricerca di nuovi farmaci antimicrobici: si modifica in maniera sistematica nuovi prodotti antimicrobici per generarne anloghi.
