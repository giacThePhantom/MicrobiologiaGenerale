\chapter{Controllo della crescita microbica}
Questi tipi di controllo tengono conto solamente delle caratteristiche biologiche dei microrganismi da selezionare e non dell'ambiente in cui vivono. \\Vi sono vari metodi:
\begin{itemize}
    \item \textbf{Sterilizzazione} $\xrightarrow{}$ uccisione o rimozione di tutti gli organismi viv all'interno di un terreno di crescita;
    \item \textbf{Inibizione} $\xrightarrow{}$ riduzione della crescita microbica causata dalla diminuzione del numero di organismi presenti o dalle alterazioni nell'ambiente microbico; 
    \item \textbf{Decontaminazione} $\xrightarrow{}$ trattamento di oggeti o superfici che le rende tali da poter essere utilizzate senza rischio di contaminazione; 
    \item \textbf{Disinfezione} $\xrightarrow{}$ processo che colpisce direttamente i microrganismi, uccidendone o inibendone la crescita;ù
    \item \textbf{Pastorizzazione} $\xrightarrow{}$ riduzione della carica microbica nei liquidi sensibili al calore con lo scopo di disruggere tutti i microrganismi patogeni e ridurre il numero dei microrganismi responsabili del deterioramento degli alimenti; 
\end{itemize}
Ci sono tre tipi di agenti o trattamenti antimicrobici a seconda dell'effetto che hanno sul batterio:
\begin{itemize}
    \item \textbf{Battericida} $\xrightarrow{}$ agente antimicrobico che uccide i microrganismi. La conta vitale subisce una deflessione, mentre la conta totale rimane costante. Questo avviene perchè lo sprettrofotometro non ha la capacità di differenziare le cellule vive da quelle morte, visto che non perdono l'integrità;
    \item \textbf{Batteriolitico} $\xrightarrow{}$ agente antimicrobico che uccide i microrganismi provocando la lisi. C'è una totale distruzione della cellula, quindi si nota sia un calo della conta vitale che nella conta totale;
    \item \textbf{Batteriostatico} $\xrightarrow{}$ agenti antimicroico che inibisce la crescita dei microrganismi. Dopo la sua aggiunta, la conta totale e la conta vitale rimangono costanti.
\end{itemize}
\section{Metodi fisici}
Si possono fare dei controlli della crescita batterica mediante trattamenti fisici solo se il controllo viene fatto su oggetti o terreni di coltura. Non si possono applicare a sistemi in vivo in quanto si porterebbe alla morte dell'organismo ospite. 
\\Questi metodi si dividono in due gruppi principali:
\begin{itemize}
    \item \textbf{Caldo umido} $\xrightarrow{}$ bollitura, autoclave, pastorizzazione e sterilizzazione; 
    \item \textbf{Caldo secco} $\xrightarrow{}$ aria calda o incenerizzazione.
\end{itemize}
Il metodo con caldo umido è più efficace, infatti l'acqua presenta una capacità calorifica maggiore dell'aria ed è così un veicolo più efficiente per trasmettere la temperatura. 
\subsection{Caldo umido}
\subsubsection{Sterilizzazione mediante calore}
La letalità dovuta all'incremento della temperatura è una funzione esponenziale. Il tempo necessario per uccidere una determinata frazione di cellule è indipendete dal numero iniziale di cellule. 
\\Il tempo di riduzione decimale (D) viene definito come il tempo necessario per ridurre di dieci volte, a una data temperatura, la densità della popolazione. 
\\Nei grafici vengono utilizzate sempre due scale diverse di misurazione:
\begin{itemize}
    \item Sull'asse delle ascisse si trova il tempo (scala lineare); 
    \item Sull'asse delle ordinate si trova il numero dei microbi (scala logaritmica).
\end{itemize}
Si ottiene così un grafico semilogaritmico. 
Su un'ampia scala temporale si nota che le sostanze o i trattamenti battericidi uccidono una percentuale costante di cellule per ogni intervallo di tempo. Dal punto di vista grafico questo si nota tracciando una retta che rappresenta una decrescita esponenziale e che indica un tasso di morte costante.
\\Tuttavia non è assicurato che applicando questi metodi, si arrivi a un numero di indiviudi pari a zero: dipende tutto dal numero di microbi iniziale. Quando una popolazione diminuisce di un certo ordine di gradezza, si può dire che è diminuita del 90$\$$. Partendo da una popolazione di 10\ap{9} e affermando che con un determinato prodotto si riesce a ridurla del 99,99$\%$, si ottiene che la popolazione finale è ancora composta da 10\ap{5} microbi. 
\\Un grafico che si basa sulla relazione tra D e la temperatura è esponenziale. La posizione della retta offre un'indicazione quantitativa della sensibilità al calore dell'organismo in esame. 
\\(GRAFICO)Se confrontiamo il comportamento di due batteri, uno mesofilo e uno termofilo, si nota che:
\begin{itemize}
    \item Nel mesofilo con un'esposizione a 100\ap{o}C il tempo di riduzione decimale è a pari a meno di 20 s; 
    \item Nel termofilo, adatto a sopravvivere a temperature elevate, con una temperatura sempre di 110\ap{o}C il tempo di riduzione decimale è di 10 minuti. 
\end{itemize}
Si necessita dell'allestimento di un numero elevato di conte vitali. 
\\Alternativamente si può utilizzare il tempo di inattivazione termica, cioè il tempo necessario per uccidere tutte le cellule di una popolazione a una data temperatura, che dipende dal numero di cellule iniziali. 
\\I batteri che presentano la capacità di fornire endospore sono quelli che hanno la maggiore resistenza al calore. 
\subsubsection{Autoclave}
L'autoclave è composto da una camera a chiusura ermetica che permette l'immissione di vapore sotto pressione. La normale procedura prevede il riscaldamento a una pressione di 1,1 kg/cm\ap{2} (= 15 pounds per square inch, psi) che consente di raggiungere una temperatura di 121\ap{o}C. 
\\La morte dei microrganismi non viene provocata dall'alta pressione, ma dall'elevata temperatura che può essere raggiunta in condizioni di vapore a una pressione superiore a quella atmosferica. Se la pressione non venisse aumentata, la temperatura di ebollizione sarebbe di 100\ap{o}C. Questa temperatura è capace di dentaturare le proteine e di distruggere le membrane, ma non di uccidere le endospore. 
\\L'autoclave è simile ad una pentola a pressione ed è composta da:
\begin{itemize}
    \item Tappo ermetico; 
    \item Tubo per l'ingresso del vapore caldo; 
    \item Sistema per abbassare la temperatura in seguito all'avvenuta operazione. Circa 20 minuti. 
\end{itemize}
(GRAFICO) Se si osserva il grafico si può vedere che le linee che rappresentano la temperatura in autoclave e la temperatura dell'oggetto sono sfasate. La temperatura dell'oggetto sterilizzato aumenta più lentamente della temperatura dell'autoclave. Normalmente il tempo di sterilazzione dura 10-15 minuti, mentre l'intero processo dura circa 60 minuti. 
\\Questo è l'unico metodo che riesce a debellare le endospore. 
\\Queste ultime sono la forma di vita più resistense in assoluto. Possono sopravvivere a valori estremi di temperatura e di pH e resistono all'azione di molte sostanze chimiche utilizzate come disinfettanti, per esempio l'etanolo al 70$\%$. Alla temperatura di 121\ap{o}C si rendono necessari 5-6 minuti per la riduzione decimale delle endospore, mentre bastano 0,1 minuti per la riduzione decimale delle cellule vegetative a 65\ap{o}C.
\\Per verificare l'avvenuta eliminazione possono essere utilizzati due metodi. 
\\Il primo prevede un nastro che diventa di colore nero se la sterilizzazione è stata svolta nel modo corretto, quindi si basa su un sistema chimico. 
\\Nel secondo, invece, è presente un'ampolla che contiene un terreno di crescita con indicatore pH colorimetrico, mentre lo strip contiene le endospore. Durante l'autoclavaggio si verifica la rottura della provetta di vetro e l'esposizione delle endospore al terreno di coltura. Se dopo il ciclo l'indicatore di colorimetrico resta rosso significa che le endospore sono morte; mentre se c'è un cambiamento di colore, significa che c'è stato un abbassamento di pH e di conseguenza l'attività metabolica delle endospore.
\subsubsection{Pastorizzazione}
La pastorizzazione è un processo che utilizza una temperatura controllata per ridurre la carica microbica nel latte e in altri alimenti particolarmente sensibili al calore. Lo scopo di questo processo è quello di prevenire la diffusione di patogeni e il procrastinare della crescita di microrganismi responsabili del deterioramento degli alimenti. Alcuni esempi di batteri che vengono prevenuti grazie all'utilizzo della pastorizzazione sono: \textit{Listeria monocytogenes}, \textit{Escherichia Coli} O15:H7, specie di \textit{Campylobacter} e \textit{Salmonella}.
\\Processi possibili:
\begin{itemize}
    \item \textbf{Pastorizzazione istantanea} $\xrightarrow{}$ piccoli volumi riscaldati a 72\ap{o}C per 15 secondi;
    \item \textbf{Pastorizzazione di massa} $\xrightarrow{}$ grandi volumi riscaldati a 63-66\ap{o}C per 30 minuti;
    \item \textbf{Pastorizzazione UHT} $\xrightarrow{}$ processo detto "flash heating" con getto di vapore a 134\ap{o}C per 1 secondo;
    \item \textbf{Sterilizzazione UHT} $\xrightarrow{}$ getto di vapore a 140\ap{o}C per 1-3 secondi, seguito da un rapido raffreddamento. 
\end{itemize}
\subsubsection{Altri metodi con temperatura}
\begin{enumerate}
    \item \textbf{Caldo secco}: viene utilizzata per sostanze che tollerano l'umidità, come ad esempio le polveri. È meno efficace rispetto al caldo umido; per esempio un'autoclave sterilizza un oggetto in 15 min a 121\ap{o}C, un forno alla stessa temperatura ha bisogno di 16 ore. Questo dipende dal fatto che l'energia termica nell'acqua ha una migliore capacità di stoccaggio. Il processo di sterilizzazione con caldo secco avviene tipicamente a 170\ap{o}C per 1 ora.
    \item \textbf{Refrigerazione (0-8\ap{o}C) e congelamento (<0\ap{o}C)}: il freddo rallenta o ferma il metabolismo microbico. La refrigerazione inibisce la crescita della maggioranza dei patogeni umani, che sono per lo più mesofili. Il congelamento lento consente la formazione di cristalli di ghiaccio che danneggia le membrane ed è per questo moitvo più efficace del congelamento rapido. Per questo motivo quando si scongela un alimento si assiste alla sua perdita di consistenza (le cellule vengono lise dai cristalli di ghiaccio).
    \item \textbf{Essicazione e liofilizzazione}: entrambi i processi inibisconno la crescita microbica perchè le reazioni metaboliche si svolgono in soluzione acquosa. La iofilizzazione combina il congelamento rapido con azoto liquido seguito dalla rimozione dell'acqua tramite sublimazione.
\end{enumerate}
\subsection{Trattamenti fisici alternativi}
\subsubsection{Sterilizzazione con radiazioni}
\begin{enumerate}
    \item \textbf{Sterilizzazione mediante radiazioni non-ionizzanti}: hanno una lunghezza d'onda maggiore di 1 nm(UV, luce visibile, infrarosso, onde radio), ma vengono utilizzati solamente i raggi UV (260 nm) sono sufficientemente energetici per essere utilizzati come agente antimicrobico. Inducono la formazione di dimeri di pirimidina nel DNA. Le radiazioni UV non penetrano nella materia e vengono quindi usate per sterilizzazione di superficie, acqua, liquidi trasparenti.
    \item \textbf{Sterilizzazione mediante radiazioni ionizzanti}: hanno una lunghezza d'onda minore di 1 nm (fascio di elettroni, raggi X, raggi gamma). Possono produrre ioni che interagiscono con le macromolecole biologiche ad esempio rompendo legami idrogeno o producendo ioni OH\ap{-}. 
    \\L'irradiazione con fascio di elettroni è molto efficace in quanto ad alta energia, ma non penetra bene nella materia; al contrario, i raggi gamma sono meno energetici ma penetrano più in profondità. Questo tipo di radiazioni vengono usate nel trattamento della frutta, infatti, se viene bombardata da radiazioni ionizzanti aumenta la sua "shelf life", ossia il tempo per il quale l'alimento può essere messo in vendita.
    \\L'energia di irraggiamento emessa dalla sorgente viene misurata per quantificarne gli effetti. Lo standard per le applicazioni biologiche come la sterilizzazione è dato dalla dose di radiazione assorbita. Questa dose viene espressa in rad (100 egr/g, dove 1 erg = 10\ap{-7} J) o in Gray (1 Gy = 100 rad). La relazione tra la frazione di soppravivenza microbica, riportata su scala logaritmica, e la dose di radiazioni è lineare. 
    \\La dose letale standard per una completa sterilizzazione con radiazioni è 12 volte D10 ed equivale a:
    \begin{itemize}
        \item 39600 Gy per \textit{Clostridium botulinum}; 
        \item 2400 Gy per \textit{Salmonella typhimurium}; 
        \item 10 Gy per l'uomo.
    \end{itemize}
\end{enumerate}ù
\subsubsection{Strerilizzazione per filtrazione}
Viene utilizzata per sterilizzare soluzioni sensibili al calore. Necessita dell'utilizzo di un dispostivo in grado di trattenere microrganismi di dimensione comprese tra0.3 e 10 micrometri. 
\begin{enumerate}
    \item \textbf{Filtri a spessore}: sono costituiti da strati fibrosi di carta, amianto o lana di vetro. Vengono usati come prefiltri per la rimozione delle particelle di maggiore dimensione che potrebbero intasare quelli utilizzati nel processo di strerilizzazione vera e propria. Questa tipologia viene utilizzata nei sistemi di condizionamento, filtri HEPA della cappe biologiche. In quest'ultime si crea un flusso di aria laminare, che evita che l'aria contaminata e quella di laboratorio si mescolino. Questo flusso passa attraverso i filtri HEPA che consentono di trattenere i microbi e di avere un sistema di ricircolo dell'aria. 
    \item \textbf{Membrane filtranti}: vengono utilizzate per la sterilizzazione di liquidi e sono costituite da dischetti di acetato di cellulosa o di nitrocellulosa. Aggiustando le condizioni di polimerizzazione durante la fabbricazione può essere controllata in maniera precisa la dimensione dei pori, normalmente da 0.1 a 10 micrometri. Esistono vari tipi di membrani filranti e quelle più utilizzate per il controllo delle crescita batterica sono da 0.45 (batteri di grandi dimensione) e da 0.22 (maggioreì parte dei batteri e i virus più grandi). 
    \item \textbf{Il filtro Nucleopore}: viene prodotto trattando un sottile strato di policarbonato con un composto chimico corrosivo. Vengono maggiormente usati per la preparazione di campioni per la microscopia elettronica. L'organismo viene facilmente rimosso dalla fase liquida e distribuito su un unico piano alla superficie del filtro. 
\end{enumerate}
\section{Metodi chimici}
\\AGGIUNGI TABELLA
\subsection{Controllo della crescita in vivo}
Nel 1929 Alexander Fleming riporta per la prima volta l'azione antibatterica della penicillina, prodotta da \textit{Penicillium chrysogenum} (fungo). Gli agenti che vengono utilizzari per controllare la crescita batterica in vivo, che sia per uso clinico o veterinario, sono detti antibiotici. Esistono antibiotici naturali, semisintetici e sintetici. Gli antibiotici agiscono solamente contro i batteri e non anche contro virus e funghi. 
\\Vengono prodotti da funghi o batteri e si possono riassumere le informazioni principali sul loro utilizzo nei seguenti punti:
\begin{enumerate}
    \item Sono farmaci salva-vita;
    \item Trattano solamente infezioni batteriche; 
    \item Alcune infezioni all'orecchio non necessitano la cura antibiotica; 
    \item La maggior parte delle infezioni alla gola non richiede cura anitbiotica;
    \item Il muco di colore verde non necessita l'utilizzo di antibiotico; 
    \item Ci sono molti rischi se vengono assunti senza prescrizione medica.
\end{enumerate}
Spesso la terapia antimicrobica è molto complesssa se applicata in vivo: è difficile trovare delle che molecole che rispettino tutte le caratteristiche necessarie. 
\\Va considerata la tossicità selettiav, quindi bisogna adottare dei metodi che prevedono l'uccisione dei microbi ma che allo stesso tempo preservino l'organismo ospite. Bisogna anche prendere in considerazione il funzionamento delle drug delivery, quindi la molecola che si deve adottare deve raggiungere la regione bersaglio senza subire alcuna alterazione. Dovrebbe passare per i vari tessuti del corpo e arrivare al sito target con la giusta concetrazione. 
\\Ci possono essere numerose complicazioni durante il loro utilizzo:
\begin{itemize}
    \item Ritezione del farmaco: l'ospite può degradare o disattivare l'antibiotico; 
    \item Utilizzo di un farmaco errato: bisogna conoscere lo spettro di resistenza prima di utilizzarlo; 
    \item Problemi di delivery; 
    \item Tossicità sull'ospite, reazioni allergiche, alterazioni del microbioma interno.
\end{itemize}
Vanno considerati i seguenti quattro aspetti principali:
\begin{enumerate}
    \item Pericolo di sviluppo di resistenza;
    \item Pericolo di rilascio di tossine dopo la lisi. Esotossine e endotossine provenienti dalla parte lipidica che portano ad un peggioramento nell'ospite. Si parla in particolare di gram-negativi;
    \item Uso corretto e per il giusto periodo;
    \item Possibilità di utilizzare varie tipologie di antibiotici in cicli sequenziali.
\end{enumerate}
Le caratteristiche dell'antibiotico ideale sono:
\begin{itemize}
    \item Disponibilità;
    \item Basso costo $\xrightarrow{}$ deve essere facile da preparare, poco dispendioso e molto vendibile;
    \item Chimicamente stabile (trasporto e stoccaggio);
    \item Semplice da somministrare;
    \item Non allergenico;
    \item Tossicità selettiva.
\end{itemize}
Nella realtà nessuna molecola ha tutte queste caratteristiche contemporaneamente.
\\L'utilizzo di antibiotici ad ampio spettro pone il problema di "super-infections" dovute all'alterazione delle comunità microbiche naturalmente associate all'ospite umano e all'eventuale comparsa di infezioni secondarie. 
\section{Misurazione dell'attività antimicrobica}
\subsection{Test di Kirby-Bauer}
Questo test viene utilizzato per determinare la suscettibilità o la resistenza di un determinato batterio a una cura con un certo antibiotico. 
\\Viene formato un tappetto uniforme fi batteri in cui vengono posti dei dischetti imbevuti di antibiotici diversi. Poi si misurano gli aloni di inibizione, che sono le zone che si trovano intorno al disco in cui i batteri non sono stati in grado di crescere. 
\\Va comunque indagaa anche la sopportazione dell'organismo all'agente usato.
\subsection{Test di Minimum Inhibitory Concentration (MIC)}
La MIC è la minore concentrazione di un antibiotico in grado di inibire la crescita batterica. 
\\Vengono poste uguali quantità di batteri in una serie di tubi in cui vengono aggiunte delle soluzioni con differenti concetrazioni di antibiotici diversi. Una volta avvenuta l'incubazione, la torbidità indica la crescita batterica e quindi la sua assenza individua i tubi dove i betteri o sono stati uccisi o la replicazione è stata bloccata. 
\\Per misurare la MIC viene utilizzato il metodo dell'assorbanza. 
\subsection{Etest}
L'Etest associa le caratteristiche del test di Kirby-Bauer con il test MIC. Si basa su una coltura in terreno solido dove viene applicato uno striscio che contiene concentrazioni differenti di antibiotico. Data la concetrazione diversa si ottiene un alone di inibizione a forma di goccia. Questo permette di avere una lettura immediata per trovare il MIC dove comincia la goccia di inibizione.
\subsection{Minimum bactericidal concentration (MBC)}
Questo test è un estensione del test MIC e determina la concetrazione minima necessaria di antibiotico per uccidere i batteri presenti. 
\\Le provette utilizzate per MIC e le si piastra su terreni di coltura ma senza antibiotico. Questo assicura di trovarsi in presenza di antibiotici battericidi che portano alla morte di tutti i batteri e non solo di quello batteriostatici, che vanno ad inibire la crescita dei microrganismi. 
\section{Meccanismi di azione degli antibiotici}
Gli antibiotici hanno diverse modalità di azione e queste dipendono anche dalle diverse specie:
\begin{itemize}
    \item Inibizione della sintesi della parete cellulare, $\beta$ lattamici e glicopeptidi;
    \item Inibizione della sintesi delle proteine (traduzione), amminoglicosidi;
    \item Inibizione delle vie metaboliche;
    \item Inibizione della sintesi di acidi nucleici;
    \item Inibizione del riconoscimento o attacco del patogeno al suo ospite, colpiscono fattori di virulenza e le patogenicità frequenti. 
\end{itemize}
Classificazione degli antibiotici:
\begin{itemize}
    \item Aminoglicosidi e tetracicline $\xrightarrow{}$ inibiscono la sintesi proteica legando la subunità 30S dei ribosomi batterici; 
    \item Beta-lattamici e glicopeptidi $\xrightarrow{}$ interferiscono con la sintesi della parete batterica;
    \item Fluoroquinolones $\xrightarrow{}$ inibiscono l'attività della DNA girasi o topoisomerasi coinvolti nella replicazione del DNA;
    \item Macrolidi $\xrightarrow{}$ inibiscono la sintesi proteica legando la subunità 50S dei ribosomi batterici;
    \item Sulfonamidi $\xrightarrow{}$ inibiscono la biosintesi dell'acido folico.
\end{itemize}
\subsection{Inibizione della sintesi della parete cellulare}
La parete cellulare dei batteri è composta da macromolecole di peptidoglicano formato a sua volta da catene di NAM-NAG che sono legate da ponti peptidici tra le subunità di NAM.
\\Al processo di sintesi di questa parete prendono parte due classi di antibiotici.
\subsubsection{$\beta$ lattamici}
Questi rappresentano (cephalosporine e penicillina) quasi la metà delle 500 tonnellate di antibiotici che vengono utilizzati ogni anno. Questi agiscono solamente sui batteri in fase di divisione. In questa classe di antibiotici sono presenti dei composti con gruppi funzionali differenti tra loro. Tuttavia, tutti presentano l'anello $\beta$ lattamico che gli impedisce di formare dei legami crociati peptidici tra subunità di NAM del peptidoglicano. 
\\Il legame con gli enzimi transpeptidasi non blocca la sintesi della parete, ma la indebolisce progressivamente visto che le catene di peptidoglicano non sono più unite tra loro. La cellula non è in grado di resistere alla pressione osmotica e va incontro alla lisi.
\\Sono divisi in due gruppi principali:
\begin{itemize}
    \item Peniciline naturali, come per esempio la Benzilpenicillina;
    \item Penicilline semisintetiche $\xrightarrow{}$ non sono presenti in natura e hanno un'attività più efficace della penicillina naturale. Le più conosciute sono la Meticillina, l'Oxacillina, l'Ampicillina e la Carbenicillina. Queste hanno vari vantaggi:
    \begin{itemize}
        \item Sono attive anche contro i gram-negativi; 
        \item Sono resistenti alla beta-lattamasi; 
        \item Sono più stabili in ambiente acido, come lo stomaco; 
        \item Vengono assorbite più facilmente dall'epitelio intestinale.
    \end{itemize}
\end{itemize}
\subsubsection{Glicopeptide}
Queste hanno il compito di legare il ponte peptidico in corrispondenza di ogni filamento prima che la transpeptidasi possa agire. Questa classe di antibiotici riconosce gli antibiotici e si legano a loro prima che gli enzimi deputati alla loro fusione possano essere in grado di assemblarli. Anche qui si ottiene una cellula debole, che non ha la capacità di far fronte alla forza osmotica:
\begin{itemize}
    \item Vancomicina $\xrightarrow{}$ interferisce con la formazione dei ponti Ala-Ala dei legami crociati nei gram positivi; 
    \item Bacitracina $\xrightarrow{}$ blocca la defosforilazione del bacctoprenolo, inibendo la secrezione delle subunità NAM e NAG del citoplasma. 
\end{itemize}
Questi antibiotici hanno efficacia solo nei confronti di cellule in divisione attiva.
\\Un batterio ha la possibilità di diventare resistente a questa classe modificando l'ultimo aminoacido della sequenza. In questo modo l'antibiotico non è capace di riconoscere più le cateme visto che si viene a creare un legame imperfetto.
\subsubsection{Isoniziade ed etambutolo}
I batteri del genere \textit{Mycobacterium}, agenti del morbo di Hansen (lebbra) e della tuberculosi, presentano una parete cellulare peculiare contenente uno strato di acidi micolici e arabinogalactan. L'isoniziade e l'etambutolo inibiscono la formazione di questo involucro. \\(AGGIUNGI IMMAGINE)
\subsection{Inibizione della sintesi proteica}
La tossicità selettiva di alcuni antibiotici si basa sulle lievi differenze strutturali tra ribosomi procariotici ed eucariotici. 
\subsubsection{Aminoglicosidi}
Alterano la struttura della subunità 30S portando alla lettura erronea dei codoni e all'icorporazione di aminoacidi scorretti.
\subsubsection{Tetracicline}
Bloccano il sito di legame del tRNA impedendo l'elongazione della proteina.
\subsubsection{Cloramfenicolo}
Il cloramfenicolo blocca l'attività enzimatica della subunità 50S impendendo la formazione dei legami peptidici. 
\subsubsection{Macrolidi}
Legano la subunità 50S interferendo con il movimento del ribosoma lungo la molecola di mRNA. Un esempio è l'eritromicina.
