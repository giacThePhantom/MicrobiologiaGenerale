\chapter{Antibiotico resistenza}

\section{Introduzione}
Si intende per resistenza la capacit\`a intrinseca o acquisita di un microorganismo di resistere agli effetti di un agente chemioterapico.


	\subsection{Esperimento della Harvard Medical School}
	L'esperimento consiste in una piastra divisa in $9$ bande con all'interno un antibiotico in gradiente di concentrazione.
	Alle estremit\`a, con concentrazione minore si sviluppano le prime colonie.
	I batteri si devono spostare nella piastra.
	Basta un solo batterio mutato per far sopravvivere in $11$ giorni la colonia nella zona della piastra dove la concentrazione dell'antibiotico \`e mille volte superiore a quella inizialmente tollerata.
	Si pu\`o tracciare inoltre un albero delle mutazioni.



	\subsection{Resistenza intrinseca}
	Diverse cause concorrono alla resistenza intrinseca.

		\subsubsection{Assenza del bersaglio}
		La resistenza intrinseca pu\`o essere dovuta all'assenza del bersaglio su cui agisce l'antibiotico: i micoplasmi non sono dotati di parete e sono resistenti alle penicilline.

		\subsubsection{Impermeabilit\`a all'antibiotico}
		La resistenza intrinseca pu\`o essere dovuta all'impermeabilit\`a dell'organismo all'antibiotico: la membrana esterna dei Gram$-$ \`e impermeabile alla penicillina $G$.

		\subsubsection{Alterazione chimica dell'antibiotico}
		La resistenza intrinseca pu\`o essere dovuta all'alterazione chimica dell'antibiotico: le $\beta$-lattamasi tagliano l'anello $\beta$-lattamico delle penicilline.

		\subsubsection{Modifica del bersaglio}
		La resistenza intrinseca pu\`o essere dovuta alla modifica del bersaglio dell'antibiotico come una subunit\`a riboosmale.

		\subsubsection{Sviluppo di una via biochimica di resistenza}
		La resistenza intrinseca pu\`o essere dovuta allo sviluppo di una via biochimica di resistenza, come l'assorbimento diretto dell'acido folico dall'ambiente che rende inefficienti i sulfonamidici che ne inibiscono la sintesi da parte del microorganismo.

		\subsubsection{Trasporto dell'antibiotico}
		La resistenza intrinseca pu\`o essere dovuta al trasporto dell'antibiotico all'esterno della cellula batterica attraverso sistemi di efflusso.

\section{Sviluppo e diffusione della resistenza nelle popolazioni microbiche}
L'antibiotico resistenza viene trasmessa attraverso trasmissione orizzontale: come trasformazione, trasduzione o coniugazione.
In assenza di pressione selettiva le cellule resistenti sono meno efficienti rispetto a quelle sensibili in quanto devono sostenere costi maggiori per la produzione e mantenimento delle funzioni di resistenza.
L'antibiotico pertanto favorisce la diffusione della resistenza delle popolazioni aumentando il fitness dei ceppi batterici.
Si nota come esiste una relazione tra antibiotici pi\`u utilizzati e resistenza ad essi.

	\subsection{Resistenza multipla e cross-resistenza}
	Un microorganismo \`e detto multiresistente quando \'`e in grado di resistere all'azione di $3$ o pi\`u classi diverse di antiboitici.
	La cross-resistenza \`e la resistenza acquisita contro diversi antibiotici della stessa classe.

	\subsection{Ritardare la comparsa della resistenza}
	Per ritardare la comparsa della resistenza si pu\`o:
	\begin{multicols}{2}
		\begin{itemize}
			\item Programmare trattamenti con concentrazioni e tempi adeguati.
			\item Utilizzare di agenti microbici in combinazione per sfruttare eventuali sinergie.
			\item Limitare l'utilizzo ai casi necessari.
			\item Sviluppare antibiotici semisintetici.
		\end{itemize}
	\end{multicols}

\section{Metagenomica dell'antibiotico resistenza in batteri ambientali}
Si intende per metagenomica lo studio delle comunit\`a microbiche complesse che si trovano nell'ambiente in cui le funzioni di resistenza sono presenti.
La resistenza clinica \`e associata a meccanismi che si trovano in modo ambientale.
Si devono pertanto considerare gli organismi non patogeni nella ricerca degli antibiotici.

	\subsection{Resistoma}
	Si intende per resistoma un collezione dei geni che codificano per resistenze antibiotiche e i loro precursori in batteri patogeni e non.
	Esistono geni di resistenza criptici in quanto compresi in un cromosoma batterico, ma che non si possono associare con l'antibiotico resistenza.

	\subsection{Batteri nel suolo}
	I batteri che vivono nel suolo vivono in continua competizione per le risorse e sono spesso in grado di sviluppare resistenze.
	Questo in quanto la guerra chimica tra i microorganismi ha richiesto la coevoluzione di meccanismi di autoprotezione dai produttori di antibiotici sia l'acquisizione della resistenza in altri procarioti.
	Questo suggerisce un'origine ambientale della maggior parte dei determinanti della resistenza clonica.

	\subsection{Trasferimento genico dei determinanti della resistenza}
	I geni che codificano le proteine che mediano il trasferimento genico orizzontale fiancheggiano molti dei geni di resistenza.
	Tuttavia molti geni di resistenza non sono collegati a tali elementi e i ruoli delle proteine associate sono generalmente sconosciuti.
	La capacit\`a dei batteri di mobilitare i geni e la pressione selettiva facilitano la distribuzione dei geni di resistenza agli antibiotici a tutte le popolazioni microbiche: il resistoma si espande anche in assenza di selezione continua.

	\subsection{Modello per l'evoluzione dell'antibiotico resistenza}
	Un modello di consenso \`e che le proteine di resistenza si evolvono da proteine con funzioni biochimiche alternative che funzionano come precursori di elementi di resistenza.
	Alcune di queste possono avere modeste funzioni di resistenza e di fronte alla pressione selettiva si evolve in un robusto meccanismo di resistenza.

\section{Future terapie antivirali}
Le future terapie antivirali nascono con lo scopo di avere come obiettivo la virulenza batterica invece di ricorrere a una terapia antimicrobica.

	\subsection{Virulenza}
	Si intende per virulenza quanto un batterio \`e aggressivo nei confronti del suo ospite.
	
	\subsection{Inibire la patogenesi}
	Questo approccio sembra inibire la patogenesi e le sue conseguenze senza porre un'immediata pressione di vita o morte sul batterio.
	Strappare ai microorganismi le propriet\`a virulente senza sottoporle a stress selettivo pu\`o essere un modo di controllare la loro crescita senza che si sentano minacciati e sviluppino resistenze.

		\subsection{Elementi di patenogenesi}
		\begin{multicols}{2}
			\begin{itemize}
				\item Adesione del batterio: per esempio in E. coli i pili di tipo $I$.
				\item Tossine batteriche: le tossine sono fattori di virulenza che infettano i tessuti.
				\item Sistema di secrezione III: spinge le tossine batteriche nella matrice extracellualre.
				\item Biofilm: ingombro fisico agli antibiotici.
				\item Quorum sensing.
			\end{itemize}
		\end{multicols}
