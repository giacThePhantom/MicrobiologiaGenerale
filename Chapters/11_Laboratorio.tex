\chapter{Laboratorio}

\section{Prima esperienza - preparazione del terreno di coltura sterile}

	\subsection{Introduzione}

		\subsubsection{Categorie di terreni batterici}

			\paragraph{Determinazione in base allo stato fisico}
			In base allo stato fisico i terreni batterici si distinguono in:
			\begin{itemize}
				\item Terreni liquidi o brodi: usati per la coltivazione batterica.
				\item Terreni solidi o gel: usati sia che per la coltivazione batterica che per l'isolamento.
					Sono terreni liquidi gelificati tramite \emph{AGAR}, sostanza polisaccaride isolata da un'alga rossa in Giappone.
					Gelifica a temperature inferiori ai $45\%$.
			\end{itemize}
	
			\paragraph{Determinazione in base alla quantit\`a di sostanze}
			In base alla quantit\`a di sostanze presenti i terreni batterici si distinguono in:
			\begin{itemize}
				\item Terreni minimi: sono utilizzati per la crescita dei soli batteri autotrofi.
					Contengono solo gli elementi essenziali $N$, $C$, $S$, $P$ come sali inorganici in composizione e quantit\`a note.
				\item Terreni sintetici o definiti: vengono preparati \emph{ad hoc} a seconda delle esigenze nutrizionali del microorganismo.
					Se ne conosce l'esatta composizione.
				\item Terreni complessi: permettono la crescita di pi\`u organismi con esigenze nutrizionali diverse.
					Non se ne conosce l'esatta composizione.
			\end{itemize}

			\paragraph{Determinazione in base alla qualit\`a delle sostanze}
			In base alla qualit\`a delle sostanze presenti i terreni batterici si distinguono in:
			\begin{itemize}
				\item Terreni nutritivi: favoriscono la crescita di microorganismi particolari dal punto di vista nutritivo:
					al terreno vengono aggiunte sostanze nutritive come siero, latte o sangue per favorire una specie specifica.
				\item Terreni selettivi: favoriscono la crescita di particolari specie batteriche grazie alla presenza di fattori che inibiscono lo sviluppo di altre specie.
					I fattori vengono detti sostanze inibenti e possono essere antibiotici, coloranti o sali.
				\item Terreni differenziali: permettono l'identificazione batterica in relazione all'attivit\`a metabolica o aspetti morfologici delle colonie.
					Questo avviene grazie a particolari substrati o indicatori in grado di dimostrare con una variazione cromatica l'azione metabolica del microorganismo ricercato.
			\end{itemize}

\section{Seconda esperienza - Determinazione della curva di crescita di un ceppo batterico (E. coli)}
	
	\subsection{Introduzione}

		\subsubsection{Fattori che influenzano la crescita batterica}
		I fattori che influenzano la crescita batterica sono:
		\begin{multicols}{2}
			\begin{itemize}
				\item Nutrienti: fonti di carbonio, energia, acqua, vitamine, azoto e oligoelementi.
				\item Concentrazioni di sale: divide i batteri in alofili, alotolleranti o alofili estremi.
				\item Ossigeno: divide i batteri in aerobi obbligati, aerobi facoltativi, microaerofili, anaerobi aerotolleranti e anaerobi obbligati.
				\item $pH$: divide i batteri in alcalinofili, basofili e neutrofili.
				\item Temperatura: divide i batteri in psicrofili ($-10$-$30\si{\degree}$), mesofili ($10$-$50\si{degree}$), termofili ($40$-$90\si{\degree}$) e termofili estremi.
			\end{itemize}
		\end{multicols}

		\subsubsection{Quantificazione}
	
			\paragraph{Fasi della crescita microbica}\mbox{}
			\begin{enumerate}
				\item Lag: adattamento del batterio alle nuove condizioni.
				\item Esponenziale: crescita continua grazie alla presenza di nutrienti.
					Il tempo di replicazione specifico del batterio influisce sulla velocit\`a: Mycobacterium tubercolosis impiega $16$ ore, E. coli $20$ minuti.
				\item Stazionaria: ci sono troppi batteri sul terreno e i nutrienti iniziano ad esaurirsi.
					Gli eventi di replicazione e morte sono in equilibrio tra di loro.
				\item Morte: avviene progressivamente la morte per mancanza di nutrienti.
			\end{enumerate}


\section{Terza esperienza - Caratterizzazione dei batteri del cavo orale}

	\subsection{Introduzione}
	Nel caso orale si trovano numerosi batteri anche simbiotici in un pattern specifico alla persona in quanto dipende da:
	\begin{multicols}{3}
		\begin{itemize}
			\item $pH$.
			\item Umidit\`a.
			\item Nutrienti della dieta.
		\end{itemize}
	\end{multicols}
	
		\subsubsection{Piastra su agar-sangue}
		Il risultato del tampone viene seminato su una piastra di agar-sangue con il $5\%$ di sangue di montone. 
		Questo in quanto alcuni batteri come lo Streptococco sono in grado di attuare emolisi, ovvero di degradare i globuli rossi.

			\paragraph{Emolisi}
			Ci sono tre tipi di emolisi:
			\begin{itemize}
				\item Emolisi $\beta$ o emolisi completa: il sangue viene completamente degradato e il terreno appare giallo.
				\item Emolisi $\alpha$ o emolisi incompleta: il sangue viene degradato parzialmente e il terreno appare verdognolo.
				\item Emolisi $\gamma$ o non-emolisi: il sangue non viene degradato e il terreno rimane rosso.
			\end{itemize}

			\paragraph{Semina sulla piastra}
			\begin{enumerate}
				\item Prima patch: si striscia il tampone in modo da occupare la parte superiore della piastra.
				\item Seconda patch: si ruota la piastra di $90\si{\degree}$ e si striscia il tampone in modo che solo qualche strisciata si sovrapponga alla prima patch.
				\item Terza patch: si ruota ancora la piastra di $90\si{\degree}$ e si ripete il procedimento.
			\end{enumerate}

		\subsubsection{Morfologia dei batteri del cavo orale}
		Per identificare le colonie le si distinguono in base a colore, forma, spessore e margini.

			\paragraph{Aspetto}\mbox{}
			\begin{multicols}{3}
				\begin{itemize}
					\item Puntiforme.
					\item Circolare.
					\item Filamentoso.
					\item Irregolare.
					\item Rizoide.
					\item Lenticolare.
				\end{itemize}
			\end{multicols}

			\paragraph{Spessore}\mbox{}
			\begin{multicols}{2}
				\begin{itemize}
					\item Rasata.
					\item Convessa.
					\item Pulvinata.
					\item Umbonata.
				\end{itemize}
			\end{multicols}

			\paragraph{Margini}\mbox{}
			\begin{multicols}{3}
				\begin{itemize}
					\item Interi.
					\item Ondulati.
					\item Lobati.
					\item Erosi.
					\item Filamentosi.
					\item Stratificati.
				\end{itemize}
			\end{multicols}

\section{Quarta esperienza - Conta standard su piastra}

	\subsection{Introduzione}

		\subsubsection{Conta microbica indiretta}
		La conta microbica indiretta avviene tramite spettrofotometro: la densit\`a ottica nella cuvetta \`e proporzionale alla quantit\`a di batteri presenti.
		Non permette per\`o di distinguere tra cellule vive e morte.

		\subsubsection{Conta vitale}
		La conta vitale su piastra permette di contare unicamente le cellule vive.

			\paragraph{Processo}
			\begin{enumerate}
				\item Si prende la provetta contenente le colonie batteriche.
				\item Si fanno diluizioni seriali su piastra in brodo di coltura o soluzione salina.
				\item Si prelevano $0.1\si{mL}$ dalla diluizione e si distribuiscono uniformemente su piastra tramite ansa a l.
				\item Si incuba ogni diluizione.
				\item Si conta la piastra in cui le colonie si distinguono correttamente e non sono ammassate (tipicamente tra le $10$ e le $200$).
			\end{enumerate}

\section{Quinta esperienza - test di aerobiosi/anaerobiosi su terreno solido}

	\subsection{Introduzione}

		\subsubsection{Distinguere i batteri in base alla richiesta di ossigeno}
		In base alla loro richiesta di ossigeno i batteri si dividono in:
		\begin{multicols}{2}
			\begin{itemize}
				\item Anaerobi obbligati: non tollerano l'ossigeno.
				\item Anaerobi facoltativi: crescono con la respirazione cellulare in presenza di ossigeno ma se \`e assente con altri metabolismi.
				\item Microaerofili: batteri che richiedono una quantit\`a di ossigeno inferiore a quella presente nell'aria: $5$-$10\%$.
				\item Aerotolleranti: sono indifferenti alla presenza di ossigeno, effettuano la fermentazione, preferiscono una quantit\`a di ossigeno inferiore al $20\%$.
				\item Aerobi obbligati: richiedono ossigeno.
			\end{itemize}
		\end{multicols}

		\subsubsection{Batteri modello}
		Per verificare l'aerobiosi o anaerobiosi di una colonia si utilizzano:
		\begin{multicols}{2}
			\begin{itemize}
				\item Citrobacter freundii: anaerobio facoltativo, Gram$-$.
				\item Micrococcus luteus: aerobio obbligato, Gram$+$.
			\end{itemize}
		\end{multicols}

		\subsubsection{Processo}
		\begin{enumerate}
			\item Si prendono due piastre di terreno solido e le si divide a met\`a.
			\item In ogni piastra si semina su una met\`a Citrobacter freundii e sull'altra Micrococcus luteus.
			\item Una piastra viene posta in condizioni di aerobiosi mentre l'altra viene posta all'interno di una giara in cui si produce una condizione di anerobiosi.
			\item Si verifica che Citrobacter freundii cresce in entrambe le piastre mentre Micrococcus luteus solo nella piastra con ossigeno.
		\end{enumerate}
		Nella giara vengono inseriti dei filtri in grado di eliminare l'ossigeno presente grazie a reazioni attuate all'interno della giara stessa.
		
\section{Sesta esperienza - test Kirby-Bauer}

	\subsection{Introduzione}
	L'abuso di antibiotici ha causato lo sviluppo di antibiotico-resistenza da parte dei batteri per alcuni antibiotici.
	Si tratta della minaccia mondiale pi\`u pericolosa dal punto di vista sanitario vista la facilit\`a del trasferimento genico orizzontale.
	
		\subsubsection{Test di Kirby-Bauer}
		Durante il test di Kirby-Bayer per la verifica dell'antibiotico-resistenza:
			
			\paragraph{Procedimento}
			\begin{enumerate}
				\item Si prende una piastra con agar solido Mueller-Hinton e si semina uniformemente una certa specie batterica.
				\item Si applicano dischetti imbevuti ognuno da un antibiotico specifico
				\item Si incubano le piastre.
			\end{enumerate}

			\paragraph{lettura dei risultati}
			La sensibilit\`a a un antibiotico viene determinata come alone di inibizione: il diametro dell'alone indica il grado di sensibilit\`a.
			L'assenza di alone indica un antibiotico-resistenza.

\section{Settima esperienza - test biochimico \emph{API 20E}}

	\subsection{Introduzione}
	Esistono molti metodi biochimici per determinare la specie di una coltura batterica.
	Uno dei pi\`u utilizzati \`e il test \emph{API 20E}.

		\subsubsection{Test \emph{API 20E}}
		Il test \emph{API 20E} si compone di $21$ reazioni metaboliche e $6$ test biochimici supplementari.
		Verr\`a utilizzato per identificare \emph{Enterobacteriaceae} e altri Gram$-$.

			\paragraph{Processo}
			\begin{multicols}{2}
				\begin{enumerate}
					\item Si preleva una colonia.
					\item Si stempera la colonia in acqua sterile.
					\item Si riempono i microtubuli o microgallerie con la colonia stemperata.
						I microtubuli sono alti $1\si{cm}$ e larghi $0.5\si{cm}$ e contengono terreno liofilizzato.
					\item Si incubano a $37\si{\degree}$ per una notte.
				\end{enumerate}
			\end{multicols}

			\paragraph{Lettura dei risultati}
			In base alla reazione chimica metabolica che avviene in ogni pozzetto o microtubulo o galleria si nota una colorazione diversa.
			Il colore denota l'avvenimento o meno della reazione.
			Moduli identificano la specie.

				\subparagraph{Lettura dei moduli}
				A ogni pozzetto corrisponde un numero.
				Se la reazione \`e avvenuta nel pozzetto considero il numero, altrimenti no.
				I pozzetti sono divisi in trittici: sommando i numeri da considerare per ogni trittico si ottiene un codice univoco per la specie.

\section{Ottava esperienza - Colorazione di Gram}
	\subsection{Introduzione}
	La colorazione di Gram differenzia batteri Gram$+$ che appaiono di color violetto e batteri Gram$-$ che appaiono rosa.
	Questo avviene in quanto viene usata la sostanza Crystal Violet che colora di violetto lo strato di peptidoglicano: essendoci nei Gram$+$ uno strato pi\`u spesso, la colorazione \`e pi\`u forte.

		\subsubsection{Procedimento}
		\begin{enumerate}
			\item Si applicano i batteri sulla piastra e si applica il Crystal Violet.
			\item Si applica iodio che funge da mordente: attacca al peptidoglicano il Crystal Violet.
			\item Si decolora tramite alcol: solo i batteri su cui il Crystal Violet ha attaccato grazie al mordente rimangono colorati (Gram$+$).
			\item Si utilizza la safranina per colorare i Gram$-$ di rosa.
		\end{enumerate}

\section{Nona esperienza - Osservazione della motilit\`a batterica di tipo ``swimming'' al microscopio ottico}

	\subsection{Introduzione}
	I Batteri si distinguono anche grazie alla loro motilit\`a, che pu\`o avvenire attraverso flagelli o pili.

		\subsubsection{Flagelli}
		I flagelli sono lunghe appendici proteiche che servono al moto della cellula batterica.
			
			\paragraph{Tipi di movimento}
			Il movimento pu\`o essere:
			\begin{multicols}{2}
				\begin{itemize}
					\item Tumbling: casuale, con continui cambi di direzione.
					\item Swimming: in una sola direzione.
				\end{itemize}
			\end{multicols}

			\paragraph{Distinzione in base a numero e posizione dei flagelli}\mbox{}\\
			\begin{multicols}{2}
				\begin{itemize}
					\item Monotrofico: un solo flagello all'estremit\`a.
					\item Lofotrico: un gruppo id flagelli all'estremit\`a.
					\item Anfitrico: flagelli a estremit\`a opposte.
					\item Peritrico: flagelli distribuiti su tutta la superficie.
				\end{itemize}
			\end{multicols}

\section{Decima esperienza - Trasformazione di batteri}

	\subsection{Introduzione}
	Per produrre batteri ricombinanti si utilizza la tecnologia del DNA ricombinante tramite inserimento di un gene in un battere per mezzo dei plasmidi.

		\subsubsection{Plasmidi}
		Si utilizzer\`a un plasmide contenente:
		\begin{multicols}{2}
			\begin{itemize}
				\item \emph{GFP}: green fluorescent protein.
				\item Gene per la resistenza all'ampicillina: creazione di un terreno selettivo.
			\end{itemize}
		\end{multicols}
		Questo verr\`a inserito in E. coli.

		\subsubsection{Processo di trasformazione}
		Si trasformano i batteri tramite shock termico, abbassando la temperatura da $43\si{\degree}$ a $3\si{\degree}$ per facilitare l'entrata nel plasmide.
		Si semina su un terreno contenente ampicillina \emph{Amp} e si incuba.
		In questo modo si eliminano le cellule non trasformate (non resistenti ad ampicillina).
		Le cellule trasformate contengono fluorescenza e resistenza ad \emph{Amp}.


