\chapter{Laboratorio}

\section{Prima esperienza - preparazione del terreno di coltura sterile}

	\subsection{Introduzione}

		\subsubsection{Categorie di terreni batterici}

			\paragraph{Determinazione in base allo stato fisico}
			In base allo stato fisico i terreni batterici si distinguono in:
			\begin{itemize}
				\item Terreni liquidi o brodi: usati per la coltivazione batterica.
				\item Terreni solidi o gel: usati sia che per la coltivazione batterica che per l'isolamento.
					Sono terreni liquidi gelificati tramite \emph{AGAR}, sostanza polisaccaride isolata da un'alga rossa in Giappone.
					Gelifica a temperature inferiori ai $45\%$.
			\end{itemize}
	
			\paragraph{Determinazione in base alla quantit\`a di sostanze}
			In base alla quantit\`a di sostanze presenti i terreni batterici si distinguono in:
			\begin{itemize}
				\item Terreni minimi: sono utilizzati per la crescita dei soli batteri autotrofi.
					Contengono solo gli elementi essenziali $N$, $C$, $S$, $P$ come sali inorganici in composizione e quantit\`a note.
				\item Terreni sintetici o definiti: vengono preparati \emph{ad hoc} a seconda delle esigenze nutrizionali del microorganismo.
					Se ne conosce l'esatta composizione.
				\item Terreni complessi: permettono la crescita di pi\`u organismi con esigenze nutrizionali diverse.
					Non se ne conosce l'esatta composizione.
			\end{itemize}

			\paragraph{Determinazione in base alla qualit\`a delle sostanze}
			In base alla qualit\`a delle sostanze presenti i terreni batterici si distinguono in:
			\begin{itemize}
				\item Terreni nutritivi: favoriscono la crescita di microorganismi particolari dal punto di vista nutritivo:
					al terreno vengono aggiunte sostanze nutritive come siero, latte o sangue per favorire una specie specifica.
				\item Terreni selettivi: favoriscono la crescita di particolari specie batteriche grazie alla presenza di fattori che inibiscono lo sviluppo di altre specie.
					I fattori vengono detti sostanze inibenti e possono essere antibiotici, coloranti o sali.
				\item Terreni differenziali: permettono l'identificazione batterica in relazione all'attivit\`a metabolica o aspetti morfologici delle colonie.
					Questo avviene grazie a particolari substrati o indicatori in grado di dimostrare con una variazione cromatica l'azione metabolica del microorganismo ricercato.
			\end{itemize}
	\subsection{Primo giorno}
	Per la preparazione di un terreno si utilizza un preparato in polvere pesato accuratamente tramite bilancia digitale.

		\subsubsection{Coltura liquida}
		Per la preparazione di una coltura liquida si misura una specifica quantit\`a di acqua distillata tramite cilindro e la si aggiunge alla polvere all'interno del contenitore in cui si vuole ottenere il terreno.
		Una volta mescolata la polvere si posiziona il contenitore all'interno dell'autoclave per la sterilizzazione del terreno mediante pressioni elevate $1.5\si{atm}$ senza bollire il terreno e rovinare il nutriente per $20$ minuti.
		Non bisogner\`a poi pi\`u aprire il contenitore se non sotto cappa, quando viene steso il terreno sulle piastre.

		\subsubsection{Terreni preparati}
		Vengono preparati $4$ terreni di coltura sterili:
		\begin{multicols}{2}
			\begin{itemize}
				\item \emph{LB agar}: terreno di base.
				\item \emph{LB agar-ampicillina}: terreno di coltura selettivo per i terreni ampicillina-resistenti.
				\item \emph{Nutrient agar}: terreno nutritivo adatto a ceppi non capaci di sintetizzare i nutrienti.
				\item \emph{Mueller-Hinton agar}: terreno di coltura che non contiene sostanze che interferiscono con antibiotici e si usa per il test per determinare l'antibiotico-resistenza.
			\end{itemize}
		\end{multicols}
\section{Seconda esperienza - Determinazione della curva di crescita di un ceppo batterico (E. coli)}
	
	\subsection{Introduzione}

		\subsubsection{Fattori che influenzano la crescita batterica}
		I fattori che influenzano la crescita batterica sono:
		\begin{multicols}{2}
			\begin{itemize}
				\item Nutrienti: fonti di carbonio, energia, acqua, vitamine, azoto e oligoelementi.
				\item Concentrazioni di sale: divide i batteri in alofili, alotolleranti o alofili estremi.
				\item Ossigeno: divide i batteri in aerobi obbligati, aerobi facoltativi, microaerofili, anaerobi aerotolleranti e anaerobi obbligati.
				\item $pH$: divide i batteri in alcalinofili, basofili e neutrofili.
				\item Temperatura: divide i batteri in psicrofili ($-10$-$30\si{\degree}$), mesofili ($10$-$50\si{degree}$), termofili ($40$-$90\si{\degree}$) e termofili estremi.
			\end{itemize}
		\end{multicols}

		\subsubsection{Quantificazione}
	
			\paragraph{Fasi della crescita microbica}\mbox{}
			\begin{enumerate}
				\item Lag: adattamento del batterio alle nuove condizioni.
				\item Esponenziale: crescita continua grazie alla presenza di nutrienti.
					Il tempo di replicazione specifico del batterio influisce sulla velocit\`a: Mycobacterium tubercolosis impiega $16$ ore, E. coli $20$ minuti.
				\item Stazionaria: ci sono troppi batteri sul terreno e i nutrienti iniziano ad esaurirsi.
					Gli eventi di replicazione e morte sono in equilibrio tra di loro.
				\item Morte: avviene progressivamente la morte per mancanza di nutrienti.
			\end{enumerate}
	\subsection{Primo giorno}

		\subsubsection{Micropipetta}
		La micropipetta \`e uno strumento in grado di prelevare volumi diversi.
		Possiede una scala graduata.

			\paragraph{Tipologie}\mbox{}\\
			\begin{multicols}{2}
				\begin{itemize}
					\item $p20$: da $2\si{\micro\litre}$ a $20\si{\micro\litre}$, i numeri letti vanno divisi per $10$.
					\item $p200$ da $20\si{\micro\litre}$ a $200\si{\micro\litre}$.
					\item $p1000$ da $200\si{\micro\litre}$ a $1000\si{\micro\litre}$, i numeri letti vanno moltiplicati per $10$.
				\end{itemize}
			\end{multicols}

			\paragraph{Utilizzo}\mbox{}\\
			\begin{multicols}{2}
				\begin{enumerate}
					\item Si inserisce un nuovo puntale ogni volta che si cambia sostanza.
					\item Si schiaccia lo stantuffo fino al primo scatto per verificare il volume da prelevare.
					\item Si inserisce nella sostanza da prelevare e si rilascia lo stantuffo superiore per prelevare il volume.
					\item Si inserisce nella sostanza il cui volume deve essere rilasciato e si preme lo stantuffo fino in fondo.
					\item Si rilascia il puntale tramite un pulsante vicino allo stantuffo.
				\end{enumerate}
			\end{multicols}

		\subsubsection{Tempo di generazione}
		Si intende per tempo di generazione il tempo di duplicazione della massa di una colonia batterica.
		Questo valore \`e specifico per ogni specie:
		\begin{multicols}{2}
			\begin{itemize}
				\item E. coli: $20$ minuti.
				\item Staphylococcus aureus: $30$ minuti.
			\end{itemize}
		\end{multicols}

		\subsubsection{Procedimento}
		\begin{enumerate}
			\item Inoculo: si prelevano $100\si{\micro\litre}$ di E. coli da una provetta preparata dagli esercitatori overnight tramite la $p200$ e la si inserisce in una beuta con $100\si{mL}$ di brodo.
				Si ottiene una diluizione $1:100$ e si mescola.
			\item Si preleva $1\si{mL}$ di coltura con una $p1000$ e la si pone nella cuvetta in modo da inserirla poi nello spettrofotometro per monitorare la crescita batterica.
				Si misura la densit\`a ottica e la si annota. Si \`e ora a $t=0$.
				Ci si deve assicurare che lo spettrofotometro sia a $OD=600\si{nm}$ e che sia stato tarato secondo il brodo sterile o bianco in modo da eliminarne il rumore.
			\item Si trasferisce la beuta in uno shaker o incubatore termostato orbitale che la mescola uniformemente e peremette una corretta areazione.
			\item Si puliscono micropipetta e cuvetta per mantenerli sterile.
				Per la cuvetta si getta il brodo nei contenitori e la si risciacqua con acqua distillata.
				La si fa asciugare a testa in giù su carta assorbente.
			\item Si procede allo stesso modo ogni $20$ minuti per ottenere un totale di $12$ misurazioni.
			\item Quando si nota una $OD=0.8$ si procede con diluizioni $1:2$ per rimanere nel range di sensibilit\`a dello spettrofotometro: si prelevano $500\si{\micro\litre}$ di coltura e $500\si{\micro\litre}$ di terreno.
				Quando necessario si fa una diluizione $1:5$ con $200\si{\micro\litre}$ di coltura e $800\si{\micro\litre}$ di terreno.
			\item Una volta terminate le misurazioni si versa la sospensione nella cuvetta nei rifiuti biologici liquidi e si cestina la cuvetta nei rifiuti biologici solidi.
				Si versa la sospensione nella bottigliata in vetro da $100\si{ml}$ nei rifiuti biologici iquidi e la si pone nel contenitore con la vetreria da lavare.
		\end{enumerate}

		\subsubsection{Analisi dei dati}
		\begin{enumerate}
			\item Si effettua un grafico a punti con la curva di crescita microbica tramite i dati ottenuti.
				Sull'asse delle $x$ si pone il tempo, mentre su quello delle $y$ la densit\`a ottica a $600\si{\nano\metre}$.
			\item Si individuano le fasi di crescita osservate durante la crescita microbica.
			\item Si calcolano il numero di generazioni e al tempo di generazione della coltura.
		\end{enumerate}
		Per calcolare il numero di generazioni $n$ dopo un intervallo si usa la formula:
		\[N_{t_f} = N_0 \cdot 2^n\]
		\[\log N_{t_f} = \log N_0 + n\log 2\]
		Dove:
		\begin{multicols}{3}
			\begin{itemize}
				\item $N_0$ \`e il numero di batteri a $t=0$.
				\item $N_{t_f}$ \`e il numero di batteri a $t_f$.
				\item $t_f$ \`e $\Delta t = t_i - t_0$.
			\end{itemize}
		\end{multicols}
		Da cui deriva che il numero di generazioni $n$ \`e:
		\[n = \dfrac{\log N_{t_f} - \log N_0}{\log 2}\]
		Si pu\`o anche calcolare il tempo di generazione $g$ avvenute in un $\Delta t = t_2 - t_1$:
		\[g = \dfrac{t_2 - t_1}{n}\]
			
			\paragraph{Dati raccolti}
			\begin{center}
				\begin{tabular}{|c|c|}
					\hline
					$N_0$ & \num{60000}\\
					\hline
					$N_{t_f}$ & \num{38000000}\\
					\hline
					$t$ & $300 min$\\
					\hline
					$n$ & $12.6$ generazioni\\
					\hline
					$g$ & $23.8 min$\\
					\hline
				\end{tabular}
			\end{center}



\section{Terza esperienza - Caratterizzazione dei batteri del cavo orale}

	\subsection{Introduzione}
	Nel caso orale si trovano numerosi batteri anche simbiotici in un pattern specifico alla persona in quanto dipende da:
	\begin{multicols}{3}
		\begin{itemize}
			\item $pH$.
			\item Umidit\`a.
			\item Nutrienti della dieta.
		\end{itemize}
	\end{multicols}
	
		\subsubsection{Piastra su agar-sangue}
		Il risultato del tampone viene seminato su una piastra di agar-sangue con il $5\%$ di sangue di montone. 
		Questo in quanto alcuni batteri come lo Streptococco sono in grado di attuare emolisi, ovvero di degradare i globuli rossi.

			\paragraph{Emolisi}
			Ci sono tre tipi di emolisi:
			\begin{itemize}
				\item Emolisi $\beta$ o emolisi completa: il sangue viene completamente degradato e il terreno appare giallo.
				\item Emolisi $\alpha$ o emolisi incompleta: il sangue viene degradato parzialmente e il terreno appare verdognolo.
				\item Emolisi $\gamma$ o non-emolisi: il sangue non viene degradato e il terreno rimane rosso.
			\end{itemize}

			\paragraph{Semina sulla piastra}
			\begin{enumerate}
				\item Prima patch: si striscia il tampone in modo da occupare la parte superiore della piastra.
				\item Seconda patch: si ruota la piastra di $90\si{\degree}$ e si striscia il tampone in modo che solo qualche strisciata si sovrapponga alla prima patch.
				\item Terza patch: si ruota ancora la piastra di $90\si{\degree}$ e si ripete il procedimento.
			\end{enumerate}

		\subsubsection{Morfologia dei batteri del cavo orale}
		Per identificare le colonie le si distinguono in base a colore, forma, spessore e margini.

			\paragraph{Aspetto}\mbox{}\\
			\begin{multicols}{3}
				\begin{itemize}
					\item Puntiforme.
					\item Circolare.
					\item Filamentoso.
					\item Irregolare.
					\item Rizoide.
					\item Lenticolare.
					\item Raggiato.
				\end{itemize}
			\end{multicols}

			\paragraph{Rilievo}\mbox{}\\
			\begin{multicols}{2}
				\begin{itemize}
					\item Rasato.
					\item Poco convesso.
					\item Convesso.
					\item Pulvinato.
					\item Umbonato.
					\item Diffuso.
					\item Rilevato.
					\item Cupuliforme.
					\item Mammellonato.
					\item Con o senza margine smussato.
				\end{itemize}
			\end{multicols}

			\paragraph{Superficie}\mbox{}\\
			\begin{multicols}{3}
				\begin{itemize}
					\item Liscia.
					\item Rugosa.
					\item Raggiata.
					\item Opaca o luccicante.
					\item Inglobata.
					\item Asciutta o umida.
				\end{itemize}
			\end{multicols}
			
			\paragraph{Margini}\mbox{}\\
			\begin{multicols}{3}
				\begin{itemize}
					\item Continui
					\item Interi.
					\item Ondulati.
					\item Lobati.
					\item Erosi.
					\item Filamentosi.
					\item Stratificati.
					\item Diffusi.
					\item Seghettati.
				\end{itemize}
			\end{multicols}

			\paragraph{Struttura}\mbox{}\\
			\begin{multicols}{2}
				\begin{itemize}
					\item Amorfa.
					\item Granulare.
					\item Filamentosa.
					\item Ondulata.
				\end{itemize}
			\end{multicols}

			\paragraph{Altri fattori}\mbox{}\\
			\begin{multicols}{3}
				\begin{itemize}
					\item Dimensione.
					\item Colore.
					\item Opacit\`a.
					\item Consistenza (cremosa, mucosa, friabile, membranosa).
				\end{itemize}
			\end{multicols}
	
	\subsection{Secondo giorno}
	All'interno della bocca sono contenuti diversi tessuti a cui sono associati diversi ceppi batterici.
	Alla nascita la cavit\`a orale \`e sterile: la prima colonizzazione avviene tra le $6$ e le $10$ ore dopo la nascita.
	
		\subsubsection{Equilibrio plastico}
		Le popolazioni microbiche si dicono in equilibrio plastico in quanto si trovano in una condizione di equilibrio dinamico, con cambiamenti dovuti ad abitudini alimentari sbalzi ormonali e altri cambiamenti che si subiscono durante la vita.

		\subsubsection{Ecosistema orale}
		L'ecosistema orale \`e formato da microorganismi orali e dalla cavit\`a orale che li ospita.
		La popolazione batterica \`e estremamente diversificata e abbondante: si contano pi\`u di $300$ specie in grado di colonizzarla.
		Questa convivenza, per lo pi\`u pacifica, si dice di simbiosi o commensalismo.
		Con l'insorgenza di fenomeni patologici si passa a rapporti opportunistici.
		Oltre ai batteri sono anche presenti virus e funghi.
		
			\paragraph{Esempi}
			Lo Streptococcus pyogenes del gruppo A \`e un fungo che pu\`o essere presente nella cavit\`a orale e patogeno: produce tossine, fattori di virulenza ed emolisine.
			Se raggiunge il cuore pu\`o produrre endocarditi gravi.

		\subsection{Procedimento}
		\begin{enumerate}
			\item Si prepara una piastra Petri agar-sangue e la si contrassegna sul bordo con data, nome gruppo, protocollo e piastra $A$.
			\item Con spatola e tampone sterile si preleva del materiale dalla parte superiore della lingua.
			\item Si usa il tampone su una porzione della piastra (un quinto) e si striscia con un'ansa sterile la popolazione microbica prelevata tramite la tecnica del quadrante.
				Grazie all'ansa si notano quattro strisciate complessiva.
				L'ultima permette di isolare le colonie.
			\item Si pone la piastra capovolta nell'incuvatore statico termostatato a $37\si{\degree}$ per $24$ ore.
			\item Il giorno successivo si notano diverse colonie batteriche che vengono distinte per fenotipo.
			\item Si selezionano due colonie con morfologia e/o emolisi diversa sulla piastra e si contrassegna una nuova piastra agar-sangue con data, nome gruppo e protocollo e piastra $B$.
			\item Attraverso lo striscio continuo si piastriano sulla piastra $B$ le colonie batteriche che si vogliono isolare e si pone la piastra capovolta nell'incubatore come la prima.
			\item Il giorno successivo si effettua un'analisi dettagliata del fenotipo batterico in riferimento alle colonie isolate.
			\item Si confrontano i risultati con la piastra $A$.
			\item Si gettano nei biobox entrambe le piastre.
		\end{enumerate}
\section{Quarta esperienza - Conta standard su piastra}

	\subsection{Introduzione}

		\subsubsection{Conta microbica indiretta}
		La conta microbica indiretta avviene tramite spettrofotometro: la densit\`a ottica nella cuvetta \`e proporzionale alla quantit\`a di batteri presenti.
		Non permette per\`o di distinguere tra cellule vive e morte.

		\subsubsection{Conta vitale}
		La conta vitale su piastra permette di contare unicamente le cellule vive.

			\paragraph{Processo}
			\begin{enumerate}
				\item Si prende la provetta contenente le colonie batteriche.
				\item Si fanno diluizioni seriali su piastra in brodo di coltura o soluzione salina.
				\item Si prelevano $0.1\si{mL}$ dalla diluizione e si distribuiscono uniformemente su piastra tramite ansa a l.
				\item Si incuba ogni diluizione.
				\item Si conta la piastra in cui le colonie si distinguono correttamente e non sono ammassate (tipicamente tra le $10$ e le $200$).
			\end{enumerate}
	\subsection{Secondo giorno}

		\subsubsection{Metodi della conta batterica}
		
			\paragraph{Conta diretta}
			La conta diretta avviene al microscopio o allo spettrofotometro e non si distinguono cellule vive e morte.

			\paragraph{Conta indiretta}
			La conta indiretta avviene in coltura e si contano solo le cellule vive, le uniche in grado di riprodursi e formare una colonia nel terreno.
			Per attuare una conta batterica indiretta si procede per diluizioni seriali provenienti da un unico brodo di coltura \emph{LB} sterile standard.
			Si attuano diluizioni seriali $1:10$.
			Avviene poi piastratura di $1\si{mL}$ di ogni soluzione e si incubano le piastre.

			\paragraph{Conta vitale}
			La conta vitale consente di contare solo cellule vive presenti in una sospensione batterica.
			Si dice anche conta su piastra o conta delle colonie.

		\subsubsection{Colonia}
		Si definisce colonia un gruppo di cellule batteriche appartenenti allo stesso ceppo o specie che ha origine da una sola cellula vitale.
		Una cellula in grado di formare una colonia \`e detta unit\`a formante colonia o \emph{UFC} o \emph{CFU}.
		Il numero di colonie contate sulla piastra corrisponde al numero di \emph{UFC} dell'inoculo.

		\subsubsection{Procedimento}
		\begin{enumerate}
			\item Si contrassegna ognuna delle $9$ provette con $900\si{\micro\litre}$ di brodo \emph{LB} sterile con la diluizione rispettiva: $10^{-1;-9}$.
			\item SI contrassegna il bordo del fondo delle $9$ piastre di \emph{LB-agar} da $90\si{mm}$ di diametro con nome gruppo, nome protocollo, diluizione della provetta corrispondente e volume piastrato.
			\item Si contrassengla la piastra rimanente con nome gruppo e non diluita, \`e dedicata alla sospensione batterica non diluita.
			\item Si agita la provetta con la coltura batterica non diluita con il vortex e si prelevano $100\si{\micro\litre}$ e la si trasferisce nella provetta $10^{-1}$.
				Questa \`e la prima diluizione seriale.
			\item Dopo aver agitato con il virtex si prelevano $100\si{\micro\litre}$ dalla diluizione e si trasferiscono nella provetta $10^{-2}$ e si mescola.
				Questa \`e la seconda diluizione seriale.
			\item Si ripete la procedura fino ad arrivare all'ultima diluizione.
			\item Si prelevano $100\si{\micro\litre}$ della sospensione batterica di partenza, si trasferiscono al centro della piastra corrispondente e si piastra con ansa sterile.
			\item Si prelevano $100\si{\micro\litre}$ dalla sospensione diluita e si trasferiscono al centro della piastra corrispondente e si piastra con ansa sterile.
			\item Si uniscono le piastre con nastro adesivo su cui si scrive il gruppo. 
				Si pongono capovolte nell'incubatore a $37\si{\degree}$ per $18$-$24$ ore.
			\item Una volta concluso si versa il contenuto delle provette in plastica nei rifiuti biologici liquidi e si buttano le provette in plastica nei rifiuti biologici solidi.
			\item Si conta e annota il numero di colonie presenti in ogni piastra.
			\item Per la conta si inizia dalla piastra con il maggior contenuto di colonie contabili.
				Si escludono le piastre con numero di colonie minore di $10$.
		\end{enumerate}

		\subsubsection{Analisi dei risultati}
		Si calcola la concentrazione della sospensione batterica tramite il calcolo della media delle \emph{CFU} ottenute nelle piastre contate.
		\[Concentrazione \dfrac{n\cdot f}{v}\dfrac{CFU}{\si{mL}}\]
		Dove:
		\begin{multicols}{3}
			\begin{itemize}
				\item $n$ \`e il numero di colonie contate su una piastra.
				\item $f$ \`e il fattore di diluizione (inverso della diluizione operata).
				\item $v$ \`e il volume di sospensione batterica piastrata.
			\end{itemize}
		\end{multicols}

\section{Quinta esperienza - test di aerobiosi/anaerobiosi su terreno solido}

	\subsection{Introduzione}

		\subsubsection{Distinguere i batteri in base alla richiesta di ossigeno}
		In base alla loro richiesta di ossigeno i batteri si dividono in:
		\begin{multicols}{2}
			\begin{itemize}
				\item Anaerobi obbligati: non tollerano l'ossigeno.
				\item Anaerobi facoltativi: crescono con la respirazione cellulare in presenza di ossigeno ma se \`e assente con altri metabolismi.
				\item Microaerofili: batteri che richiedono una quantit\`a di ossigeno inferiore a quella presente nell'aria: $5$-$10\%$.
				\item Aerotolleranti: sono indifferenti alla presenza di ossigeno, effettuano la fermentazione, preferiscono una quantit\`a di ossigeno inferiore al $20\%$.
				\item Aerobi obbligati: richiedono ossigeno.
			\end{itemize}
		\end{multicols}

		\subsubsection{Batteri modello}
		Per verificare l'aerobiosi o anaerobiosi di una colonia si utilizzano:
		\begin{multicols}{2}
			\begin{itemize}
				\item Citrobacter freundii: anaerobio facoltativo, Gram$-$.
				\item Micrococcus luteus: aerobio obbligato, Gram$+$.
			\end{itemize}
		\end{multicols}

		\subsubsection{Processo}
		\begin{enumerate}
			\item Si prendono due piastre di terreno solido e le si divide a met\`a.
			\item In ogni piastra si semina su una met\`a Citrobacter freundii e sull'altra Micrococcus luteus.
			\item Una piastra viene posta in condizioni di aerobiosi mentre l'altra viene posta all'interno di una giara in cui si produce una condizione di anerobiosi.
			\item Si verifica che Citrobacter freundii cresce in entrambe le piastre mentre Micrococcus luteus solo nella piastra con ossigeno.
		\end{enumerate}
		Nella giara vengono inseriti dei filtri in grado di eliminare l'ossigeno presente grazie a reazioni attuate all'interno della giara stessa.

	\subsection{Seconda giornata}
	Si verifica su terreno solido l'aerobiosi o anaerobiosi di due ceppi batterici:
	\begin{multicols}{2}
		\begin{itemize}
			\item Citrobacter freundii: bacillo anaerobio facoltativo Gram$-$, in piastra di agar-sangue.
			\item Micrococcus luteus: cocco aerobio obbligato Gram$+$, in piastra \emph{TLC}.
		\end{itemize}
	\end{multicols}
	I due ceppi rimangono ignoti e detti ceppo $1$ e ceppo $2$.

		\subsubsection{Creazione di condizione di anaerobiosi}
		Per creare una condizione di anaerobiosi si utilizzano giare, contenitori a chiusura ermetica in cui vengono inseriti sistemi \emph{GAS-PACK}, sacchetti contenenti reagenti chimici con la capcait\`a di eseguire una reazione chimica che elimina l'ossigeno nell'ambiente.
		Si aggiungono pochi $\si{mL}$ di acqua al gas-pack in modo da attivare la reazione con liberazione di \emph{$CO_{2(g)}$} e \emph{$H_{2(g)}$}.
		Sul tappo della giara \`e presente un catalizzatore al palladio sulla quale si forma acqua.
		
			\paragraph{Verificare l'assenza di ossigeno}
			Per verificare l'assenza di ossigeno si usano indicatori come:
			\begin{itemize}
				\item Resazurina: azzurra in presenza di ossigeno e rosa in sua assenza.
				\item Blu di metilene: blu in presenza e incolore in assenza di ossigeno.
			\end{itemize}

		\subsubsection{Procedimento}
		\begin{enumerate}
			\item Si disegnano due quadranti sul fondo di due piastre di Nutrient-agar con un pennarello.
				Si marca ogni quadrante con una cifra e si scrive si una piastra aerobiosi e sull'altra anaerobiosi.
			\item Si preleva $1$ colonia dalla piastra del ceppo $1$ e la si deposita sul quadrante corrispondente di entrambe le piastre con ansa.
			\item Si distribuisce la colonia con lo striscio continuativo non uscendo dal quadrante.
			\item Si ripete l'operazione con l'altro ceppo. 
			\item Si dispone la piastra anaerobiosi vicino alla giara in cui gli esercitatori inseriscono blu di metilene, bustina del catalizzatore.
				Avviene incubazione a $37\si{\degree}$.
				Si attendono $24$ ore.
			\item Il giorno successivo si paragona il fenotipo di crescita di ogni ceppo batterico in condizioni aerobiche e anaerobiche e si annotano le osservazioni.
			\item Si classificano i $2$ ceppi in base al risultato del test.
		\end{enumerate}
		
\section{Sesta esperienza - test Kirby-Bauer}

	\subsection{Introduzione}
	L'abuso di antibiotici ha causato lo sviluppo di antibiotico-resistenza da parte dei batteri per alcuni antibiotici.
	Si tratta della minaccia mondiale pi\`u pericolosa dal punto di vista sanitario vista la facilit\`a del trasferimento genico orizzontale.
	
		\subsubsection{Test di Kirby-Bauer}
		Durante il test di Kirby-Bayer per la verifica dell'antibiotico-resistenza:
			
			\paragraph{Procedimento}
			\begin{enumerate}
				\item Si prende una piastra con agar solido Mueller-Hinton e si semina uniformemente una certa specie batterica.
				\item Si applicano dischetti imbevuti ognuno da un antibiotico specifico
				\item Si incubano le piastre.
			\end{enumerate}

			\paragraph{lettura dei risultati}
			La sensibilit\`a a un antibiotico viene determinata come alone di inibizione: il diametro dell'alone indica il grado di sensibilit\`a.
			L'assenza di alone indica un antibiotico-resistenza.

\section{Settima esperienza - test biochimico \emph{API 20E}}

	\subsection{Introduzione}
	Esistono molti metodi biochimici per determinare la specie di una coltura batterica.
	Uno dei pi\`u utilizzati \`e il test \emph{API 20E}.

		\subsubsection{Test \emph{API 20E}}
		Il test \emph{API 20E} si compone di $21$ reazioni metaboliche e $6$ test biochimici supplementari.
		Verr\`a utilizzato per identificare \emph{Enterobacteriaceae} e altri Gram$-$.

			\paragraph{Processo}
			\begin{multicols}{2}
				\begin{enumerate}
					\item Si preleva una colonia.
					\item Si stempera la colonia in acqua sterile.
					\item Si riempono i microtubuli o microgallerie con la colonia stemperata.
						I microtubuli sono alti $1\si{cm}$ e larghi $0.5\si{cm}$ e contengono terreno liofilizzato.
					\item Si incubano a $37\si{\degree}$ per una notte.
				\end{enumerate}
			\end{multicols}

			\paragraph{Lettura dei risultati}
			In base alla reazione chimica metabolica che avviene in ogni pozzetto o microtubulo o galleria si nota una colorazione diversa.
			Il colore denota l'avvenimento o meno della reazione.
			Moduli identificano la specie.

				\subparagraph{Lettura dei moduli}
				A ogni pozzetto corrisponde un numero.
				Se la reazione \`e avvenuta nel pozzetto considero il numero, altrimenti no.
				I pozzetti sono divisi in trittici: sommando i numeri da considerare per ogni trittico si ottiene un codice univoco per la specie.

\section{Ottava esperienza - Colorazione di Gram}
	\subsection{Introduzione}
	La colorazione di Gram differenzia batteri Gram$+$ che appaiono di color violetto e batteri Gram$-$ che appaiono rosa.
	Questo avviene in quanto viene usata la sostanza Crystal Violet che colora di violetto lo strato di peptidoglicano: essendoci nei Gram$+$ uno strato pi\`u spesso, la colorazione \`e pi\`u forte.

		\subsubsection{Procedimento}
		\begin{enumerate}
			\item Si applicano i batteri sulla piastra e si applica il Crystal Violet.
			\item Si applica iodio che funge da mordente: attacca al peptidoglicano il Crystal Violet.
			\item Si decolora tramite alcol: solo i batteri su cui il Crystal Violet ha attaccato grazie al mordente rimangono colorati (Gram$+$).
			\item Si utilizza la safranina per colorare i Gram$-$ di rosa.
		\end{enumerate}

\section{Nona esperienza - Osservazione della motilit\`a batterica di tipo ``swimming'' al microscopio ottico}

	\subsection{Introduzione}
	I Batteri si distinguono anche grazie alla loro motilit\`a, che pu\`o avvenire attraverso flagelli o pili.

		\subsubsection{Flagelli}
		I flagelli sono lunghe appendici proteiche che servono al moto della cellula batterica.
			
			\paragraph{Tipi di movimento}
			Il movimento pu\`o essere:
			\begin{multicols}{2}
				\begin{itemize}
					\item Tumbling: casuale, con continui cambi di direzione.
					\item Swimming: in una sola direzione.
				\end{itemize}
			\end{multicols}

			\paragraph{Distinzione in base a numero e posizione dei flagelli}\mbox{}\\
			\begin{multicols}{2}
				\begin{itemize}
					\item Monotrofico: un solo flagello all'estremit\`a.
					\item Lofotrico: un gruppo id flagelli all'estremit\`a.
					\item Anfitrico: flagelli a estremit\`a opposte.
					\item Peritrico: flagelli distribuiti su tutta la superficie.
				\end{itemize}
			\end{multicols}

\section{Decima esperienza - Trasformazione di batteri}

	\subsection{Introduzione}
	Per produrre batteri ricombinanti si utilizza la tecnologia del DNA ricombinante tramite inserimento di un gene in un battere per mezzo dei plasmidi.

		\subsubsection{Plasmidi}
		Si utilizzer\`a un plasmide contenente:
		\begin{multicols}{2}
			\begin{itemize}
				\item \emph{GFP}: green fluorescent protein.
				\item Gene per la resistenza all'ampicillina: creazione di un terreno selettivo.
			\end{itemize}
		\end{multicols}
		Questo verr\`a inserito in E. coli.

		\subsubsection{Processo di trasformazione}
		Si trasformano i batteri tramite shock termico, abbassando la temperatura da $43\si{\degree}$ a $3\si{\degree}$ per facilitare l'entrata nel plasmide.
		Si semina su un terreno contenente ampicillina \emph{Amp} e si incuba.
		In questo modo si eliminano le cellule non trasformate (non resistenti ad ampicillina).
		Le cellule trasformate contengono fluorescenza e resistenza ad \emph{Amp}.


