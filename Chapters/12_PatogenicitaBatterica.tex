\chapter{Patogenicit\`a batterica}

\section{Introduzione}
Sia i batteri che gli eucarioti possono essere patogeni e ad ogni microorganismo viene associata una priorit\`a nel contenerlo e nel trattarlo.

\section{Simbiosi}

	\subsection{Commensalismo}
	Il commensalismo \`e un'interazione simbiotica non obbligatoria tra due esseri viventi in cui solo uno approfitta del nutrimento o degli scarti dell'altro senza procurare sofferenza o disturbo.

	\subsection{Mutualismo}
	Il mutualismo \`e una forma di interazione tra individui di specie diversa grazie alla quale entrambi riescono a ricevere dei benefici.

	\subsection{Parassitismo}
	Il parassitismo \`e una forma di interazione biologica, tipicamente di natura trofica fra due specie di organismi in cui uno \`e detto parassita e l'altro ospite.
	Il parassita trae vantaggi a spese dell'ospite creandogli un danno biologico.

		\subsubsection{Caratteristiche}
		\begin{multicols}{2}
			\begin{itemize}
				\item Il parassita \`e privo di vita autonoma e dipende dall'ospite a cui \`e pi\`u o meno intimamente legato da una relazione anatomica e fisiologica.
				\item Il parassita ha una struttura anatomica e morfologica semplificata rispetto all'ospite.
				\item Il ciclo vitale del parassita \`e pi\`u breve di quello dell'ospite e si conclude prima della morte dell'ospite.
				\item Il parassita ha rapporti con un solo ospite.
			\end{itemize}
		\end{multicols}
	
	\subsection{Interazioni dannose tra microorganismi e uomo}
		
		\subsubsection{Infezione}
		Si intende per infezione la penetrazione e moltiplicazione di microorganismi in un ospite.

		\subsubsection{Malattia}
		Si intende per malattia l'alterazione dello stato fisiologico e psicologico di un organismo capace di riprodurre, modificare negativamente o eliminare le funzionalit\`a normali del corpo.

		\subsubsection{Patogenicit\`a}
		Si intende per patogenicit\`a la capacit\`a da parte di un organismo di causare una malattia.

		\subsubsection{Virulenza}
		Si intende per virulenza una misura della patogenicit\`a con un valore quantitativo.
		Permette di capire quanto un organismo \`e pericoloso nei confronti dell'ospite.

\section{Determinare quantitativamente la virulenza}
Per determinare la virulenza quantitativamente si misura la $LD_{50}$, ovvero la dose necessaria per uccidere il $50\%$ delle cellule saggiate.
Pi\`u la dose \`e alta pi\`u organismi vengono uccisi.
Nei patogeni altamente virulenti la $LD_{50}$ \`e vicina alla $LD_{100}$, in quelli moderatamente virulenti si distinguono per $2$ o $3$ ordini di grandezza.

\section{Fattori di virulenza}

	\subsection{Heliobacter pilori}
	L'heliobacter pilori \`e  un microorganismo che vive nello stomaco ma pu\`o creare lesioni, ulcere e forme di cancro.
	Produce ureasi che idrolizzano l'urea provocando un locale innalzamento del $pH$.
	I lipopolisaccaridi gli permettono di aderire alle cellule stimolando l'infiammazione.
	L'adesina indica tutte le proteine importanti nel processo di adesione, gli effettori consentono un invasione dei tessuti tramite un sistema di secrezione di tipo $IV$.
	Attraverso esotossine vengono causati danni alle cellule permettendo al batterio di vivere in un ambiente particolare dello stomaco riducendo la viscosit\`a del muco aumentando la persistenza dei batteri.

	\subsection{Salmonella}
	La salmonella \`e un batterio patogeno per l'uomo.
	Si trovano lipopolisaccaridi che consentono di produrre un'infiammazione.
	Produce inoltre tossine come enterotossina e citotossina.
	I siderofori consentono di catturare il ferro, cofattore enzimatico che riesce a strappare dalle cellule.
	Fimbrie di tipo $I$ e le appendici consentono aderenza, mentre la motilit\`a \`e importante in quanto gli permette di svuggire dall'antigene $O$.
	Il flagello e l'antigene della capsula $IV$ lo proteggono dalla fagocitosi.

	\subsection{Infezioni da Pseudomonas aeruginosa in pazienti con fibrosi cistica}
	I pazienti affetti da fibrosi cistica hanno un epitelio polmonare con un muco viscoso e spesso.
	Si creano squilibri ionici che lo portano ad essere denso e disidratato, permettendo infezioni batteriche non eliminabili.
	Pseudomonas aeruginosa all'inizio del ciclo \`e aggressivo: produce una serie di fattori di virulenza, dal flagello per la motilit\`a, attivazione del quorum sensing, espulsione di enzimi con propriet\`a degradative.
	Nel frattempo cambia metabolismo producendo muco e biofilm riduce i fattori di virulenza, assume resistenza agli antibiotici e aumentano le mutazioni.

	\subsection{Genoma accessorio}
	Molti fattori di virulenza sono associati al genoma accessorio.
	Il genoma accessorio \`e flessibile e assunto attraverso trasferimento orizzontale.
	Le isole genomiche sono circoscrizioni del genoma di chilobasi con funzioni specifiche come le isole di patogenicit\`a.
	Le isole di patogenicit\`a tipicamente si inseriscono vicino ai geni codificanti il tRNA.

		\subsubsection{Evoluzione dei genomi batterici e delle funzioni associate alla patogenicit\`a e virulenza}
		Lo stile di vita dell'organismo influenza l'evoluzione del genoma: i patogeni obbligati tendono ad avere meno geni, mentre i facoltativi ne acquisiscono di nuovi.
		Avvengono sempre mutazioni e riarrangiamenti.

\section{Caratteristiche di un batterio patogeno}

	\subsection{Caratteristiche}
	\begin{multicols}{2}
		\begin{itemize}
			\item Acquisizione di DNA alieno: funzioni necessarie per diventare patogeno.
			\item Quorum sensing con capacit\`a di percepire l'ambiente come densit\`a cellulare.
			\item Raggiungere il sito di infezione.
			\item Aderire al sito bersaglio.
			\item Trovare i nutrienti e moltiplicarsi.
			\item Colonizzare o invadere il tessuto.
			\item Sopravvivere allo stress come la risposta immunitaria.
			\item Sovvertire la fisiologia del tessuto bersaglio.
			\item Danneggiare l'ospite.
			\item Diffondersi nell'organismo ospite.
			\item Ritornare nei reservoir o nella nicchia ecologica per poi riprendere un nuovo processo infettivo.
		\end{itemize}
	\end{multicols}

	\subsection{Fattori di adesione}
	Le infezioni iniziano a livello di mucose o ferite della pelle.
	L'aderenza \`e importante in quanto definisce la specificit\`a di un patogeno.
	Neisseria gonorrhoeae aderisce pi\`u fortemente alle cellule dell'epitelio urogeneitale attraverso \emph{Opa} che lega il recettore \emph{CD66}.

		\subsubsection{Sistemi di adesione}

			\paragraph{Pili o fimbrie}
			I pili o fimbrie sono adesine che legano le glicoproteine espresse sulla superficie dell'ospite.

			\paragraph{Adesine afimbriali}
			Le adesine fimbriali interagiscono con un recettore specifico.
			L'aderenza iniziale pu\`o avvenire tramite proteine della matrice extracellulare come fibronectina.


	\subsection{Esotossine}
	Le esotossine sono proteine solubili e termolabili che vengono rilasciate dal patogeno nell'ambiente.
	Possono raggiungere, diffondendo, lo stato di infezione in siti distanti dal sito d'infezione primario.
	Sono sostanze letali e immunogeniche: stimolano la produzione di anticorpi neutralizzanti.
	Possono essere inattivate chimicamente per la produzione di vaccini.

		\subsubsection{Meccanismi di trasporto}

			\paragraph{Tossine \emph{AB}}
			Le tossine \emph{AB} si trovano in forma eterodimerica.
			Il dominio \emph{B} porta al legame con un recettore di membrana, mentre il dominio \emph{A} \`e la tossina.
			Il \emph{B} si lega al recettore causando un cambio conformazionale attraverso il quale \emph{A} entra nella cellula.
			\emph{B} si stacca dal recettore e viene rigenerato il sito di legame.

			\paragraph{Endocitosi recettore mediata}
			L'endocitosi recettore mediata porta all'endocitosi dell'intero eterodimero.
			Il legame dell'esotossina forma un complesso recettore-ligando che viene internalizzato attraverso un'invaginazione rivestita di clatrina che si richiude.
			Si forma una vescicola e si depolimerizza l'involucro di clatrina.
			L'abbassamento del $pH$ provoca la disgiunzione di \emph{A} e \emph{B}.
			\emph{B} viene riciclato verso la superficie cellulare, mentre \emph{A} esercita la sua azione tossica.


		\subsubsection{Esempi}

			\paragraph{Tossina difterica}
			La tossina difterica \`e prodotta da Corynebacterium diphterium.
			I batteriofagi sono utili al batterio per la veicolazione della sua patogenicit\`a.
			Il gene \emph{tox} codifica la tossina viene fornito al batterio esclusivamente dal batteriofago $\beta$ lisogenico.
			La tossina \emph{AB} viene secreta dal patogeno come singolo polipeptide.
			Quando \emph{A} entra nel citoplasma della cellula ospite catalizza \emph{EF-2} che normalmente legano tRNA e i ribosomi non riescono a sintetizzare proteine correttamente.

			\paragraph{Tossina botulinica}
			La tossina botulinica \`e prodotta da Clostridium botulinum.
			Questa tossina \`e estremamente velenosa, con $LD_{50} = 3\cdot 10^{-12}\si{g}$.
			Consiste di una serie di $7$ tossine \emph{AB} di cui $2$ son sintetizzate a partire da geni localizzati su un profago.
			Il complesso si lega alle membrane presinaptiche a livello delle terminazioni dei neuroni in corrispondenza delle giunzioni neuromuscolari.
			Blocca il rilascio dell'acetilcolina che consente la contrazione muscolare, portando a paralisi e morte.
			Attacca le proteine \emph{SNARE}.

			\paragraph{Tossina tetanica}
			La tossina tetanica \`e prodotta da C. tetani, \`e di tipo \emph{AB} e viene trasportata attraverso i motoneuroni nel midollo spinale.
			Qui si lega a lipidi dei neuroni inibitori che rilasciano normalmente \emph{G} che blocca la liberazione di acetilcolina inibendo la contrazione.
			La tossina blocca il rilascio e il rilassamento muscolare portando a una contrazione incontrollata e a una paralisi spastica.

			\paragraph{Tossina colerica}
			La tossina colerica \`e prodotta da Vibrio cholerae.
			\`E formata da $5$ subunit\`a \emph{A} e \emph{B}.
			La tossina lega il recettore \emph{GM1} che porta a una cascata di segnalazioni quando \emph{A} entra nel citoplasma delle cellule epiteliali.
			Attiva adenilatociclasi inducendo la conversione da \emph{ATP} a \emph{cAMP} bloccando il trasporto di \emph{$Na^+$}.
			L'uscita massiccia di ioni dal sangue porta a un accumulo di acqua nel lumen portando a disidratazione.

			\paragraph{Emolisine}
			Le emolisine sono esotossine che agiscono a livello della membrana citoplasmatica. 
			Non sono di tipo \emph{AB} ma formano canali.
			Uccidono eitrociti e leucociti senza degradare la membrana ma ponendosi al suo interno e facendo uscire il contenuto della cellula incorporando acqua.

				\subparagraph{$\mathbf{\beta}$-emolisi}
				La $\beta$-emolisi causa una completa lisi degli eritrociti.

				\subparagraph{$\mathbf{\alpha}$-emolisi}
				La $\alpha$-emolisi causa una parziale lisi degli eritrociti.
				

	\subsection{Endotossine}
	Le endotossine sono i lipopolisaccaridi \emph{LPS} della membrana esterna dei Gram$-$.
	Sono parte integrante della cellula batterica e vengono rilasciate in grande quantit\`a quando va incontro a lisi provocando febbre stimolando la produzione di pirogeni.
	La tossicit\`a \`e inferiore rispetto alle esotossine e mediata dalla frazione lipidica, mentre la polisaccaridica rende il complesso idrosolubile e immunogenico.


