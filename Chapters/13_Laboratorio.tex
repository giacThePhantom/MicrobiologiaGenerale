\chapter{Laboratorio}

\section{Prima esperienza - preparazione del terreno di coltura sterile}

	\subsection{Introduzione}

		\subsubsection{Categorie di terreni batterici}

			\paragraph{Determinazione in base allo stato fisico}
			In base allo stato fisico i terreni batterici si distinguono in:
			\begin{itemize}
				\item Terreni liquidi o brodi: usati per la coltivazione batterica.
				\item Terreni solidi o gel: usati sia che per la coltivazione batterica che per l'isolamento.
					Sono terreni liquidi gelificati tramite \emph{AGAR}, sostanza polisaccaride isolata da un'alga rossa in Giappone.
					Gelifica a temperature inferiori ai $45\%$.
			\end{itemize}
	
			\paragraph{Determinazione in base alla quantit\`a di sostanze}
			In base alla quantit\`a di sostanze presenti i terreni batterici si distinguono in:
			\begin{itemize}
				\item Terreni minimi: sono utilizzati per la crescita dei soli batteri autotrofi.
					Contengono solo gli elementi essenziali $N$, $C$, $S$, $P$ come sali inorganici in composizione e quantit\`a note.
				\item Terreni sintetici o definiti: vengono preparati \emph{ad hoc} a seconda delle esigenze nutrizionali del microorganismo.
					Se ne conosce l'esatta composizione.
				\item Terreni complessi: permettono la crescita di pi\`u organismi con esigenze nutrizionali diverse.
					Non se ne conosce l'esatta composizione.
			\end{itemize}

			\paragraph{Determinazione in base alla qualit\`a delle sostanze}
			In base alla qualit\`a delle sostanze presenti i terreni batterici si distinguono in:
			\begin{itemize}
				\item Terreni nutritivi: favoriscono la crescita di microorganismi particolari dal punto di vista nutritivo:
					al terreno vengono aggiunte sostanze nutritive come siero, latte o sangue per favorire una specie specifica.
				\item Terreni selettivi: favoriscono la crescita di particolari specie batteriche grazie alla presenza di fattori che inibiscono lo sviluppo di altre specie.
					I fattori vengono detti sostanze inibenti e possono essere antibiotici, coloranti o sali.
				\item Terreni differenziali: permettono l'identificazione batterica in relazione all'attivit\`a metabolica o aspetti morfologici delle colonie.
					Questo avviene grazie a particolari substrati o indicatori in grado di dimostrare con una variazione cromatica l'azione metabolica del microorganismo ricercato.
			\end{itemize}
	\subsection{Primo giorno}
	Per la preparazione di un terreno si utilizza un preparato in polvere pesato accuratamente tramite bilancia digitale.

		\subsubsection{Coltura liquida}
		Per la preparazione di una coltura liquida si misura una specifica quantit\`a di acqua distillata tramite cilindro e la si aggiunge alla polvere all'interno del contenitore in cui si vuole ottenere il terreno.
		Una volta mescolata la polvere si posiziona il contenitore all'interno dell'autoclave per la sterilizzazione del terreno mediante pressioni elevate $1.5\si{atm}$ senza bollire il terreno e rovinare il nutriente per $20$ minuti.
		Non bisogner\`a poi pi\`u aprire il contenitore se non sotto cappa, quando viene steso il terreno sulle piastre.

		\subsubsection{Terreni preparati}
		Vengono preparati $4$ terreni di coltura sterili:
		\begin{multicols}{2}
			\begin{itemize}
				\item \emph{LB agar}: terreno di base.
				\item \emph{LB agar-ampicillina}: terreno di coltura selettivo per i terreni ampicillina-resistenti.
				\item \emph{Nutrient agar}: terreno nutritivo adatto a ceppi non capaci di sintetizzare i nutrienti.
				\item \emph{Mueller-Hinton agar}: terreno di coltura che non contiene sostanze che interferiscono con antibiotici e si usa per il test per determinare l'antibiotico-resistenza.
			\end{itemize}
		\end{multicols}
\section{Seconda esperienza - Determinazione della curva di crescita di un ceppo batterico (E. coli)}
	
	\subsection{Introduzione}

		\subsubsection{Fattori che influenzano la crescita batterica}
		I fattori che influenzano la crescita batterica sono:
		\begin{multicols}{2}
			\begin{itemize}
				\item Nutrienti: fonti di carbonio, energia, acqua, vitamine, azoto e oligoelementi.
				\item Concentrazioni di sale: divide i batteri in alofili, alotolleranti o alofili estremi.
				\item Ossigeno: divide i batteri in aerobi obbligati, aerobi facoltativi, microaerofili, anaerobi aerotolleranti e anaerobi obbligati.
				\item $pH$: divide i batteri in alcalinofili, basofili e neutrofili.
				\item Temperatura: divide i batteri in psicrofili ($-10$-$30\si{\celsius}$), mesofili ($10$-$50\si{\celsius}$), termofili ($40$-$90\si{\celsius}$) e termofili estremi.
			\end{itemize}
		\end{multicols}

		\subsubsection{Quantificazione}
	
			\paragraph{Fasi della crescita microbica}\mbox{}
			\begin{enumerate}
				\item Lag: adattamento del batterio alle nuove condizioni.
				\item Esponenziale: crescita continua grazie alla presenza di nutrienti.
					Il tempo di replicazione specifico del batterio influisce sulla velocit\`a: Mycobacterium tubercolosis impiega $16$ ore, E. coli $20$ minuti.
				\item Stazionaria: ci sono troppi batteri sul terreno e i nutrienti iniziano ad esaurirsi.
					Gli eventi di replicazione e morte sono in equilibrio tra di loro.
				\item Morte: avviene progressivamente la morte per mancanza di nutrienti.
			\end{enumerate}
	\subsection{Primo giorno}

		\subsubsection{Micropipetta}
		La micropipetta \`e uno strumento in grado di prelevare volumi diversi.
		Possiede una scala graduata.

			\paragraph{Tipologie}\mbox{}\\
			\begin{multicols}{2}
				\begin{itemize}
					\item $p20$: da $2\si{\micro\litre}$ a $20\si{\micro\litre}$, i numeri letti vanno divisi per $10$.
					\item $p200$ da $20\si{\micro\litre}$ a $200\si{\micro\litre}$.
					\item $p1000$ da $200\si{\micro\litre}$ a $1000\si{\micro\litre}$, i numeri letti vanno moltiplicati per $10$.
				\end{itemize}
			\end{multicols}

			\paragraph{Utilizzo}\mbox{}\\
			\begin{multicols}{2}
				\begin{enumerate}
					\item Si inserisce un nuovo puntale ogni volta che si cambia sostanza.
					\item Si schiaccia lo stantuffo fino al primo scatto per verificare il volume da prelevare.
					\item Si inserisce nella sostanza da prelevare e si rilascia lo stantuffo superiore per prelevare il volume.
					\item Si inserisce nella sostanza il cui volume deve essere rilasciato e si preme lo stantuffo fino in fondo.
					\item Si rilascia il puntale tramite un pulsante vicino allo stantuffo.
				\end{enumerate}
			\end{multicols}

		\subsubsection{Tempo di generazione}
		Si intende per tempo di generazione il tempo di duplicazione della massa di una colonia batterica.
		Questo valore \`e specifico per ogni specie:
		\begin{multicols}{2}
			\begin{itemize}
				\item E. coli: $20$ minuti.
				\item Staphylococcus aureus: $30$ minuti.
			\end{itemize}
		\end{multicols}

		\subsubsection{Procedimento}
		\begin{enumerate}
			\item Inoculo: si prelevano $100\si{\micro\litre}$ di E. coli da una provetta preparata dagli esercitatori overnight tramite la $p200$ e la si inserisce in una beuta con $100\si{mL}$ di brodo.
				Si ottiene una diluizione $1:100$ e si mescola.
			\item Si preleva $1\si{mL}$ di coltura con una $p1000$ e la si pone nella cuvetta in modo da inserirla poi nello spettrofotometro per monitorare la crescita batterica.
				Si misura la densit\`a ottica e la si annota. Si \`e ora a $t=0$.
				Ci si deve assicurare che lo spettrofotometro sia a $OD=600\si{nm}$ e che sia stato tarato secondo il brodo sterile o bianco in modo da eliminarne il rumore.
			\item Si trasferisce la beuta in uno shaker o incubatore termostato orbitale che la mescola uniformemente e peremette una corretta areazione.
			\item Si puliscono micropipetta e cuvetta per mantenerli sterile.
				Per la cuvetta si getta il brodo nei contenitori e la si risciacqua con acqua distillata.
				La si fa asciugare a testa in giù su carta assorbente.
			\item Si procede allo stesso modo ogni $20$ minuti per ottenere un totale di $12$ misurazioni.
			\item Quando si nota una $OD=0.8$ si procede con diluizioni $1:2$ per rimanere nel range di sensibilit\`a dello spettrofotometro: si prelevano $500\si{\micro\litre}$ di coltura e $500\si{\micro\litre}$ di terreno.
				Quando necessario si fa una diluizione $1:5$ con $200\si{\micro\litre}$ di coltura e $800\si{\micro\litre}$ di terreno.
			\item Una volta terminate le misurazioni si versa la sospensione nella cuvetta nei rifiuti biologici liquidi e si cestina la cuvetta nei rifiuti biologici solidi.
				Si versa la sospensione nella bottiglietta in vetro da $100\si{ml}$ nei rifiuti biologici liquidi e la si pone nel contenitore con la vetreria da lavare.
		\end{enumerate}

		\subsubsection{Analisi dei dati}
		\begin{enumerate}
			\item Si effettua un grafico a punti con la curva di crescita microbica tramite i dati ottenuti.
				Sull'asse delle $x$ si pone il tempo, mentre su quello delle $y$ la densit\`a ottica a $600\si{\nano\metre}$.
			\item Si individuano le fasi di crescita osservate durante la crescita microbica.
			\item Si calcolano il numero di generazioni e al tempo di generazione della coltura.
		\end{enumerate}
		Per calcolare il numero di generazioni $n$ dopo un intervallo si usa la formula:
		\[N_{t_f} = N_0 \cdot 2^n\]
		\[\log N_{t_f} = \log N_0 + n\log 2\]
		Dove:
		\begin{multicols}{3}
			\begin{itemize}
				\item $N_0$ \`e il numero di batteri a $t=0$.
				\item $N_{t_f}$ \`e il numero di batteri a $t_f$.
				\item $t_f$ \`e $\Delta t = t_i - t_0$.
			\end{itemize}
		\end{multicols}
		Da cui deriva che il numero di generazioni $n$ \`e:
		\[n = \dfrac{\log N_{t_f} - \log N_0}{\log 2}\]
		Si pu\`o anche calcolare il tempo di generazione $g$ avvenute in un $\Delta t = t_2 - t_1$:
		\[g = \dfrac{t_2 - t_1}{n}\]
			
			\paragraph{Dati raccolti}
			\begin{center}
				\begin{tabular}{|c|c|}
					\hline
					$N_0$ & \num{60000}\\
					\hline
					$N_{t_f}$ & \num{38000000}\\
					\hline
					$t$ & $300 min$\\
					\hline
					$n$ & $12.6$ generazioni\\
					\hline
					$g$ & $23.8 min$\\
					\hline
				\end{tabular}
			\end{center}



\section{Terza esperienza - Caratterizzazione dei batteri del cavo orale}

	\subsection{Introduzione}
	Nel cavo orale si trovano numerosi batteri anche simbiotici in un pattern specifico alla persona in quanto dipende da:
	\begin{multicols}{3}
		\begin{itemize}
			\item $pH$.
			\item Umidit\`a.
			\item Nutrienti della dieta.
		\end{itemize}
	\end{multicols}
	
		\subsubsection{Piastra su agar-sangue}
		Il risultato del tampone viene seminato su una piastra di agar-sangue con il $5\%$ di sangue di montone. 
		Questo in quanto alcuni batteri come lo Streptococco sono in grado di attuare emolisi, ovvero di degradare i globuli rossi.

			\paragraph{Emolisi}
			Ci sono tre tipi di emolisi:
			\begin{itemize}
				\item Emolisi $\beta$ o emolisi completa: il sangue viene completamente degradato e il terreno appare giallo.
				\item Emolisi $\alpha$ o emolisi incompleta: il sangue viene degradato parzialmente e il terreno appare verdognolo.
				\item Emolisi $\gamma$ o non-emolisi: il sangue non viene degradato e il terreno rimane rosso.
			\end{itemize}

			\paragraph{Semina sulla piastra}
			\begin{enumerate}
				\item Prima patch: si striscia il tampone in modo da occupare la parte superiore della piastra.
				\item Seconda patch: si ruota la piastra di $90\si{\degree}$ e si striscia il tampone in modo che solo qualche strisciata si sovrapponga alla prima patch.
				\item Terza patch: si ruota ancora la piastra di $90\si{\degree}$ e si ripete il procedimento.
			\end{enumerate}

		\subsubsection{Morfologia dei batteri del cavo orale}
		Per identificare le colonie le si distinguono in base a colore, forma, spessore e margini.

			\paragraph{Aspetto}\mbox{}\\
			\begin{multicols}{3}
				\begin{itemize}
					\item Puntiforme.
					\item Circolare.
					\item Filamentoso.
					\item Irregolare.
					\item Rizoide.
					\item Lenticolare.
					\item Raggiato.
				\end{itemize}
			\end{multicols}

			\paragraph{Rilievo}\mbox{}\\
			\begin{multicols}{2}
				\begin{itemize}
					\item Rasato.
					\item Poco convesso.
					\item Convesso.
					\item Pulvinato.
					\item Umbonato.
					\item Diffuso.
					\item Rilevato.
					\item Cupuliforme.
					\item Mammellonato.
					\item Con o senza margine smussato.
				\end{itemize}
			\end{multicols}

			\paragraph{Superficie}\mbox{}\\
			\begin{multicols}{3}
				\begin{itemize}
					\item Liscia.
					\item Rugosa.
					\item Raggiata.
					\item Opaca o luccicante.
					\item Inglobata.
					\item Asciutta o umida.
				\end{itemize}
			\end{multicols}
			
			\paragraph{Margini}\mbox{}\\
			\begin{multicols}{3}
				\begin{itemize}
					\item Continui
					\item Interi.
					\item Ondulati.
					\item Lobati.
					\item Erosi.
					\item Filamentosi.
					\item Stratificati.
					\item Diffusi.
					\item Seghettati.
				\end{itemize}
			\end{multicols}

			\paragraph{Struttura}\mbox{}\\
			\begin{multicols}{2}
				\begin{itemize}
					\item Amorfa.
					\item Granulare.
					\item Filamentosa.
					\item Ondulata.
				\end{itemize}
			\end{multicols}

			\paragraph{Altri fattori}\mbox{}\\
			\begin{multicols}{3}
				\begin{itemize}
					\item Dimensione.
					\item Colore.
					\item Opacit\`a.
					\item Consistenza (cremosa, mucosa, friabile, membranosa).
				\end{itemize}
			\end{multicols}
	
	\subsection{Secondo giorno}
	All'interno della bocca sono contenuti diversi tessuti a cui sono associati diversi ceppi batterici.
	Alla nascita la cavit\`a orale \`e sterile: la prima colonizzazione avviene tra le $6$ e le $10$ ore dopo la nascita.
	
		\subsubsection{Equilibrio plastico}
		Le popolazioni microbiche si dicono in equilibrio plastico in quanto si trovano in una condizione di equilibrio dinamico, con cambiamenti dovuti ad abitudini alimentari, sbalzi ormonali e altri cambiamenti che si subiscono durante la vita.

		\subsubsection{Ecosistema orale}
		L'ecosistema orale \`e formato da microorganismi orali e dalla cavit\`a orale che li ospita.
		La popolazione batterica \`e estremamente diversificata e abbondante: si contano pi\`u di $300$ specie in grado di colonizzarla.
		Questa convivenza, per lo pi\`u pacifica, si dice di simbiosi o commensalismo.
		Con l'insorgenza di fenomeni patologici si passa a rapporti opportunistici.
		Oltre ai batteri sono anche presenti virus e funghi.
		
			\paragraph{Esempi}
			Lo Streptococcus pyogenes del gruppo A \`e un fungo che pu\`o essere presente nella cavit\`a orale e patogeno: produce tossine, fattori di virulenza ed emolisine.
			Se raggiunge il cuore pu\`o produrre endocarditi gravi.

		\subsection{Procedimento}
		\begin{enumerate}
			\item Si prepara una piastra Petri agar-sangue e la si contrassegna sul bordo con data, nome gruppo, protocollo e piastra $A$.
			\item Con spatola e tampone sterile si preleva del materiale dalla parte superiore della lingua.
			\item Si usa il tampone su una porzione della piastra (un quinto) e si striscia con un'ansa sterile la popolazione microbica prelevata tramite la tecnica del quadrante.
				Grazie all'ansa si notano quattro strisciate complessiva.
				L'ultima permette di isolare le colonie.
			\item Si pone la piastra capovolta nell'incuvatore statico termostatato a $37\si{\celsius}$ per $24$ ore.
			\item Il giorno successivo si notano diverse colonie batteriche che vengono distinte per fenotipo.
			\item Si selezionano due colonie con morfologia e/o emolisi diversa sulla piastra e si contrassegna una nuova piastra agar-sangue con data, nome gruppo e protocollo e piastra $B$.
			\item Attraverso lo striscio continuo si piastriano sulla piastra $B$ le colonie batteriche che si vogliono isolare e si pone la piastra capovolta nell'incubatore come la prima.
			\item Il giorno successivo si effettua un'analisi dettagliata del fenotipo batterico in riferimento alle colonie isolate.
			\item Si confrontano i risultati con la piastra $A$.
			\item Si gettano nei biobox entrambe le piastre.
		\end{enumerate}
\section{Quarta esperienza - Conta standard su piastra}

	\subsection{Introduzione}

		\subsubsection{Conta microbica indiretta}
		La conta microbica indiretta avviene tramite spettrofotometro: la densit\`a ottica nella cuvetta \`e proporzionale alla quantit\`a di batteri presenti.
		Non permette per\`o di distinguere tra cellule vive e morte.

		\subsubsection{Conta vitale}
		La conta vitale su piastra permette di contare unicamente le cellule vive.

			\paragraph{Processo}
			\begin{enumerate}
				\item Si prende la provetta contenente le colonie batteriche.
				\item Si fanno diluizioni seriali su piastra in brodo di coltura o soluzione salina.
				\item Si prelevano $0.1\si{mL}$ dalla diluizione e si distribuiscono uniformemente su piastra tramite ansa a l.
				\item Si incuba ogni diluizione.
				\item Si conta la piastra in cui le colonie si distinguono correttamente e non sono ammassate (tipicamente tra le $10$ e le $200$).
			\end{enumerate}
	\subsection{Secondo giorno}

		\subsubsection{Metodi della conta batterica}
		
			\paragraph{Conta diretta}
			La conta diretta avviene al microscopio o allo spettrofotometro e non si distinguono cellule vive e morte.

			\paragraph{Conta indiretta}
			La conta indiretta avviene in coltura e si contano solo le cellule vive, le uniche in grado di riprodursi e formare una colonia nel terreno.
			Per attuare una conta batterica indiretta si procede per diluizioni seriali provenienti da un unico brodo di coltura \emph{LB} sterile standard.
			Si attuano diluizioni seriali $1:10$.
			Avviene poi piastratura di $1\si{mL}$ di ogni soluzione e si incubano le piastre.

			\paragraph{Conta vitale}
			La conta vitale consente di contare solo cellule vive presenti in una sospensione batterica.
			Si dice anche conta su piastra o conta delle colonie.

		\subsubsection{Colonia}
		Si definisce colonia un gruppo di cellule batteriche appartenenti allo stesso ceppo o specie che ha origine da una sola cellula vitale.
		Una cellula in grado di formare una colonia \`e detta unit\`a formante colonia o \emph{UFC} o \emph{CFU}.
		Il numero di colonie contate sulla piastra corrisponde al numero di \emph{UFC} dell'inoculo.

		\subsubsection{Procedimento}
		\begin{enumerate}
			\item Si contrassegna ognuna delle $9$ provette con $900\si{\micro\litre}$ di brodo \emph{LB} sterile con la diluizione rispettiva: $10^{-1;-9}$.
			\item Si contrassegna il bordo del fondo delle $9$ piastre di \emph{LB-agar} da $90\si{mm}$ di diametro con nome gruppo, nome protocollo, diluizione della provetta corrispondente e volume piastrato.
			\item Si contrassegna la piastra rimanente con nome gruppo e non diluita, \`e dedicata alla sospensione batterica non diluita.
			\item Si agita la provetta con la coltura batterica non diluita con il vortex e si prelevano $100\si{\micro\litre}$ e la si trasferisce nella provetta $10^{-1}$.
				Questa \`e la prima diluizione seriale.
			\item Dopo aver agitato con il virtex si prelevano $100\si{\micro\litre}$ dalla diluizione e si trasferiscono nella provetta $10^{-2}$ e si mescola.
				Questa \`e la seconda diluizione seriale.
			\item Si ripete la procedura fino ad arrivare all'ultima diluizione.
			\item Si prelevano $100\si{\micro\litre}$ della sospensione batterica di partenza, si trasferiscono al centro della piastra corrispondente e si piastra con ansa sterile.
			\item Si prelevano $100\si{\micro\litre}$ dalla sospensione diluita e si trasferiscono al centro della piastra corrispondente e si piastra con ansa sterile.
			\item Si uniscono le piastre con nastro adesivo su cui si scrive il gruppo. 
				Si pongono capovolte nell'incubatore a $37\si{\celsius}$ per $18$-$24$ ore.
			\item Una volta concluso si versa il contenuto delle provette in plastica nei rifiuti biologici liquidi e si buttano le provette in plastica nei rifiuti biologici solidi.
			\item Si conta e annota il numero di colonie presenti in ogni piastra.
			\item Per la conta si inizia dalla piastra con il maggior contenuto di colonie contabili.
				Si escludono le piastre con numero di colonie minore di $10$.
		\end{enumerate}

		\subsubsection{Analisi dei risultati}
		Si calcola la concentrazione della sospensione batterica tramite il calcolo della media delle \emph{CFU} ottenute nelle piastre contate.
		\[Concentrazione=\dfrac{n\cdot f}{v}\dfrac{CFU}{\si{mL}}\]
		Dove:
		\begin{multicols}{3}
			\begin{itemize}
				\item $n$ \`e il numero di colonie contate su una piastra.
				\item $f$ \`e il fattore di diluizione (inverso della diluizione operata).
				\item $v$ \`e il volume di sospensione batterica piastrata.
			\end{itemize}
		\end{multicols}

\section{Quinta esperienza - test di aerobiosi/anaerobiosi su terreno solido}

	\subsection{Introduzione}

		\subsubsection{Distinguere i batteri in base alla richiesta di ossigeno}
		In base alla loro richiesta di ossigeno i batteri si dividono in:
		\begin{multicols}{2}
			\begin{itemize}
				\item Anaerobi obbligati: non tollerano l'ossigeno.
				\item Anaerobi facoltativi: crescono con la respirazione cellulare in presenza di ossigeno ma se \`e assente con altri metabolismi.
				\item Microaerofili: batteri che richiedono una quantit\`a di ossigeno inferiore a quella presente nell'aria: $5$-$10\%$.
				\item Aerotolleranti: sono indifferenti alla presenza di ossigeno, effettuano la fermentazione, preferiscono una quantit\`a di ossigeno inferiore al $20\%$.
				\item Aerobi obbligati: richiedono ossigeno.
			\end{itemize}
		\end{multicols}

		\subsubsection{Batteri modello}
		Per verificare l'aerobiosi o anaerobiosi di una colonia si utilizzano:
		\begin{multicols}{2}
			\begin{itemize}
				\item Citrobacter freundii: anaerobio facoltativo, Gram$-$.
				\item Micrococcus luteus: aerobio obbligato, Gram$+$.
			\end{itemize}
		\end{multicols}

		\subsubsection{Processo}
		\begin{enumerate}
			\item Si prendono due piastre di terreno solido e le si divide a met\`a.
			\item In ogni piastra si semina su una met\`a Citrobacter freundii e sull'altra Micrococcus luteus.
			\item Una piastra viene posta in condizioni di aerobiosi mentre l'altra viene posta all'interno di una giara in cui si produce una condizione di anerobiosi.
			\item Si verifica che Citrobacter freundii cresce in entrambe le piastre mentre Micrococcus luteus solo nella piastra con ossigeno.
		\end{enumerate}
		Nella giara vengono inseriti dei filtri in grado di eliminare l'ossigeno presente grazie a reazioni attuate all'interno della giara stessa.

	\subsection{Seconda giornata}
	Si verifica su terreno solido l'aerobiosi o anaerobiosi di due ceppi batterici:
	\begin{multicols}{2}
		\begin{itemize}
			\item Citrobacter freundii: bacillo anaerobio facoltativo Gram$-$, in piastra di agar-sangue.
			\item Micrococcus luteus: cocco aerobio obbligato Gram$+$, in piastra \emph{TLC}.
		\end{itemize}
	\end{multicols}
	I due ceppi rimangono ignoti e detti ceppo $1$ e ceppo $2$.

		\subsubsection{Creazione di condizione di anaerobiosi}
		Per creare una condizione di anaerobiosi si utilizzano giare, contenitori a chiusura ermetica in cui vengono inseriti sistemi \emph{GAS-PACK}, sacchetti contenenti reagenti chimici con la capacit\`a di eseguire una reazione chimica che elimina l'ossigeno nell'ambiente.
		Si aggiungono pochi $\si{mL}$ di acqua al gas-pack in modo da attivare la reazione con liberazione di \emph{$CO_{2(g)}$} e \emph{$H_{2(g)}$}.
		Sul tappo della giara \`e presente un catalizzatore al palladio sulla quale si forma acqua.
		
			\paragraph{Verificare l'assenza di ossigeno}
			Per verificare l'assenza di ossigeno si usano indicatori come:
			\begin{itemize}
				\item Resazurina: azzurra in presenza di ossigeno e rosa in sua assenza.
				\item Blu di metilene: blu in presenza e incolore in assenza di ossigeno.
			\end{itemize}

		\subsubsection{Procedimento}
		\begin{enumerate}
			\item Si disegnano due quadranti sul fondo di due piastre di Nutrient-agar con un pennarello.
				Si marca ogni quadrante con una cifra e si scrive si una piastra aerobiosi e sull'altra anaerobiosi.
			\item Si preleva $1$ colonia dalla piastra del ceppo $1$ e la si deposita sul quadrante corrispondente di entrambe le piastre con ansa.
			\item Si distribuisce la colonia con lo striscio continuativo non uscendo dal quadrante.
			\item Si ripete l'operazione con l'altro ceppo. 
			\item Si dispone la piastra anaerobiosi vicino alla giara in cui gli esercitatori inseriscono blu di metilene, bustina del catalizzatore.
				Avviene incubazione a $37\si{\celsius}$.
				Si attendono $24$ ore.
			\item Il giorno successivo si paragona il fenotipo di crescita di ogni ceppo batterico in condizioni aerobiche e anaerobiche e si annotano le osservazioni.
			\item Si classificano i $2$ ceppi in base al risultato del test.
		\end{enumerate}
		
\section{Sesta esperienza - test Kirby-Bauer}

	\subsection{Introduzione}
	L'abuso di antibiotici ha causato lo sviluppo di antibiotico-resistenza da parte dei batteri per alcuni antibiotici.
	Si tratta della minaccia mondiale pi\`u pericolosa dal punto di vista sanitario vista la facilit\`a del trasferimento genico orizzontale.
	
		\subsubsection{Test di Kirby-Bauer}
		Durante il test di Kirby-Bayer per la verifica dell'antibiotico-resistenza:
			
			\paragraph{Procedimento}
			\begin{enumerate}
				\item Si prende una piastra con agar solido Mueller-Hinton e si semina uniformemente una certa specie batterica.
				\item Si applicano dischetti imbevuti ognuno da un antibiotico specifico
				\item Si incubano le piastre.
			\end{enumerate}

			\paragraph{lettura dei risultati}
			La sensibilit\`a a un antibiotico viene determinata come alone di inibizione: il diametro dell'alone indica il grado di sensibilit\`a.
			L'assenza di alone indica un antibiotico-resistenza.
	\subsection{Secondo giorno}

		\subsubsection{Resistenza}
		Si definisce resistenza lo sviluppo della capacit\`a di un microorganismo di sopravvivere a farmaci che dovrebbero ucciderlo o indebolirlo.
		Se la resistenza diventa a diversi farmaci, trattare l'infezione pu\`o diventare difficile.
		I microorganismi resistenti tendono a essere trasmessi da persona a persona.
		In questo modo infezioni difficili da trattare possono diffondersi con conseguenze serie fino alla morte.

			\paragraph{Sviluppo della resistenza}
			Se la cura antibiotica non si protrae per il giusto periodo di cura, alcuni batteri che stanno sviluppando la resistenza non vengono uccisi e insorgono.

			\paragraph{Superbatteri}
			Si definiscono superbatteri batteri resistenti a pi\`u antibiotici contemporaneamente, a volte a tutti gli antibiotici conosciuti.

			\paragraph{Identificazione della resistenza}
			La suscettibilit\`a batterica ai farmaci pu\`o essere individuata in due modi:
			\begin{multicols}{2}
				\begin{itemize}
					\item Test Kirby-Bauer su terreno solido.
					\item Test di determinazione della minima concentrazione inibente \emph{MIC} su terreno liquido.
				\end{itemize}
			\end{multicols}

		\subsubsection{Procedimento}
		\begin{enumerate}
			\item Si marca la piastra di \emph{MH-agar} con nome gruppo e protocollo, data e tipo di Gram.
			\item Si disegnano sul fondo della piastra le due diagonali e quattro linee aggiuntive.
			\item Si prelevano $1$-$2$ colonie con un diametro maggiore di $1.5\si{mm}$ dalla piastra Petri con un bastoncino cotonato sterile e si stemperano nella fiala di soluzione fisiologica sterile.
			\item Si rimuove l'eccesso di liquido dal bastoncino cotonato premendo e ruotandolo vigorosamente sulle pareti interne della fiala.
			\item Si distribuiscono i batteri su ogni millimetro della piastra Petri di \emph{MH-agar} con lo stesso bastoncino.
				Per garantire una distribuzione uniforme si ruota la piastra di $45\si{\degree}$ e la si striscia nuovamente con lo stesso bastoncino.
			\item Si lascia asciugare la superficie dell'agar sul bancone per $3$-$5$ minuti.
			\item Si sterilizza una pinzetta in etanolo al $70\%$ e lo si lascia evaporare.
			\item Si depone con la pinzetta sulla superficie di agar il dischetto \emph{CAZ/CLA} al centro della piastra e si preme delicatamente per assicurarsi il contatto completo.
			\item Si dispongono gli altri dischetti da almeno $15\si{mm}$ dalla piastra.
				In senso orario \emph{E}, \emph{AM}, \emph{VA} e \emph{GM}.
			\item Si dispongono le piastre capovolte all'interno dell'incubatore statico termostatato a $37\si{\celsius}$ per $16$-$18$ ore.
			\item A fine protocollo si svuota il contenuto della fiala in vetro nei rifiuti biologici liquidi e si cestina quella in vetro nei rifiuti biologici taglienti.
			\item Il giorno successivo si verifica la presenza degli aloni di inibizione: in caso positivo si misurano i diametri e si annota il diametro per ogni antibiotico e il tipo di Gram a disposizione.
			\item Si consultano i breakpoint \emph{EUCAST} per determinare a quali antibiotici il microorganismo \`e sensibile, intermedio o resistente.
				Si annota il profilo di resistenza del ceppo batterico.
		\end{enumerate}

		\subsubsection{Breakpoint \emph{EUCAST}}
		\begin{center}
			\begin{tabular}{|c|c|c|c|}
				\hline
				\multirow{2}{*}{Antibiotico} & \multicolumn{3}{c|}{Diametro di alone di inibizione ($\si{mm}$)} \\
				\cline{2-4}
							     & Resistente & Intermedio & Sensibile \\
				\hline
				Ampicillina Gram$-$ & $\le 13$ & $14$-$16$ & $\ge 16$ \\
				\hline
				Ampicillina Gram$+$ & $\le 28$ & & $\ge 29$ \\
				\hline
				Eritromicina & $\le 13$ & $14$-$22$ & $\ge 23$ \\
				\hline
				Gentamicina & $\le 12$ & $13$-$14$ & $\ge 15$ \\
				\hline
				Ceftazidime/Acido clavulanico & $\le 13$ & $14$-$17$ & $\ge 18$ \\
				\hline
				Vancomicina Gram$-$ & $\le 3$ & & $> 3$ \\
				\hline
				Vancomicina Gram$+$ & $\le 14$ & & $\ge 15$ \\
				\hline
			\end{tabular}
		\end{center}
\section{Settima esperienza - test biochimico \emph{API 20E}}

	\subsection{Introduzione}
	Esistono molti metodi biochimici per determinare la specie di una coltura batterica.
	Uno dei pi\`u utilizzati \`e il test \emph{API 20E}.

		\subsubsection{Test \emph{API 20E}}
		Il test \emph{API 20E} si compone di $21$ reazioni metaboliche e $6$ test biochimici supplementari.
		Verr\`a utilizzato per identificare \emph{Enterobacteriaceae} e altri Gram$-$.

			\paragraph{Processo}
			\begin{multicols}{2}
				\begin{enumerate}
					\item Si preleva una colonia.
					\item Si stempera la colonia in acqua sterile.
					\item Si riempono i microtubuli o microgallerie con la colonia stemperata.
						I microtubuli sono alti $1\si{cm}$ e larghi $0.5\si{cm}$ e contengono terreno liofilizzato.
					\item Si incubano a $37\si{\celsius}$ per una notte.
				\end{enumerate}
			\end{multicols}

			\paragraph{Lettura dei risultati}
			In base alla reazione chimica metabolica che avviene in ogni pozzetto o microtubulo o galleria si nota una colorazione diversa.
			Il colore denota l'avvenimento o meno della reazione.
			Moduli identificano la specie.

				\subparagraph{Lettura dei moduli}
				A ogni pozzetto corrisponde un numero.
				Se la reazione \`e avvenuta nel pozzetto considero il numero, altrimenti no.
				I pozzetti sono divisi in trittici: sommando i numeri da considerare per ogni trittico si ottiene un codice univoco per la specie.
	\subsection{Secondo giorno}
	
		\subsubsection{Caratteristiche individuate}
		Il test biochimico \emph{API 20E} permette di identificare il profilo biochimico di una data specie batterica individuando:
		\begin{multicols}{3}
			\begin{itemize}
				\item Richiesta di ossigeno per la crescita.
				\item Utilizzo di zuccheri come fonte di carbonio.
				\item Produzione di substrati.
				\item Utilizzo del citrato.
				\item Produzione di enzimi.
				\item Decarbossilazione di amminoacidi.
			\end{itemize}
		\end{multicols}

		\subsubsection{Procedimento}
		\begin{enumerate}
			\item Si contrassegna il bordo del contenitore di incubazione e il suo coperchio con nome gruppo, data e nome microorganismo.
			\item Si distribuisce con la spruzzetta acqua distillata sul fondo del contenitore di incubazione fino a riempire tutti i pozzetti per creare un ambiente umido ed evitare il disseccamento dei test durante l'incubazione in termostato.
			\item Si rimuove l'eventuale eccesso di acqua con la micropipetta $p1000$ inclinando il fondo del contenitore.
			\item Si blocca con il nastro di carta il contenitore di incubazione al bancone e si blocca il coperchio verso il basso al bancone in modo da creare una corsia tra contenitore e coperchio per disporre la galleria.
			\item Si pone la galleria \emph{API 20E} inclinata verso l'alto nella corsia tra il contenitore e il coperchio come supporto.
			\item Si apre una fiala con $3\si{mL}$ di soluzione salina sterile $0.85\%$ applicando pressione verso l'esterno sulla zona zigrinata del coperchio in modo da rompere la punta in vetro.
			\item Si preleva sterilmente con un bastoncino cotonato sterile $4$ colonie dalla piastra di \emph{LB-agar}.
			\item Si stemperano le colonie nella soluzione salina ruotando il bastoncino cotonato sulle pareti interne della fiala.
			\item Si risospendono le cellule batteriche pipettando pi\`u volte con la $p1000$.
			\item Si prelevano $300\si{\micro\litre}$ di sospensione batterica e si riempiono cambiando puntale per ogni test le microprovette e le cupole corrispondenti ai test \emph{$|$CIT$|$}, \emph{$|$VP$|$}, \emph{$|$GEL$|$}.
				Si appoggia la punta del puntale sul bordo laterale della cupola per evitare la formazione di bolle d'aria all'interno della microprovetta.
			\item Si prelevano con la $p200$ $100\si{\micro\litre}$ di sospensione batterica e si riempiono, cambiando puntale le microprovette di \emph{ADH, LDC, ODC, H2S e URE}.
				Si appoggia la punta del puntale sul bordo laterale della cupola per evitare la formazione di bolle d'aria all'interno della microprovetta.
			\item Si aggiungono nei $5$ test gocce d'olio di paraffina fino a riempire la cupola per far avvenire le reazioni in anaerobiosi.
			\item Si prelevano con la $p200$ $120\si{\micro\litre}$ di sospensione batterica e si riempiono le microprovette dei test rimasti.
			\item Si dispone la galleria \emph{API 20E} orizzontalmente all'interno del contenitore di incubazione e si chiude con il coperchio.
				Si mette la galleria nell'incubatore statico termostatato a $37\si{\celsius}$.
			\item Si svuota il contenuto della fiala in vetro con la sospensione microbica nei rifiuti biologici liquidi vicino al lavandino e la fiala in vetro nei rifiuti taglienti.
			\item Si determinano quali test sono positivi e quali negativi tranne \emph{TDA, VP, IND}:
				\begin{itemize}
					\item Se \emph{GLU} \`e negativo (blu, blu-verde) e ci sono meno di $3$ reazioni positivi ci si ferma in quanto l'organismo non \`e un enterobatterio e necessita di un tempo di incubazione pi\`u lungo.
					\item Se il test \emph{GLU} \`e positivo (giallo) o ci sono pi\`u di $3$ reazioni positive si procede con il sistema.
				\end{itemize}
			\item Si aggiunge il reagente \emph{TDA} (cloruro di ferro $10\%$) al test \emph{TDA} della galleria.
				La reazione, se positiva \`e istantanea ed \`e rosso mattone.
			\item Si aggiunge una goccia di reattivo di James al test \emph{IND} della galleria.
				Una reazione positiva di colore rosa intenso o rossa avviene nell'arco di due minuti.
				L'acido nel reagente pu\`o reagire con la cupola di plastica e produrre un viraggio verso il rosso marrone che non indica positivit\`a.
			\item Si aggiunge una goccia di reattivo \emph{VP1} e \emph{VP2} nel test \emph{VP}.
				Il colore rosa chiaro immediato non \`e indice di positivit\`a: la reazione impiega $10$ minuti e risulta rosa intenso o rosso.
			\item Si annota sulla scheda di lettura il risultato di ogni test della galleria.
				Si annota la cifra data dalla somma dei valori dei test positivi e il codice a $7$ cifre ottenuto.
			\item Si individua nell'indice di profilo analitico il codice ottenuto.
				Si annota il risultato e le reazioni di dubbia interpretazione.
			\item Si annotano i codici alternativi possibili e le specie corrispondenti.
			\item Si utilizza il codice per l'identificazione.
			\item Si cestina la galleria nei rifiuti biologici solidi, si svuota la fiala con la sospensione batterica nei rifiuti biologici liquidi e la fiala nel contenitore per materiale acuminato infetto.
		\end{enumerate}

\section{Ottava esperienza - Colorazione di Gram}
	\subsection{Introduzione}
	La colorazione di Gram differenzia batteri Gram$+$ che appaiono di color violetto e batteri Gram$-$ che appaiono rosa.
	Questo avviene in quanto viene usata la sostanza Crystal Violet che colora di violetto lo strato di peptidoglicano: essendoci nei Gram$+$ uno strato pi\`u spesso mentre nei Gram$-$ uno strato pi\`u sottile contenuto tra le membrane.

		\subsubsection{Procedimento}
		\begin{enumerate}
			\item Si applicano i batteri sulla piastra e si applica il Crystal Violet.
			\item Si applica iodio che funge da mordente: attacca al peptidoglicano il Crystal Violet.
			\item Si decolora tramite alcol: solo i batteri su cui il Crystal Violet ha attaccato grazie al mordente rimangono colorati (Gram$+$).
			\item Si utilizza la safranina per colorare i Gram$-$ di rosa.
		\end{enumerate}

		\subsubsection{Identificare i Gram$\mathbf{+}$}
		I batteri vengono anche raggruppati in base alla forma che appare al microscopio come:
		\begin{multicols}{2}
			\begin{itemize}
				\item I cocchi Gram$+$ sono stafilococchi (in gruppi) o streptococchi (in catene).
				\item I bastoncelli Gram$+$ includono Bacillus, Clostridium, Corynebacterium e Listeria.
			\end{itemize}
		\end{multicols}

		\subsubsection{Identificare i Gram$\mathbf{-}$}
		I Gram$-$ vengono classificati in tre gruppi:
		\begin{multicols}{3}
			\begin{itemize}
				\item Cocchi, dalla forma sferica sono pi\`u comunemente Neisseria.
				\item Bastoncelli, allungati e sottili sono E. coli, Enterobacter, Klebsiella, Citrobacter, Serratia, Proteus, Salmonella, Shigella, Pseudomonas e altri.
					Il Vibrio cholerae pu\`o essere a forma di normale bastoncello o ricurvo.
				\item Coccoidi o coccobacilli, di forma intermedia tra cocchi e bacilli sono Bordetella, Brucella, Haemophilus e Pasteurella.
			\end{itemize}
		\end{multicols}

	\subsection{Terzo giorno}

		\subsubsection{Procedimento}

			\paragraph{Preparazione del vetrino}
			La preparazione avviene al bancone e sotto cappa biologica:
			\begin{enumerate}
				\item Si marca una provetta da microcentrifuga da $2\si{mL}$ sterile con id campione.
				\item Si trasferisce con la $p200$ $100\si{\micro\litre}$ di acqua distillata sterile nella provetta da $2\si{mL}$.
				\item Si preleva una colonia batterica sterilmente con un'ansa da inoculazione sterile e la si trasferisce nei $100\si{\micro\litre}$ di acqua distillata sterile.
				\item Si risposende pipettando su e giu.
				\item Si pulisce e sgrassa con alcol $70\%$ il vetrino portaoggetti e lo si asciuga con carta assorbente.
				\item Si scrive a matita sullo spazio laterale satinato del vetrino portaoggetti nome gruppo e nome campione.
				\item Si trasferisce $20\si{\micro\litre}$ di sospensione batterica al centro del vetrino portaoggetti.
				\item Si striscia delicatamente la sospensione batterica con l'ausilio di un'ansa sterile fino ad occupare $1$-$2\si{cm}$ al centro del vetrino.
				\item Si lascia asciugare completamente il vetrino per evaporazione sul bancone o sotto cappa biologica.
			\end{enumerate}

			\paragraph{Fissazione del preparato}
			La fissazione del preparato avviene sotto cappa chimica.
			\begin{enumerate}
				\item Si copre la parte centrale del vetrino con lo striscio con $1$-$2$ gocce di etanolo o acido acetico con la pipetta Pasteur, lavorando sopra il becker per i rifiuti chimici.
				\item Si rimuove l'eccesso di etanolo o acido acetico su carta assorbente e si fa asciugare all'aria sotto cappa chimica fino a completa evaporazione.
			\end{enumerate}

			\paragraph{Colorazione del preparato}
			La colorazione del preparato avviene sotto cappa chimica.
			\begin{enumerate}
				\item Si pipetta sul vetrino la soluzione di crystal violet fino a coprire lo striscio e la si lascia agire per $1$ minuto.
				\item Si rimuove il colorante in eccesso sciacquando con acqua distillata sopra il becker per i rifiuti chimici fino a che il preparato non rilascia pi\`u colore.
				\item Si pipetta sul vetrino il reattivo di Lugol e si lascia agire per $1$ minuto.
				\item Si rimuove il reattivo di Lugol in eccesso sciacquando con acqua distillata sopra il becker per i rifiuti chimici.
				\item Si versa con il contagocce la soluzione decolorante per Gram sul preparato.
					Una decolorazione troppo prolungata pu\`o rimuovere il colorante anche dai Gram$+$, si procede per al massimo $30$ secondi.
				\item Si rimuove il decolorante sciacquando con acqua distillata sopra il becker per i rifiuti chimici.
				\item Si pipetta sul vetrino la soluzione di safranina e la si lascia agire per $1$ minuto.
				\item Si rimuove il colorante in eccesso sciacquando con acqua distillata sopra il becker per i rifiuti chimici fino a che il preparato non rilascia pi\`u colore.
				\item Si sgocciola il vetrino su carta assorbente e lo si lascia asciugare all'aria.
				\item Si pone il vetrino sul tavolino portaoggetti del microscopio.
				\item Si osserva al microscopio dimensione, forma e colore dei batteri, si annota il tipo di colorazione e la disposizione delle cellule prevalente.
			\end{enumerate}


\section{Nona esperienza - Trasformazione di batteri}

	\subsection{Introduzione}
	Per produrre batteri ricombinanti si utilizza la tecnologia del DNA ricombinante tramite inserimento di un gene in un battere per mezzo dei plasmidi.

		\subsubsection{Trasformazione batterica}
		La trasformazione batterica \`e un processo di trasferimento genico orizzontale con cui batteri competenti acquisiscono frammenti di DNA dall'ambiente.
		Questa propriet\`a viene usata per il clonaggio genico, tecniche di mutagenesi e per la produzione di proteine ricombinanti.
		Esistono procedure per trasformare batteri naturalmente non competenti.

			\paragraph{Trasformazione chimica}
			Non richiede strumenti dedicati e consiste nel pre-trattamento delle cellule con alte concentrazioni di cloruro di calcio per renderle competenti.
			Un trattamento termico rapido o heat-shock in presenza delle molecole di DNA permette l'internalizzazione del DNA.

			\paragraph{Trasformazione elettrica}
			\`E pi\`u efficiente di quella chimica ma richiede l'elettroporatore.
			Gli impulsi elettrici permeabilizzano la membrana cellulare permettendo al DNA di entrare nella cellula.

		\subsubsection{Plasmidi}
		I plasmidi sono molecole di DNA extracromosomico in grado di replicarsi indipendentemente dal cromosoma della cellula.
		Contengono tipicamente un marcatore di selezione, un sito di restrizione enzimatica multipla e un marcatore per la visualizzazione di plasmide.
		Si utilizzer\`a un plasmide contenente:
		\begin{multicols}{2}
			\begin{itemize}
				\item \emph{GFP}: green fluorescent protein.
				\item Gene per la resistenza all'ampicillina: creazione di un terreno selettivo.
			\end{itemize}
		\end{multicols}
		Questo verr\`a inserito in E. coli.

		\subsubsection{Processo di trasformazione}
		Si trasformano i batteri tramite shock termico, abbassando la temperatura da $43\si{\celsius}$ a $3\si{\celsius}$ per facilitare l'entrata nel plasmide.
		Si semina su un terreno contenente ampicillina \emph{Amp} e si incuba.
		In questo modo si eliminano le cellule non trasformate (non resistenti ad ampicillina).
		Le cellule trasformate contengono fluorescenza e resistenza ad \emph{Amp}.

	\subsection{Terzo giorno}

		\subsubsection{Procedimento}
		\begin{enumerate}
			\item Si marca una provetta da $1.5\si{mL}$ con \emph{TR-NEG} e la si raffredda in ghiaccio.
			\item Marcare la provetta con il DNA con \emph{TR} e la si mantiene in ghiaccio.
			\item Si aliquotano $48\si{\micro\litre}$ di cellule competenti in \emph{TR-NEG} e la si mantiene in ghiaccio.
			\item Si aliquotano $48\si{\micro\litre}$ di cellule competenti in \emph{TR} e si pipetta per mescolare e la si mantiene in ghiaccio.
			\item Si incubano le due provette con cellule competenti in ghiaccio per $30$ minuti.
			\item Si marcano le piastre con:
				\begin{multicols}{2}
					\begin{itemize}
						\item Piastra Petri di \emph{LB-agar}: nome gruppo, nome protocollo \emph{DH5$\alpha$}.
						\item Piastra Petri di \emph{LB-agar} e ampicillina: nome gruppo, nome protocollo \emph{TR-NEG}.
						\item Piastra Petri di \emph{LB-agar} e ampicillina: nome gruppo, nome protocollo \emph{TR-$100\si{\micro\litre}$}.
						\item Piastra Petri di \emph{LB-agar} e ampicillina: nome gruppo, nome protocollo \emph{TR-resto}.
					\end{itemize}
				\end{multicols}
			\item Si preleva il volume restante di cellule competenti dal tubo di partenza e li si pipetta nella piastra Petri \emph{LB-agar} \emph{DH5$\alpha$}.
			\item Si distribuisce con ansa sterile per inoculazione la sospensione batterica.
			\item Si pone la piastra capovolta nell'incubatore termostatato a $37\si{\celsius}$.
			\item Si esegue lo shock termico incubando le due provette in un termoblocco a $42\si{\celsius}$ per $90$ secondi.
			\item Si raffredda in ghiaccio per $2$ minuti.
			\item Si aggiunge $750\si{\micro\litre}$ di brodo \emph{LB} a ciascuna provetta.
			\item Si incuba nell'incubatore ad agitazione orbitale per $37\si{\celsius}$ per $1$ ora.
			\item Si risospendono le cellule batteriche con il vortex.
			\item Si prelevano $100\si{\micro\litre}$ a \emph{TR-NEG} e li si piastra in \emph{TR-NEG}.
				I restanti vengono scartati.
			\item Si prelevano $100\si{\micro\litre}$ da \emph{TR} e li si piastra sulla piastra \emph{TR-$100\si{\micro\litre}$}.
			\item Si centrifuga la provetta da $1.5\si{mL}$ con la restante coltura batterica a $4000\times g$ per $3$ minuti.
			\item Si verifica che sia avvenuta la separazione tra i batteri e brodo di coltura.
			\item Si rimuovono $600\si{\micro\litre}$ di surnatante da \emph{TR} con la $p1000$.
			\item Si risospende il pellet nei $100\si{\micro\litre}$ di surnatante rimasti con la $p200$.
			\item Si piastra con un'ansa a $L$ sterile i $10\si{\micro\litre}$ rimanenti sulla piastra \emph{TR-resto}.
			\item Si incubano le $3$ piastre di \emph{LB-agar} e ampicillina e la piastra con lo striscio a quadrante di E. coli competenti unite da un pezzo di nastro adesivo a $37\si{\celsius}$ per $18$-$24$ ore.
		\end{enumerate}
				

\section{Decima esperienza - Osservazione della motilit\`a batterica di tipo ``swimming'' al microscopio ottico}

	\subsection{Introduzione}
	I Batteri si distinguono anche grazie alla loro motilit\`a, che pu\`o avvenire attraverso flagelli o pili.

		\subsubsection{Flagelli}
		I flagelli sono lunghe appendici proteiche che servono al moto della cellula batterica.
							
			\paragraph{Tipi di movimento}
			Il movimento pu\`o essere:
			\begin{multicols}{2}
				\begin{itemize}
					\item Tumbling: casuale, con continui cambi di direzione.
					\item Swimming: in una sola direzione.
				\end{itemize}
			\end{multicols}

			\paragraph{Distinzione in base a numero e posizione dei flagelli}\mbox{}\\
			\begin{multicols}{2}
				\begin{itemize}
					\item Monotrofico: un solo flagello all'estremit\`a.
					\item Lofotrico: un gruppo di flagelli all'estremit\`a.
					\item Anfitrico: flagelli a estremit\`a opposte.
					\item Peritrico: flagelli distribuiti su tutta la superficie.
				\end{itemize}
			\end{multicols}

	\subsection{Terza giornata}
	In piastra si possono osservare le motilit\`a tramite un alone trasparente attorno alla colonia inoculata.
	Si attua un'analisi tramite microscopio ottico su terreno liquido in modo da poter distinguere i moti browniani da moti di swimming.

		\subsubsection{Procedimento}
		\begin{enumerate}
			\item Si pulisce e sgrassa il vetrino con alcol $70\%$.
			\item Si agita la sospensione batterica e si prelevano $10\si{\micro\litre}$ con la $p20$.
			\item Si depone la goccia al centro del vetrino portaoggetti.
			\item Si pone il vetrino coprioggetti sopra la goccia.
			\item Si rimuove l'eccesso di liquido tamponando con della carta.
			\item Si pone il preparato a fresco sul tavolino portaoggetti del microscopio ottico.
			\item Si osserva la motilit\`a batterica e si annotano le osservazioni come il tipo di movimenti e la quantit\`a di cellule motili.
			\item Si cestina il vetrino nell'apposito contenitore per rifiuti taglienti e acuminati.
		\end{enumerate}
